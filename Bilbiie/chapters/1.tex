%1

\section{La phrase simple}
L'objectif de cette thèse est la description et l'analyse de deux constructions elliptiques qui mettent en jeu une coordination ou une subordination de phrases. L'unité syntaxique phrase est donc au c{\oe}ur de cette thèse. Comme la notion de phrase ne reçoit pas toujours une définition claire, la première partie du chapitre (sections 1.1 et 1.2) est consacrée à l'étude de cette notion, en prenant en compte trois dimensions : syntaxique, sémantique et pragmatique.  

La deuxième partie de ce chapitre (section \ref{sec:1.3}) étudie les principaux phénomènes syntaxiques, nécessaires à la compréhension du fonctionnement de la phrase simple en roumain. Je présente ainsi le complexe verbal du roumain, le phénomène de non-réalisation du sujet (en particulier, le pro-drop), la distribution du sujet par rapport au verbe tête et par rapport aux autres dépendants verbaux et, plus généralement, l'ordre des mots.  

La troisième partie de ce chapitre (section \ref{sec:1.4}) propose une analyse formelle des faits observés dans les sections précédentes dans le cadre de la grammaire syntagmatique guidée par les têtes (angl. \textit{Head-driven Phrase Structure Grammar}), abrégée dorénavant HPSG.

\subsection{La notion de phrase}
\label{bkm:Ref299302849}Une conception répandue analyse la phrase comme l'articulation d'un syntagme nominal sujet et d'un syntagme verbal prédicat, ce que capte la règle de réecriture ou le schéma de constituance SN SV.\footnote{La notion de syntagme verbal prédicat est utilisée par la plupart des théories linguistiques contemporaines, sans avoir toujours de justification empirique (voir une discussion à cet égard dans Abeillé (2002) pour le français). Cette bipartition des structures phrastiques rappelle le découpage classique d'une proposition en un sujet et un prédicat, la notion de syntagme verbal regroupant en gros tout ce qui n'est pas le sujet dans la phrase (en particulier, les compléments du verbe).}  Cependant, ce schéma est loin de capter toutes les occurrences phrastiques. Dans la première partie de ce chapitre, je rappelle les critères nécessaires pour une définition formelle de la notion de phrase, en m'appuyant sur des recherches récentes faites sur le français (\textit{GGF} \textit{en prép.}) et l'anglais (Ginzburg \& \citet{Sag2000}, Huddleston \& \citet{Pullum2002}). 

Dans le cadre HPSG ou des approches qui s'en inspirent, la phrase est un syntagme comportant une tête prédicative dont la valence est saturée. La saturation de la valence est généralement mise en relation avec la fonction syntaxique de sujet. Syntaxiquement, on définit ainsi la phrase comme un syntagme qui n'attend pas de sujet, soit parce que le sujet attendu est réalisé, soit parce que la tête n'attend pas de sujet (\citet{Sag1997}, Ginzburg \& \citet{Sag2000}). 

Crucialement, la phrase contient toujours une tête prédicative. La façon dont on définit cette tête dans les grammaires traditionnelles du roumain est assez restrictive. Elle est dans la plupart du temps mise en relation avec les marques dites de prédicativité, c.-à-d. les marques de mode, temps, personne et nombre, qui permettent l'ancrage situationnel de l'unité syntaxique en question (\textit{GALR} (2005), \textit{GBLR} (2010)). Dans cette perspective, la tête prédicative par excellence est le verbe à un mode fini et, donc, les phrases sont des unités syntaxiques maximales contenant un verbe fini. Selon cette définition, est une phrase toute unité à tête verbale finie, comme \REF{ex:1:1}a, mais ne sont pas des phrases les unités à tête verbale non-finie\footnote{La distinction qu'on peut faire pour le français entre forme finie et forme tensée ne vaut pas a priori pour le roumain. En français, les participes présents et les infinitifs se comportent différemment en ce qui concerne le placement de la négation, cf. \textbf{\textit{ne}}\textit{ venant} \textbf{\textit{pas}} vs. \textbf{\textit{ne pas}}\textit{ venir {\textbar} *}\textbf{\textit{ne}}\textit{ venir} \textbf{\textit{pas}}\textbf{,} ce qui rapproche le participe présent des formes tensées comme l'indicatif ou le subjonctif. En roumain, ce critère n'est pas pertinent, car la négation s'exprime par un seul morphème (pouvant avoir plusieurs réalisations), qui précède toujours le verbe, cf. \textbf{\textit{ne}}\textit{venind} (forme niée de participe présent pour le verbe \textit{a veni} `venir') vs. \textit{a} \textbf{\textit{nu}}\textit{ veni} (forme niée de l'infinitif).} , comme \REF{ex:1:1}b-c, ou encore les unités à tête non-verbale, comme \REF{ex:1:1}d-e. Je me sépare de cette tradition et j'admets que toutes les unités de \REF{ex:1:1} sont des phrases. Autrement dit, la tête prédicative de la phrase peut appartenir à des catégories diverses~et elle est généralement saturée pour son sujet.


\begin{enumerate}
\item \label{bkm:Ref283825287}a  [Maria \textbf{citeşte} un roman].


\end{enumerate}
{\itshape
Maria lit un roman}

  b  [\textbf{A nu se călca} iarba].

{\itshape
Ne pas marcher sur la pelouse} 

  c  [Ploaia \textbf{încetând}], am putut în sfârşit să ies în parc.

{\itshape
La pluie s'arrêtant, j'ai pu enfin sortir dans le parc}

  d  [Toată lumea \textbf{afară}] !

{\itshape
Tout le monde dehors}

  e  [\textbf{Probabil} că mâine plouă].

    \textit{Peut-être que demain il pleut}

Toute phrase appartient à un type phrastique (déclaratif, exclamatif, interrogatif, désidératif\footnote{Je suis Marandin \textit{en prép.} en adoptant la terminologie \textit{type désidératif} au lieu de \textit{type impératif}, afin d'avoir une description uniforme des phrases racines et des phrases subordonnées ne contenant pas de forme verbale à l'impératif.}) qui associe à un ensemble de formes un type de contenu spécifique. Le contenu associé à une phrase est de type \textit{message} (Ginzburg \& \citet{Sag2000}). Je retiens ici la typologie des contenus sémantiques établie par \citet{Marandin2008} : proposition (pour les déclaratives et les exclamatives), question (pour les interrogatives) et visée (pour les impératives).\footnote{Ginzburg \& \citet{Sag2000} ont aussi le type sémantique \textit{fait} pour les exclamatives, mais je ne le retiens pas ici, cf. \citet{Marandin2008}. Plus de détails sur cette typologie sémantique dans la section \ref{sec:1.2.2}.} Par conséquent, le contenu associé à une phrase doit être déterminé. Sur la base de ce critère, on ne peut pas analyser par exemple les interjections comme des phrases, car elles n'ont pas de contenu stable. 

Une phrase peut être indépendante \REF{ex:1:2}a, si elle n'entretient pas de relation syntaxique avec une autre phrase, ou bien liée à une autre phrase par des relations de coordination \REF{ex:1:2}b ou de subordination \REF{ex:1:2}c. Le terme de phrase subordonnée couvre toutes les phrases qui ont une fonction de sujet, complément ou ajout par rapport à une tête lexicale ou syntagmatique de la phrase racine\footnote{La plupart des grammaires traditionnelles utilisent la notion de \textit{phrase principale} pour désigner la phrase tête ou la phrase qui contient le mot tête. Elle peut être séparée d'une subordonnée ayant la fonction ajout (i)a, en revanche elle n'a pas d'autonomie syntaxique sans son complément phrastique (i)b. Comme Abeillé \textit{en prép.} suggère, le terme de \textit{phrase racine} rend mieux compte de cette distribution.
(i)  a  Nu voi veni (dacă plouă). 
    \textit{Je ne viendrai pas (s'il pleut)}
  b  Ion mi-a spus ??(că plouă).
    \textit{Ion m'a dit (qu'il pleuvait)}}. Une classification des phrases en fonction des relations possibles est donnée en \REF{ex:1:3}.


\begin{enumerate}
\item \label{bkm:Ref283825471}a  La Bucureşti ninge.


\end{enumerate}
{\itshape
A Bucarest il neige}

  b  [La Bucureşti ninge], [\textbf{iar} la Braşov plouă].

{\itshape
A Bucarest il neige, et à Braşov il pleut} 

  c  [Ion mi-a spus [\textbf{că} la Bucureşti ninge]].

{\itshape
Ion m'a dit qu'à Bucarest il neigeait}


\begin{enumerate}
\item   \label{bkm:Ref299124709}Classification des phrases (cf. Abeillé et al. \textit{en prép.})


\end{enumerate}
{   [Warning: Image ignored] % Unhandled or unsupported graphics:
%\includegraphics[width=4.5902in,height=0.6516in,width=\textwidth]{fe443409cd384d3fb0f6390ffd77f513-img1.svm}
} 

Toute phrase racine, verbale ou non, peut être un énoncé, c.-à-d. une expression linguistique associée à un acte illocutoire. Il faut noter toutefois que l'inverse n'est pas toujours vrai : un énoncé n'est pas toujours une phrase. Ainsi, les exemples en \REF{ex:1:4} peuvent être considérés comme des énoncés, car chacun a une valeur d'acte illocutoire (exclamation en \REF{ex:1:4}a et injonction en \REF{ex:1:4}b). En revanche, aucun des deux énoncés n'est une phrase, car le contenu associé à ces unités est très peu déterminé.\footnote{L'anglais dispose d'une terminologie moins ambiguë : \textit{utterance // sentence / clause}. \citet{Huddleston1994} parle du phénomène de \textit{desententialization} des subordonnées : une \textit{clause} perd son statut de \textit{sentence} (perte de la valeur d'acte illocutoire, perte de la finitude parfois). En français, on a \textit{énoncé // phrase // proposition} et en roumain \textit{enunț // frază // propoziție}. Dans ces deux langues, le problème réside dans l'ambiguïté du terme \textit{proposition} qui est emprunté à la logique et qui a été spécialisé en sémantique.} 


\begin{enumerate}
\item \label{bkm:Ref283828126}a  Ura !


\end{enumerate}
{\itshape
Hurrah} 

  b  In  picioare !

    dans  pieds

{\itshape
Debout}

\subsection{Types de phrases et types d'énoncés}
\label{bkm:Ref299353216}\label{bkm:Ref299380411}Dans cette section, je discute les types de phrases en lien avec trois dimensions linguistiques : syntaxique, sémantique et pragmatique. Je montre que la corrélation traditionnelle entre types de phrases et types d'actes illocutoires est problématique et qu'on a besoin d'une typologie des phrases en fonction du type de contenu sémantique (cf. Ginzburg \& \citet{Sag2000}). Cette dimension sémantique est la seule qui permette une corrélation constante avec les types de phrase, c.-à-d. à chaque type phrastique est associé de façon biunivoque un type de contenu. Cette approche s'applique de manière uniforme aux phrases racines et aux phrases subordonnées. 

\subsubsection{Typologie des phrases à l'interface syntaxe -- pragmatique} 
Pour définir la phrase, on fait souvent appel à une association un-à-un entre des formes syntaxiques et des usages discursifs ; dans cette perspective, une phrase serait une unité ayant un certain type phrastique (déclaratif, exclamatif, interrogatif, désidératif) et un certain acte illocutoire (assertion, exclamation, interrogation, injonction). Cependant, définir une phrase par une relation biunivoque entre un type phrastique et un type d'acte illocutoire pose beaucoup de problèmes. En particulier, ce genre d'analyse ne permet pas de définir ce qui fait l'identité d'un type.

\paragraph[Syntaxe~: les types phrastiques]{Syntaxe : les types phrastiques}
La notion de type de phrase regroupe des unités qui sont hétérogènes du point de vue syntaxique. Ainsi, le type désidératif regroupe deux sous-types : les phrases finies avec une tête verbale à l'impératif \REF{ex:1:5}a ou au subjonctif \REF{ex:1:5}b, et les phrases non-finies avec une tête prédicative au participe \REF{ex:1:5}c ou à l'infinitif \REF{ex:1:5}d. 


\begin{enumerate}
\item \label{bkm:Ref299127024}a  \textbf{Nu fumați} în această încăpere !


\end{enumerate}
{\itshape
Ne fumez pas dans cette salle}

  b  (Vreau) \textbf{Să se fumeze} afară.

{\itshape
(Je veux) qu'on fume dehors}

  c  Fumatul  \textbf{interzis~}!

    le-fumer  interdit

{\itshape
Interdit de fumer}

  d  \textbf{A nu se fuma} în spațiile publice.

{\itshape
Ne pas fumer dans les espaces publiques}

Le type exclamatif n'est pas homogène non plus. Il regroupe deux sous-types majeurs : (i) le premier présente une expression de degré en position canonique (p.ex. les adverbiaux \textit{aşa {\textbar} atât} `tellement' {\textbar} `si' suivis de la préposition \textit{de}, cf. \REF{ex:1:6}a), et (ii) le deuxième sous-type présente une expression \textit{qu-} extraite (p.ex. les adverbiaux \textit{ce {\textbar} cât de}\footnote{L'expression \textit{cât de} `comme' (suivie d'un adverbe ou d'un adjectif) permet l'extraposition du syntagme en \textit{de} en (i)b.
(i)  a  \textbf{Cât  de  bine}  îi  merge  Mariei !
    \textsc{qu}  de  bien  lui  va  Maria.\textsc{dat}
    \textit{Qu'est-ce qu'elle va bien Maria}
 b  \textbf{Cât}  îi  merge  Mariei  \textbf{de  bine} !
    \textsc{qu}  lui  va  Maria\textsc{.dat}  de  bien
    \textit{Qu'est-ce qu'elle va bien Maria}} `comme' {\textbar} `que de', cf. \REF{ex:1:6}b). On peut ajouter un troisième sous-type qui ne présente aucune expression de degré ou expression \textit{qu-}, mais qui se distingue par l'antéposition d'un adverbe ou d'un adjectif et par la présence (optionnelle) de l'adverbial \textit{mai} (clitique sur le verbe), cf. \REF{ex:1:6}c. 


\begin{enumerate}
\item \label{bkm:Ref299128099}a  Ioana e \{\textbf{atât} {\textbar} \textbf{aşa}\} \textbf{de} frumoasă când râde !


\end{enumerate}
{\itshape
Ioana est si belle quand elle rit}

  b  \{\textbf{Ce} {\textbar} \textbf{cât}  \textbf{de}\}  frumoasă  e  Ioana  când  râde !

    \textsc{qu } de  belle  est  Ioana  quand  rit

{\itshape
Comme Ioana est belle quand elle rit} 

  c  Frumoasă  \textbf{mai}  e  Ioana  când  râde ! 

    Belle  \textsc{adv } est  Ioana  quand  rit

{\itshape
Comme Ioana est belle quand elle rit}

La situation est encore plus délicate avec le type interrogatif et le type déclaratif. Généralement, on admet que le type interrogatif subsume deux sous-types, à savoir les interrogatives partielles, ayant comme identifieur lexical une expression \textit{qu-} qui impose un certain ordre des éléments dans la phrase (comparer \REF{ex:1:7}a-b-c) et les interrogatives totales, qui peuvent avoir comme identifieur le complémenteur \textit{dacă} `si' en emploi subordonné, cf. \REF{ex:1:7}d.  


\begin{enumerate}
\item \label{bkm:Ref299128534}a  \textbf{Ce } a  mâncat  Ion ?


\end{enumerate}
\textsc{qu}  a  mangé  Ion

{\itshape
Qu'est-ce que Ion a mangé~}

  b  *\textbf{Ce } Ion  a  mâncat ?

\textsc{qu}  Ion  a  mangé

{\itshape
Qu'est-ce que Ion a mangé~}

  c  Ion  \textbf{ce } a  mâncat ?

Ion\textsc{  qu}  a  mangé

{\itshape
Ion, qu'est-ce qu'il a mangé~}

  d  Ion  mă  întreabă  \textbf{dacă}  va  veni  Maria.

    Ion  me  demande  si  va  venir  Maria

{\itshape
Ion me demande si Maria viendra}

En revanche, il n'est pas évident de distinguer formellement l'interrogative totale en emploi racine et la phrase déclarative en roumain, car les deux types ne présentent pas d'identifieurs lexicaux ou morphosyntaxiques spécifiques.\footnote{Voir les données de l'italien, qui conduisent vers un même type d'analyse (cf. \citet{Petrone2008}).} Deux différences sont évoquées dans la littérature et doivent être étudiées systématiquement : la position du sujet et la prosodie. Les grammaires du roumain précisent que la position postverbale du sujet est massive avec les interrogatives (par rapport aux déclaratives), mais une étude détaillée doit être faite afin de valider cette hypothèse pour les interrogatives totales en emploi racine.\footnote{S'il s'avère que les interrogatives ont une préférence pour la postposition par rapport aux déclaratives, cela concerne non seulement le sujet, mais aussi les autres dépendants.}  On pourrait bien imaginer que la postposition en phrase interrogative et celle en phrase déclarative soient associées à des propriétés différentes, c'est ce que prédit l'approche constructionnelle des types de phrase. 

Contrairement aux travaux traditionnels, je considère (toujours à la suite de Marandin \textit{en prép.}) que la ponctuation ne dit rien sur les types phrastiques.\footnote{La ponctuation n'est pas un fait de langue, mais plutôt une convention du codage écrit des énoncés.}  Si elle donne une indication, il s'agit plutôt de l'acte illocutoire de l'énoncé en question : le point marque généralement une assertion, le point d'exclamation marque une injonction ou une exclamation, alors que le point d'interrogation indique une interrogation. Ainsi, une même séquence peut être interprétée comme mettant en jeu une assertion \REF{ex:1:8}a, une exclamation \REF{ex:1:8}b ou une interrogation \REF{ex:1:8}c. De même, pour les énoncés en \REF{ex:1:9}a-b (c.-à-d. respectivement assertion et interrogation).


\begin{enumerate}
\item \label{bkm:Ref299129450}a  Va  veni  Maria.


\end{enumerate}
va  venir  Maria

{\itshape
Maria va venir~}

  b  Va veni Maria !

{\itshape
Maria va venir~}

  c  Va veni Maria ?

{\itshape
Maria, viendra-t-elle~}


\begin{enumerate}
\item \label{bkm:Ref299129513}a  Maria va veni mâine. 


\end{enumerate}
{\itshape
Maria va venir demain}

  b  Maria va veni mâine ?

{\itshape
Maria va venir demain}

\paragraph[Pragmatique~: les actes illocutoires]{Pragmatique : les actes illocutoires}
La phrase appartient au domaine de l'énonciation : chaque énoncé racine est associé à un acte illocutoire. On peut ainsi définir la notion d'\textit{énoncé} comme étant le plus petit événement linguistique auquel on associe un acte illocutoire, c.-à-d. l'attitude du locuteur lorsqu'il produit un énoncé. Les types d'actes proposés traditionnellement sont l'assertion (c.-à-d. proposer une proposition à l'assentiment de l'interlocuteur), l'exclamation (c.-à-d. exprimer un jugement à propos d'un individu ou d'une situation), l'interrogation (c.-à-d. soumettre une question à l'interlocuteur) et l'injonction (c.-à-d. demander à l'interlocuteur de faire en sorte qu'une certaine situation devienne effective), cf. Marandin \textit{en prép}.  

La dimension discursive comme critère typologique des phrases est problématique. D'une part, uniquement les phrases en emploi racine ont une valeur d'acte ; les phrases enchâssées (comme la complétive \textit{că mâine va ninge} `que demain il va neiger' en \REF{ex:1:10}a) ne sont pas associées à un acte illocutoire. En même temps, les actes illocutoires peuvent être accomplis par des unités comme en \REF{ex:1:10}b qui ne constituent pas des phrases selon les critères syntaxiques qu'on a discutés plus haut dans la section \ref{sec:1.1}. 


\begin{enumerate}
\item \label{bkm:Ref299093531}a  Maria crede [că mâine va ninge].


\end{enumerate}
{\itshape
Maria croit que demain il va neiger}

  b  Foc !

{\itshape
Feu}

Il est généralement admis (cf. \citet{Gazdar1981}) que les types phrastiques sont {\guillemotleft}~polyfonctionnels~{\guillemotright}, c.-à-d. un même type phrastique (par exemple, une déclarative comme en \REF{ex:1:11}) peut donner lieu en contexte à une assertion, une interrogation ou une injonction, ce qui est mis en évidence par la réponse de l'interlocuteur B\textsubscript{1}, B\textsubscript{2} et respectivement B\textsubscript{3} en \REF{ex:1:11}. Par conséquent, on ne peut pas associer de manière biunivoque et constante un type phrastique et un type d'acte. 


\begin{enumerate}
\item \label{bkm:Ref299093669}\label{bkm:Ref299093611}A~  Vei pleca mâine la prima oră\{. {\textbar} ? {\textbar} !\}


\end{enumerate}
{\itshape
Tu partiras demain à la première heure}

  B\textsubscript{1 } Cine ți-a spus ?

{\itshape
Qui te l'a dit}

  B\textsubscript{2}  Nu ştiu încă sigur.

{\itshape
Je ne suis pas encore sûr}

  B\textsubscript{3}  Ok, sunt de acord.

{\itshape
Ok, je suis d'accord}

Inversement, il y a des cas qui combinent simultanément un acte principal et un acte secondaire dit {\guillemotleft}~indirect~{\guillemotright} obtenu par inférence. C'est par exemple le cas des interrogatives injonctantes : une interrogative en contexte comme \REF{ex:1:12}a doit être prise par l'interlocuteur comme une injonction \REF{ex:1:12}b. La valeur d'acte associé à un type phrastique est généralement conventionalisée dans ces cas.


\begin{enumerate}
\item \label{bkm:Ref299095055}a  Fiți amabil, ați putea să-mi spuneți cât este ceasul ?


\end{enumerate}
{\itshape
Pourriez-vous me dire l'heure, s'il vous plaît}

  b  Spuneți-mi cât este ceasul !~

{\itshape
Dites-moi l'heure}

\subsubsection{La sémantique : les types de contenu}
\label{bkm:Ref299091554}Pour éviter les problèmes mentionnés plus haut, on adopte, à la suite de Ginzburg \& \citet{Sag2000}, une triple classification des phrases. Chaque phrase se définit ainsi par : (i) un type phrastique, (ii) un type de contenu sémantique, et (iii) un type d'acte illocutoire\textbf{} (le dernier uniquement pour les phrases racines). Dans cette approche tridimensionnelle, une place centrale est donnée aux types de contenu sémantique, étant les seuls à pouvoir rendre compte de l'unité d'un type phrastique. S'il n'y a pas d'association stable entre un type phrastique et un type d'acte illocutoire, on peut tout de même postuler (cf. Ginzburg \& \citet{Sag2000}, Beyssade \& \citet{Marandin2006}) qu'à chaque type phrastique est associé de façon biunivoque un type de contenu stable (c.-à-d. un type de \textit{message}), qu'il s'agisse d'une phrase racine ou d'une phrase subordonnée : \textit{{\guillemotleft}~The clause is a special kind of construction that correlates a particular syntactic combination with a kind of message. Messages are the semantic kinds most fundamental to communication. {\guillemotright}} (Ginzburg \& \citet[10]{Sag2000}). Dans le tableau suivant, je liste les types distingués pour chacun des trois paramètres mentionnés précédemment : 


\begin{table}


\begin{tabular}{lll}

TYPE PHRASTIQUE & CONTENU SEMANTIQUE & ACTE ILLOCUTOIRE\\
Déclaratif & Proposition & Assertion\\
Exclamatif & Poposition-exclamative & Exclamation\\
Interrogatif & Question & Interrogation\\
Désidératif & Visée & Injonction\\
\end{tabular}

\caption{}
%\label{}
\end{table}

On considère qu'à chaque type et sous-type de phrase peuvent correspondre plusieurs structures syntaxiques, plusieurs types d'actes illocutoires, mais un seul type de contenu. 

Une phrase déclarative a un contenu de type \textit{proposition}, c.-à-d. une description de situation qui est susceptible d'être vraie ou fausse. Une proposition n'obéit à aucune restriction sur le contenu ou la source du contenu.

Le contenu d'une phrase exclamative est aussi \textit{propositionnel}, avec la mention qu'il s'agit d'une quantification de haut degré, la source du contenu étant l'intime conviction du locuteur (cf. \citet{Marandin2008} pour le français).

Une phrase interrogative a un contenu de type \textit{question}, c.-à-d. une description de situation dont la valeur de vérité est laissée en suspens (= \textit{question totale {\textbar} polaire {\textbar} fermée}) ou dont un paramètre est {\guillemotleft}~abstrait~{\guillemotright}, c.-à-d. il n'est pas spécifié (= \textit{question partielle {\textbar} qu- {\textbar} ouverte}). Les questions sont résolvables (dans le monde actuel) ou non.

Une phrase impérative a un contenu de type \textit{visée}, c.-à-d. une description de situation dont le paramètre temporel n'est pas spécifié ; elle est future par rapport à la situation présente. Les visées sont réalisables (dans un monde possible) ou non. 

Dans une théorie de la logique des prédicats, cela revient à dire que les types déclaratif et exclamatif sont sémantiquement des \textit{propositions}, alors que les types interrogatif et impératif sont des \textit{abstractions propositionnelles}, n'ayant pas de contenu susceptible d'être vrai ou faux. Dans ces derniers cas, le contenu est modélisé comme des abstractions simultanées sur un ensemble de paramètres, dont la cardinalité n'est pas contrainte. L'abstraction propositionnelle peut se faire sur 0 paramètres (et on obtient ainsi le contenu d'une question polaire), sur 1 paramètre (c'est le cas d'une question unaire et d'une visée) ou bien sur 2 paramètres ou plus (comme pour les questions binaires ou multiples).

On observe ainsi que le contenu d'une phrase n'est pas toujours une proposition. Le fait de postuler un type sémantique pour chaque type phrastique est fondé empiriquement sur quatre critères.

(i) Un premier critère concerne la sélection des compléments phrastiques par le verbe de la phrase racine (\citet{Grimshaw1979}, Ginzburg \& \citet{Sag2000}, \citet{Marandin2008}). On observe ainsi que le verbe de la phrase racine se caractérise par des restrictions sélectionnelles quant au type de son complément phrastique. En français par exemple, un verbe comme \textit{penser} prend seulement un complément de type \textit{que +} Phrase (il ne sélectionne donc jamais de phrase interrogative) \REF{ex:1:13}a, tandis qu'un verbe comme \textit{se demander} prend uniquement des interrogatives \REF{ex:1:13}b. Les mêmes propriétés de sélection caractérisent les verbes en roumain : le verbe \textit{a crede} `penser' sélectionne une phrase dont le contenu sémantique est une proposition \REF{ex:1:14}a, le verbe \textit{a admira} `admirer' prend un complément phrastique dont le contenu est une proposition exclamative \REF{ex:1:14}b, le verbe \textit{a întreba} `demander' n'est compatible qu'avec une question \REF{ex:1:14}c, tandis que le verbe \textit{a vrea} `vouloir' prend une subordonnée dont le contenu est une visée \REF{ex:1:14}d.


\begin{enumerate}
\item \label{bkm:Ref299268979}a  Jean \textbf{pense} \{que Paul a souffert {\textbar} *comme Paul a souffert {\textbar} *si Paul a souffert\}.


\end{enumerate}
  b  Jean \textbf{se demande} \{*que Paul a souffert {\textbar} *comme Paul a souffert {\textbar} si Paul a souffert\}.


\begin{enumerate}
\item \label{bkm:Ref299269517}a  Ion \textbf{crede} \{că va ploua {\textbar} *ce frumos plouă {\textbar} *când va ploua {\textbar} *să plouă\}.


\end{enumerate}
{\itshape
Ion croit \{qu'il pleuvra {\textbar} comme il pleut {\textbar} quand il pleuvra {\textbar} qu'il pleuve\}}

  b  \textbf{Admiră} \{*că va ploua {\textbar} ce frumos plouă {\textbar} *când va ploua {\textbar} *să plouă\} !

{\itshape
    Admire \{qu'il pleuvra {\textbar} comme il pleut {\textbar} quand il pleuvra {\textbar} qu'il pleuve\} } 

  c  Ion \textbf{întreabă} \{*că va ploua {\textbar} *ce frumos plouă {\textbar} când va ploua {\textbar} *să plouă\}.

{\itshape
    Ion demande \{qu'il pleuvra {\textbar} comme il pleut {\textbar} quand il pleuvra {\textbar} qu'il pleuve\}}

  d  Ion \textbf{vrea} \{*că va ploua {\textbar} *ce frumos plouă {\textbar} *când va ploua {\textbar} să plouă\}.

{\itshape
    Ion veut \{qu'il pleuvra {\textbar} comme il pleut {\textbar} quand il pleuvra {\textbar} qu'il pleuve\}}

(ii) Un deuxième test, utilisé pour l'anglais (Ginzburg \& \citet{Sag2000}), repose sur l'emploi classifiant des termes : \textit{claim, fact, question, outcome ...} (en anglais). Le test utilisant les noms peut être étendu aux données du roumain. Ainsi, un nom comme \textit{convingerea} `la conviction' n'est compatible qu'avec une proposition \REF{ex:1:15}a ; un nom comme \textit{întrebarea} `la question' exige une subordonnée ayant un contenu de type question \REF{ex:1:15}b, alors qu'un nom comme \textit{dorința} `le désir' prend une subordonnée au subjonctif ayant un contenu de type visée \REF{ex:1:15}c.


\begin{enumerate}
\item \label{bkm:Ref299272048}a  \textbf{Convingerea} mea e \{că vom reuşi {\textbar} *cine va reuşi {\textbar} *să reuşeşti\}.


\end{enumerate}
{\itshape
Ma conviction est \{que nous réussirons {\textbar} qui réussira {\textbar} que tu réussisses\}}

  b  \textbf{Intrebarea} e \{*că vom reuşi {\textbar} cine va reuşi {\textbar} *să reuşeşti\}.

{\itshape
    La question est \{que nous réussirons {\textbar} qui réussira {\textbar} que tu réussisses\} } 

  c  \textbf{Dorința} mea e \{*că vom reuşi {\textbar} *cine va reuşi {\textbar} să reuşeşti\}.

    \textit{Mon désir est \{que nous réussirons {\textbar} qui réussira {\textbar} que tu réussisses\}}

(iii) Un troisième critère touche aux propriétés de sélection sémantique des adverbes évaluatifs (Bonami \& \citet{Godard2005}, \citet{Laurens2007}),~comme \textit{heureusement} ou \textit{malheureusement} en français. Ces adverbes modifient toujours des phrases ayant comme contenu sémantique une proposition ; par conséquent, on observe qu'ils sont compatibles uniquement avec des phrases déclaratives et exclamatives \REF{ex:1:16}a-b. Les mêmes observations tiennent pour les expressions adverbiales correspondantes en roumain ; une expression comme \textit{din păcate} `malheureusement' se combine avec une phrase déclarative (racine \REF{ex:1:17}a ou enchâssée \REF{ex:1:17}b) ou bien avec une phrase exclamative \REF{ex:1:17}c, mais pas avec une phrase interrogative \REF{ex:1:17}d ou désidérative \REF{ex:1:17}e. Ce test a l'avantage de s'appliquer à une variété de phrases, à la différence du premier test qui s'applique uniquement aux syntagmes compléments de verbes.


\begin{enumerate}
\item \label{bkm:Ref299273432}a  \textbf{Heureusement}, Marie viendra.


\end{enumerate}
  b  \textbf{Heureusement}, quelle bonne idée elle a eue !

  c  \#\textbf{Heureusement}, Marie va-t-elle venir ?

  d  \#\textbf{Heureusement}, viens ici !


\begin{enumerate}
\item \label{bkm:Ref299273362}a  \textbf{Din păcate}, Ion nu poate veni.


\end{enumerate}
{\itshape
Malheureusement, Ion ne peut pas venir}

  b  Maria mi-a spus că, \textbf{din păcate}, Ion nu poate veni.

{\itshape
Maria m'a dit que, malheureusement, Ion ne pourrait pas venir}

  c  Ce neşansă a avut Ion, \textbf{din păcate} !

{\itshape
Quelle malchance a eu Ion, malheureusement}

  d  \#\textbf{Din păcate}, de ce nu poate veni Ion ?

{\itshape
Malheureusement, pourquoi Ion ne peut pas venir~}

  e  \#\textbf{Din păcate}, să doarmă Ion !

{\itshape
Malheureusement, que Ion dorme~}

(iv) Un dernier critère qu'on peut mentionner concerne les propriétés de sélection de certains ajouts illocutoires (Beyssade \& \citet{Marandin2006}). Ainsi, en français, l'ajout \textit{sans indiscrétion} est compatible avec des phrases dont le contenu sémantique est une question, ce qui explique l'acceptabilité de \REF{ex:1:18}a et \REF{ex:1:18}c et l'inacceptabilité de \REF{ex:1:18}b et \REF{ex:1:18}d ; de même, l'ajout \textit{n'est-ce pas} se combine avec des phrases déclaratives \REF{ex:1:19}a et exclamatives \REF{ex:1:19}b, mais pas avec des interrogatives \REF{ex:1:19}c ou désidératives \REF{ex:1:19}d (cf. \citet{Laurens2007}). 


\begin{enumerate}
\item \label{bkm:Ref299274566}a  \textbf{Sans indiscrétion}, Marie viendra ?


\end{enumerate}
  b  \#\textbf{Sans indiscrétion}, quel beau livre elle a ! 

  c  \textbf{Sans indiscrétion}, Marie va-t-elle venir ?

  d  \#\textbf{Sans indiscrétion}, viens ici.


\begin{enumerate}
\item \label{bkm:Ref299274591}a  Marie viendra, \textbf{n'est-ce pas} ?


\end{enumerate}
  b  Quel beau livre elle a, \textbf{n'est-ce pas} ?

  c  \#Marie va-t-elle venir, \textbf{n'est-ce pas} ? 

  d  \#Viens ici, \textbf{n'est-ce pas} ?

\subsubsection{Propriétés illocutoires des types phrastiques}
Si on veut tout de même établir une association entre un type phrastique et un type d'acte, on doit redéfinir la notion d'acte illocutoire, en prenant en compte la dimension intéractive (ou dialogique) d'un énoncé, et non seulement l'action spécifique du locuteur (cf. Beyssade \& \citet{Marandin2006}, Marandin \textit{en prép.}). Dans la perspective de ces auteurs, le locuteur s'engage sur le contenu de son énoncé et attend de son interlocuteur qu'il réagisse par rapport au contenu de l'énoncé en question. Ce qui justifie la décomposition de l'acte illocutoire en deux parties : l'engagement du locuteur (angl. \textit{commitment}) et l'appel au destinataire (angl. \textit{call-on-addressee}).\footnote{Cette bipartition de l'acte illocutoire rend mieux compte des actes indirects. Selon Marandin \textit{en prép.}, le type de phrase ne détermine pas un type d'acte, mais l'attitude dialogique du locuteur. Par conséquent, un acte indirect du type (i) serait un acte où l'engagement du locuteur n'est pas identique à l'appel au destinataire.
(i)  Nu vă supărați, aveți un ceas ? (= Cât este ceasul, vă rog?) 
  \textit{Excusez-moi, vous avez l'heure} (= \textit{Quelle heure il est})
} Je reprends la typologie telle qu'elle a été établie par Marandin \textit{en prép~}:





\begin{tabular}{lll}

Type phrastique & Type d'engagement du locuteur & Type d'appel à l'interlocuteur\\
Déclaratif & prêt à ajouter le contenu de la proposition à l'univers de discours partagé & faire entrer le contenu de la proposition dans l'univers de discours partagé\\
Exclamatif & intimement convaincu / garant de la vérité de la proposition & être le témoin de l'énonciation\\
Interrogatif & intéressé par la résolution de la question & être intéressé par la résolution de la question\\
Désidératif & favorable à la réalisation de la visée & être favorable à la réalisation de la visée\\
\end{tabular}


La structure bipartite d'un acte illocutoire peut être justifiée empiriquement. D'une part, le fait de postuler une typologie des engagements du locuteur explique pourquoi les énoncés en \REF{ex:1:20} sont inappropriés : dans la deuxième partie de l'énoncé, le locuteur se contredit en déniant l'engagement qu'il vient de faire en énonçant la première phrase.


\begin{enumerate}
\item \label{bkm:Ref299275188}a  \#Maria e foarte isteață. Pe de altă parte, nu cred că e foarte isteață.


\end{enumerate}
{\itshape
Maria est très intelligente. D'ailleurs, je ne crois pas qu'elle soit très intelligente}

  b  \#Cât de isteață e Maria ! Pe de altă parte, nu sunt convins că e isteață.

{\itshape
Comme Maria est intelligente. D'ailleurs, je ne suis pas convaincu qu'elle soit intelligente}

  c  \#Maria e oare isteață ? Pe de altă parte, nu mă interesează să ştiu dacă e isteață.

{\itshape
Est-ce que Maria est intelligente ? D'ailleurs, je ne suis pas intéressé de savoir si elle est intelligente} 

  d  \#Fii isteață ! Pe de altă parte, nu vreau să fii isteață.

{\itshape
Sois intelligente ! D'ailleurs, je ne veux pas que tu sois intelligente}

D'autre part, l'existence d'une typologie des appels à l'interlocuteur explique les propriétés de sélection de certains ajouts illocutoires (Beyssade \& \citet{Marandin2006}) qu'on a vus dans la section précédente (voir les exemples \REF{ex:1:18} et \REF{ex:1:19}) : il y a des ajouts de catégories diverses qui sont sensibles au type d'appel à l'interlocuteur dans un énoncé. Ainsi, en roumain, l'ajout \textit{n'est-ce pas}~`n'est-ce pas' en \REF{ex:1:21} est compatible avec tout énoncé de type propositionnel (phrase déclarative ou exclamative), avec une valeur questionnante (le locuteur demande à son interlocuteur de traiter son énoncé comme une interrogation, il s'agit en particulier d'une demande de confirmation). De même, l'adverbial \textit{oare} `est-ce que' en \REF{ex:1:22} est compatible avec tout énoncé dont la valeur d'acte est questionnante (il a une distribution libre dans une phrase déclarative ou interrogative, mais est exclu dans les exclamatives et les impératives).\footnote{L'adverbial \textit{oare} a une distribution libre dans la phrase et il est compatible avec les deux types de questions, totales ou partielles.
(i)  (Oare)  Va  veni  (oare)  Ion  (oare)  la  petrecere  (?oare) ?
  \textsc{interr}  va  venir  \textsc{interr}  Ion  \textsc{interr}  à  fête  \textsc{interr}
  \textit{Est-ce que Ion viendra à la fête}
(ii)  (Oare)  Cine  (oare)  va veni  (oare)  la  petrecere  (?oare) ?
  \textsc{interr}  qui  \textsc{interr}  va venir  \textsc{interr}  à  fête  \textsc{interr}
\textsc{ } \textit{Qui viendra à la fête}}


\begin{enumerate}
\item \label{bkm:Ref299275662}a  (Ion) va veni (Ion), \textbf{nu-i aşa} ?


\end{enumerate}
{\itshape
Ion viendra, n'est-ce pas}

  b  Ninge atât de frumos afară, \textbf{nu-i a}şa ?

{\itshape
Il neige tellement bien, n'est-ce pas}

  c  \#Cine va veni, \textbf{nu-i aşa} ?

{\itshape
Qui va venir, n'est-ce pas}

  d  \#Vino mâine, \textbf{nu-i aşa} ?

{\itshape
Viens demain, n'est-ce pas~}


\begin{enumerate}
\item \label{bkm:Ref299275711}a  (\textbf{Oare})  va veni  (\textbf{oare})  Ion  (\textbf{oare}) ?


\end{enumerate}
\textsc{(interr) } va venir  (\textsc{interr)}  Ion  (\textsc{interr)} 

{\itshape
Ion viendra-t-il} 

  b  \#Vino acum \textbf{oare} !

{\itshape
Viens maintenant}

  c  \#Ce frumos ninge \textbf{oare} !

{\itshape
Comme il neige bien}

La prosodie aussi semble être sensible au type d'appel à l'interlocuteur dans un énoncé. A priori, la prosodie non-descendante spécifierait une valeur questionnante, alors que la prosodie descendante s'associerait avec une valeur assertante, mais il reste à vérifier cette hypothèse par une description précise de ces traits prosodiques.

Ces deux facettes de l'acte illocutoire permettent aussi de rendre compte des actes indirects, lorsque l'engagement du locuteur et celui de l'interlocuteur ne sont pas symétriques. Ainsi, un énoncé interrogatif comme \REF{ex:1:23} doit être interprété par l'interlocuteur comme une injonction, et non pas comme une interrogation.


\begin{enumerate}
\item \label{bkm:Ref299096317}Puteți să închideți fereastra, vă rog ?


\end{enumerate}
{\itshape
  Pouvez-vous fermer la fenêtre, s'il vous plaît}

En conclusion, la phrase en tant qu'énoncé possible fait intervenir trois dimensions linguistiques : la syntaxe, la sémantique et la pragmatique. Les phrases se regroupent en plusieurs types phrastiques, chacun ayant associé un certain contenu sémantique. En même temps, pour chaque type on peut définir un type d'engagement du locuteur et un type d'appel à l'interlocuteur par rapport au contenu de l'énoncé en question. Une typologie des phrases basée sur ces trois niveaux permet de rendre compte non seulement des phrases-énoncés typiques, mais aussi des phrases qui ne présentent pas d'acte illocutoire.

\subsection{Syntaxe de la phrase simple en roumain}
\label{bkm:Ref299354108}Dans cette section, je discute brièvement les principales particularités syntaxiques du roumain. 

Dans un premier temps, je présente les éléments qui font partie du complexe verbal en roumain : formes verbales, formes pronominales, formes adverbiales et marques modales. En ce qui concerne les formes verbales, je montre le statut syntaxique différent (i) des auxiliaires de temps et modes composés, (ii) des verbes modaux et (iii) des verbes attributifs (en particulier, l'auxiliaire de la voix passive), qui imposent trois analyses différentes : (i) une structure à complexe verbal (c.-à-d. les formes verbales composées forment un sous-constituant en syntaxe), (ii) une structure plate (c.-à-d. tous les constituants sont au même niveau) et (iii) une structure hiérarchique (c.-à-d. il y a un verbe tête qui sous-catégorise les autres constituants). En ce qui concerne les formes pronominales, je fais la distinction entre les pronoms au nominatif et les pronoms à l'accusatif ou datif. Les premiers sont des pronoms {\guillemotleft}~forts~{\guillemotright}, alors que les autres sont~{\guillemotleft}~faibles~{\guillemotright} et donc doivent être analysés comme des affixes verbaux. La même analyse s'étend aux autres éléments restants (la négation \textit{nu}, les adverbiaux dits d'intensité et les marques modales de l'infinitif et respectivement du subjonctif) : ils ont un statut affixal. 

Dans un deuxième temps, je présente les situations dans lesquelles un sujet n'est pas réalisé en roumain (y compris le phénomène de pro-drop), la position préverbale et postverbale du sujet et, plus généralement, l'ordre des dépendants verbaux. 

Les points théoriques de la section \ref{sec:1.3} sont les suivants :

- A part le complexe verbal, le roumain se caractérise par un ordre relativement libre des mots dans la phrase.

- La linéarisation des constituants n'est pas corrélée aux fonctions syntaxiques. En particulier, la position préverbale semble être corrélée à des facteurs discursifs plutôt que syntaxiques. 

- Il n'y a pas de motivation empirique pour considérer le sujet dans une position {\guillemotleft}~hiérarchique~{\guillemotright} par rapport aux autres dépendants verbaux.

- Il n'y a pas de motivation empirique pour postuler un syntagme verbal fini, comme catégorie intermédiaire entre le verbe lexical et la phrase.

- Rien ne nous empêche de proposer une structure {\guillemotleft}~plate~{\guillemotright} pour la phrase simple roumaine, le sujet et les compléments se trouvant au même niveau.

\subsubsection{Le complexe verbal en roumain}
\label{bkm:Ref299309451}Le complexe verbal roumain est constitué par une séquence d'éléments qui s'organisent autour d'un verbe dans un ordre rigide et qui semblent se comporter comme une seule unité en syntaxe (Dobrovie-Sorin (1987, 1994), Barbu (1999, 2004), Monachesi (1999, 2000, 2005), Abeillé \& \citet{Godard2003}). Maximalement, cette séquence d'éléments contient les deux marques de mode (la marque \textit{a} pour l'infinitif et la marque \textit{să} pour le subjonctif), la négation \textit{nu}, les pronoms atones de datif et accusatif, les auxiliaires et quelques adverbes d'intensité. La linéarisation de tous ces éléments est donnée en \REF{ex:1:24}\footnote{Voir \citet{Barbu1999}, note 4, pour les exceptions à cette règle d'ordre linéaire.}. On observe ainsi que le complexe verbal roumain est un système plus riche que celui des autres langues romanes, et en quelque sorte similaire à celui rencontré dans les langues slaves du sud, comme le bulgare et le macédonien (cf. \citet{Monachesi2005}). 


\begin{enumerate}
\item \label{bkm:Ref276915670}a  MARQ\textsc{\textsubscript{mode}} {\textless} NEG {\textless} CL\textsc{\textsubscript{dat}} {\textless} CL\textsc{\textsubscript{acc}} {\textless} AUX {\textless} ADV\textsc{\textsubscript{degre}} {\textless} \{fi\} {\textless} verbe lexical


\end{enumerate}
  b  MARQ\textsc{\textsubscript{mode}} = \{\textit{a, să}\}  

  c  ADV\textsc{\textsubscript{degre}} = \{\textit{cam} `un peu'\textit{, mai} `encore'\textit{, şi} `aussi'\textit{, tot} `encore'\textit{, prea} `très'\}


\begin{enumerate}
\item a  Nu  l-am  mai fi  văzut.


\end{enumerate}
    \textsc{neg cl}\textsc{\textsubscript{acc}}\textsc{-aux adv perf} vu

    \textit{Je ne l'aurais plus vu}  

  b  Să     nu  i-l     mai  dea ! 

    \textsc{mrq}\textsc{\textsubscript{subj}}\textsc{ neg cl}\textsc{\textsubscript{dat}}\textsc{-cl}\textsc{\textsubscript{acc}}\textsc{ adv} donner.\textsc{subj.3sg}

    \textit{Qu'il ne le lui donne plus}  

En plus de suivre une linéarisation rigide, ces éléments se comportent comme une unité compacte ayant comme propriété l'adjacence stricte, ce qui explique l'impossibilité d'insérer un autre élément entre les items mentionnés en \REF{ex:1:24}. Bien que le roumain soit une langue à ordre de mots relativement libre, on ne peut pas insérer un sujet \REF{ex:1:26}a, un adverbe ordinaire \REF{ex:1:26}b ou un pronom quantifieur \REF{ex:1:26}c parmi les éléments du complexe verbal. 


\begin{enumerate}
\item \label{bkm:Ref276918916}a  (Ioana)  i-a  (*Ioana)  mai  (*Ioana)  dat  (Ioana)  ceva  (Ioana).


\end{enumerate}
(Ioana)  \textsc{cl-aux } (Ioana)  \textsc{adv  (}Ioana)  donné  (Ioana)  quelque-chose (Ioana)

    \textit{Ioana lui avait encore donné quelque chose}  

  b  (Mereu)  i-a  (*mereu)  tot  (*mereu)  dat  (mereu)  câte ceva  (mereu).

    (toujours)  \textsc{cl-aux  (}toujours)  \textsc{adv } (toujours)  donné  (tjrs)  qq-ch  (tjrs)

    \textit{Il lui a toujours donné quelque chose}

  c  (Toți)  i-am  (*toți)  mai  (*toți)  cântat  (toți)  un  cântec  (toți).

    (tous)  \textsc{cl-aux } (tous)  \textsc{adv } (tous)  chanté  (tous)  une  chanson  (tous)

    \textit{On lui a tous chanté encore une chanson}

Pour rendre compte des propriétés spécifiques des éléments précédant le verbe lexical dans un complexe verbal, on a souvent fait appel à la notion de \textit{clitique} (Dobrovie-\citet{Sorin1994}, Barbu (1999, 2004), Monachesi (1999, 2000, 2005), Miller \& \citet{Monachesi2003}, Abeillé \& \citet{Godard2003}). Traditionnellement, on considère les clitiques comme une classe intermédiaire entre les mots et les affixes. Ils semblent avoir une plus grande autonomie que les affixes, mais en même temps, contrairement aux mots, ils s'attachent phonologiquement à un hôte, formant un seul mot prosodique avec celui-ci. Cette notion trop générale de clitique est rendue plus explicite dans la distinction établie par \citet{Zwicky1977} entre les clitiques \textit{simples} et les clitiques \textit{spéciaux}, distinction reprise par Dobrovie-\citet{Sorin1994} qui oppose les clitiques \textit{phonologiques} aux clitiques \textit{syntaxiques}. Les clitiques simples ou phonologiques sont des items avec une {\guillemotleft}~prosodie défaillante~{\guillemotright}, qui sont obligatoirement adjacents à leur hôte syntaxique et qui ont généralement une contrepartie non-clitique avec les mêmes propriétés syntaxiques. En revanche, les clitiques spéciaux ou syntaxiques sont des items avec une distribution spécifique, ayant des propriétés syntaxiques différentes par rapport à leurs contreparties libres autonomes. Par conséquent, les deux types de clitiques ne reçoivent pas la même analyse syntaxique : les clitiques simples restent des mots indépendants sur le plan syntaxique, avec la particularité qu'ils s'attachent phonologiquement à un hôte syntaxique~pour former un seul mot prosodique ; les clitiques spéciaux perdent leur autonomie lexicale, ils ne forment pas d'unité syntaxique, donc ils se comportent comme des affixes, ce qui impose un traitement en morphologie et non en syntaxe. Pour l'analyse du complexe verbal roumain, j'ai besoin de cette dichotomie que, par simplicité, je réduirai aux notions de \textit{clitique} (pour tout élément attaché phonologiquement à un hôte syntaxique) et \textit{affixe} (uniquement pour les éléments qui perdent l'autonomie et la distribution spécifiques à la catégorie syntaxique en question).  

\paragraph[Les formes verbales composées]{Les formes verbales composées}
Toute étude sur le complexe verbal doit rendre compte de la distribution des formes verbales composées, qui forment ce que Abeillé \& \citet{Godard2003} appellent un prédicat complexe (c.-à-d. une suite de verbes fonctionnant comme un seul domaine vis-à-vis de certaines propriétés syntaxiques ou sémantiques).

Le test syntaxique utilisé par Abeillé \& \citet{Godard2003} pour l'identification de ces formes verbales composées dans les langues romanes est la {\guillemotleft}~montée~{\guillemotright} du clitique (c.-à-d. il s'agit d'un clitique qui apparaît sur un hôte, ici un verbe, qui ne le sous-catégorise pas directement, mais qui est la tête du verbe qui le sous-catégorise). Dans tous les exemples roumains donnés en \REF{ex:1:27}, le verbe qui sous-catégorise ces clitiques {\guillemotleft}~montés~{\guillemotright} est toujours une forme verbale non-finie (infinitif, participe passé ou supin)\footnote{On pourrait y ajouter le gérondif (= le participe présent) utilisé pour le mode présomptif, que je n'aborderai pas ici.} . Il y a principalement trois types de structures dans lesquelles on trouve des formes verbales composées avec la {\guillemotleft}~montée~{\guillemotright} du clitique\footnote{Tous les clitiques apparaissent sur la première forme verbale, sauf le clitique accusatif féminin singulier \textit{o} qui apparaît en distribution quasi-complémentaire : tantôt sur la première forme verbale (\textit{am citit-}\textbf{\textit{o}} `je l'ai lue'\textit{ / aş citi-}\textbf{\textit{o}} `je la lirais'), tantôt sur le verbe lexical (\textbf{\textit{o}}\textit{ voi citi} `je vais la lire'\textit{ /} \textbf{\textit{o}}\textit{ pot citi} `je peux la lire'\textit{ /} \textbf{\textit{o}}\textit{ am de citit} `je l'ai à lire'). Pour des explications sur cette distribution, consulter \citet{Monachesi2005}.} : (i) les modes ou temps composés \REF{ex:1:27}a obtenus à l'aide de l'auxiliaire \textit{a avea} `avoir' (s'il est suivi d'un participe passé, on obtient le passé composé de l'indicatif ; s'il est suivi d'un verbe à l'infinitif, on obtient le conditionnel présent) et \textit{a vrea} `vouloir' (suivi d'un infinitif pour former le futur), (ii) les modaux \REF{ex:1:27}b\textit{ a putea} `pouvoir' et \textit{a avea de} `avoir à', et enfin (iii) l'auxiliaire \textit{a fi} `être' utilisé pour la voix passive \REF{ex:1:27}c.\footnote{Je laisse de côté la forme \textit{fi} `être' qui est employée comme marque aspectuelle (du parfait) ou modale (du présomptif). La marque aspectuelle \textit{fi} est toujours adjacente au verbe lexical non-fini (elle peut être séparée de celui-ci uniquement par deux affixes adverbiaux \textit{tot} `encore' et \textit{şi} `aussi'). \citet{Barbu1999} l'analyse comme un affixe du verbe non-fini, qui est la réalisation du trait [ASPECT \textit{parfait}].}  


\begin{enumerate}
\item \label{bkm:Ref276988213}a  L-\textbf{am}  citit.     / L-\textbf{aş}     citi.  / Il   \textbf{voi}     citi.


\end{enumerate}
\textsc{cl.acc-}avoir.\textsc{aux } lire.\textsc{part / cl.acc-}avoir.\textsc{aux} lire.\textsc{inf / cl.acc} vouloir.\textsc{aux} lire.\textsc{inf}

    \textit{Je l'ai lu / Je le lirais / Je vais le lire}   

  b  Il     \textbf{pot}     citi. / Il    \textbf{am}    de  citit.

    \textsc{cl.acc} peux.\textsc{1sg} lire / \textsc{cl.acc} avoir.\textsc{aux  mrq} lire.\textsc{supin}

    \textit{Je peux le lire / Je l'ai à lire}

  c  Cartea  îi     \textbf{este} recomandată  de  profesor.

    livre.\textsc{def cl.dat} est  recommandée par professeur

    \textit{Le livre lui est recommandé par le professeur}

Bien que toutes ces formes verbales composées soient compatibles avec la {\guillemotleft}~montée~{\guillemotright} du clitique, elles n'ont pas un comportement homogène vis-à-vis de certaines propriétés structurales comme l'extraction, la portée sur une coordination, la possibilité d'avoir une ellipse du verbe lexical ou encore l'insertion d'un autre élément (p.ex. sujet inversé, adverbe ordinaire, quantifieur) entre les deux formes verbales (Abeillé \& Godard~(2003), Dobrovie-\citet{Sorin1994}).

Ces tests permettent d'établir le fonctionnement complètement différent des auxiliaires de temps, par rapport aux modaux et à l'auxiliaire du passif. Les exemples \REF{ex:1:28} à \REF{ex:1:33} sont constitués des paires de trois exemples où le premier contient l'auxiliaire \textit{a avea} `avoir' du passé composé, le deuxième contient le verbe modal \textit{a putea} `pouvoir' et le troisième contient l'auxiliaire \textit{a fi} `être' du passif. En appliquant les tests mentionnés précédemment, on observe que les auxiliaires de temps ne permettent pas la topicalisation du verbe lexical (avec ses dépendants), cf. \REF{ex:1:28}a, ils n'ont pas de portée sur une coordination de verbes, cf. \REF{ex:1:29}a, et ils ne permettent pas l'omission du verbe lexical, cf. \REF{ex:1:30}a.


\begin{enumerate}
\item \label{bkm:Ref277002017}a  *Mâncat  fructe,  Maria  \textbf{a}.


\end{enumerate}
manger.\textsc{part } fruits,  Maria  avoir.\textsc{aux}

    \textit{Manger des fruits, Maria l'a}   

  b  ??Spăla  vase,  Maria  \textbf{poate}.\footnote{L'extraction est possible si le verbe modal se combine avec un verbe au subjonctif. 
(i)  Maria  poate  să  spele  vase.
  Maria  peut  \textsc{mrq } laver.\textsc{subj.3}  vaisselle.\textsc{pl}
  \textit{Maria peut laver la vaisselle}
(ii)  Să  spele  vase,  Maria  poate.
  \textsc{mrq } laver.\textsc{subj.3}  vaisselle.\textsc{pl},  Maria  peut
  \textit{Qu'elle lave la vaisselle, Maria peut}}

    laver.\textsc{inf } vaisselle\textsc{.pl, } Maria  peut

    \textit{Laver la vaisselle, Maria peut}

  c  Apreciată  de  soț,  Maria  \textbf{a  fost}  întotdeauna.

    apprécier.\textsc{part } par mari,  Maria  a été  toujours 

{\itshape
Appréciée par son mari, Maria l'a été toujours}


\begin{enumerate}
\item \label{bkm:Ref299280142}a  *Maria \textbf{a} [cumpărat această carte şi citit primul capitol].


\end{enumerate}
    \textit{Maria a acheté ce livre et lu le premier chapitre}  

  b  Ion \textbf{poate} [cumpăra această carte şi citi primul capitol].

    \textit{Ion peut acheter ce livre et lire le premier chapitre}

  c  Ion \textbf{este} [iubit de profesor şi urât de colegi].

    \textit{Ion est aimé par le professeur et haï par ses collègues}


\begin{enumerate}
\item \label{bkm:Ref299280145}a  *Maria  \textbf{a } venit  la  petrecere,  dar  Ion  nu  \textbf{a}.


\end{enumerate}
Maria  \textsc{aux}  venu à  fête,  mais  Ion  \textsc{neg } a

    \textit{Maria est venue à la fête, mais Ion n'est pas}  

  b  Maria  \textbf{poate}  veni  la  petrecere,  dar  Ion  nu  \textbf{poate}.

    Maria  peut  venir  à  fête,  mais  Ion  \textsc{neg } peut

    \textit{Maria peut venir à la fête, mais Ion ne peut pas}

  c  Maria  \textbf{e}  apreciată  de  toată  lumea,  dar  Ion  nu  \textbf{e}.  

    Maria  est  appréciée  par  tout  le-monde,  mais  Ion  \textsc{neg } est

{\itshape
Maria est appréciée par tout le monde, mais Ion ne l'est pas } 

De plus, les auxiliaires de temps ne permettent aucun type d'insertion (p.ex. sujet inversé \REF{ex:1:31}a, adverbe ordinaire \REF{ex:1:32}a ou quantifieur \REF{ex:1:33}a) entre les formes verbales composées, excepté~les éléments qu'on analyse comme clitiques syntaxiques, donnés en \REF{ex:1:24}. De ce point de vue, le roumain se rapproche des langues comme le bulgare ou le macédonien (cf. \citet{Legendre2000}).


\begin{enumerate}
\item \label{bkm:Ref299281523}a  *\textbf{A}  Maria  \textbf{venit}  la  petrecere ?


\end{enumerate}
\textsc{aux } Maria  venir.\textsc{part } à  fête

    \textit{Maria est-elle venue à la fête}  

  b  O  \textbf{poate}  Ion  \textbf{fotografia} ?

    \textsc{cl.acc } peut  Ion  photographier

    \textit{Ion, peut-il la photographier}

  c  \textbf{E}  Maria  \textbf{admirată}  de  toată  lumea ?

est  Maria  admirée  par  tout  le-monde

    \textit{Maria est-elle admirée par tout le monde}


\begin{enumerate}
\item \label{bkm:Ref299281528}a  *Maria \textbf{a} mereu \textbf{fost} lângă mine.


\end{enumerate}
    \textit{Maria a toujours été près de moi}

  b  Se  \textbf{poate}  mereu  \textbf{observa}  progresul.

    \textsc{cl } peut  toujours  observer  le-progrès

    \textit{Le progrès peut être toujours observé}  

  c  Maria \textbf{e} mereu \textbf{lăudată} de profesor.

    \textit{Maria est toujours appréciée par le professeur}


\begin{enumerate}
\item \label{bkm:Ref299279328}a  *Copiii \textbf{au} toți \textbf{venit} la petrecere.


\end{enumerate}
{\itshape
Les enfants sont tous venus à la fête}

  b  Prietenii  mei  şi-au  \textbf{putut } toți  \textbf{lua } case  pe  credit.

    amis  mes  \textsc{cl-}ont  pu  tous  prendre  maisons  sur  crédit

{\itshape
Mes amis ont tous pu s'acheter une maison avec un crédit}

  c  Copiii \textbf{sunt} toți \textbf{apreciați} de profesor.

    \textit{Les enfants sont tous appréciés par le professeur}

Ce test oppose le roumain au français, qui permet toujours l'insertion d'un adverbe \REF{ex:1:34}a ou d'un quantifieur \REF{ex:1:34}b entre l'auxiliaire et le passé composé. Une autre différence entre le roumain et le français concerne l'accord du participe passé avec l'auxiliaire. Si tous les participes passés à la voix passive s'accordent avec le sujet dans les deux langues, en français il y a un accord aux temps composés à la voix active si l'auxiliaire est \textit{être} \REF{ex:1:34}c, alors qu'il n'y a jamais d'accord en roumain entre le participe passé et le sujet à la voix active. 


\begin{enumerate}
\item \label{bkm:Ref276465206}a  Marie a \textbf{toujours} été près de moi.  


\end{enumerate}
  b  Les enfants sont \textbf{tous} venus à la fête.

  c  Marie est venu\textbf{e} en voiture.

En même temps, si on revient aux exemples en \REF{ex:1:28} où la deuxième forme verbale (c.-à-d. le verbe lexical à l'infinitif) est topicalisée avec son complément, on observe que les modaux du roumain, contrairement à l'auxiliaire du passif, ne permettent pas cette extraction \REF{ex:1:28}b. L'absence d'extraction du verbe lexical avec ses compléments est un argument contre une structure hiérarchique dans laquelle le verbe modal est la tête qui sous-catégorise un complément verbal de type syntagmatique.  

Les propriétés distributionnelles mentionnées plus haut nous amènent à postuler trois types de structures~pour les formes verbales composées du roumain :

(i) Pour les auxiliaires de temps et modes composés (suivis d'une forme verbale non-finie) comme en \REF{ex:1:27}a, on adopte une structure à complexe verbal, c.-à-d. les deux formes verbales forment un sous-constituant en syntaxe et un seul mot prosodique en phonologie (cf. \citet{Barbu1999}, Monachesi (1999, 2005), Abeillé \& \citet{Godard2003}). Pour l'exemple en \REF{ex:1:35}, on aura donc la représentation arborescente figurant en \REF{ex:1:36}.\footnote{La même analyse vaut pour les auxiliaires du futur et du conditionnel suivis du verbe lexical à l'infinitif. P.ex. \textit{voi citi} (`je vais lire') ou \textit{aş citi} (`je lirais').}


\begin{enumerate}
\item \label{bkm:Ref299284113}Am citit cartea.


\end{enumerate}
  \textsc{aux} lu\textsc{ } le-livre

  \textit{J'ai lu le livre}    


\begin{enumerate}
\item \label{bkm:Ref299284150}Structure à complexe verbal\footnote{Ces arbres ne contiennent pas de n{\oe}ud correspondant au syntagme verbal fini. Voir la discussion concernant ce point dans la section \ref{sec:1.3.5} de ce chapitre.}  pour les temps et modes composés du roumain


\end{enumerate}
{   [Warning: Image ignored] % Unhandled or unsupported graphics:
%\includegraphics[width=2.8673in,height=1.6957in,width=\textwidth]{fe443409cd384d3fb0f6390ffd77f513-img2.svm}
} 

La structure à complexe verbal est bien adaptée pour bloquer les coordinations et l'insertion du sujet, des adverbes ordinaires et des quantifieurs. Dans cette perspective, les auxiliaires en question sont des clitiques simples (ou phonologiques), car ils sont dépendants phonologiquement d'un mot adjacent. Leur statut de clitique simple les autorise, dans certains registres du roumain, à s'attacher en enclise au verbe lexical (Dobrovie-\citet{Sorin1994}).  


\begin{enumerate}
\item a  Plecat-\textbf{am}       nouă din Vaslui.


\end{enumerate}
partir.\textsc{part-aux.passe.1pl} neuf  de  Vaslui

{\itshape
On est parti neuf personnes de Vaslui } 

  b  Lua-te-\textbf{ar}         dracul !\footnote{Au conditionnel présent, l'auxiliaire peut s'attacher en enclise au verbe non-fini uniquement s'il y a des clitiques pronominaux entre les deux formes verbales (cf. raisons phonologiques). Comparer \textit{lua-}\textbf{\textit{te}}\textit{-ar} vs. \textit{*lua-ar}.}~

    prendre.\textsc{inf}-\textsc{cl.2sg-aux.cond.3sg} le-diable

{\itshape
    Que tu ailles au diable}

  c  Pleca-\textbf{voi}.~

    partir.\textsc{inf}-\textsc{aux.fut.1sg}

{\itshape
    Je partirai}

De plus, ils se comportent comme des clitiques spéciaux (ou syntaxiques), vu le fait que les auxiliaires ont des formes réduites (toujours monosyllabiques), et donc différentes de leurs contreparties lexicales : l'auxiliaire \textit{a avea} `avoir' de l'indicatif passé composé \REF{ex:1:38}a et du conditionnel présent \REF{ex:1:38}b ne partage que la moitié des formes du verbe d'origine \REF{ex:1:38}c ; l'auxiliaire du futur \textit{a vrea} `vouloir' en \REF{ex:1:39}a a une seule forme identique à celle du verbe équivalent \REF{ex:1:39}b.  


\begin{enumerate}
\item \label{bkm:Ref307433669}a  \{am, ai, a, am, ați, au\} mâncat \}  


\end{enumerate}
{\itshape
\{ai, as, a, avons, avez, ont\} mangé}

b  \{aş, ai, ar, am, ați, ar\} mânca

{\itshape
\{aurais, aurais, aurait, aurions, auriez, auraient\} mangé} 

c  \{am, ai, are, avem, aveți, au\}

{\itshape
\{ai, as, a, avons, avez, ont\}}


\begin{enumerate}
\item \label{bkm:Ref307433880}a  \{voi, vei, va, vom, veți, vor\} mânca  


\end{enumerate}
{\itshape
\{vais, vas, vas, allons, allez, vont\} manger}

b  \{vreau, vrei, vrea, vrem, vreți, vor\}

    \textit{\{veux, veux, veut, voulons, voulez, veulent\}}

(ii) Pour les modaux suivis d'un verbe non-fini, comme en \REF{ex:1:27}b, on a une structure plate, où le verbe non-fini est au même niveau que le modal tête et les constituants qu'il sous-catégorise (cf. Monachesi (1999, 2005), Abeillé \& \citet{Godard2003}).\footnote{Cette analyse en termes d'une structure plate est adéquate aussi pour les auxiliaires des temps composés en français (Abeillé \& \citet{Godard2003}) et en italien (\citet{Monachesi2005}).} Par conséquent, le verbe non-fini et ses dépendants ne forment pas un seul constituant, ce qui est montré par l'impossibilité d'extraire cette séquence si le verbe fini est un modal. En revanche, cette structure prévoit l'acceptabilité des coordinations de séquences de compléments (c.-à-d. coordinations de non-constituants), ainsi que la possibilité d'insérer un sujet, un adverbe ordinaire ou un quantifieur entre les formes verbales. L'exemple \REF{ex:1:40} aura donc comme représentation l'arbre donné en \REF{ex:1:41}.\footnote{La même analyse en termes de structure plate s'applique à la séquence \textit{are de citit cartea} `il a à lire le livre' (l'auxiliaire \textit{a avea} `avoir', le verbe \textit{a citi} `lire' au supin et le complément \textit{cartea} `le livre' se trouvent au même niveau).}


\begin{enumerate}
\item \label{bkm:Ref299285753}Pot citi cartea.


\end{enumerate}
  peux lire\textsc{} le-livre

  \textit{Je peux lire le livre}    


\begin{enumerate}
\item \label{bkm:Ref299285785}Structure plate pour le modal suivi d'un infinitif en roumain


\end{enumerate}
{   [Warning: Image ignored] % Unhandled or unsupported graphics:
%\includegraphics[width=2.3189in,height=1.202in,width=\textwidth]{fe443409cd384d3fb0f6390ffd77f513-img3.svm}
} 

(iii) Pour les verbes attributifs et en particulier les formes au passif, comme \REF{ex:1:27}c, on a une structure hiérarchique, où le (premier) verbe est une tête qui prend un complément syntagmatique, cf. \citet{Barbu1999}, Abeillé \& \citet{Godard2003}. Dans le cas des formes au passif, comme \REF{ex:1:42}, ce complément syntagmatique contient le participe passé et les compléments de celui-ci. Une représentation est donnée en \REF{ex:1:43}. 


\begin{enumerate}
\item \label{bkm:Ref299286510}Este citită de  Maria.


\end{enumerate}
  est  lue  par Maria

  \textit{Elle est lue par Maria}  


\begin{enumerate}
\item \label{bkm:Ref299286745}Structure hiérarchique pour les verbes attributifs


\end{enumerate}
{   [Warning: Image ignored] % Unhandled or unsupported graphics:
%\includegraphics[width=2.9835in,height=1.6673in,width=\textwidth]{fe443409cd384d3fb0f6390ffd77f513-img4.svm}
} 

\paragraph[Affixes du complexe verbal]{Affixes du complexe verbal}
{\bfseries
Pronoms}

En roumain, on doit distinguer entre les formes pronominales au nominatif et les formes pronominales au datif et à l'accusatif. Les premières, contrairement aux clitiques sujet du français et aux clitiques compléments du roumain, doivent être analysées comme des pronoms forts, sur la base des propriétés suivantes : un pronom sujet a la même distribution qu'un syntagme nominal \REF{ex:1:44}a, il n'est pas nécessairement contigu au verbe (il peut être séparé du verbe par un ajout ou un complément, cf. \REF{ex:1:44}b), il peut être modifié \REF{ex:1:44}c ou coordonné \REF{ex:1:44}d, il peut avoir portée large sur une coordination \REF{ex:1:44}e, il peut apparaître seul dans un énoncé elliptique \REF{ex:1:44}f. 


\begin{enumerate}
\item \label{bkm:Ref299292497}a  \{\textbf{Eu}  / copiii\}  vin  mâine.  


\end{enumerate}
\{\textsc{pro.nom.1sg}  / les-enfants\}  venir.\textsc{ind.pres.1sg/3pl}  demain

{\itshape
Je viens demain / Les enfants viennent demain}

b  \textbf{Eu}  mâine  voi  veni.  / \textbf{Eu}  MEre\footnote{Les majuscules indiquent une accentuation prosodique.}  vreau.

\textsc{pro}  demain  vais  venir  / \textsc{pro}  pommes  veux

  \textit{Je viendrai demain / Je veux des pommes}

c  \{Și  \textbf{eu}  / \textbf{eu}  de asemenea\}  voi  veni.

  \{aussi  \textsc{pro}  / \textsc{pro}  aussi\}  vais  venir.

  \textit{Moi aussi je viendrai}

d  [\textbf{Eu}  şi  Maria]  vom  veni  mâine.

  \textsc{pro}  et  Maria  allons  venir  demain

  \textit{Moi et Maria viendrons demain}

e  \textbf{Eu}  [vin  azi  şi  plec  mâine].

  \textsc{pro}  viens  aujourd'hui  et  pars  demain

  \textit{Je viens aujourd'hui et je pars demain}

f  A : - Cine  vine  cu  mine ?  B :  - Eu.

  qui  vient  avec  moi    \textsc{pro} 

  \textit{A : - Qui vient avec moi ? B : - Moi} 

En revanche, les propriétés syntaxiques et morphosyntaxiques des clitiques au datif et à l'accusatif plaident en faveur d'une analyse en termes d'affixes flexionnels du verbe (cf. Monachesi (1998, 2000, 2005), \citet{Barbu1999}, Miller \& \citet{Monachesi2003}). Je me limite ici à lister leurs propriétés majeures (pour les exemples en roumain, voir les références citées précédemment). Ils se comportent comme des éléments morphologiques : les pronominaux datifs et accusatifs s'attachent phonologiquement à un hôte (qui peut être un verbe, la négation \textit{nu}, un complémenteur, un nom ou une forme \textit{qu-}) ; ils n'apparaissent pas dans les mêmes positions que les syntagmes correspondants ; ils suivent un ordre spécifique, rigide, qui ne correspond pas à l'ordre des syntagmes pleins correspondants ; ils doivent être contigus au verbe (ils peuvent être séparés du verbe uniquement par des clitiques adverbiaux qu'on analyse comme des affixes du verbe) ; ils peuvent précéder le verbe (proclitiques) ou le suivre (enclitiques) dans des conditions bien précises ; ils ne peuvent pas apparaître dans une phrase où le verbe dont ils dépendent est absent ; ils ne peuvent pas être coordonnés ; ils n'ont pas de portée sur une coordination d'hôtes ; ils présentent des idiosyncrasies morphophonologiques et parfois des lacunes dans leurs paradigmes (voir la distribution du clitique accusatif féminin singulier \textit{o}). Il s'agit donc de clitiques spéciaux au sens de \citet{Zwicky1977}. 

Si on a une forme verbale composée comme en \REF{ex:1:45}, les pronoms clitiques sont affixés au verbe fini et non au verbe qui les sous-catégorise.\footnote{La seule exception est la forme pronominale de l'accusatif féminin singulier \textit{o}, qui est affixée au verbe lexical non-fini, s'il s'agit d'une forme verbale composée.} Dans ces cas, la {\guillemotleft}~montée~{\guillemotright} du clitique peut recevoir une analyse syntaxique sans passer par~le phénomène de mouvement. On peut facilement en rendre compte si on suppose une opération lexicale d'héritage des arguments d'un verbe par un verbe {\guillemotleft}~supérieur~{\guillemotright} (en termes de structure ou de linéarisation), c'est ce qu'on appelle une composition d'arguments (Hinrichs \& \citet{Nakazawa1994}). C'est ainsi que l'auxiliaire hérite tous les arguments (y compris les clitiques) du verbe non-fini (Abeillé \& \citet{Godard2003}, \citet{Monachesi2005}) dans toutes les langues romanes.


\begin{enumerate}
\item \label{bkm:Ref299293811}a  Noi  i-am  spus  adevărul.


\end{enumerate}
nous  \textsc{cl.dat-aux}  dit  la-vérité

\textit{Nous lui avons dit la vérité}  

  b  Maria  ți-l  va  da  dacă  vei  fi  cuminte.

Maria\textsc{  cl.dat-cl.acc}  va  donner si  \textsc{aux.fut.2sg}  être  sage

{\itshape
    Maria te le donnera si tu es sage}

Contrairement au français et à l'italien (où il y a une distribution complémentaire entre les pronoms clitiques et les syntagmes pleins correspondants), en roumain les affixes pronominaux peuvent être cooccurrents avec un complément syntagmatique, phénomène connu sous le nom de redoublement clitique (\citet{Farkas1978}, Dobrovie-\citet{Sorin1994}). Le redoublement clitique est contraint syntaxiquement, sémantiquement et pragmatiquement (p.ex. catégorie, position, et surtout [+/- spécifique], [+/- humain]). Pour les pronoms forts, le redoublement clitique est obligatoire. \citet{Monachesi2005} analyse le clitique qui redouble un complément syntagmatique comme une marque d'accord. De ce point de vue, on peut rapprocher le redoublement clitique \REF{ex:1:46}a du pro-drop du sujet \REF{ex:1:46}b (cf. \citet{Barbu1999}) : quand le sujet n'est pas exprimé, l'information se trouve dans la marque flexionnelle de personne sur le verbe ; de même, quand il n'y a pas de syntagme complément dans une phrase, il est {\guillemotleft}~inclus~{\guillemotright} dans le clitique pronominal affixé au verbe. 


\begin{enumerate}
\item \label{bkm:Ref299294746}a  (Merele)  le  mănânci  dimineața.


\end{enumerate}
les-pommes  \textsc{cl.acc.3pl}  manger.\textsc{ind.pres.2sg}  le-matin

\textit{Les pommes, tu les manges le matin}  

  b  (Tu)  mănânci  mereu.

\textsc{pro.nom.2sg}  manger.\textsc{ind.pres.2sg}  toujours

{\itshape
    Tu manges tout le temps}

\textbf{Adverbe de négation phrastique} \textbf{\textit{nu} }

L'adverbe de négation \textit{nu}, qui réalise la négation de phrase, suit un ordre fixe, précédant toujours les clitiques pronominaux dans le complexe verbal, cf. \REF{ex:1:24}. Il est en distribution complémentaire avec le préfixe \textit{ne-~}: la forme \textit{nu} est employée si l'adverbe est adjacent à une forme verbale finie (ou un infinitif avec la marque \textit{a}), tandis que \textit{ne-} est utilisé s'il se combine avec un verbe au gérondif (= participe présent), participe (passé) ou supin. Il ne peut avoir portée sur une coordination de verbes ; il doit être répété devant chaque terme coordonné. Il ne peut apparaître dans une phrase sans son hôte syntaxique. Il présente des idiosyncrasies morphophonologiques (\textit{nu} peut se réduire à \textit{n-} s'il précède un verbe ou un auxiliaire qui commence par la voyelle \textit{a} ou s'il précède le clitique pronominal \textit{o~}: \textit{n-aude} `il n'entend pas'\textit{, n-aş merge} `je n'irais pas'\textit{, n-o aud} `je ne l'entends pas').\footnote{Pour des exemples illustrant toutes les propriétés mentionnées ici, consulter Barbu (1999, 2003).}  

Sur la base de ces propriétés, Barbu (1999, 2003) analyse la négation phrastique en roumain comme affixe lexical, formant un mot avec le verbe. Il faut ajouter que \textit{nu} peut être hôte phonologique des clitiques pronominaux \REF{ex:1:47}a ou des formes réduites du verbe \textit{a fi} `être' \REF{ex:1:47}b ; de plus, il bloque l'inversion verbe-auxiliaire \REF{ex:1:47}c.


\begin{enumerate}
\item \label{bkm:Ref299296824}a  Nu-mi  spune adevărul. 


\end{enumerate}
\textsc{neg-cl} dit  la-vérité 

{\itshape
Il ne me dit pas la vérité}

b  Nu-i bun.

  \textsc{neg-}est bon

  \textit{Ce n'est pas bon} 

c  plecat-am {\textbar} *nu  plecat-am {\textbar} nu  am  plecat

  parti-\textsc{aux} {\textbar}  \textsc{neg} parti-\textsc{aux} {\textbar} \textsc{neg} \textsc{aux} parti

{\itshape
  nous sommes partis {\textbar} nous ne sommes pas partis}

La négation de phrase \textit{nu} a une forme homonyme qui ne partage pas les mêmes propriétés. Il s'agit de la négation de constituant dont le comportement est celui d'un adverbe modifieur. Il peut modifier toute catégorie syntagmatique, à condition qu'il précède toujours l'élément modifié. Il est toujours accentué et associatif : il a un rôle {\guillemotleft}~emphatique~{\guillemotright} (cf. \citet{Barbu2003}) et, au niveau sémantique, il est associé à un ensemble d'alternatives. Par conséquent, il ne peut jamais se réduire aux formes \textit{n-} ou \textit{ne-}. De plus, il a portée large sur une coordination d'unités lexicales et il peut fonctionner comme mot pro-phrase (il peut remplacer tout un énoncé). Avec les mêmes propriétés, il peut avoir comme substituts lexicaux les adverbes négatifs \textit{nicidecum, deloc, defel, în niciun caz} (et \textit{ba}) `pas du tout' qui sont indubitablement des mots lexicaux.  

Dans cette thèse, je garde donc la distinction faite par Barbu (1999, 2003) entre les deux \textit{nu} en roumain : un \textit{nu} affixal pour la négation de phrase \REF{ex:1:48}a, et un \textit{nu} lexical pour la négation de constituant \REF{ex:1:48}b et aussi pour la négation pro-phrase \REF{ex:1:48}c. 


\begin{enumerate}
\item \label{bkm:Ref299297299}a  Ion  \textbf{nu}  a  venit. 


\end{enumerate}
Ion  \textsc{neg}  \textsc{aux}  venu 

{\itshape
Ion n'est pas venu}

b  \textbf{Nu}  Ion  a  venit.

  \textsc{neg } Ion  \textsc{aux}  venu

  \textit{Ce n'est pas Ion qui est venu} 

c  A : - A  venit  Ion ? B : - \textbf{Nu}.

  A : \textsc{aux}  venu  Ion  B : non

{\itshape
A : - Ion est-il venu ? B : - Non.}

\textbf{Adverbes d'intensité / de degré} \{\textit{cam, mai, prea, şi, tot}\}

Selon la linéarisation qu'on avait établie en \REF{ex:1:24}, on observe que les cinq adverbes mentionnés ont une distribution très contrainte : ils précèdent toujours le verbe lexical, qu'il s'agisse d'un verbe simple ou d'une forme verbale composée. Par conséquent, ce sont les seuls éléments (avec la marque aspectuelle \textit{fi}) qui peuvent s'insérer entre les formes verbales composées.

Leur distribution présente beaucoup d'idiosyncrasies syntaxiques. Ils ne peuvent pas s'insérer (sauf l'adverbial \textit{şi}) entre les modaux et le verbe non-fini, bien que d'autres insertions (sujet, adverbe ordinaire, quantifieur) soient possibles pour les modaux. De plus, certains adverbiaux ont une distribution particulière : distribution variable de \textit{prea} (\textit{prea} s'insère entre l'auxiliaire et le verbe non-fini uniquement s'il y a une négation phrastique ; sinon, il précède l'auxiliaire), distribution bizarre de \textit{şi} (\textit{şi} apparaît uniquement entre le modal et le verbe non-fini, bien que les autres soient interdits dans cette position), distribution variable avec la marque aspectuelle \textit{fi} (les adverbes \textit{mai} et\textit{ cam} apparaissent toujours avant la marque aspectuelle \textit{fi~}; les adverbes \textit{tot} et\textit{ şi} apparaissent entre la marque aspectuelle et le verbe non-fini).

Tous ces items adverbiaux ne peuvent être coordonnés. Ils ont des contreparties lexicales (avec des propriétés syntaxiques et sémantiques différentes) qui sont en alternance avec d'autres adverbes ; cependant, les substituts lexicaux de leurs contreparties ne peuvent pas apparaître à la place de ces cinq adverbiaux. 

Dobrovie-\citet{Sorin1994}, Barbu (1999, 2003) et \citet{Monachesi2000}) les analysent comme des affixes du verbe, car : (i) ils peuvent apparaître entre les clitiques pronominaux et le verbe \REF{ex:1:49}a ; (ii) certains apparaissent entre le préfixe négatif \textit{ne-} et le verbe non-fini \REF{ex:1:49}b. 


\begin{enumerate}
\item \label{bkm:Ref299297884}a  Il  \textbf{mai}  văd. 


\end{enumerate}
\textsc{cl  adv}  vois

{\itshape
Je le vois encore}

b  ne\textbf{mai}auzind  {\textbar} ne\textbf{mai}auzit

  \textsc{neg.adv.}entendre.\textsc{gerond}  {\textbar} \textsc{neg.adv.}entendre.\textsc{part}

  \textit{ne l'entendant plus {\textbar} inouï} 

\textbf{Marques modales} \textbf{\textit{a} }\textbf{et} \textbf{\textit{să}}

Le statut de~la marque de l'infinitif \textit{a} en \REF{ex:1:50}a et du subjonctif \textit{să} en \REF{ex:1:50}b est assez controversé. Selon Dobrovie-\citet{Sorin1994}, ces éléments oscillent entre un rôle syntaxique (de complémenteur) et un rôle morphologique (de marque flexionnelle). Si la grammaire traditionnelle considère l'infinitif et le subjonctif en concurrence (\citet{Avram2001}), il faut préciser que le subjonctif gagne de plus en plus du terrain par rapport à l'infinitif avec \textit{a}.  


\begin{enumerate}
\item \label{bkm:Ref307434333}a  Copilul  începu  \textbf{a}  plânge.


\end{enumerate}
l'enfant  commença  \textsc{mrq } pleurer.\textsc{inf}

\textit{L'enfant commença à pleurer}  

  b  Copilul  începu  \textbf{să}  plângă.

l'enfant  commença  \textsc{mrq } pleurer.\textsc{subj.3sg}

{\itshape
    L'enfant commença à pleurer}

La marque du subjonctif \textit{să} est listée par la plupart des ouvrages dans la série des complémenteurs (ou conjonctions de subordination, dans les termes traditionnels), à côté du complémenteur \textit{că} `que' qui demande dans la subordonnée un verbe à l'indicatif ou au conditionnel (comparer \REF{ex:1:51}a-b). Le choix entre \textit{să} et \textit{că} semble être lié au type sémantique du verbe dans la phrase racine. Ainsi, \textit{să} est sélectionné par des verbes de désir ou de volonté (p.ex. \textit{a vrea} `vouloir', \textit{a dori} `désirer'), alors que \textit{că} est sélectionné par des verbes d'attitude propositionnelle (p.ex. \textit{a crede} `croire',\textit{ a gândi} `penser'). Par conséquent, l'interprétation d'une subordonnée introduite par \textit{să} serait non-factuelle (le contenu de la complétive est de type visée), alors que celle d'une subordonnée introduite par \textit{că} serait factuelle (le contenu de la complétive est une proposition). Cependant, on doit noter que syntaxiquement \textit{că} et \textit{să} n'ont pas la même distribution : si le complémenteur \textit{că} apparaît toujours en début de phrase (il précède tous les autres constituants de la subordonnée), la marque \textit{să} ne peut pas être séparée du verbe par un élément lexical (voir la linéarisation des éléments appartenant au complexe verbal en \REF{ex:1:24}). Si dans la subordonnée il y a un syntagme qui précède le verbe, elle doit être introduite par \textit{ca}, tandis que la forme \textit{să} restera toujours adjacente au verbe (comparer \REF{ex:1:51}c-d).\footnote{Dans ces cas, les ouvrages traditionnels parlent d'un morphème discontinu \textit{ca ... să}.} De plus, on observe que la marque \textit{să} peut être cooccurrente avec un élément relatif \REF{ex:1:51}e. Par conséquent, une subordonnée au subjonctif en roumain peut être introduite par le complémenteur \textit{ca} \REF{ex:1:51}d ou un élément relatif \REF{ex:1:51}e, ou bien elle n'est introduite par aucun complémenteur \REF{ex:1:51}b. Si la subordonnée au subjonctif est {\guillemotleft}~libre~{\guillemotright} (c.-à-d. elle n'a aucun introducteur, à part la marque \textit{să}), elle obéit à une contrainte de linéarisation forte : le verbe (ou complexe verbal) doit occuper la première position.\footnote{\citet{Monachesi2005} rapproche le roumain \textit{să} de la marque du subjonctif en grec (\textit{na}) et en bulgare (\textit{da}). De plus, on pourrait mentionner le cas du salentin (un dialecte de l'italien du sud) qui distingue entre le complémenteur \textit{ka} et la forme hybride \textit{ku} (cf. \citet{Monachesi1998}). Mais, contrairement au roumain \textit{să}, le salentin \textit{ku} n'est pas compatible avec une expression \textit{qu-}. Il y a d'autres langues qui posent le même problème vis-à-vis du statut syntaxique d'un élément qui semble avoir un double rôle. C'est le cas des particules préverbales \textit{aL} et \textit{goN} en irlandais, qui, selon \citet{Sells1984}, ne doivent pas être analysées comme des complémenteurs phrastiques, mais comme des marqueurs qui sélectionnent une tête verbale (en HPSG).}


\begin{enumerate}
\item \label{bkm:Ref299298883}a  Sper  \textbf{că}  Ion  vine  astăzi.


\end{enumerate}
espére.1\textsc{sg}  que  Ion  vient.\textsc{ind } aujourd'hui

\textit{J'espère que Ion vient aujourd'hui}   

  b  Sper  \textbf{să}  vină  Ion  astăzi.

espére.1\textsc{sg}  \textsc{mrq}  vienne.\textsc{subj } Ion  aujourd'hui

\textit{    J'espère que Ion vient aujourd'hui}    

  c  *Sper  \textbf{să}  Ion  vină  astăzi.

espére.1\textsc{sg}  \textsc{mrq}  Ion  vienne.\textsc{subj } aujourd'hui

\textit{    J'espère que Ion vient aujourd'hui}  

  d  Sper  *(\textbf{ca})  Ion  \textbf{să}  vină  astăzi.

espére.1\textsc{sg}  que  Ion  \textsc{mrq}  vienne.\textsc{subj } aujourd'hui

{\itshape
    J'espère que Ion vient aujourd'hui}

  e  Caut  o  fată  cu  \textbf{care}  \textbf{să}  plec  la  munte.

cherche.1\textsc{sg}  une  fille  avec  \textsc{rel}  \textsc{mrq}  parte.\textsc{subj.1sg } à  montagne

{\itshape
    Je cherche une fille avec laquelle je parte à la montagne}

Je suis toujours l'analyse de \citet{Barbu1999} et je considère les deux marques \textit{a} et \textit{să} comme des affixes du verbe. Les propriétés générales~des deux affixes~sont les suivantes : fort degré de sélection de leur hôte syntaxique (les deux sélectionnent toujours respectivement l'infinitif et le subjonctif), ordre et distribution rigides (toujours en première position dans le complexe verbal ; séparés du verbe uniquement par d'autres affixes), pas de portée large sur une coordination d'hôtes. Les deux peuvent être des hôtes phonologiques pour les clitiques pronominaux (p.ex. \textit{a-l cere} `le demander'\textit{, să-l cer} `que je le demande'). Parmi leurs propriétés particulières, on note les aspects suivants : l'affixe \textit{a} est en distribution avec d'autres clitiques qui sélectionnent l'infinitif (voir les formes des auxiliaires \textit{a avea} `avoir' et \textit{a vrea} `vouloir') ; l'affixe \textit{să} comporte des idiosyncrasies morphophonologiques : la forme \textit{să} se réduit à \textit{s-} devant un verbe qui commence par la voyelle \textit{a} (p.ex. \textit{s-avem} `que nous ayions') ou devant le clitique pronominal \textit{o} (p.ex. \textit{s-o cauți} `que tu la cherches') ; la forme \textit{să} bloque l'inversion verbe-clitique pronominal (p.ex. \textit{bată-l} `qu'il le frappe' vs. *\textit{să bată-l / să-l bată} `qu'il le frappe').

En conclusion, tous les éléments précédant le verbe lexical en \REF{ex:1:24}, c.-à-d. les marques de l'infinitif et subjonctif, la négation, les clitiques pronominaux de datif et accusatif, les auxiliaires des temps et modes composés, ainsi que les affixes adverbiaux font partie de ce qu'on appelle le complexe verbal. Leurs propriétés montrent qu'on a affaire à des affixes attachés au verbe. La notion de complexe verbal explique les propriétés particulières des formes verbales composées et de tous les éléments pronominaux et adverbiaux qui apparaissent uniquement en relation avec un verbe.  

\subsubsection{Sujets non-réalisés} 
Le roumain se distingue du français par le fait qu'il permet plus souvent la non-réalisation du sujet. Les travaux traditionnels en distinguent trois contextes majeurs dans lesquels il n'y a pas de sujet réalisé.  

(i) Premièrement, le roumain se caractérise comme étant une langue \textit{pro-drop}, qui a la possibilité de ne pas réaliser lexicalement un sujet référentiel, récupérable morphologiquement (par la flexion du verbe) et discursivement (par deixis ou anaphore). 


\begin{enumerate}
\item a  (Tu)  Vii  mâine.  


\end{enumerate}
(tu)  viens.\textsc{ind}  demain

{\itshape
Tu viens demain}

b  (Tu)  Vino  mâine !

(tu)  venir.\textsc{imper}  demain

{\itshape
Viens demain}

c  Ioana n-a venit la muncă. (Eu) Cred că (ea) e bolnavă.

Ioana \textsc{neg aux} venu à travail. (je) crois que (elle) est malade

{\itshape
Ioana n'est pas venue au travail. Je crois qu'elle est malade}

(ii) Deuxièmement, il y a des phrases comme en \REF{ex:1:53} dans lesquelles, bien que le verbe soit apte à recevoir un sujet, le sujet n'est pas réalisé et de plus sa référence ne peut pas être récupérée : c'est le cas des sujets appelés {\guillemotleft}~non-déterminés~{\guillemotright} ou {\guillemotleft}~non-spécifiés~{\guillemotright} (cf. Pană \citet{Dindelegan2003}, \textit{GALR} (2005)) dans une construction active (avec un verbe personnel agentif) ; la référence de ce type de sujet est soit non-identifiable, soit non-pertinente du point de vue informationnel.


\begin{enumerate}
\item \label{bkm:Ref307434626}a  Scrie  în  ziare.  


\end{enumerate}
écrire.\textsc{ind.pres.3sg } dans  journaux

{\itshape
Les journaux écrivent}

b  Au  adus  fructe  tropicale  (la piață).

ont  apporté  fruits  tropicaux  (à marché)

  \textit{On a apporté des fruits tropicaux au marché}

c  Mă cheamă Gabriela.\footnote{Le patron d'un verbe \textit{dicendi} (\textit{a spune, a zice} `dire'\textit{, a chema} `appeler') sans sujet est lexicalisé en roumain, avec un sens distinct de celui des verbes homonymes utilisés avec un sujet référentiel. Leur nouveau sens est `s'appeler'.}

m'appeler.\textsc{ind.pres.3} Gabriela

{\itshape
Je m'appelle Gabriela}

(iii) Troisièmement, il y a certaines formes verbales qui ne peuvent pas sous-catégoriser de sujet, quel que soit son type. Rentrent dans cette catégorie les verbes impersonnels intrinsèques dits {\guillemotleft}~météorologiques~{\guillemotright} \REF{ex:1:54}a (p.ex. \textit{a} \textit{ploua} `pleuvoir',\textit{ a ninge} `neiger', etc.), les verbes intransitifs à la tournure impersonnelle \REF{ex:1:54}b (p.ex. \textit{se vorbeşte} `on parle'\textit{, se lucrează} `on travaille'\textit{, se doarme} `on dort'), et les expressions verbales non-agentives {\guillemotleft}~psychologiques~{\guillemotright} \REF{ex:1:54}c ou de~{\guillemotleft}~sensation physique~{\guillemotright} \REF{ex:1:54}d~(p.ex. \textit{a-i arde de ceva} `avoir envie de quelque chose'\textit{, a-i păsa de cineva~}`se soucier de quelqu'un'\textit{, a i se căşuna pe cineva} `s'en prendre à quelqu'un'\textit{, a i se urî de ceva} `en avoir assez'\textit{, a-l durea undeva} `avoir mal quelque part', \textit{a-l mânca undeva} `avoir des démangeaisons quelque part', \textit{a-l arde undeva} `avoir des brûlures quelque part'). Les valents de ces verbes ne peuvent jamais avoir accès à la fonction sujet. Le dernier type de verbes mentionné (c.-à-d. les verbes non-agentifs désignant un état mental ou émotionnel) assigne le datif ou l'accusatif à la place du nominatif (comme c'est le cas du polonais ou du russe, cf. Mc\citet{Shane2005}). Ainsi, en \REF{ex:1:54}c le verbe \textit{a păsa} `se soucier' sous-catégorise un complément nominal [+ humain] au datif (p.ex. \textit{Mariei}), alors qu'en \REF{ex:1:54}d le verbe \textit{a durea} `avoir mal' sous-catégorise un complément nominal [+ animé] à l'accusatif (p.ex. \textit{pe Maria}). 


\begin{enumerate}
\item \label{bkm:Ref299303333}a  Plouă. 


\end{enumerate}
pleuvoir.\textsc{ind.pres.3}

{\itshape
Il pleut}

b  Se  lucrează  prea mult  în  zilele  noastre. 

\textsc{cl.acc.3 } travailler.\textsc{ind.pres.3}  trop  dans  jours  nos

{\itshape
On travaille trop aujourd'hui}

c  \textbf{Mariei  îi}  pasă  de  problemele  lui Ion.  

  Maria.\textsc{dat  cl.dat  vb.ind.pres}  de  les-problèmes  Ion.\textsc{gen}

{\itshape
Maria se soucie des problèmes de Ion}

d  \textbf{Pe  Maria  o}  doare  în  gât. 

\textsc{    mrq.acc } Maria\textsc{  cl.acc}  \textsc{vb.ind.pres.3sg } dans  gorge

  \textit{J'ai mal à la gorge}

Selon Pană \citet{Dindelegan2003} et \textit{GALR II} (2005 : 314), le patron des phrases sans sujet (qu'il s'agisse d'une non-lexicalisation du sujet ou bien d'une inexistence de sujet) est très bien représenté en roumain, ayant une fréquence assez importante dans le langage courant. 

{\bfseries
Pro-drop du sujet}

Parmi les trois situations mentionnées plus haut, le phénomène le plus discuté dans la littérature est le pro-drop (ou le paramètre du sujet nul, dans les travaux transformationnels). En roumain (comme en espagnol ou italien), un sujet anaphoriquement ou déictiquement récupérable peut ne pas être exprimé lexicalement (cf. Dobrovie-\citet{Sorin1987}, sujets {\guillemotleft}~phonétiquement nuls~{\guillemotright}), mais simplement marqué dans la flexion du verbe (via les marques de personne et de nombre). Cela distingue le roumain \REF{ex:1:55} des langues comme le français \REF{ex:1:56}, dont la phrase a généralement besoin d'un syntagme sujet, en dehors de la tête.


\begin{enumerate}
\item \label{bkm:Ref299305113}a  Vine mâine.  


\end{enumerate}
vient demain

{\itshape
Il vient demain}

b  El vine mâine.

  \textit{Il vient demain}


\begin{enumerate}
\item \label{bkm:Ref299305055}a  *Vient demain.  


\end{enumerate}
b  \{Marie / elle\} vient demain.

Selon \citet{Gutman2004}, le roumain est une langue typiquement pro-drop, car il permet le pro-drop avec toutes les personnes du verbe ; contrairement à l'hébreu ou au finnois, il permet facilement le pro-drop à la troisième personne même avec un antécédent contextuellement inféré \REF{ex:1:57}a ou avec un antécédent ayant un cas différent du nominatif (p.ex. en \REF{ex:1:56}b, l'antécédent du sujet de la subordonnée est au datif). Par conséquent, le pro-drop est permis à condition qu'on ait accès à un référent. Selon Vainikka \& \citet{Levy1999}, la nature référentielle de la première et la deuxième personne est différente de la troisième personne. Dans le premier cas, le référent se trouve directement dans la situation d'énonciation. Dans le dernier cas, le référent se trouve dans le contexte linguistique (c.-à-d. il a été déjà mentionné auparavant). 


\begin{enumerate}
\item \label{bkm:Ref299305264}a  A  reuşit  la  examen.


\end{enumerate}
a  réussi  à  examen

\textit{Il / Elle a réussi son examen}   

  b  Ion şi Maria i-au dat telefon \textbf{Danei} fix în momentul în care ieşea din țară.

Ion et Maria \textsc{cl.dat-}ont donné téléphone Dana.\textsc{dat} exactement au moment où sortait du pays

\textit{    Ion et Maria ont donné un coup de fil à Dana}\textit{\textsubscript{i}}\textit{ exactement au moment où elle}\textit{\textsubscript{i}}\textit{ quittait le pays}  

Les pronoms sujets sont réalisés lexicalement avec des effets contrastifs \REF{ex:1:58}b ou emphatiques~\REF{ex:1:58}c :


\begin{enumerate}
\item \label{bkm:Ref299305473}a  Am  cumpărat  casa.


\end{enumerate}
\textsc{aux.1sg}  acheté  la-maison

\textit{J'ai acheté la maison}   

  b  Eu  am  cumpărat  casa,  nu  tu.

\textsc{pro.1sg}  \textsc{aux.1sg}  acheté  la-maison,  non  \textsc{pro.2sg}

\textit{    C'est moi qui ai acheté la maison, et pas toi}  

  c  Eu  însumi  am  cumpărat  casa.

\textsc{pro.1sg}  \textsc{refl.1sg.masc}  \textsc{aux.1sg}  acheté  la-maison

{\itshape
    Moi-même j'ai acheté la maison}

Selon \textit{GALR II} (2005 : 48), toutes les valences d'un verbe tête sont mises à jour uniquement à l'intérieur du syntagme qui contient la tête, le seul type de relation permise étant les relations anaphoriques. Cela nous permet de considérer l'exemple en \REF{ex:1:59}a comme une coordination de phrases, et non une coordination de syntagmes verbaux \REF{ex:1:59}b ou encore la coordination d'une phrase avec un syntagme verbal \REF{ex:1:59}c.


\begin{enumerate}
\item \label{bkm:Ref299305605}a  [Elevul îmi \textbf{trimite} cartea prin poştă]\textsubscript{S} şi [mă \textbf{anunță} imediat de asta]\textsubscript{S}.


\end{enumerate}
l'élève \textsc{cl.dat.1sg} envoie le-livre par poste et \textsc{cl.acc.1sg} annonce immédiatement de cela 

\textit{L'élève m'envoie le livre par la poste et m'annonce immédiatement cela}  

  b  Elevul [îmi \textbf{trimite} cartea prin poştă]\textsubscript{VP} şi [mă \textbf{anunță} imediat de asta]\textsubscript{VP}.  

\textit{    L'élève m'envoie le livre par la poste et m'annonce immédiatement cela}  

  c  [Elevul îmi \textbf{trimite} cartea prin poştă]\textsubscript{S} şi [mă \textbf{anunță} imediat de asta]\textsubscript{VP}.  

{\itshape
    L'élève m'envoie le livre par la poste et m'annonce immédiatement cela}

Les deux dernières possibilités d'analyse sont problématiques, car comme on verra dans la section \ref{sec:1.3.5}, on n'a pas de motivation empirique pour postuler un syntagme verbal fini en roumain. En revanche, supposer qu'on a affaire à une coordination de phrases nous permet une analyse unitaire des trois exemples donnés en \REF{ex:1:60} : coordination d'une phrase avec sujet réalisé et d'une phrase pro-drop \REF{ex:1:60}a, coordination avec deux phrases pro-drop \REF{ex:1:60}b, et coordination de deux phrases où le sujet implicite du deuxième conjoint est coréférent à un complément prépositionnel dans le premier conjoint \REF{ex:1:60}c.


\begin{enumerate}
\item \label{bkm:Ref277687288}a  Ieri, Ion n-a fost atent şi a spart vaza preferată a mamei.


\end{enumerate}
    \textit{Hier, Ion n'a pas fait attention et a cassé la vase préférée de sa mère}

  b  N-am fost atent şi am spart o vază.

\textsc{    neg} ai été attentif et ai cassé un vase

\textit{Je n'ai pas fait attention et j'ai cassé une vase}  

  c  Am fost \textbf{la dentist} şi mi-a spus că am paradontoză.

\textsc{aux.1} été chez le dentiste et m'a dit que avoir.\textsc{pres.1sg} paradonthose

\textit{J'ai été chez le dentiste et il m'a dit que j'avais la paradonthose} 

\subsubsection{Distribution du sujet}
\label{bkm:Ref299309324}S'il est réalisé, le sujet peut occuper plusieurs positions. Il peut précéder le verbe \REF{ex:1:61}a, il peut suivre immédiatement le verbe lexical \REF{ex:1:61}b ou encore il peut suivre un complément du verbe \REF{ex:1:61}c, mais il ne peut jamais s'insérer entre l'auxiliaire et le participe passé (ou à l'intérieur du complexe verbal, d'une manière plus générale) \REF{ex:1:61}d, cf. Dobrovie-\citet{Sorin1994}, Monachesi (1999, 2005).


\begin{enumerate}
\item \label{bkm:Ref299308425}a  \textbf{Maria}  a făcut  o  prăjitură.  


\end{enumerate}
Maria  a fait  un  gâteau

{\itshape
Maria a préparé un gâteau}

b  A  făcut  (ieri)  \textbf{Maria}  o  prăjitură.

  a  fait  (hier)  Maria  un  gâteau

  \textit{(Hier,})\textit{ Maria a préparé un gâteau}

c  A  făcut  o  prăjitură  \textbf{Maria}.

  a  fait  un  gâteau  Maria

  \textit{Maria a préparé un gâteau}

d  *A  \textbf{Maria}  făcut  o  prăjitură.

  a  Maria  fait  un  gâteau

  \textit{Maria a préparé un gâteau}

Cependant, la question se pose de savoir quelle est la position canonique du sujet en roumain. Intuitivement, on peut croire que le roumain privilégie un positionnement préverbal du sujet (donc, un ordre SVO). Des travaux récents semblent infirmer cette intuition : le sujet est préféré dans une position postverbale (après l'auxilaire et le verbe lexical). Selon Dobrovie-Sorin (1987, 1994), Pană \citet{Dindelegan2003}, etc., l'ordre canonique du roumain serait de type VSO.

Si on regarde les données, on observe qu'il y a un gradient de situations allant des restrictions les plus fortes aux préférences. Le seul cas obligatoire de position préverbale du sujet concerne les phrases interrogatives avec un mot \textit{qu-} en position sujet (comparer \REF{ex:1:62}a et \REF{ex:1:62}b). Les seuls cas de position postverbale obligatoire concernent le sujet des incises de citation directe \REF{ex:1:63}a, le sujet des infinitives \REF{ex:1:63}b, le sujet des interrogatives avec un mot \textit{qu-} non-sujet \REF{ex:1:63}c, et le sujet des phrases impératives avec un conditionnel ou subjonctif inversé dans les formules invectives \REF{ex:1:63}d. 


\begin{enumerate}
\item \label{bkm:Ref283930871}a  \textbf{Ce carte} a apărut la editura Seuil ?  


\end{enumerate}
{\itshape
Quel livre est paru au Seuil?}

b  *A apărut \textbf{ce carte} la editura Seuil ?

  a paru quel livre à l'édition Seuil

  \textit{Quel livre est paru au Seuil}


\begin{enumerate}
\item \label{bkm:Ref283931010}a  {\guillemotleft}~Vino aici~{\guillemotright}, spuse \textbf{Ioana}. 


\end{enumerate}
  {\guillemotleft}~\textit{Viens ici~{\guillemotright}, dit Ioana}

  a'  *{\guillemotleft}~Vino aici~{\guillemotright}, \textbf{Ioana} spuse.

    viens ici, Ioana dit 

  {\guillemotleft}~\textit{Viens ici~{\guillemotright}, dit Ioana}

b  Ion a intrat în clasă [înainte de a pleca \textbf{profesorul}].

Ion est entré en classe avant de partir le-professeur

  \textit{Ion est entré dans la classe avant que le professeur parte}

b'  *Ion a intrat în clasă [înainte de \textbf{profesorul} a pleca].

Ion est entré en classe avant de le-professeur partir

  \textit{Ion est entré dans la classe avant que le professeur parte}

c  Unde merge \textbf{Ion} ?

  où va Ion

  \textit{Où Ion va-t-il}

c'  *Unde \textbf{Ion} merge ?

  où Ion va

  \textit{Où Ion va-t-il}

d  Lua-te-ar \textbf{dracul~}!

  prendre.\textsc{inf}-\textsc{cl.2sg-aux.cond.3sg} le-diable  

  \textit{Que tu ailles au diable}

d'  *\textbf{Dracul} lua-te-ar\textbf{~}!

  le-diable prendre.\textsc{inf}-\textsc{cl.2sg-aux.cond.3sg}   

  \textit{Que tu ailles au diable}  

En dehors de ces cas, le choix entre position préverbale ou postverbale du sujet n'est pas fixé rigidement. Des considérations d'ordre informationnel ou discursif semblent déterminer la réalisation pré- vs. postverbale du sujet. Ainsi, si le sujet est non-personnel et non-agentif comme en \REF{ex:1:64}, on a une préférence nette pour un positionnement postverbal si l'énoncé en question est \textit{all focus}, c.-à-d. un énoncé dont l'articulation fond-focus présente un fond vide (p.ex. les énoncés produits hors contexte). En revanche, les mêmes sujets peuvent apparaître en position préverbale s'ils constituent le focus étroit (angl. \textit{narrow focus}) de la phrase, c.-à-d. le sujet en question ne fait pas partie du fond dans l'articulation fond-focus, et dans ce cas ils reçoivent une saillance prosodique (marquée par des majuscules dans les exemples en \REF{ex:1:64}).  


\begin{enumerate}
\item \label{bkm:Ref283930992}a  Mă  doare  \textbf{capul}.


\end{enumerate}
me  fait-mal\textsc{ } la-tête

{\itshape
J'ai mal à la tête}

a'  \textbf{CApul } mă  doare.

la-tête\textsc{ } me  fait-mal\textsc{} 

  \textit{C'est à la tête que j'ai mal}

b  Imi  place  \textbf{cartea}.

me  plaît\textsc{ } le-livre

  \textit{Le livre me plaît (=J'aime lire)}

b'  \textbf{CARtea } îmi  place.

le-livre  me  plaît

{\itshape
C'est le livre qui me plaît (=C'est lire que j'aime)}

  c  S-a  întâmplat  \textbf{o  tragedie}.

s'est  passé  une  tragédie

{\itshape
Il s'est passé une tragédie}

  c'  \textbf{O  trageDIe } s-a  întâmplat.

une  tragédie  s'est  passé 

{\itshape
C'est une tragédie qui s'est passée}

d  Este  \textbf{multă  suferință}  pe  pământ.

  est  beaucoup.\textsc{adj}  souffrance  sur  terre

{\itshape
Il y a beaucoup de souffrances sur la terre}

d'  \textbf{MULtă  sufeRINță}  este  pe  pământ !

  beaucoup.\textsc{adj}  souffrance  est  sur  terre

  \textit{Il y a beaucoup de souffrances sur la terre}

De même, on observe une préférence nette pour le positionnement préverbal du sujet s'il a la fonction de \textit{sorting key} au sens de \citet{Kuno1982}, c.-à-d. dans un couple question-réponse, l'élément distingué pour répondre à une question ; il donne une indication de comment attaquer la résolution de la question. 


\begin{enumerate}
\item A :  \textbf{-} Ce studii au \textbf{copiii tăi} ?  


\end{enumerate}
{\itshape
Où travaillent tes enfants}

B :  - \textbf{Băiatul} a terminat medicina la Lyon, iar \textbf{cele două fete} fac dreptul la Paris.

  \textit{Le garçon a étudié la médecine à Lyon, et les deux filles étudient le droit à Paris}

De manière générale, le syntagme qui a une fonction de topique\footnote{Les termes \textit{topique, topic} et \textit{thème} sont ici interchangeables.} est préverbal. Il arrive qu'en roumain la position du topique soit souvent occupée par le sujet, mais elle est disponible aussi pour d'autres fonctions syntaxiques. Par conséquent, la position préverbale dans une phrase racine du roumain est dédiée non pas à une fonction syntaxique (p.ex. sujet), mais à un rôle discursif (p.ex. topique).

Contrairement au français, il est donc inadéquat d'utiliser pour le roumain le terme d'\textit{inversion du sujet}, c.-à-d. placement postverbal du premier argument du verbe (cf. \citet{Marandin2003}). On va parler plutôt de position préverbale ou postverbale du sujet.

\subsubsection{Ordre des mots}
\label{bkm:Ref299315187}Si l'on prend en compte l'ordre du sujet, du verbe prédicatif et de l'objet, il est difficile d'attribuer au roumain un type structural bien précis, cf. les discussions de la section \ref{sec:1.3.3}. Les ordres prototypiques sont SVO et VSO, mais les autres ordres sont possibles aussi.

On doit donc dire que le roumain se caractérise par un ordre relativement libre, par rapport au français qui a un ordre plus rigide. On constate une distribution libre du sujet par rapport à la tête (position préverbale ou postverbale, avec la possibilité d'être inséré parmi les autres dépendants du verbe), distribution libre de l'adjectif par rapport à la tête nominale, et plus généralement un positionnement libre des compléments. Le seul domaine qui n'est pas flexible est le complexe verbal, qui impose une distribution rigide aux clitiques pronominaux et adverbiaux, comme on l'a observé dans la section \ref{sec:1.3.1}.

Etant donné l'ordre relativement libre, on peut dire qu'en roumain la position d'un constituant ne joue pas de rôle syntaxique distinctif. Les fonctions syntaxiques sont marquées essentiellement par des moyens morphosyntaxiques (accord, cas, préposition) et moins par l'ordre des constituants. Il semble que l'ordre des mots a un rôle discursif plutôt que syntaxique. De ce point de vue, le roumain se distingue du français (Kerleroux \& \citet{Marandin2001}) et se rapproche des langues comme le hongrois (\citet{Kiss1995}), où le discours détermine la position syntaxique, mais pas la fonction grammaticale. 

En ce qui concerne la distinction des langues à tête initiale vs. langues à tête finale, le roumain se comporte prototypiquement plutôt comme une langue à tête initiale, ce qui explique les distributions suivantes : présence des prépositions plutôt que postpositions, position postnominale du déterminant défini et des adjectifs, possibilité d'avoir en postposition les possessifs et les démonstratifs et aussi la postposition des compléments par rapport à la tête verbale.

Dans une approche discursive de la linéarisation des constituants dans une phrase, on identifie les positions {\guillemotleft}~thématiques~{\guillemotright} ou {\guillemotleft}~topicalisées~{\guillemotright}, occupées par des constituants se trouvant en position préverbale. Le roumain permet la topicalisation de tous les constituants syntagmatiques, indépendamment de leur fonction (sujet \REF{ex:1:66}a, complément objet \REF{ex:1:66}b, complément prépositionnel \REF{ex:1:66}c, complément oblique \REF{ex:1:66}d, ajout \REF{ex:1:66}e). 


\begin{enumerate}
\item \label{bkm:Ref299310131}a  \textbf{Ion}  când  vine ? 


\end{enumerate}
Ion  quand  vient

\textit{Ion quand est-ce qu'il vient}  

  b  \textbf{Banii}  ți-i  dau  mâine. 

    argent.\textsc{pl } te-les  rends  demain

    \textit{L'argent je te le rends demain}

  c  \textbf{In  Ion}  nu  te  poți  încrede.

    en  Ion  \textsc{neg  cl } peux  avoir-confiance\textsc{} 

    \textit{On ne peut pas avoir confiance en Ion}

  d  \textbf{Mariei}  nu-i  dă  nimeni  bani. 

    Maria\textsc{.dat  neg-}lui  donne  personne  argent.\textsc{pl} 

    \textit{A Marie, personne ne lui donne d'argent}

  e  \textbf{La  şcoală } înveți  foarte  multe  lucruri  bune. 

    à  école  apprends  très  beaucoup.\textsc{adj}  choses  bonnes

    \textit{A l'école, on apprend beaucoup de bonnes choses}

Les ouvrages traditionnels identifient deux types de thématisation~en roumain (cf. Pană \citet{Dindelegan2003}, \textit{GALR} (2005)), les deux mettant en jeu une position préverbale des constituants dans la phrase. D'une part, on a la~{\guillemotleft}~thématisation forte~{\guillemotright}, qui se caractérise par une prosodie incidente des constituants préverbaux (marquée à l'oral par des pauses et à l'écrit par des virgules), et éventuellement par un emploi d'une construction spécifique avec une préposition ou locution prépositionnelle (p.ex. \textit{cât despre} `quant à'\textit{, în ce priveşte} `en ce qui concerne'\textit{, în privința, în materie de} `en matière de', plus une construction spécifique au roumain :\textit{ de} suivi d'un adjectif ou d'un verbe au supin). D'autre part, on a la {\guillemotleft}~thématisation faible~{\guillemotright}, qui se caractérise par une prosodie intégrée des constituants préverbaux. Les exemples mentionnés en \REF{ex:1:66} présentent tous des constituants mettant en jeu une thématisation faible (p.ex. pour les compléments, l'antéposition implique souvent le redoublement clitique). 

Les deux types mentionnés ont un rôle discursif, c.-à-d. celui d'identifier le topique (de manière générale, c'est l'élément qui indique la perspective adoptée par le locuteur pour organiser l'enchaînement discursif). En même temps, Dobrovie-\citet{Sorin1987} fait la différence entre les deux (ses terms étant \textit{dislocation gauche} vs.\textit{ thématisation}), car selon elle la thématisation forte peut être utilisée pour changer le thème d'un discours. De plus, les éléments disloqués ne font pas partie de l'organisation syntaxique de la phrase, alors que les éléments thématisés ont un rôle syntaxique à l'intérieur de la phrase.

\subsubsection{Vers une structure plate} 
\label{bkm:Ref299284285}Généralement, la distribution d'une phrase est décrite selon le modèle des grammaires transformationnelles qui postulent une coïncidence entre la constituance et l'ordre dans lequel les constituants apparaissent. Par conséquent, dans ce genre d'analyses, l'ordre est directement encodé dans la structure en constituants, toutes les variations d'ordre étant expliquées en termes de mouvements (\citet{Kayne1994}). On dérive ainsi les fonctions syntaxiques des positions dans un arbre de constituants.

Si en revanche on adopte une analyse des fonctions syntaxiques comme propriétés indépendantes de l'arbre de constituants, on n'est pas obligé de postuler une structure hiérarchique. Certains modèles théoriques (comme HPSG) ont les moyens de distinguer entre la représentation de la valence et la représentation de la structure syntagmatique. Si on reconnaît l'indépendance des deux dimensions d'organisation syntaxique (la constituance et l'ordre), cf. Gazdar \textit{et al.} (1985), on peut alors supposer une structure plate où le sujet est au même niveau que le verbe lexical et ses compléments. 

Vu l'ordre libre des constituants (en dehors du complexe verbal), on observe qu'en roumain les fonctions peuvent être définies de manière indépendante par rapport à la position dans l'arbre syntaxique, grâce à un système assez riche de marques morphosyntaxiques. Ainsi, le sujet est le constituant qui s'accorde avec le verbe en nombre et personne, il lie le clitique réfléchi \textit{se}, et habituellement il ne comporte pas de redoublement clitique ou une autre marque casuelle\footnote{Le nominatif roumain n'a pas de marque spéciale (cf. Pană \citet{Dindelegan2003}).}. En revanche, les compléments objet (à l'accusatif), oblique (au datif) et prépositionnel (accusatif ou datif) disposent de plusieurs moyens d'identification fonctionnelle : le redoublement clitique (très fréquent en roumain), marques prépositionnelles, marques casuelles, la marque \textit{pe} de l'objet direct (qui a, en dehors de son rôle syntaxique, un rôle sémantico-lexical, c.-à-d. \textit{pe} sélectionne des expressions nominales ayant les traits [+ personne], [+ détermination définie]).

Qu'il y ait sujet ou non, une phrase doit comporter une tête prédicative (cf. section \ref{sec:1.1}). Si on pense au schéma de la phrase canonique en français ou en anglais, il comporte un syntagme de type sujet-tête, dont les catégories sont le plus souvent un syntagme nominal et un syntagme verbal (ce dernier regroupant le verbe tête et ses dépendants). La notion de syntagme verbal, comme constituant intermédiaire, apparaît dans la plupart des descriptions linguistiques, mais les arguments empiriques pour le postuler sont remis en cause dans les analyses contemporaines. En anglais ou en français, on pourrait considérer que la distribution des adverbes est un argument en faveur de cette hypothèse, car on ne trouve pas d'adverbe intégré prosodiquement entre le verbe et l'objet en anglais \REF{ex:1:67}, ou encore entre le sujet et le verbe en français \REF{ex:1:68}a, à moins qu'il soit incident \REF{ex:1:68}b (voir \citet{Pollock1989} pour le français). Néanmoins, pour le français, Abeillé (2002) et \citet{Creissels2006}) argumentent que ce constituant intermédiaire n'a pas de fondement (il n'y a pas de raison phonétique ou prosodique pour faire de la séquence verbe-compléments un constituant ; l'application de tests syntaxiques ne le justifie pas non plus). 


\begin{enumerate}
\item \label{bkm:Ref299313205}a  John has (\textbf{often}) read (*\textbf{often}) this book (\textbf{often}).  


\end{enumerate}
b  John (\textbf{often}) reads (*\textbf{often}) this book (\textbf{often}).


\begin{enumerate}
\item \label{bkm:Ref299314164}a  *Jean \textbf{souvent} boit de l'eau pendant le repas.


\end{enumerate}
b  Jean, \textbf{souvent}, boit de l'eau pendant le repas.

Quant au roumain, l'utilité d'un n{\oe}ud intermédiaire entre le verbe lexical et la phrase n'est pas évidente non plus, car on a des arguments qui montrent que le sujet est au même niveau que les compléments. Premièrement, contrairement à l'anglais ou au français, la distribution des adverbes n'est pas contrainte, le roumain permettant l'insertion des adverbes~intégrés prosodiquement entre le sujet préverbal et le verbe.


\begin{enumerate}
\item a  Ion (\textbf{adesea}) cântă (\textbf{adesea}) la pian (\textbf{adesea}).


\end{enumerate}
  Ion (souvent) joue (souvent) à piano (souvent) 

\textit{Ion joue souvent du piano}  

  b  (\textbf{Mereu})\textbf{ } noi  (\textbf{mereu})  ne spălăm  (\textbf{mereu}) pe mâini (\textbf{mereu}) înainte de masă.

    (\textsc{adv})  nous  (\textsc{adv})  \textsc{cl} lavons  (\textsc{adv}) \textsc{mrq}  mains (\textsc{adv}) avant de repas 

  \textit{Nous lavons toujours les mains avant le repas}

Deuxièmement, on observe que sujets et compléments peuvent être mélangés sans restriction dans la phrase simple. D'une part, comme on l'a vu dans la section \ref{sec:1.3.4}, la position préverbale n'est pas dédiée au sujet, mais elle peut concerner aussi les compléments du verbe. Ainsi, en \REF{ex:1:70}a, le constituant préverbal est un complément oblique ; en \REF{ex:1:70}b, il s'agit d'un complément objet, alors qu'en \REF{ex:1:70}c et \REF{ex:1:70}d, on a à la fois un complément et le sujet en position préverbale. D'autre part, le sujet peut se trouver entre deux compléments en position postverbale \REF{ex:1:71}, s'il s'agit d'un verbe à deux compléments.  


\begin{enumerate}
\item \label{bkm:Ref299315318}a  Ioanei  îi  plac  tartele  cu  pere.


\end{enumerate}
  Ioana\textsc{.dat  cl.dat}  plaisent  les-tartes.\textsc{nom}  avec  poires 

\textit{Ioana aime les tartes aux poires}  

  b  Pe  mine  mă  cheamă  Gabriela.

    \textsc{mrq.acc  pro.acc  cl.acc}  appelle  Gabriela 

    \textit{Je m'appelle Gabriela}

  c  Ioana  MEre  mănâncă,  şi  nu  pere.

    Ioana  pommes  mange,  et  non  poires 

    \textit{Ioana mange des pommes et non des poires}

  d  Pe  mine  mama  mă  strigă  Gabi.

    \textsc{mrq.acc  pro.acc } maman  \textsc{cl.acc}  appelle  Gabi 

    \textit{Ma mère m'appelle Gabi}


\begin{enumerate}
\item \label{bkm:Ref299317110}a  Să trimită  un  mail  \textbf{careva  dintre  voi}  firmei  din  Anglia !


\end{enumerate}
  envoyer.\textsc{subj}  un  mél  quelqu'un  parmi  vous  la-firme.\textsc{dat}  de  Angleterre

\textit{Que quelqu'un parmi vous envoie un mél à la firme anglaise}  

  b  I-a  dat  vreun  cadou  \textbf{Ion}  maică-sii ?

    \textsc{cl.dat}-a  donné  un  cadeau  Ion  mère.\textsc{dat-poss} 

  \textit{Ion a-t-il offert un cadeau à sa mère}

Troisièmement, on observe que, tout comme les compléments, les sujets permettent (dans certains contextes) le redoublement, employé comme marque d'accord, cf. Barbu \citet{Mititelu2003}. La différence entre le redoublement du sujet et le redoublement des compléments consiste dans le statut du pronom qui {\guillemotleft}~redouble~{\guillemotright} la fonction en question : dans le premier cas, il s'agit d'un pronom fort, alors que dans le deuxième cas, il s'agit d'un pronom faible. En \REF{ex:1:72}a, on a le redoublement du sujet \textit{tata} `papa' par le pronom au nominatif \textit{el~}; en \REF{ex:1:72}b, le redoublement du complément objet direct \textit{pe Maria} par le clitique pronominal à l'accusatif \textit{o}, et en \REF{ex:1:72}b, le redoublement du complément oblique \textit{Mariei} par le clitique au datif \textit{îi}.


\begin{enumerate}
\item \label{bkm:Ref299317634}a  Vine  \textbf{el}  tata  imediat. 


\end{enumerate}
vient\textsc{  pro.nom}  papa  tout-de-suite

\textit{  Papa viendra tout de suite}  

  b  Ion  a  văzut-\textbf{o}  pe  Maria.

    Ion  a  vu-\textsc{cl.acc  mrq.acc}  Maria

    \textit{Ion a vu Maria}

  c  Ion  \textbf{i}-a  dat  Mariei  o  carte. 

    Ion  \textsc{cl.dat-}a  donné  Maria.\textsc{dat}  un  livre

  \textit{Ion a donné à Maria un livre}

Enfin, on note que, tout comme les compléments, les sujets permettent la~{\guillemotleft}~montée~{\guillemotright} sur le verbe d'un clitique datif à interprétation possessive, indiquant une relation de possession à l'intérieur d'un syntagme nominal (Dobrovie-\citet{Sorin1987})\footnote{Le roumain autorise que certaines informations grammaticales internes à un syntagme nominal (p.ex. la possession) puissent apparaître sur sa tête verbale. Donc, le verbe tête collecte des informations syntaxiques et sémantiques sur ses valents (clitiques pronominaux habituels), mais aussi sur les dépendents de ses valents (grâce aux clitiques pronominaux {\guillemotleft}~possessifs~{\guillemotright} au datif). 
} : 


\begin{enumerate}
\item \label{bkm:Ref307435262}a  \textbf{Mi}-a  murit  câinele. 


\end{enumerate}
    \textsc{cl.dat-aux}  mort  le-chien

\textit{  Mon chien est mort}  

  b  Câinele  \textbf{meu}  a  murit.

    le-chien  \textsc{poss  aux}  mort

    \textit{Mon chien est mort}


\begin{enumerate}
\item \label{bkm:Ref307435265}a  Ion  \textbf{şi}-a  văzut  părinții. 


\end{enumerate}
Ion  \textsc{cl.dat-}a  vu  les-parents

{\itshape
  Ion a vu ses parents}

  b  Ion  i-a  văzut  pe  părinții  \textbf{săi}.

    Ion  \textsc{cl.acc-}a  vu  \textsc{mrq.acc}  les-parents  \textsc{poss}

    \textit{Ion a vu ses parents}


\begin{enumerate}
\item \label{bkm:Ref307435268}a  Ion  \textbf{îmi}  stă  la  dispoziție. 


\end{enumerate}
Ion\textsc{  cl.dat } est  à  disposition

\textit{  Ion est à ma disposition}  

  b  Ion  stă  la  dispoziția  \textbf{mea}.

    Ion  est  à  la-disposition  \textsc{poss}

    \textit{Ion est à ma disposition}


\begin{enumerate}
\item \label{bkm:Ref307435270}a  Ion  \textbf{îmi}  este  cumnat. 


\end{enumerate}
Ion\textsc{  cl.dat } est  beau-frère

\textit{  Ion est mon beau-frère}  

  b  Ion  este  cumnatul  \textbf{meu}.

    Ion  est  le-beau-frère  \textsc{poss}

    \textit{Ion est mon beau-frère}

Ces clitiques datifs indiquent le possesseur pour tout valent du verbe, qu'il s'agisse d'un sujet \REF{ex:1:73}, d'un complément objet \REF{ex:1:74}, d'un complément prépositionnel \REF{ex:1:75} ou bien d'un attribut \REF{ex:1:76}, à condition que le clitique datif et le syntagme nominal dont il dépend appartiennent à la même phrase (\citet{Steriade1980}), ce qui explique leur comportement différent dans les constructions avec un infinitif \REF{ex:1:77}a par rapport à celles avec un subjonctif \REF{ex:1:77}b-c.


\begin{enumerate}
\item \label{bkm:Ref307435457}a  [Nu-\textbf{ți}  pot  vedea  fața]. 


\end{enumerate}
\textsc{  neg-cl.dat}  peux  voir\textsc{.inf}  le-visage

\textit{  Je ne peux voir ton visage}  

  b  *Nu-\textbf{ți}  pot  să  văd  fața.

    \textsc{neg-cl.dat}  peux  \textsc{mrq.subj } vois le-visage

    \textit{Je ne peux voir ton visage}

  c  [Nu  pot  [să-\textbf{ți}  văd  fața]]. 

    \textsc{neg } peux  \textsc{mrq.subj-cl.dat } vois  le-visage

    \textit{Je ne peux voir ton visage}

Sur la base de ces arguments empiriques, je propose donc une structure plate avec le verbe et les dépendants au même niveau (sujet, compléments, ajouts). Tous les valents sont ainsi réalisés dans le même arbre local. Cette analyse est d'ailleurs en accord avec les travaux traditionnels sur le roumain (\textit{GALR II} (2005 : 48)), qui ne distinguent pas syntagme verbal fini et phrase.\footnote{\textit{GBLR} (2010) choisit aussi une structure plate pour la phrase simple en roumain. La projection maximale est pour eux un syntagme verbal, et non une phrase.} La représentation simplifiée d'une phrase comme \REF{ex:1:78} sera donc \REF{ex:1:79}.


\begin{enumerate}
\item \label{bkm:Ref307172433}Ioana  mănâncă  mere  dimineața.


\end{enumerate}
  Ioana  mange  pommes  le-matin

  \textit{Ioana mange des pommes le matin}


\begin{enumerate}
\item \label{bkm:Ref307172546}Arbre simplifié de la phrase \REF{ex:1:78}


\end{enumerate}
{   [Warning: Image ignored] % Unhandled or unsupported graphics:
%\includegraphics[width=3.2146in,height=1.1992in,width=\textwidth]{fe443409cd384d3fb0f6390ffd77f513-img5.svm}
} 

\subsection{Formalisation HPSG}
\label{bkm:Ref299354376}Dans cette section, je présente les propriétés générales du modèle théorique que j'utilise dans cette thèse et l'architecture de la phrase dans ce modèle. 

\subsubsection{Propriétés générales du modèle HPSG}
Le cadre HPSG est un formalisme grammatical qui rend compte de l'ensemble des structures bien formées d'une langue en spécifiant une hiérarchie de contraintes de bonne formation, ce qui justifie sa place parmi les grammaires \textit{génératives} \textit{à base de contraintes}.\footnote{Voir la distinction faite par Pullum \& \citet{Scholz2001} entre les approches qu'ils appellent \textit{generative-enumerative syntax}, qui voient la grammaire comme l'énumération des ensembles d'expressions, et les approches qu'ils appellent \textit{model-theoretic syntax}, qui voient la grammaire plutôt comme un ensemble fini de contraintes sur la structure des expressions individuelles. Les modèles de ces contraintes sont les expressions décrites par la grammaire. Une expression est bien formée uniquement si elle est un modèle de la théorie.}  

HPSG est aussi un modèle \textit{lexicaliste}. Dans le sens strict du terme, cela veut dire que la syntaxe et le lexique sont séparés. Par conséquent, on ne combine pas en syntaxe les unités inférieures aux mots (qui sont traitées en morphologie),\footnote{Cf. le principe d'intégrité lexicale (Miller \& \citet{Sag1997}).}  ce qui distingue ce modèle des grammaires dérivationnelles qui manipulent les affixes dans la structure syntaxique. Dans un sens plus large, le fait d'être une grammaire lexicaliste implique qu'au moins certaines généralisations linguistiques sont directement encodées dans le lexique. Donc, le lexique n'est pas simplement une liste d'exceptions. Ainsi, de nombreux phénomènes syntaxiques (p.ex. extraction, phénomènes de {\guillemotleft}~montée~{\guillemotright}, alternances de valence) peuvent être décrits au moyen des règles lexicales, sans passer par une opération de transformation (ou dérivation). On peut donc conclure que HPSG est une grammaire \textit{non}-\textit{dérivationnelle} (c.-à-d. il n'y a pas d'opération qui dérive une structure à partir d'une autre) et \textit{non-transformationnelle} (c.-à-d. pas d'opérations de réarrangement, insertion ou effacement).

HPSG constitue un modèle \textit{surfaciste} et \textit{monostratal} dans lequel les expressions linguistiques sont modélisées sous forme de structures de traits typées, permettant l'organisation dans une notation commune d'informations linguistiques hétérogènes ; on peut ainsi capter de façon unitaire des informations phonologiques, morphologiques, syntaxiques, sémantiques, discursives et éventuellement prosodiques sur les mots ou les syntagmes, sans postuler d'isomorphie entre les différents niveaux d'analyse (en particulier la syntaxe et la sémantique). Une structure de traits traite en parallèle les informations provenant des niveaux linguistiques hétérogènes, sans passer par un mécanisme de dérivation d'un niveau à l'autre. A l'aide de ces structures de traits, on représente non seulement les catégories, mais aussi les structures en constituants et les règles de grammaire. Ce type de grammaire opère avec des représentations lexicales riches et des représentations syntaxiques générales qui peuvent être sous-spécifiées. Les syntagmes sont caractérisés par la fonction grammaticale de leurs constituants immédiats (\textit{tête, sujet, complément,} etc.), et non par leur catégorie ou par leur ordre.

HPSG est aussi une grammaire \textit{à base d'unification}, qui est le mécanisme permettant de combiner plusieurs contraintes superposées, afin d'obtenir la description d'un objet linguistique. Grossièrement, l'unification de deux structures est la plus petite structure qui contient uniquement les informations compatibles de l'une et de l'autre.

Le modèle HPSG, dans ses versions récentes, peut être considéré comme une grammaire \textit{constructionnelle} (cf. \citet{Sag1997}, Ginzburg \& \citet{Sag2000}, Sag \textit{et al.} (2003), Sag \textit{à paraître}), qui permet de représenter simultanément~les propriétés générales communes à une famille de constructions. Il possède une hiérarchie de constructions, qui permet de représenter non seulement les propriétés communes, mais aussi les éventuelles propriétés idiosyncratiques (généralement non-compositionnelles) des expressions. 

Le modèle que je retiens ici est une version constructionnelle du modèle HPSG (cf. \citet{Sag1997}, Ginzburg \& \citet{Sag2000}, Abeillé (2007)).

\subsubsection{Architecture générale de la phrase dans le modèle HPSG}
En HPSG, chaque unité linguistique est un objet d'un certain type. La liste de types d'objets linguistiques est organisée selon une hiérarchie. La hiérarchie de base est celle des \textit{signes} \REF{ex:1:80}. Les signes peuvent être des mots (angl. \textit{words}) ou des syntagmes (angl. \textit{phrases}).


\begin{enumerate}
\item \label{bkm:Ref299358974}Hiérarchie de signes


\end{enumerate}
{   [Warning: Image ignored] % Unhandled or unsupported graphics:
%\includegraphics[width=3.1043in,height=1.5661in,width=\textwidth]{fe443409cd384d3fb0f6390ffd77f513-img6.svm}
} 

Chaque signe linguistique peut être représenté sous la forme d'une structure de traits, utilisée comme cadre unique pour représenter des informations linguistiques hétérogènes (phonologiques, syntaxiques, sémantiques, discursives). La structure de traits correspondant à un signe linguistique figure en \REF{ex:1:81}. Et les mots et les syntagmes ont un contenu phonologique (représenté sous l'attribut PHON) et une variété de propriétés syntaxiques et sémantiques (regroupées sous l'attribut SYNSEM).\footnote{Les traits PHON et SYNSEM décrivant le signe en HPSG rappellent la dichotomie saussurienne \textit{signifiant} vs. \textit{signifié}.} 


\begin{enumerate}
\item \label{bkm:Ref299359517}La structure de traits d'un signe linguistique  


\end{enumerate}
  [Warning: Image ignored] % Unhandled or unsupported graphics:
%\includegraphics[width=4.0311in,height=2.5047in,width=\textwidth]{fe443409cd384d3fb0f6390ffd77f513-img7.svm}
 

En plus de ces traits qui s'appliquent simultanément aux mots et aux syntagmes, il y a des traits qui sont spécifiques à certain sous-types. Ainsi, les mots, contrairement aux syntagmes, comportent une structure argumentale (cf. le trait ARG-ST) qui regroupe dans une seule liste d'objets \textit{synsem} tous les éléments qu'ils sous-catégorisent. Les syntagmes, et non les mots, ont un trait DAUGHTERS (abrégé DTRS), dont la valeur est une liste de \textit{signes}, qui enregistre les constituants immédiats. 

Les synsems qui apparaissent sur la structure argumentale d'un mot peuvent être canoniques (\textit{canonical}) ou non-canoniques (\textit{non-canonical}). Les synsems canoniques figurent non seulement dans la structure argumentale d'un mot, mais aussi dans ses traits de valence, contrairement aux synsems non canoniques, qui n'apparaissent que dans la structure argumentale, comme l'indique le Principe de conservation des arguments en \REF{ex:1:82}. Selon ce principe, les arguments sous-catégorisés apparaissent à l'identique sur les traits de valence du prédicat (cf. la coïndiciation des variables), à l'exception des arguments typés comme non-canoniques. 


\begin{enumerate}
\item \label{bkm:Ref290386869}Principe de conservation des arguments


\end{enumerate}
  [Warning: Image ignored] % Unhandled or unsupported graphics:
%\includegraphics[width=3.6209in,height=0.811in,width=\textwidth]{fe443409cd384d3fb0f6390ffd77f513-img8.svm}
 

La hiérarchie d'objets \textit{synsem} est donnée en \REF{ex:1:83}. En dehors des arguments canoniques (réalisés localement), on identifie plusieurs sous-types d'arguments non canoniques (cf. Miller \& \citet{Sag1997}, \citet{Monachesi1999}, Ginzburg \& \citet{Sag2000}) : (i) le sous-type \textit{gap} concerne les éléments extraits dans les dépendances à distance\textit{~}; (ii) le sous-type \textit{pro} peut être utilisé pour le phénomène de pro-drop\footnote{Voir l'analyse alternative proposée par \citet{Ionescu1999} pour le phénomène de pro-drop, qui postule l'existence d'un certain type d'affixe (appelé affixe d'argument), sans réalisation argumentale, dans la structure morphologique des formes verbales. L'attribut SUJ du verbe aurait ainsi une valeur non-vide. Finalement, il y a un partage d'information concernant le cas et les indices réferentiels entre cet affixe d'argument et le SYNSEM représenté par la valeur SUJ du verbe.}  en roumain, ainsi que pour le sujet des impératifs et de certains infinitifs (donc, pour des éléments n'ayant pas de réalisation phonologique) ; (iii) le sous-type \textit{pron-affix} concerne la réalisation des clitiques pronominaux au datif et à l'accusatif, qui, bien qu'ils aient une réalisation phonologique, doivent être analysés comme des affixes verbaux, et (iv) \textit{adv-affix} concerne la réalisation des adverbiaux apparaissant à l'intérieur du complexe verbal, qui sont analysés eux aussi comme des affixes verbaux. Les éléments ainsi typés n'apparaissent pas dans les traits de valence, ils apparaissent uniquement sur la structure argumentale de la tête qui les sous-catégorise. Ils n'ont donc pas de réalisation en syntaxe. 


\begin{enumerate}
\item \label{bkm:Ref299374044}Hiérarchie de sous-types pour les valeurs \textit{synsem}


\end{enumerate}
{   [Warning: Image ignored] % Unhandled or unsupported graphics:
%\includegraphics[width=4.2591in,height=1.5654in,width=\textwidth]{fe443409cd384d3fb0f6390ffd77f513-img9.svm}
} 

Maintenant, on peut rendre compte de toutes les possibilités de sous-catégorisation d'un verbe roumain. Ainsi, un verbe comme \textit{vouloir}, qui sous-catégorise un sujet et un complément, a comme catégorie la représentation donnée en \REF{ex:1:84}, si les deux arguments sont réalisés lexicalement. En revanche, si son complément est réalisé par un clitique pronominal (p.ex. \textit{îl} `le'), la catégorie du verbe sera \REF{ex:1:85}. Si son complément est réalisé par un syntagme nominal, mais son sujet n'est pas réalisé (c.-à-d. pro-drop du sujet), la catégorie du verbe est dans ce cas \REF{ex:1:86}. Finalement, si les deux arguments sous-catégorisés ne sont pas réalisés, le verbe aura comme catégorie la représentation en \REF{ex:1:87}.  


\begin{enumerate}
\item \label{bkm:Ref299440983}Ion  \textbf{vrea}  un  măr.


\end{enumerate}
  Ion  veut\textsc{.3sg}  une  pomme

{\itshape
  Ion veut une pomme}

  [Warning: Image ignored] % Unhandled or unsupported graphics:
%\includegraphics[width=1.6228in,height=0.8402in,width=\textwidth]{fe443409cd384d3fb0f6390ffd77f513-img10.svm}
 


\begin{enumerate}
\item \label{bkm:Ref299441053}Ion  îl  \textbf{vrea}.


\end{enumerate}
  Ion \textsc{cl.acc}  veut\textsc{.3sg} 

{\itshape
  Ion le veut}

  [Warning: Image ignored] % Unhandled or unsupported graphics:
%\includegraphics[width=1.739in,height=0.7965in,width=\textwidth]{fe443409cd384d3fb0f6390ffd77f513-img11.svm}
 


\begin{enumerate}
\item \label{bkm:Ref299441141}\textbf{Vreau}  un  măr.


\end{enumerate}
  veux\textsc{.1sg}  une  pomme

{\itshape
  Je veux une pomme}

  [Warning: Image ignored] % Unhandled or unsupported graphics:
%\includegraphics[width=1.6236in,height=0.8102in,width=\textwidth]{fe443409cd384d3fb0f6390ffd77f513-img12.svm}
 


\begin{enumerate}
\item \label{bkm:Ref299441213}A : - Vrei  un  măr ?  B : - \textbf{Vreau}.


\end{enumerate}
    veux\textsc{.2sg}  une  pomme    veux\textsc{.1sg}

{\itshape
  A : - Tu veux une pomme ?  B : - J'en veux}

  [Warning: Image ignored] % Unhandled or unsupported graphics:
%\includegraphics[width=1.7382in,height=0.7382in,width=\textwidth]{fe443409cd384d3fb0f6390ffd77f513-img13.svm}
 

Revenons aux syntagmes, qui, comme on l'a déjà précisé, ont un trait DAUGHTERS qui enregistre les constituants immédiats. A la suite de \citet{Sag1997} et Ginzburg \& \citet{Sag2000}, on représente les différents types de syntagmes dans une hiérarchie à deux dimensions, avec un type phrastique (CLAUSALITY) et un type combinatoire (HEADEDNESS), comme en \REF{ex:1:88}. Un syntagme hérite ainsi non seulement d'une construction phrastique (c.-à-d. un sous-type de \textit{clause}) ou non-phrastique (c.-à-d. un sous-type de \textit{non-clause}), mais aussi d'une construction endocentrique (c.-à-d. un sous-type de \textit{headed-ph}) ou exocentrique (c.-à-d. un sous-type de \textit{non-headed-ph}).  


\begin{enumerate}
\item \label{bkm:Ref290391771}Classification des syntagmes


\end{enumerate}
{   [Warning: Image ignored] % Unhandled or unsupported graphics:
%\includegraphics[width=5.7634in,height=2.0874in,width=\textwidth]{fe443409cd384d3fb0f6390ffd77f513-img14.svm}
} 

Les syntagmes sans tête comportent un attribut NON-HEAD-DTRS où sont enregistrés les constituants immédiats non-têtes \REF{ex:1:89}a. En revanche, les syntagmes avec tête présentent un attribut HEAD-DTRS où figure leur constituant immédiat tête \REF{ex:1:89}b. Tout syntagme endocentrique obéit au Principe des traits de tête généralisé (\textit{Generalized Head Feature Principle}, cf. Ginzburg \& \citet{Sag2000}), qui dit que tous les traits syntaxiques et sémantiques (c.-à-d. la valeur de l'attribut SYNSEM) sont partagés par défaut (cf. le symbole /) entre un syntagme et sa tête, ce que note la coïndiciation de la variable [1] en \REF{ex:1:88}.


\begin{enumerate}
\item \label{bkm:Ref290396507}a  \textit{phrase} ={\textgreater} [NON-HEAD-DTRS \textit{list(sign)}]  


\end{enumerate}
  b  \textit{headed-phrase} ={\textgreater} [HEAD-DTR \textit{sign}]

Maintenant qu'on a défini les syntagmes avec tête, on peut définir la notion de phrase. La phrase est un signe syntagmatique à tête saturée (ses traits de valence ont pour valeur la liste vide notée {\textless} {\textgreater}). De plus, selon Ginzburg \& \citet{Sag2000}, le contenu d'une phrase doit être un sous-type de \textit{message}. On arrive ainsi à la représentation suivante :


\begin{enumerate}
\item Représentation de la phrase


\end{enumerate}
  [Warning: Image ignored] % Unhandled or unsupported graphics:
%\includegraphics[width=2.478in,height=0.8409in,width=\textwidth]{fe443409cd384d3fb0f6390ffd77f513-img15.svm}
 

Les types de contenu sémantique sont, comme on l'a observé dans la section \ref{sec:1.2}, associés à un type de phrase. Les types de phrase apparaissent dans la hiérarchie donnée en \REF{ex:1:91}, cf. Ginzburg \& \citet{Sag2000}. Les sous-types de messages sont donnés en \REF{ex:1:92}.


\begin{enumerate}
\item \label{bkm:Ref299380534}Hiérarchie des types de phrase 


\end{enumerate}
{   [Warning: Image ignored] % Unhandled or unsupported graphics:
%\includegraphics[width=3.6244in,height=1.1154in,width=\textwidth]{fe443409cd384d3fb0f6390ffd77f513-img16.svm}
} 


\begin{enumerate}
\item \label{bkm:Ref299381388}Hiérarchie des types de contenu


\end{enumerate}
{   [Warning: Image ignored] % Unhandled or unsupported graphics:
%\includegraphics[width=4.2535in,height=1.1035in,width=\textwidth]{fe443409cd384d3fb0f6390ffd77f513-img17.svm}
} 

On a vu dans la section \ref{sec:1.3.5} qu'on n'avait pas d'arguments pour justifier l'existence d'un syntagme verbal fini en roumain, comme n{\oe}ud intermédiaire. D'ailleurs, c'est ce qu'on observe aussi dans d'autres langues (gallois, cf. Borsley \textit{et al.} (2007) ; persan, cf. Bonami \& \citet{Samvelian2009} ; français, cf. Abeillé (2002, 2007)).~Par conséquent, j'adopte une structure {\guillemotleft} plate~{\guillemotright}, dans laquelle la tête prédicative se trouve au même niveau que son sujet et ses compléments. Cette approche nécessite un type spécial de syntagme, un syntagme tête-sujet-compléments, qui obéit à la contrainte donnée en \REF{ex:1:93}. La même analyse a été proposée pour les auxiliaires anglais à sujet inversé (Pollard \& \citet{Sag1994}, Ginzburg \& \citet{Sag2000}), ainsi que pour d'autres langues (gallois, cf. Borsley \textit{et al.} (2007), ou persan, cf. Bonami \& \citet{Samvelian2009}).  


\begin{enumerate}
\item \label{bkm:Ref299382031}Syntagme de type tête-sujet-compléments


\end{enumerate}
  [Warning: Image ignored] % Unhandled or unsupported graphics:
%\includegraphics[width=3.9453in,height=1.7811in,width=\textwidth]{fe443409cd384d3fb0f6390ffd77f513-img18.svm}
 

 Ainsi, à la phrase en \REF{ex:1:94} correspond l'arbre simplifié donné en \REF{ex:1:95}. 


\begin{enumerate}
\item \label{bkm:Ref299384615}Ioana mănâncă mere.


\end{enumerate}
  \textit{Ioana mange des pommes}


\begin{enumerate}
\item \label{bkm:Ref299384650}Arbre simplifié de \REF{ex:1:94}


\end{enumerate}
{   [Warning: Image ignored] % Unhandled or unsupported graphics:
%\includegraphics[width=2.1319in,height=2.4339in,width=\textwidth]{fe443409cd384d3fb0f6390ffd77f513-img19.svm}
} 


