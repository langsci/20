%5

\section{Les subordonnées fragmentaires : les ajouts relatifs averbaux}
\subsection{Qu'est-ce qu'un ajout relatif averbal ?}
Il existe en roumain et en français des ajouts averbaux, exemplifiés en \REF{ex:5:1} et \REF{ex:5:2}, qui partagent des propriétés formelles avec certaines phrases relatives. Ces ajouts ont été décrits comme des phrases relatives elliptiques (voir \citet{Grevisse1993} pour le français, Gheorghe (2004, 2005) pour le roumain). A la suite des travaux de Bîlbîie \& Laurens (2009, 2010), je les appelle ici \textit{ajouts relatifs averbaux} (angl. \textit{Verbless Relative Adjuncts} ou VRA) en raison de cette ressemblance formelle sans pour autant les assimiler à des phrases relatives.  


\begin{enumerate}
\item \label{bkm:Ref292713923}a  La întâlnire au venit trei persoane, [\{\textbf{printre {\textbar} între}\} \textbf{care} (şi) Maria]. 


\end{enumerate}
  à rendez-vous\textsc{ aux} venu trois personnes, \{parmi {\textbar} parmi\} lesquelles (aussi) Maria

  \textit{Au rendez-vous, trois personnes sont venues, parmi lesquelles Maria (aussi)}

  b  Au venit trei persoane, [\textbf{dintre care} una ieri].

    \textsc{aux} venu trois personnes, parmi lesquelles une hier

    \textit{Plusieurs personnes sont venues, dont une hier}


\begin{enumerate}
\item \label{bkm:Ref292713928}a  Trois personnes, [\textbf{parmi lesquelles} Jean], sont venues.


\end{enumerate}
  b  Trois personnes sont venues, [\textbf{dont} une hier].

Le domaine empirique est le suivant : un ajout relatif averbal (désormais VRA) est un syntagme caractérisé par un constituant initial qui est soit un syntagme prépositionnel contenant une forme \textit{qu-} (roum.\textit{ printre} \textbf{\textit{care}}\textit{, între} \textbf{\textit{care}}\textit{, dintre} \textbf{\textit{care}}, fr. \textit{parmi} \textbf{\textit{lesquel(le)s}}), soit la forme \textit{dont} en français. Le syntagme initial est suivi d'un ou de plusieurs constituants à l'exclusion de la configuration \textit{syntagme initial + syntagme verbal fini}. Cette définition du domaine empirique exclut toute phrase relative finie, comme en \REF{ex:5:3} et \REF{ex:5:4}.


\begin{enumerate}
\item \label{bkm:Ref292718532}a  Autobuzul cu refugiați, [\textbf{printre care} se aflau şi patru români]\textsubscript{REL}, a fost întors în Gaza, din motive de securitate. 


\end{enumerate}
{\itshape
Le bus avec des réfugiés, parmi lesquels il y avait aussi quatre roumains, a été renvoyé à Gaza, par des raisons de sécurité}

  b  Datele statistice arată că în acest oraş lucrează aproximativ 160~000 de persoane, [\emph{\textbf{\textup{dintre}}}\emph{\textbf{} }\emph{\textbf{\textup{care}}} aproape 130~000 activează în domeniul privat]\textsubscript{REL}.

{\itshape
Les données statistiques montrent que dans cette ville travaillent environ 160~000 de personnes, parmi lesquelles presque 130~000 ont une activité dans le secteur privé}


\begin{enumerate}
\item \label{bkm:Ref292718535}a  Le recrutement a repris : 3~500 personnes [\textbf{dont} près de 800 cadres ont été embauchés  à l'extérieur de  France-Télécom]\textsubscript{REL}.


\end{enumerate}
  b  Il a écrit plusieurs romans, [\textbf{dont} deux ont été publiés le mois dernier]\textsubscript{REL}.

On trouve des constructions similaires dans d'autres langues romanes, comme l'italien \REF{ex:5:5} ou l'espagnol~\REF{ex:5:6} :


\begin{enumerate}
\item \label{bkm:Ref292827451}a  Quattro persone sono state arrestate, [\{\textbf{tra} {\textbar} \textbf{fra} {\textbar} *\textbf{di}\} \textbf{cui} Maria]\footnote{L'agrammaticalité de la préposition \textit{di} dans cet exemple est due à la sémantique de cette préposition, qui est incompatible avec une interprétation exemplifiante, comme on verra dans la section \ref{sec:5.3.2.2}.}. 


\end{enumerate}
{\itshape
Quatre personnes ont été arrêtées, parmi lesquelles Maria}

  b  Quattro persone sono state arrestate, [\{\textbf{tra} {\textbar} \textbf{fra} {\textbar} \textbf{di}\} \textbf{cui} due ieri].

{\itshape
Quatre personnes ont été arrêtées, parmi lesquelles deux hier}

  c  Queste spiagge offrono rifugio a molte specie d'animali, [\textbf{fra queste}, in particolare, (a) degli orsetti].

{\itshape
Ces plages offrent un refuge à plusieurs espèces d'animaux, parmi lesquelles, en particulier, (à) des oursins}


\begin{enumerate}
\item \label{bkm:Ref292827461}a  En esta foto, puedes ver varias casas, [\textbf{entre las cuales} la nuestra]. 


\end{enumerate}
dans cette photo, peux.2\textsc{sg} voir différentes maisons, parmi les quelles la nôtre

{\itshape
Dans cette photo, tu peux voir différentes maisons, parmi lesquelles la nôtre}

  b  Los doce est\'an presentes, [dos \textbf{de los cuales} representados por sus presidentes].

    les douze sont présents, deux de les quels représentés par leur président

{\itshape
Les douze sont présents, dont deux représentés par leur président} 

Ce chapitre est consacré à l'analyse des VRA et développe en particulier les aspects suivants. Premièrement, les propriétés syntaxiques et sémantiques des VRA sont très différentes de celles des phrases relatives. Une analyse qui tente de réduire les VRA à des phrases elliptiques dérivées des phrases relatives ordinaires ne permet pas de rendre compte de ces différences. Deuxièmement, les VRA sont des fragments qui peuvent contenir des clusters de constituants. Troisièmement, les VRA peuvent avoir deux types de sémantique : une interprétation exemplifiante ou bien une interprétation partitionnante.

Afin de faciliter la lecture, j'utilise la terminologie suivante : j'appelle \textit{hôte} (ou \textit{source}\footnote{Cependant, les termes de \textit{hôte} et \textit{source} ne sont pas utilisés ici en variation libre, mais en fonction de la perspective qu'on a sur le VRA : si on parle du VRA en termes purement syntaxiques comme étant un ajout incident, la phrase le contenant est l'hôte ; en revanche, si on parle du VRA en termes plutôt sémantiques comme étant un fragment phrastique (dont l'interprétation n'est pas autonome), la phrase qui offre le matériel nécessaire à l'interprétation est étiquetée comme source.})  la phrase ou l'énoncé contenant directement ou non le VRA ; le VRA est composé d'un \textit{introducteur} et d'un \textit{corps}. Les VRA introduisent toujours une relation partitive entre une expression de leur hôte (\textit{l'antécédent} ou \textit{le légitimeur}) et une expression de leur corps (\textit{l'élément distingué}).


\begin{enumerate}
\item a  [Au venit trei persoane]\textit{\textsubscript{Hôte}}, [dintre care una ieri]\textsubscript{VRA}. 


\end{enumerate}
  \textsc{aux} venu trois personnes, parmi lesquelles une hier

  \textit{Plusieurs personnes sont venues, dont une hier}

  b  Au venit trei persoane, [[dintre care]\textit{\textsubscript{Introducteur}} [una ieri]\textit{\textsubscript{Corps}}].

  c  Au venit [trei persoane]\textit{\textsubscript{Antécédent}}, dintre care [una]\textit{\textsubscript{Elément distingué}} ieri.

Les résultats de cette étude se basent essentiellement sur des exemples attestés. Pour le français, une partie des données utilisées dans cette thèse provient du Corpus Arboré de Paris 7 (Abeillé, Clément \& \citet{Toussenel2003}). Le corpus est composé d'extraits du journal \textit{Le Monde} (de 1989 à 1993). Il contient 138 instances de VRA sur 21~560 segments taggés comme {\guillemotleft}~phrases relatives~{\guillemotright}. Quant aux données du roumain, elles proviennent en grande partie des textes de presse, vu le manque d'un corpus similaire en roumain. J'ai recueilli 202 occurrences, dont 81 avec \textit{dintre care}, 63 avec \textit{între care} et 58 avec \textit{printre care}. En français, on observe une forte asymétrie entre le nombre d'instances faisant intervenir l'introducteur \textit{dont} (127 occurrences)  et celles faisant intervenir un introducteur prépositionnel avec une forme \textit{qu-} (11 occurrences). Cette asymétrie est peut-être liée au fait que la syntaxe et la sémantique des VRA avec \textit{dont} en français sont moins contraintes que celles faisant intervenir des syntagmes prépositionnels avec une forme \textit{qu-}. En revanche, les données du roumain ne disent \textit{a priori} rien sur la fréquence des trois syntagmes prépositionnels dans l'usage contemporain.\footnote{Comme le note Olivier Bonami (c.p.), la différence pourrait aussi être due au mode de constitution très différent du corpus, voire aux conventions d'annotation du Corpus Arboré de Paris 7.}

\subsection{Propriétés syntaxiques}
Dans cette section, on s'intéresse à la constituance des VRA, en regardant les propriétés distributionnelles de l'introducteur, ainsi que celles du corps du VRA. En particulier, on veut voir si le \textit{dont} dans les VRA est le même que le complémenteur \textit{dont} dans les phrases relatives ordinaires. On présente ensuite les modalités de constitution pour le corps d'un VRA, c.-à-d. si le corps contient uniquement l'élément distingué ou s'il contient d'autres constituants immédiats. Parallèlement, on observe si l'élément distingué dans le corps du VRA permet le même marquage (prépositionnel ou casuel) que l'antécédent dans la phrase hôte. Enfin, on étudie les propriétés de linéarisation du VRA par rapport à l'antécédent, pour voir s'il s'agit d'une adjacence stricte ou non.

\subsubsection{L'introducteur du VRA}
En roumain, l'introducteur du VRA est toujours un syntagme prépositionnel contenant une préposition et une forme \textit{qu-} ; en français, on trouve le plus souvent la forme \textit{dont}, mais aussi des syntagmes prépositionnels.

L'introducteur du VRA, comme l'indique son nom, précède toujours le corps du VRA, tout comme les syntagmes extraits ou les complémenteurs dans les phrases relatives ordinaires.


\begin{enumerate}
\item a  Mai multe țări sud-americane, [\textbf{printre care} şi Brazilia], exportă cafea în Europa. 


\end{enumerate}
{\itshape
Plusieurs pays sud-américains, parmi lesquels aussi le Brésil, exportent du café en Europe}

  b  *Mai multe țări sud-americane, [şi Brazilia \textbf{printre care}], exportă cafea în Europa.


\begin{enumerate}
\item a  Plusieurs personnes sont venues, [\{\textbf{parmi lesquelles {\textbar} dont}\} Jean].


\end{enumerate}
  b  *Plusieurs personnes sont venues, [Jean \{\textbf{parmi lesquelles {\textbar} dont}\}].

Le début de l'introducteur du VRA coïncide avec le début du VRA, ce qui explique l'impossibilité d'avoir un introducteur précédé par un adverbial comme roum. \textit{în mod special} `en particulier' ou fr. \textit{notamment}.   


\begin{enumerate}
\item \label{bkm:Ref296073257}a  Mai multe țări, [\textbf{printre care} în mod special Brazilia], exportă cafea în Europa. 


\end{enumerate}
{\itshape
Plusieurs pays, parmi lesquels en particulier le Brésil, exportent du café en Europe}

  b  *Mai multe țări, [în mod special \textbf{printre care} Brazilia], exportă cafea în Europa.


\begin{enumerate}
\item \label{bkm:Ref296073260}a  Plusieurs personnes sont venues, [\{\textbf{parmi lesquelles {\textbar} dont}\} notamment Jean].


\end{enumerate}
  b  *Plusieurs personnes sont venues, [notamment \{\textbf{parmi lesquelles {\textbar} dont}\} Jean].

Par la suite, je m'intéresse à la distribution des introducteurs des VRA, en discutant séparément les syntagmes prépositionnels contenant une forme \textit{qu-} et la forme \textit{dont}. Il est bien admis que les relatives mettent en jeu une diversité syntaxique impressionnante (\citet{Godard1988}, Abeillé \& \citet{Godard2006}) et qu'elles peuvent être introduites par un syntagme contenant un pronom relatif ou bien par un complémenteur.

Les critères définitoires pour la distinction entre pronom relatif et complémenteur retenus par Abeillé \& \citet{Godard2006} sont les suivants : (i) au niveau morphologique, seuls les pronoms peuvent varier en genre et en nombre ; (ii) au niveau sémantique, seuls les pronoms ont un indice référentiel, leur permettant de contraindre leur antécédent (p.ex. à dénoter un être animé) ; (iii) seuls les complémenteurs contraignent le mode de la phrase qu'ils introduisent ; (iv) seuls les pronoms peuvent être complément d'une préposition. Sur la base de ces propriétés, on doit décider si les introducteurs des VRA, les formes \textit{qu-} d'un côté et la forme \textit{dont} de l'autre, ont tous la même catégorie.  

\paragraph[Syntagmes prépositionnels]{Syntagmes prépositionnels}
En dehors des trois syntagmes prépositionnels du roumain\textit{ printre care, între care} et \textit{dintre}\footnote{En roumain non-standard, on utilise la préposition \textit{din} à la place de la préposition \textit{dintre}. Dans cette thèse, je laisse de côté ces exemples.}\textit{ care} et le syntagme prépositionnel \textit{parmi lesquel(le)s} en français, on trouve d'autres expressions exprimant l'appartenance à un ensemble, plus éventuellement un ordre d'importance au sein de cet ensemble : on a ainsi des formes comme roum. \textit{în rândul cărora, în mijlocul cărora, în fruntea cărora} et fr. \textit{au nombre desquels, au} \{\textit{sein {\textbar} centre}\}\textit{ desquels, au sommet desquels, au} \{\textit{premier {\textbar} second}\} \textit{rang} \textit{desquels}. En français, on enregistre aussi le syntagme nominal complexe\textit{ parmi les plus important(e)s desquel(le)s}, qui est assez rare, mais possible dans la mesure où la relation partitive entre l'antécédent et l'élément distingué subsiste \REF{ex:5:13}b. Ces expressions, qui habituellement dénotent des relations spatiales en dehors de leur emploi dans les VRA, sont toujours utilisées avec un sens partitif abstrait dans les VRA, ce qui explique l'agrammaticalité des exemples en \REF{ex:5:14} et \REF{ex:5:15}.  


\begin{enumerate}
\item \label{bkm:Ref292836798}Un număr de zece state, [\textbf{în rândul cărora} şi România], au semnat un acord cu ONU privind ajutorarea refugiaților. 


\end{enumerate}
{\itshape
Un nombre de dix états, au nombre desquels aussi la Roumanie, ont signé un accord avec ONU concernant le secours des réfugiés} 


\begin{enumerate}
\item \label{bkm:Ref292835771}a  Les paysages urbains et la vie en ville dépendent de multiples facteurs, [\textbf{au nombre desquels} la culture et l'histoire, le cadre naturel, les activités, la situation démographique et le niveau de développement].


\end{enumerate}
  b  Par contre, le recul perçu, ce que ressent le tireur, est une chose éminemment subjective qui est influencée par différents facteurs, [\textbf{parmi les plus importants desquels} la forme et l'ajustement de la crosse].


\begin{enumerate}
\item \label{bkm:Ref294168267}*Am amenajat mai multe camere, [\textbf{în mijlocul cărora} câte o masă]. 


\end{enumerate}
{\itshape
On a aménagé plusieurs chambres, au milieu desquelles une table} 


\begin{enumerate}
\item \label{bkm:Ref294168269}*La montagne, [\textbf{au sommet de laquelle} Jean], s'appelle le Cervin.


\end{enumerate}
Les syntagmes prépositionnels que je retiens dans ce chapitre contiennent une préposition (roum. \textit{printre, între} ou \textit{dintre} et fr. \textit{parmi}) et une forme \textit{qu-} qui est un élément anaphorique dont l'antécédent se trouve dans la phrase hôte. La forme \textit{qu-} est donc coréférente avec l'antécédent dans la phrase hôte, ce qui est signalé par l'accord morphologique de la forme \textit{qu-} avec l'antécédent en nombre (toujours pluriel : fr. \textit{lesquels {\textbar} desquels}) et en genre (cf. en français, la distinction masculin \textit{lesquels {\textbar} desquels} vs. féminin \textit{lesquelles {\textbar} desquelles}). En roumain, l'accord en nombre n'est pas {\guillemotleft}~visible~{\guillemotright} avec la forme nominatif-accusatif \textit{care} (qui présente la même forme au singulier et au pluriel), mais il est observé avec la forme datif-génitif au pluriel \textit{cărora} en \REF{ex:5:12}. En revanche, aucune de ces deux formes ne permet de distinction en fonction du genre.

On observe donc que les formes \textit{qu-} \textit{lesquel(le)s} en français et \textit{care} en roumain satisfont le premier critère (mentionné plus haut) attribué aux pronoms relatifs, à savoir elles sont des formes fléchies. De plus, les formes \textit{qu-} qu'on retrouve dans les VRA sont toujours des compléments de préposition : en français, \textit{lesquel(le)s} est le complément de la préposition \textit{parmi}, en roumain \textit{care} est le complément de la préposition \textit{printre, între} ou \textit{dintre}. Sur la base de ces deux propriétés, on peut analyser les formes \textit{qu-} comme des pronoms relatifs.   

\paragraph[La forme dont en français]{La forme \textit{dont} en français}
\label{bkm:Ref298795556}En dehors de son usage dans les VRA, la forme \textit{dont} n'apparaît qu'à l'initiale de phrases relatives (y compris coda de clivée et de pseudoclivée). Dans les relatives ordinaires, cette forme est analysée comme un complémenteur et non comme une forme \textit{qu-} prépositionnelle (Godard (1988, 1989), Abeillé, Godard \& Sag  (2003), Abeillé \& Godard (2006, 2007)), sur la base des propriétés suivantes. Premièrement, \textit{dont} dans les phrases relatives ordinaires contraint le mode de la relative ; il se combine uniquement avec des phrases finies \REF{ex:5:16}a-b, alors que les relatives introduites par une forme \textit{qu-} ne sont pas toujours finies \REF{ex:5:16}c. Deuxièmement, la forme \textit{dont} ne peut être employée comme complément de nom \REF{ex:5:17}a ou de préposition \REF{ex:5:17}c dans un syntagme extrait complexe, contrairement à ce qui se passe avec une forme \textit{qu-} \REF{ex:5:17}b-d. Troisièmement, le complémenteur \textit{dont} introduit (i) soit des phrases relatives contenant un constituant manquant (gap) marqué par la forme \textit{de} (voir dans ce sens le contraste entre l'exemple \REF{ex:5:18}a et \REF{ex:5:18}c), (ii) soit des relatives contenant un pronom coréférent avec l'antécédent (pronom  résomptif), à condition qu'il soit enchâssé sous un prédicat d'attitude propositionnelle, comme en \REF{ex:5:19}a (on explique ainsi l'agrammaticalité de l'exemple \REF{ex:5:19}b par le fait qu'il manque un pronom résomptif approprié, alors que l'inacceptabilité de l'exemple \REF{ex:5:19}c est due à la présence du prédicat \textit{n'ont pas empêché} à la place d'une expression d'attitude propositionnelle). Quatrièmement, on peut ajouter l'absence de variation en genre et en nombre pour la forme \textit{dont}, contrairement à des formes \textit{qu-} fléchies (p.ex. \textit{duquel} {\textbar} \textit{de laquelle {\textbar} desquels {\textbar} desquelles}).\footnote{J'ajoute ici le fait que, contrairement à des formes \textit{qu-} comme \textit{duquel}, \textit{dont} peut être immédiatement suivi des syntagmes disloqués (i) ou des ajouts initiaux (ii). Selon Olivier Bonami (c.p.), cet effet pourrait être dû à un conflit de registre sociolinguistique entre l'emploi de \textit{duquel} (qui apparaît plutôt dans le registre formel) et la présence d'un disloqué ou d'un ajout en début d'enchâssée (qui relève plutôt du registre informel).
(i)  Je connais quelqu'un [\textbf{dont} le numéro, il commence par 04].
(ii)  Je cherche le titre d'une chanson [\textbf{dont} à un endroit des paroles, ça dit : {\guillemotleft}~T'as pas été invité sur le plateau de la méthode Cauet~{\guillemotright} ou un truc comme ça], c'est du rock. } 


\begin{enumerate}
\item \label{bkm:Ref292871616}a  C'est bien de faire sourire la personne [\{\textbf{dont {\textbar} de laquelle}\} on est amoureux].


\end{enumerate}
  b  *Voici un livre [\textbf{dont} parler à nos enfants].

  c  Voici un nouveau produit [\textbf{duquel} parler sur nos blogs].


\begin{enumerate}
\item \label{bkm:Ref292874282}a  *Paul, 19 ans, [\textbf{contre le frère dont} le procureur avait requis 8 à 10 ans de prison], a tenté de forcer le dispositif de sécurité.


\end{enumerate}
  b  Paul, 19 ans, [\textbf{contre le frère duquel} le procureur avait requis 8 à 10 ans de prison], a tenté de forcer le dispositif de sécurité.

  c  *Mon directeur est une personne [\textbf{près dont} on se sent bien].

  d  Mon directeur est une personne [\textbf{près de laquelle} on se sent bien].


\begin{enumerate}
\item \label{bkm:Ref292876063}a  Voici un poème [\textbf{dont} je me souviens].


\end{enumerate}
  b  Je me souviens \textbf{de} ce poème.

  c  *Voici un roman [\textbf{dont} je relis régulièrement].

  d  Je relis ce roman régulièrement.


\begin{enumerate}
\item \label{bkm:Ref292876147}a  Voici un livre\textsubscript{i} [\textbf{dont} il est évident qu'il\textsubscript{i} coûte cher].


\end{enumerate}
  b  *C'est une question [\textbf{dont} il est évident que nous aurons quelques problèmes].

  c  ??un tremblement\textsubscript{i} de terre [\textbf{dont} les nouvelles normes de construction n'ont pas empêché qu'il\textsubscript{i} fasse des ravages]. 

La forme \textit{dont} dans les VRA semble partager une partie des propriétés observées \textit{plus haut} pour le complémenteur \textit{dont} dans les relatives ordinaires. En particulier, elle ne peut pas être enchâssée dans un introducteur complexe \REF{ex:5:20}a, contrairement au comportement d'une forme \textit{qu-} \REF{ex:5:20}b dans les VRA.


\begin{enumerate}
\item \label{bkm:Ref292878562}a  *On a auditionné plusieurs candidats, [\textbf{parmi les plus importants dont} Jean Pataut].


\end{enumerate}
  b  On a auditionné plusieurs candidats, [\textbf{parmi les plus importants desquels} Jean Pataut].

Cependant, il n'est pas clair que toutes les propriétés de sélection du complémenteur \textit{dont} dans les relatives ordinaires  s'appliquent aussi à la forme \textit{dont} dans les VRA. En particulier, si l'on suppose que le complémenteur \textit{dont} dans les relatives ordinaires n'a pas de contribution sémantique, on ne peut pas maintenir cette affirmation dans les VRA, car \textit{dont} dans ces contextes force une interprétation partitive (cf. 5.3.2), c.-à-d. l'élément distingué dans le VRA doit être une sous-partie de l'ensemble dénoté par l'antécédent dans la phrase hôte.  


\begin{enumerate}
\item a  Au total, dix livres ont été commandés, [(*\textbf{dont}) tous pour toi].


\end{enumerate}
  b  Au total, dix livres ont été commandés, [(\textbf{dont}) deux pour toi].

De plus, la forme \textit{dont} n'a pas le même comportement distributionnel par rapport à la coordination dans les deux contextes. Si les relatives ordinaires autorisent la répétition de \textit{dont} devant chaque phrase coordonnée \REF{ex:5:22}a, cela n'est pas autorisé dans la coordination de VRA \REF{ex:5:22}b. De ce point de vue, la forme \textit{dont} dans les VRA se rapproche des conjonctions (de coordination), dont une des propriétés prototypiques est de ne pas pouvoir se combiner entre elles. En dépit de cette ressemblance, une étude détaillée reste à faire pour voir si la forme \textit{dont} dans les VRA peut être analysée comme conjonction.\footnote{La forme \textit{dont} dans les VRA peut encore être rapprochée des items comme \textit{y compris}, \textit{sauf} ou \textit{excepté} (pouvant être considérés comme des prépositions~{\guillemotleft} a-sélectives~{\guillemotright} (\citet{Melis2001}) qui ne sélectionnent pas la catégorie de leur complément\textit{.} On doit cependant préciser qu'il y a des différences notables entre \textit{dont} des VRA et \textit{y compris} par exemple (cf. \citet{Laurens2007}) : (i) \textit{y compris} est plus mobile que \textit{dont}; (ii) l'élément distingué dénotant la sous-partie peut être absent dans le modifieur partitif introduit par \textit{y compris}, et (iii)  la méronymie est possible avec \textit{y compris}. 
(i)  a  Plusieurs personnes sont venues, \{\textbf{y compris} Marie {\textbar} Marie \textbf{y compris}\}.
  b  Plusieurs personnes sont venues, \{\textbf{dont} Marie {\textbar} *Marie \textbf{dont}\}.
(ii)  Il a vendu beaucoup de livres, \{\textbf{y compris} {\textbar} *\textbf{dont}\} sur la plage.
(iii)  Il aime les suédoises, \{\textbf{y compris} {\textbar} *\textbf{dont}\} leurs cheveux.} 


\begin{enumerate}
\item \label{bkm:Ref294029258}a  Tous mes amis [\textbf{dont} je t'ai parlé et \textbf{dont} on a discuté toute la journée] sont morts.


\end{enumerate}
  b  *Tous mes amis, [\textbf{dont} Marie et \textbf{dont} Pierre], sont venus à la fête.

Finalement, comme je le montre dans la section \ref{sec:5.4}, une analyse qui dérive les VRA des relatives verbales ordinaires n'est pas empiriquement adéquate, donc rien ne nous empêche de poser deux \textit{dont} différents en français : un complémenteur \textit{dont} qui introduit les relatives verbales ordinaires et un \textit{dont} qui introduit les VRA.

\subsubsection{Le corps du VRA} 
\label{bkm:Ref298950479}La constitution du corps d'un VRA se présente sous deux formes : i) soit un seul constituant, ii) soit plusieurs constituants formant un cluster, c.-à-d. un type de syntagme sans tête explicite, contenant une suite de constituants qui ne sont pas liés entre eux par des relations fonctionnelles.

\paragraph[Un seul constituant]{Un seul constituant}
\label{bkm:Ref298953034}Dans le premier cas de figure, l'élément distingué est le seul constituant du corps. Il peut être un syntagme nominal non-marqué (c.-à-d. un syntagme sans marquage prépositionnel en français et en roumain, ou bien un syntagme sans marquage casuel spécifique en roumain) ou bien un syntagme de toute catégorie, parallèle à son légitimeur dans la phrase hôte, donc un syntagme qui reçoit un marquage casuel (en roumain) ou prépositionnel.

Tous les VRA ne permettent pas les deux options. On observe des asymétries entre le roumain et le français, mais aussi à l'intérieur d'une même langue.~

En français : de manière générale, si on a un seul élément distingué, il ne peut pas recevoir le marquage prépositionnel ; par conséquent, on ne peut pas avoir la préposition \textit{à} dans le VRA introduit par \textit{parmi lesquelles} en \REF{ex:5:23}a ou bien la préposition \textit{avec} dans le VRA introduit par \textit{dont} en \REF{ex:5:23}b. Cependant, on doit noter qu'en dehors du Corpus Arboré de Paris 7 on trouve des exemples attestés dans lesquels les VRA introduits par \textit{dont} permettent un élément distingué marqué par une préposition, mais avec une acceptabilité variable selon les locuteurs. L'acceptabilité du marquage quand on utilise la forme \textit{dont} peut être améliorée si on ajoute un adverbe comme \textit{notamment} \REF{ex:5:23}c.


\begin{enumerate}
\item \label{bkm:Ref292976815}a  J'ai parlé à plusieurs personnes, [\textbf{parmi lesquelles}~(*à) Marie].


\end{enumerate}
  b  J'ai parlé avec certaines personnes, [\textbf{dont} (*avec) Marie].

  c  Un jeune homme annonce à divers protagonistes sa mort prochaine, [\{*\textbf{parmi lesquels} {\textbar} \%\textbf{dont}\} notamment à un psychiatre qui se sent dans l'obligation de l'aider]. 

En roumain : les introducteurs \textit{printre care} et \textit{între care}\footnote{Il n'y a \textit{a priori} aucune différence majeure entre \textit{printre care} et \textit{între care}. C'est pour cela que je garde dans mes exemples une des deux formes, c.à.d. \textit{printre care}. }   ne posent aucun problème pour le marquage casuel \REF{ex:5:24}a ou prépositionnel \REF{ex:5:24}b de l'élément distingué, pourvu qu'il soit précédé de l'adverbial \textit{şi} `aussi'. En revanche, \textit{dintre care} est généralement incompatible avec une marque casuelle \REF{ex:5:25}a ou prépositionnelle \REF{ex:5:25}b, à moins que l'élément distingué soit précédé, comme dans le cas du français \textit{dont}, des modifieurs d'éventualité\footnote{D'autres auteurs utilisent la notion de \textit{situation}.  Je préfère l'emploi de la notion d'\textit{éventualité}.  } , comme les adverbiaux \textit{mai ales} `notamment' \REF{ex:5:25}c ou \textit{cel mai mult} `le plus'.


\begin{enumerate}
\item \label{bkm:Ref292979888}a  Ion a oferit flori mai multor fete, [\textbf{printre care} şi \{Maria {\textbar} Mariei\}]. 


\end{enumerate}
  Ion a offert fleurs plusieurs\textsc{.dat} filles, parmi lesquelles aussi \{Maria {\textbar} Maria.\textsc{dat}\} 

  \textit{Ion a offert des fleurs à plusieurs filles, parmi lesquelles Maria} 

  b  Ion a vorbit cu mai multe fete, [\textbf{printre care} şi (cu) Maria].

    Ion a parlé avec plusieurs filles, parmi lesquelles aussi (avec) Maria

    \textit{Ion a parlé avec plusieurs filles, parmi lesquelles Maria}


\begin{enumerate}
\item \label{bkm:Ref292982403}a  Impactul a dus la spitalizarea mai multor persoane, [\textbf{dintre care} (*a) şapte români]. 


\end{enumerate}
l'impact a mené à l'hospitalisation plusieurs\textsc{.gen} personnes, dont \textsc{mrq.gen} sept roumains 

  \textit{L'impact a mené à l'hospitalisation de plusieurs personnes, dont sept roumains } 

  b  Dragoş lucrează cu şapte medici, [\textbf{dintre care} (*cu) doi israelieni].

    Dragoş travaille avec sept médecins, dont (avec) deux israéliens

    \textit{Dragoş travaille avec sept médecins, dont deux israéliens}

  c  Dragoş lucrează cu şapte medici, [\textbf{dintre care} mai ales cu doi israelieni].

    Dragoş travaille avec sept médecins, dont notamment avec deux israéliens

    \textit{Dragoş travaille avec sept médecins, dont notamment avec deux israéliens}

Tous les éléments distingués observés jusqu'ici étaient soit des syntagmes nominaux, soit des syntagmes prépositionnels. On doit noter qu'on trouve des exemples dans lesquels l'élément distingué correspond à une catégorie non-nominale, p.ex. une phrase, et cela dans les VRA ayant comme introducteur roum. \textit{printre care} ou fr. \textit{dont}, c.-à-d. les introducteurs les moins contraignants (tant au niveau syntaxique qu'au niveau sémantique, cf. la section \ref{sec:5.3.2.2}) dans les deux langues.


\begin{enumerate}
\item a  William Steele riscă închisoarea pe viață pentru o serie de acuzații, [\textbf{printre care} că a ajutat inamicul].  


\end{enumerate}
William Steele risque la-prison à vie pour une série d'accusations, parmi lesquelles que a.3\textsc{sg} aidé l'ennemi

{\itshape
William Steele risque la prison à vie pour une série d'accusations, dont celle d'avoir aidé l'ennemi } 

  b  Cunoscută şi drept acid ascorbic, vitamina C are diverse funcții, [\textbf{printre care} că ne apără de agenții oxidanți].

connue aussi comme acide ascorbique, la-vitamine C a différentes fonctions, parmi lesquelles que nous protège.3\textsc{sg} d'agents oxydants

{\itshape
Connue aussi comme acide ascorbique, la vitamine C a différentes fonctions, dont celle de protection contre les agents oxydants} 


\begin{enumerate}
\item a  Sur le site de Porsche, on apprend plusieurs choses, [\textbf{dont} notamment que la compagnie offre ses propres formations spécialisées pour les diplômés et professionnels de l'ingénierie].


\end{enumerate}
  b  Cette exonération est sujette à certaines conditions, [\textbf{dont} notamment que la transmission soit initiée et destinée à un utilisateur de l'Internet, et que le contenu transmis à travers le réseau de communication ne soit pas modifié]. 

Jusqu'à maintenant, on a analysé uniquement les cas dans lesquels le corps du VRA avait comme constituant immédiat un seul syntagme. Cependant, le corps du VRA peut comporter plus qu'un constituant immédiat, comme le montre l'emploi du pronom \textit{una} `l'une' en \REF{ex:5:28}a (à la place du déterminant \textit{un} `un', cf. \REF{ex:5:28}b) ou bien la présence des adjectifs prédicatifs \REF{ex:5:29}a. Dans ces deux situations, les constituants du corps ne peuvent jamais former un seul syntagme nominal, cf. \REF{ex:5:28}c-\REF{ex:5:29}b. 


\begin{enumerate}
\item \label{bkm:Ref293417995}a  ... patru persoane, [\textbf{dintre care} [[una]\textsubscript{NP} [cetățean american]]\textsubscript{CLUSTER}] ...


\end{enumerate}
{\itshape
quatre personnes, dont l'une citoyen américain}

  b  ... patru persoane, [\textbf{printre} \textbf{care} [un cetățean american]\textsubscript{NP}] ...

{\itshape
quatre personnes, parmi lesquelles un citoyen américain}

  c  \{Un {\textbar} *una\} cetățean american a fost găsit mort într-un hotel din Cluj.

{\itshape
Un citoyen américain a été trouvé mort dans un hôtel de Cluj}


\begin{enumerate}
\item \label{bkm:Ref293418658}a  Mi-a dat diverse motive, [\textbf{dintre care} cele mai multe stupide].


\end{enumerate}
{\itshape
Il m'a donné plusieurs arguments, dont la plupart stupides} 

  b  *[cele mai multe stupide]\textsubscript{NP}

{\itshape
la plupart stupides}

\paragraph[Un cluster de constituants]{Un cluster de constituants}
Dans le deuxième cas de figure, l'élément distingué n'est pas le seul constituant du corps du VRA, mais il est accompagné par un syntagme prédicatif. En fonction du statut syntaxique de ce syntagme prédicatif, on distingue trois types majeurs de clusters.

Le type I met en jeu des clusters composés d'un élément distingué généralement non-marqué (dénotant une sous-partie de l'antécédent) et d'un modifieur de l'élément distingué, qui restreint la dénotation de celui-ci, cf. les exemples en \REF{ex:5:30} pour le roumain et les exemples en \REF{ex:5:31} pour le français. 


\begin{enumerate}
\item \label{bkm:Ref293333940}a  Ion a oferit flori mai multor persoane, [\textbf{dintre care} majoritatea fete].


\end{enumerate}
    \textit{Ion a offert des fleurs à plusieurs personnes, dont la plupart des filles}

  b  Maria a îngrijit de şapte bolnavi, [\{\textbf{dintre {\textbar} printre}\} \textbf{care} doi copii în stare foarte gravă].

    \textit{Maria a soigné sept malades, dont deux enfants en état très grave}

\textit{ } c  Mi-a dat diverse motive, [\textbf{dintre care} cele mai multe stupide].

    \textit{Il m'a donné plusieurs arguments, dont la plupart stupides}


\begin{enumerate}
\item \label{bkm:Ref293333959}a  Je vends seize jeux, [\textbf{dont} la plupart encore dans leur boîte].


\end{enumerate}
  b  Au total, près de cent vingt films seront projetés en dix jours, [\textbf{dont} une soixantaine, inédits, en compétition]. 

  c  J'ai vu beaucoup d'enfants, [\textbf{dont} la plupart plus âgés que toi].

Les clusters de type II présentent un élément distingué et un modifieur d'éventualité, c.-à-d. le syntagme prédicatif dénote une propriété de l'éventualité, cf. \REF{ex:5:32} en roumain et \REF{ex:5:34} en français. A ce type de clusters s'ajoutent les cas spéciaux avec catégories {\guillemotleft}~mixtes~{\guillemotright} pour certains participes passés (p.ex. roum. \textit{răniți} `blessés', fr. \textit{blessés} ou \textit{tués}) ou adjectifs (p.ex. roum. \textit{morți} `morts') employés comme noms dans les exemples \REF{ex:5:33} et \REF{ex:5:35}. Ces clusters sont construits comme si la catégorie {\guillemotleft}~mixte~{\guillemotright} en question intervenait dans la relation d'éventualité du cluster et non dans l'attribution d'une propriété à un participant, ce qui explique pourquoi on utilise un adverbe (p.ex. roum. \textit{grav} `grièvement' et fr. \textit{grièvement}) pour modifier la relation, et non un adjectif.  


\begin{enumerate}
\item \label{bkm:Ref293337225}a  Media precipitațiilor anuale este de circa 1~000 mm, [\textbf{dintre care} între 50 şi 60\% vara].  


\end{enumerate}
{\itshape
La moyenne des précipitations anuelles est d'environ 1~000 mm, dont entre 50\% et 60\% l'été}

  b  M-am întâlnit cu mai multe persoane, [\textbf{printre care} şi cu Maria ieri].

    \textit{J'ai rencontré plusieurs personnes, dont Maria hier}

  c  Anul trecut au murit 25~000 de persoane, [\textbf{dintre care} 70\% de cancer].

    \textit{L'an passé sont morts 25~000 personnes, dont 70\% de cancer} 

  d  In noaptea aceasta, s-au născut trei copii, [\textbf{dintre care} unul mort].

    \textit{Cette nuit sont nés trois enfants, dont un mort}


\begin{enumerate}
\item \label{bkm:Ref293351070}a  Bilanțul accidentului se ridică la opt răniți, [\textbf{dintre care} doi foarte grav].  


\end{enumerate}
{\itshape
Le bilan de l'accident s'élève à huit blessés, dont deux très grièvement}

  b  Am consemnat 84 de morți în avalanşe în munții noştri, [\textbf{dintre} \textbf{care} 23 în avalanşa catastrofală de la Bâlea din 17 aprilie 1977.

{\itshape
Nous avons consigné 84 morts dans les avalanches dans nos montagnes, dont 23 dans l'avalanche catastrophique de Bâlea, le 17 avril 1977}


\begin{enumerate}
\item \label{bkm:Ref293337231}a  Les entreprises ont créé beaucoup d'emplois de cadres, [\textbf{dont} 170~000 au cours des deux dernières années].


\end{enumerate}
  b  J'ai parlé à plusieurs personnes, [\{\textbf{dont} {\textbar} *\textbf{parmi lesquelles}\} Marie hier]. 

  c  Plusieurs personnes sont mortes dans mon entourage, [\textbf{dont} deux de cancer du côlon].

  d  Sept hommes ont été mis en examen et incarcérés, [\textbf{dont} l'un pour meurtre avec préméditation].


\begin{enumerate}
\item \label{bkm:Ref293351072}a  On dénombre 200 blessés, [\textbf{dont} la plupart trop grièvement pour être soignés sur place]. 


\end{enumerate}
  b  L'Europe arrive en tête du triste palmarès des vingt pays concernés, avec 25 tués, [\textbf{dont} 12 en Turquie et 11 en ex-Yougoslavie]. 

Dans les clusters de type III, l'élément distingué est suivi d'un syntagme parallèle à un argument sous-catégorisé du verbe de la phrase hôte. Chaque syntagme du cluster doit avoir le même marquage que son correspondant dans la phrase hôte, c.-à-d. l'antécédent. Le type III comme le type II peuvent donner des clusters qui imitent la syntaxe de l'hôte (c.-à-d. ils se comportent comme s'il s'agissait d'une phrase contenant une forme verbale du même lexème que le verbe de la phrase hôte) et qui se rapprochent des constructions elliptiques comme le gapping ou la coordination de séquences. Dans les clusters similaires aux constructions à gapping \REF{ex:5:36} et \REF{ex:5:37}, les syntagmes du cluster correspondent à des syntagmes légitimeurs encadrant le verbe dans la phrase hôte. Dans les clusters similaires aux coordinations de séquences \REF{ex:5:38} et \REF{ex:5:39}, les syntagmes du cluster correspondent à des syntagmes légitimeurs se trouvant à droite du verbe dans l'hôte.


\begin{enumerate}
\item \label{bkm:Ref293353980}a  La petrecere, mai toți au vorbit cu câte cineva, [\textbf{printre care} şi Dan cu Ioana]. 


\end{enumerate}
{\itshape
A la fête, presque tous ont parlé avec quelqu'un, parmi lesquels Dan avec Ioana}

  b  Mai mulți prieteni s-au stabilit în străinătate, [\textbf{dintre} \textbf{care} doi la Roma].

{\itshape
Plusieurs amis se sont établis à l'étranger, dont deux à Rome}

  c  In România, trăiesc aproximativ 8~000 de evrei, [\textbf{dintre care} jumătate în Bucureşti].

{\itshape
En Roumanie, vivent environ 8~000 juifs, dont la moitié à Bucarest}


\begin{enumerate}
\item \label{bkm:Ref293353982}a  Certains ont parlé à mes amis, [\textbf{dont} Marie à Marc].


\end{enumerate}
  b  Plusieurs de mes amis ont acheté des boîtes de pâté, [\textbf{dont} Marie deux de porc].

  c  Tous les ministres mentent à leurs électeurs, [\textbf{dont} notamment Dupont aux français]. 


\begin{enumerate}
\item \label{bkm:Ref293353999}a  De 8 Martie, Dan a oferit flori mai multor colege, [\textbf{printre care} şi Mariei un buchet de ghiocei]. 


\end{enumerate}
{\itshape
Pour le 8 Mars, Dan a offert des fleurs à plusieurs collègues, dont à Marie un bouquet de perce-neige} 

  b  Am primit cadouri de la mai mulți, [\textbf{printre care} şi o rochie de la Maria].

{\itshape
J'ai reçu des cadeaux de la part de plusieurs, dont une robe de la part de Maria}

  c  Cât a fost în armată, Ion a trimis 60 de scrisori familiei, [\textbf{dintre care} mai mult de jumătate iubitei sale].

{\itshape
Pendant son service militaire, Ion a envoyé 60 lettres à sa famille, dont plus de lam moitié à sa bien-aimée}


\begin{enumerate}
\item \label{bkm:Ref293354001}a  Paul a vendu des gâteaux à plusieurs personnes, [\textbf{dont} un à Marie hier sur la plage].


\end{enumerate}
  b  Jean a donné plein d'excuses à sa famille, [\textbf{dont} la plupart à sa s{\oe}ur].

  c  Il a envoyé trois lettres à ses amis, [\textbf{parmi} \textbf{lesquelles} deux à Marie].

\subsubsection{Propriétés de linéarisation}
De manière générale, le VRA est adjacent à l'antécédent dans la phrase hôte. Cependant, comme les phrases relatives extraposées, les VRA peuvent ne pas être strictement adjacents à leur antécédent dans la phrase hôte. Dans cette section, je m'intéresse aux contraintes qui pèsent sur la linéarisation de ces ajouts averbaux par rapport à l'hôte.

Les propriétés de linéarisation des VRA dépendent essentiellement de trois facteurs : la linéarisation relative de l'antécédent, la forme syntaxique du corps du VRA et la fonction de l'antécédent au sein de la phrase hôte. On obtient ainsi les généralisations suivantes : Dans tous les cas, l'antécédent précède toujours le VRA, ce qui explique l'agrammaticalité des exemples dans lesquels l'antécédent suit le VRA.


\begin{enumerate}
\item a  *[\textbf{Printre care} şi cea din Bucureşti], 18 grădini zoologice vor fi închise pe perioada iernii. 


\end{enumerate}
parmi lesquels aussi celui de Bucarest, 18 jardins zoologiques seront fermés pour la période de l'hiver

{\itshape
18 jardins zoologiques, parmi lesquels celui de Bucarest, seront fermés pendant l'hiver} 

  b  *Am văzut, [\textbf{printre care} şi cea din Bucureşti], 18 grădini zoologice închise

    \textsc{aux.1sg} vu, parmi lesquels aussi celui de Bucarest, 18 jardins zoologiques fermés

    \textit{J'ai vu 18 jardins zoologiques fermés, parmi lesquels celui de Bucarest} 

  c  18 grădini zoologice, [\textbf{printre care} şi cea din Bucureşti], vor fi închise pe perioada iernii.

{\itshape
18 jardins zoologiques, parmi lesquels celui de Bucarest, seront fermés pendant l'hiver} 

  d  18 grădini zoologice vor fi închise, [\textbf{printre care} şi cea din Bucureşti].

{\itshape
18 jardins zoologiques seront fermés, parmi lesquels celui de Bucarest}


\begin{enumerate}
\item a  *[\textbf{Dont} Marie], plusieurs personnes sont venues. 


\end{enumerate}
  b  *J'ai vu, [\textbf{dont} Marie], plusieurs personnes.

  c  Plusieurs personnes, [\textbf{dont} Marie], sont venues.

  d  Plusieurs personnes sont venues, [\textbf{dont} Marie].

De même, dans les VRA avec clusters, si le VRA contient un constituant parallèle à l'un des constituants de l'hôte, le VRA doit suivre le constituant de l'hôte en question. Ainsi, dans les exemples suivants, le VRA dont le corps est un cluster doit suivre et le légitimeur de l'élément distingué, et le syntagme parallèle au deuxième constituant du corps du VRA.\footnote{Une étude plus détaillée reste à faire sur un échantillon plus important d'exemples, afin de vérifier s'il y a une contrainte plus forte sur la linéarisation de clusters, à savoir s'ils doivent être en fin de phrase, et pas seulement suivre les syntagmes parallèles, cf. le contraste (i)-(ii) observé par Olivier Bonami (c.p.) :
(i)  Plusieurs personnes ont offert un cadeau au fils de Paul, [\textbf{dont} Marie un livre].
(ii)  *Plusieurs personnes ont offert un cadeau, [\textbf{dont} Marie un livre], au fils de Paul.} 


\begin{enumerate}
\item a  *Am primit cadouri, [\textbf{printre} \textbf{care} şi o rochie de la Maria], de la mai mulți.


\end{enumerate}
ai reçu cadeaux, parmi lesquels aussi une robe de Maria, de plusieurs

{\itshape
J'ai reçu des cadeaux de la part de plusieurs personnes, parmi lesquels une robe de la part de Maria} 

  b  Am primit cadouri de la mai mulți, [\textbf{printre} \textbf{care} şi o rochie de la Maria]. 

{\itshape
J'ai reçu des cadeaux de la part de plusieurs, parmi lesquels une robe de la part de Maria}


\begin{enumerate}
\item a  *Plusieurs personnes, [\textbf{dont} Marie un livre], m'ont offert un cadeau. 


\end{enumerate}
  b  Plusieurs personnes m'ont offert un cadeau, [\textbf{dont} Marie un livre]. 

Un des critères les plus importants dans la linéarisation des VRA concerne le statut syntaxique de l'antécédent, c.-à-d. dépendant direct ou non de la tête de la phrase hôte. Si l'antécédent n'est pas un dépendant direct de la tête de l'hôte, le VRA doit suivre directement l'antécédent. De même, un antécédent se trouvant dans une phrase enchâssée (cf. \REF{ex:5:44}d-\REF{ex:5:45}d) n'est pas accessible ; de ce point de vue, la relation entre un VRA et son antécédent ressemble à d'autres relations non-locales à droite, comme l'extraposition ou la dislocation droite, qui obéissent à la \textit{Right Roof Constraint} (\citet{Ross1967}, Soames \& \citet{Perlmutter1979}).


\begin{enumerate}
\item \label{bkm:Ref293501720}a  Reprezentanții mai multor țări, ([\textbf{printre care} şi Brazilia]), s-au reunit ieri, (*[\textbf{printre care} şi Brazilia]). 


\end{enumerate}
Les-représentants plusieurs.\textsc{gen} pays, (parmi lesquels aussi Brésil), se sont réunis hier, (parmi lesquels aussi Brésil)

{\itshape
Les représentants de plusieurs pays, dont le Brésil, se sont réunis hier} 

  b  După primirea invitaților, ([\textbf{printre care} şi Maria şi Ion]), m-am întors la bucătărie, (*[\textbf{printre care} şi Maria şi Ion]).

    après la-réception invités.\textsc{gen,} (parmi lesquels aussi Maria et Ion), \textsc{refl}{}-suis retourné dans cuisine, (parmi lesquels aussi Maria et Ion)

{\itshape
Après la réception des invités, parmi lesquels Maria et Ion, je suis retourné à la cuisine}

  c  Sugestiile făcute de mai mulți dintre prietenii tăi, ([\textbf{printre} \textbf{care} şi Ion]), m-au ajutat enorm, ([\textbf{printre care} şi *(cea făcută de) Ion]).

    les-suggéstions faites de plusieurs de amis \textsc{poss}, (parmi lesquels aussi Ion), m'ont aidé énormément, (parmi lesquels aussi celle faite par Ion)

{\itshape
Les suggéstions faites par plusieurs de tes amis, dont Ion, m'ont énormément aidé}

  d  Că lipsesc mai multe persoane, ([\textbf{printre care} şi Maria]), nu mă şochează, (*[\textbf{printre care} şi Maria]).

    que manquent plusieurs personnes, (parmi lesquelles aussi Maria), \textsc{neg} me choque, (parmi lesquelles aussi Maria)

{\itshape
Qu'il manque plusieurs personnes, parmi lesquelles Maria, ne me choque pas}


\begin{enumerate}
\item \label{bkm:Ref293501723}a  Des représentants de plusieurs pays, ([\textbf{dont} le Brésil]), se sont réunis hier, (*[\textbf{dont} le Brésil]). 


\end{enumerate}
  b  Après avoir reçu les invités, ([\textbf{parmi lesquels} Marie et Jean]), je suis retourné à la cuisine, (*[\textbf{parmi lesquels} Marie et Jean]).

  c  Les idées proposées par plusieurs de tes amis, ([\textbf{dont} Jean]), m'ont beaucoup aidé, ([\textbf{dont} *(celle de) Jean]).

  d  Que seulement deux personnes soient venues, ([\textbf{dont} Marie]), ne devrait pas t'étonner, (*[\textbf{dont} Marie]).

Sinon, le VRA peut être linéarisé n'importe où dans l'hôte après l'antécédent, modifiant tout dépendant direct (non-incident) de la tête de l'hôte, qu'il s'agisse d'un sujet, d'un complément ou d'un ajout, ce qui nous amène à conclure que la linéarisation des VRA implique une adjacence non-stricte par rapport à l'antécédent dans la phrase hôte.


\begin{enumerate}
\item a  Mai multe conturi de Twitter au fost hack-uite, [\textbf{printre care} Obama, Fox şi Britney Spears]. 


\end{enumerate}
{\itshape
Plusieurs comptes Twitter ont été piratés, parmi lesquels Obama, Fox et Britney Spears}

  b  I-am rugat pe câțiva prieteni să m-ajute, [\textbf{printre} \textbf{care} şi pe Ion].

{\itshape
J'ai demandé à quelques amis de m'aider, dont à Ion}

  c  Am lucrat douăsprezece ore ieri, [\textbf{dintre care} opt în fața calculatorului].

{\itshape
J'ai travaillé douze heures hier, dont huit devant l'ordinateur}


\begin{enumerate}
\item a  Plusieurs personnes sont venues, [\textbf{dont} Marie]. 


\end{enumerate}
  b  On estime à 2 milliards l'aide nécessaire, [\textbf{dont} 8 millions pour la Russie].

  c  J'ai attendu trois heures lundi, [\textbf{dont} deux sous la pluie].

Sur la base des propriétés d'occurrence dans la phrase, on peut faire l'hypothèse que le VRA est un ajout incident. L'incidence est une propriété syntaxique, qui a des conséquences prosodiques (Bonami \& Godard (2007c, 2008a)). Comme je l'ai mentionné dans le chapitre 2 (section \ref{sec:2.8}), les incidents s'interpolent parmi les constituants majeurs d'un hôte phrastique ou non-phrastique. Leur placement dans la phrase est relativement libre. Au niveau prosodique, les incidents sont des constituants qui sont isolés prosodiquement du reste de la phrase dans laquelle ils apparaissent, contrairement aux constituants ordinaires qui reçoivent une prosodie integrée. On peut ainsi considérer les VRA comme des ajouts incidents en vertu des propriétés suivantes : l'adjacence non-stricte observée \textit{plus haut}, le caractère optionnel du VRA, ainsi que le {\guillemotleft}~détachement~{\guillemotright} prosodique. En ce qui concerne ce dernier critère, on observe que le VRA est séparé de la phrase hôte par des pauses à l'oral et par des virgules à l'écrit, ce qui est catalogué dans la littérature comme {\guillemotleft}~comma intonation~{\guillemotright}. Ces propriétés justifient aussi le terme utilisé par les grammaires descriptives du roumain (voir \citet{Gheorghe2005}) pour décrire les VRA, à savoir les relatives partitives périphériques (isolées).

\subsubsection{Synthèse}
Dans cette section, on a observé le comportement syntaxique des VRA. L'introducteur, qu'il s'agisse d'un syntagme prépositionnel (contenant une préposition et une forme \textit{qu-} coréférente avec un antécédent dans l'hôte) ou bien de la forme \textit{dont} en français, est toujours en première position dans le VRA. Après avoir inventorié les propriétés générales de la forme \textit{dont}, on a montré qu'on avait besoin de deux \textit{dont} différents en français : un \textit{dont} complémenteur (cf. Godard (1988, 1989), Abeillé \& Godard (2006, 2007)) et un \textit{dont} marquant la construction VRA. Le corps du VRA peut être composé d'un seul constituant (qui coïncide avec l'élément distingué) ou bien d'un cluster de constituants (contenant au moins l'élément distingué et un syntagme prédicatif). L'élément distingué sous ces deux formes est généralement non-marqué, avec des différences en français et en roumain : si en français le marquage prépositionnel est impossible avec \textit{parmi lesquel(le)s} et, pour la plupart des locuteurs, avec \textit{dont} aussi, en roumain le marquage prépositionnel ou casuel est possible avec \textit{printre care} et \textit{între care}. En ce qui concerne les propriétés de linéarisation, le VRA doit, dans tous les cas, suivre l'antécédent et, dans le cas des clusters, il doit suivre aussi tout syntagme parallèle à un des constituants du cluster. Le VRA est strictement adjacent à l'antécédent si celui-ci n'est pas un dépendent direct de la tête de la phrase hôte ou s'il se trouve dans une phrase enchâssée. A part ces contraintes, l'adjacence du VRA par rapport à la phrase hôte est non-stricte, ce qui valide l'hypothèse selon laquelle le VRA est un ajout incident.

\subsection{Propriétés sémantiques}
Dans cette section, je regarde les propriétés sémantiques des VRA. Dans un premier temps, on s'intéresse à savoir si les VRA présentent des propriétés communes avec les relatives restrictives ou non-restrictives (en particulier, s'ils ont une interprétation intersective et/ou s'ils font partie du contenu asserté de l'hôte). Dans un deuxième temps, on étudie en détail la sémantique partitive mise en jeu dans les VRA, pour voir quelles sont les contraintes sémantiques qui pourraient expliquer les préférences des locuteurs pour certains introducteurs dans certaines configurations des VRA en roumain et en français. 

\subsubsection{Sémantique non-restrictive}
\label{bkm:Ref295412149}A la distinction syntaxico-prosodique entre les constituants intégrés et les constituants incidents se superpose une opposition sémantique, classique dans les travaux sur les phrases relatives, entre les relatives restrictives et les relatives non-restrictives ou appositives (Arnold (2004, 2007), Arnold \& \citet{Borsley2008})\footnote{En lien avec le fonctionnement syntaxique d'ajout, Arnold \& \citet{Borsley2008} précisent que les relatives non-restrictives peuvent modifier soit directement leur antécédent soit toute la phrase hôte, alors que les relatives restrictives modifient uniquement leur antécédent. } . 

Les relatives restrictives sont interprétées comme des modifieurs intersectifs : elles restreignent l'ensemble dénoté par l'antécédent à un sous-ensemble particulier, ce qui explique pourquoi les relatives restrictives ne sont pas compatibles avec les noms propres. L'effet de cette interprétation intersective est d'introduire un ensemble complémentaire implicite, qui est rendu accessible par des anaphores comme \textit{les autres} (\citet{Arnold2004}). En revanche, les relatives non-restrictives ont une interprétation non-intersective : elles ne restreignent pas la dénotation de leur antécédent, mais ajoutent tout simplement une information sur celui-ci (ce qui explique le terme de relative `supplémentaire' utilisé par Huddleston \& \citet{Pullum2002}), information dont on n'a pas besoin pour délimiter l'ensemble dénoté par l'antécédent.

Cette interprétation non-intersective a conduit les chercheurs à considérer que les relatives non-restrictives sont syntaxiquement subordonnées, mais que, sémantiquement, elles se comportent comme des phrases indépendantes, ayant leur propre valeur de vérité et leur propre force illocutoire (voir \citet{Peterson2004b}, \citet{Arnold2004}). Ainsi, on considère que le contenu d'une relative non-restrictive est parenthétique, c.-à-d. il ne contribue pas aux valeurs de vérité de la phrase hôte (Huddleston \& \citet{Pullum2002}).~Les parenthétiques sont forcément véridicaux (la phrase contenant une relative non-restrictive implique la phrase sans la relative non-restrictive), parce qu'ils ne contribuent pas à l'acte illocutoire principal. Par conséquent, les relatives non-restrictives expriment des implicatures conventionnelles (\citet{Potts2005}), alors que les relatives restrictives font partie du contenu principal.

En ce qui concerne l'intersectivité de l'interprétation, les VRA se comportent comme les relatives non-restrictives, c.-à-d. le VRA n'est pas un modifieur restrictif de l'antécédent. Ainsi, le VRA dans les phrases \REF{ex:5:48}b-c ne restreint pas l'ensemble des amis de Marie à celui incluant Dan (en \REF{ex:5:48}b) ou bien à celui incluant les cinq policiers (en \REF{ex:5:48}c). La présence d'un VRA ne rend pas accessible un ensemble complémentaire auquel on peut référer avec l'expression anaphorique \textit{les autres.} En revanche, dans l'exemple \REF{ex:5:48}a, la relative restreint l'ensemble des amis de Maria à un sous-ensemble spécifique (\textit{Les amis de Maria dont je t'ai parlé}). Cette relative restrictive permet l'introduction d'un ensemble contrastif mis en évidence par l'expression \textit{les autres amis de Maria}. 


\begin{enumerate}
\item \label{bkm:Ref293947017}a  Prietenii Mariei [despre care ți-am vorbit] au plecat la mare. Ceilalți prieteni ai Mariei au plecat la munte.  


\end{enumerate}
{\itshape
Les amis de Maria dont je t'ai parlé sont partis à la mer. Les autres amis de Maria sont partis à la montagne}

  b  Prietenii Mariei, [\textbf{printre care} şi Dan], au murit într-o explozie. \#Ceilalți prieteni ai Mariei n-au pățit nimic.

{\itshape
Les amis de Maria, parmi lesquels Dan, sont morts dans une explosion. Les autres amis de Maria n'ont rien subi}

  c  Prietenii Mariei, [\textbf{dintre care} cinci polițişti], au murit într-o explozie. \#Ceilalți prieteni ai Mariei n-au pățit nimic.

{\itshape
Les amis de Maria, dont cinq policiers, sont morts dans une explosion. Les autres amis de Maria n'ont rien subi}


\begin{enumerate}
\item a  Les amis [dont je t'ai parlé] sont venus. Les autres viendront demain. 


\end{enumerate}
  b  Les amis de Marie, [\{\textbf{dont {\textbar} parmi lesquels}\}\textbf{} Jean], sont venus. \#Les autres amis de Marie viendront demain. 

Cependant, contrairement aux subordonnées relatives non-restrictives (voir Arnold (2004, 2007), Arnold \& \citet{Borsley2008}), le contenu du VRA fait partie du contenu asserté de l'énoncé qui l'inclut. Ainsi, on observe que les VRA se comportent différemment des autres modifieurs non-restrictifs en termes de portée des verbes d'attitude propositionnelle. Dans l'exemple \REF{ex:5:50}a, le contenu de la relative non-restrictive ne fait pas partie des croyances de Paul, mais c'est plutôt une assertion du locuteur. Par conséquent, la phrase \REF{ex:5:50}a n'entraîne pas la vérité de \REF{ex:5:50}b. En revanche, dans l'exemple \REF{ex:5:51}a, le VRA est interprété sous la portée du verbe d'attitude propositionnelle. Ainsi, la phrase \REF{ex:5:51}a entraîne la vérité de \REF{ex:5:51}b, mais pas celle de \REF{ex:5:51}c. Les mêmes observations tiennent pour le VRA introduit par \textit{dintre care} en \REF{ex:5:52}. On peut ainsi conclure que les VRA n'ont pas une portée large sur les verbes d'attitude propositionnelle, donc ils n'expriment pas d'implicatures conventionnelles.


\begin{enumerate}
\item \label{bkm:Ref293686362}a  Paul crede că regele Mihai, [care de altfel nici nu candidează], va câştiga alegerile. 


\end{enumerate}
{\itshape
Paul croit que le roi Mihai, qui d'ailleurs ne candidate même pas, va gagner les élections}

${\neq}${\textgreater}  b  Paul crede că regele Mihai nu candidează la alegeri.

{\itshape
Paul croit que le roi Mihai ne candidate pas aux élections}

={\textgreater}  c  Regele Mihai nu candidează la alegeri.

{\itshape
Le roi Mihai ne candidate pas aux élections}


\begin{enumerate}
\item \label{bkm:Ref293686484}a  Paul crede că anumite plante, [\textbf{printre care} şi sunătoarea], vindecă ulcerul. 


\end{enumerate}
{\itshape
Paul croit que certaines plantes, dont la verveine, soigne les ulcères}

={\textgreater}  b  Paul crede că sunătoarea vindecă ulcerul.

{\itshape
Paul croit que la verveine soigne les ulcères}

${\neq}${\textgreater}  c  Sunătoarea vindecă ulcerul.

{\itshape
La verveine soigne les ulcères}


\begin{enumerate}
\item \label{bkm:Ref294004599}a  Paul crede că accidentul s-a soldat cu opt răniți, [\textbf{dintre care} doi în stare gravă]. 


\end{enumerate}
{\itshape
Paul croit que l'accident s'est soldé par huit blessés, dont deux en état très grave}

={\textgreater}  b  Paul crede că doi dintre cei opt răniți sunt în stare foarte gravă.

{\itshape
Paul croit que deux parmi les huit blessés sont en état très grave}

${\neq}${\textgreater}  c  Doi dintre cei opt răniți sunt în stare foarte gravă.

{\itshape
Deux parmi les huit blessés sont en état très grave}

Par conséquent, bien que les VRA partagent certaines propriétés avec les autres modifieurs non-restrictifs (p.ex. interprétation non-intersective), le contenu des VRA n'est pas parenthétique, contrairement au contenu des autres modifieurs non-restrictifs, comme le montre la possibilité de remettre en cause le contenu du VRA en utilisant \textit{c'est faux} ou \textit{ce n'est pas vrai} (Jayez \& \citet{Rossari2004}). 


\begin{enumerate}
\item a  Anumite plante, [\textbf{printre care} şi sunătoarea], vindecă afecțiuni ale ficatului.  


\end{enumerate}
{\itshape
Le bilan de l'accident s'élève à huit blessés, dont deux très grièvement}

  b  E fals, eu ştiu că sunătoarea vindecă ULcerul, nu afecțiuni ale ficatului !

{\itshape
C'est faux, je sais que la verveine soigne les ulcères, et non les affections hépatiques } 


\begin{enumerate}
\item a  Bilanțul accidentului se ridică la opt răniți, [\textbf{dintre care} doi foarte grav].  


\end{enumerate}
{\itshape
Le bilan de l'accident s'élève à huit blessés, dont deux très grièvement}

  b  Nu e adevărat, dintre cei opt răniți PAtru sunt în stare foarte gravă, şi nu doi !

{\itshape
Ce n'est pas vrai, parmi les huit blessés quatre sont en état très grave, et pas deux}

\subsubsection{Sémantique partitive}
\label{bkm:Ref298317596}\paragraph[Elément distingué et antécédent]{Elément distingué et antécédent}
\label{bkm:Ref295410024}Les VRA dans les deux langues ont une sémantique partitive. D'une part, l'antécédent d'un VRA doit être une entité plurielle fractionable, exprimant une somme dont les sous-parties soient accessibles (angl. \textit{sum individual}, cf. \citet{Lasersohn1995}). D'autre part, l'élément distingué dans un VRA doit être interprété comme une sous-partie de l'entité fractionable dénotée par l'antécédent. Une entité A est une fraction d'une entité B si : (i) A entre dans la composition de B, et (ii) si la description de B (excepté~la cardinalité) s'applique à A. Ainsi, en \REF{ex:5:55}a et \REF{ex:5:56}a, l'entité dénotée par l'élément distingué (\textit{Marie}) est une fraction de l'entité dénotée par l'antécédent (\textit{plusieurs personnes}), car les deux conditions mentionnées \textit{plus haut} sont respectées. L'entité fractionable n'est pas toujours composée de parties atomiques, ce qui explique pourquoi un syntagme nominal contenant un nom massique peut fonctionner comme antécédent d'un VRA en \REF{ex:5:55}b et \REF{ex:5:56}b. Cependant, dans la plupart des exemples attestés, l'antécédent est un syntagme nominal quantifié au pluriel.


\begin{enumerate}
\item \label{bkm:Ref294027332}a  Au venit mai multe persoane, [\textbf{printre} \textbf{care} şi Maria].  


\end{enumerate}
{\itshape
Plusieurs personnes sont venues, parmi lesquelles Maria}

  b  Vulcanul din Islanda emană o mare cantitate de gaz, [\textbf{din}\footnote{Avec un nom massique, la préposition \textit{din} est plus appropriée que la préposition \textit{dintre}, cf. leur emploi en dehors des VRA.
(i)   unul \textbf{dintre} elevi
  \textit{l'un des élèves}
(ii)   unul \textbf{din} grup
  \textit{l'un du groupe } }\textbf{ care} 25\% dioxid de carbon].

{\itshape
Le volcan d'Islande rejette une grande quantité de gaz, dont 25\% de dioxyde de carbon } 


\begin{enumerate}
\item \label{bkm:Ref294027336}a  Plusieurs personnes sont venues, [\textbf{dont} Marie]. 


\end{enumerate}
  b  Une grande quantité de gaz s'est échappée du cratère, [\textbf{dont} 25\% de CO\textsubscript{2}].   

En revanche, les syntagmes nominaux quantifiés qui ne dénotent pas des entités fractionables ne sont pas des antécédents adéquats pour les VRA. Ainsi, on peut expliquer le contraste qui s'établit entre un syntagme nominal contenant un quantifieur (au singulier) avec une force quantificationnelle universelle comme \textit{aucun des X} ou encore \textit{tout X} et un syntagme nominal contenant le quantifieur universel \textit{chaque}. Dans le premier cas, le syntagme quantifié n'est pas un bon antécédent pour un VRA, car on ne peut pas inférer une somme, alors que le syntagme quantifié par \textit{chaque} semble être acceptable.\footnote{Olivier Bonami me fait observer qu'on a le même comportement pour l'anaphore ordinaire. \textit{Chaque X} permet facilement une reprise anaphorique par un pronom pluriel (i), alors que cela est marginal avec \textit{tout X} (ii)\textit{.}
(i)  Chaque étudiant est venu. Ils étaient bien contents.
(ii)  ??Tout étudiant viendra. Ils seront bien contents.}


\begin{enumerate}
\item \label{bkm:Ref298235479}a  *Niciunul dintre studenți, [\textbf{printre care} şi Ion], n-a venit la cursul practic ieri.  


\end{enumerate}
{\itshape
Aucun des étudiants, parmi lesquels Ion, n'est venu au cours pratique hier} 

  b  *Orice student, [\textbf{printre care} şi Maria], trebuie să vină la cursurile practice.

{\itshape
Tout étudiant, dont Maria, doit venir aux cours pratiques} 

  c  ?Fiecare student a primit câte ceva, [\textbf{printre care} şi Maria o carte].

{\itshape
Chaque étudiant a reçu quelque chose, dont Maria un livre}


\begin{enumerate}
\item \label{bkm:Ref298235482}a  *Aucun des étudiants n'est venu, [\textbf{dont} Marie]. 


\end{enumerate}
  b  *Tout étudiant doit venir, [\textbf{dont} Marie].

  c  Chaque étudiant est venu, [\textbf{dont} Marie].

En dehors de cette relation partitive entre une entité fractionable et une sous-partie, les VRA n'autorisent pas d'autre relation sémantique entre l'antécédent et l'élément distingué du VRA. Cela exclut des relations telles que la possession (qui viole la première contrainte (i) mentionnée \textit{plus haut}) et la méronymie (qui n'obéit pas à la deuxième condition)\textit{~}: dans une relation de possession \REF{ex:5:59}a-\REF{ex:5:60}a, l'élément distingué (\textit{leurs} \textit{amis}) n'entre pas dans la composition de l'antécédent (\textit{plusieurs personnes}) ; dans une relation de méronymie \REF{ex:5:59}b-\REF{ex:5:60}b, la description de l'entité fractionable (\textit{plusieurs de ces chaises}) ne s'applique pas à la partie (\textit{le pied d'une d'entre elles}).


\begin{enumerate}
\item \label{bkm:Ref294168580}a  \#Mai multe persoane\textsubscript{i}, [\textbf{printre care} şi prietenii lor\textsubscript{i}], au venit la petrecere.  


\end{enumerate}
{\itshape
Plusieurs personnes, parmi lesquelles leurs amis, sont venus à la fête} 

  b  *Mi-au plăcut mai multe scaune, [\textbf{dintre care} mai ales piciorul unuia dintre ele].

{\itshape
J'ai aimé plusieurs chaises, dont surtout le pied d'une d'entre elles} 


\begin{enumerate}
\item \label{bkm:Ref294168582}a  \#Plusieurs personnes\textsubscript{i} sont venues, [\textbf{dont} leurs\textsubscript{i} amis]. 


\end{enumerate}
  b  *J'aime beaucoup les suédoises, [\textbf{dont} leurs cheveux].

Si le corps du VRA est composé d'un cluster ayant plus d'un constituant immédiat, les syntagmes qu'on trouve dans ces clusters réalisent deux fonctions au niveau sémantique : soit ils dénotent une sous-partie de l'antécédent du VRA, soit ils fonctionnent simplement comme des restricteurs sur la sous-partie introduite.

\paragraph[Interprétation exemplifiante et partitionnante]{Interprétation exemplifiante et partitionnante}
\label{bkm:Ref294028383}Les travaux antérieurs traitent ensemble les différentes constructions avec VRA, sans faire de distinction, bien qu'on note l'existence de {\guillemotleft}~certains aspects qui différencient les structures avec \textit{dintre care} des structures avec \textit{printre care} et \textit{între care}~{\guillemotright} en roumain (\citep[271]{Gheorghe2004}). Le but de cette sous-section est de montrer qu'il y a effectivement deux sémantiques différentes dans les constructions avec VRA et de préciser les critères qui nous permettent d'opérer cette distinction.

La relation qui s'établit entre la sous-partie et l'entité fractionable se définit de manière plus précise selon deux critères s'appliquant à la sous-partie : la référentialité et l'exhaustivité. Selon le premier critère, l'élément distingué est identifiable ou non indépendamment de la référence de l'antécédent. Selon le deuxième, la somme des éléments distingués est coextensive ou non à l'entité fractionable dénotée par l'antécédent.  

En fonction de ces deux critères, on arrive ainsi à deux interprétations différentes des VRA : 

(i) Interprétation exemplifiante : Dans certains VRA, les éléments distingués sont référentiels, c.-à-d. ils sont identifiables indépendamment de la référence de l'antécédent (noms propres, syntagmes nominaux définis, démonstratifs ou possessifs, ou encore certains syntagmes nominaux indéfinis avec emploi référentiel\footnote{Pour une analyse détaillée des syntagmes nominaux indéfinis, voir Dobrovie-Sorin \& \citet{Beyssade2004}.}). Ce type de VRA \textsc{nomme} un ou plusieurs éléments appartenant à l'ensemble dénoté par l'antécédent. Par conséquent, la relation sémantique qui s'établit entre l'antécédent et l'élément distingué est plutôt une relation entre un ensemble et \textsc{un élément ou une liste d'éléments} extrait(s) de cet ensemble. 

Ainsi, on observe qu'avec ce type d'interprétation le roumain autorise \textit{printre care} et \textit{între care}, mais pas\textit{ dintre care} \REF{ex:5:61}a-b-c. De plus, on remarque ici la présence (généralement optionnelle\footnote{Dans les exemples attestés, on remarque la présence de l'adverbe \textit{şi} `aussi' surtout dans les VRA dont l'élément distingué n'est pas coordonné. S'il s'agit d'une coordination, cet adverbe est généralement omis.}) de l'adverbe additif inclusif \textit{şi} `aussi', qui signale une présupposition~sur l'existence d'au moins une alternative vraie dans l'ensemble des alternatives de l'associé (Roussarie \textit{en prép.}). En revanche, en français les deux items \textit{dont} et \textit{parmi lesquel(le)s} sont compatibles avec une interprétation exemplifiante \REF{ex:5:62}a-b-c. Par ailleurs, ce type d'interprétation n'accepte pas une sous-partie exhaustive dans les deux langues, cf. \REF{ex:5:61}d en roumain et \REF{ex:5:62}d en français.


\begin{enumerate}
\item \label{bkm:Ref295148219}a  Paul a citit mai multe cărți, [\{\textbf{printre {\textbar} între {\textbar} *dintre}\} \textbf{care} (şi) Biblia].  


\end{enumerate}
{\itshape
Paul a lu plusieurs livres, parmi lesquels (aussi) la Bible}

  b  La reuniune, au fost prezenți mai mulți oficiali europeni, [\{\textbf{printre {\textbar} între {\textbar} *dintre}\}\textbf{ care} (şi) preşedintele României].

{\itshape
A la réunion ont été présents plusieurs officiels européens, parmi lesquels (aussi) le président de la Roumanie}

  c  20 de țări, [\{\textbf{printre {\textbar} între {\textbar} *dintre\} care} Rusia, Franța şi Germania], spionează intens Marea Britanie.

{\itshape
20 pays, parmi lesquels la Russie, la France et l'Allemagne, espionnent intensivement la Grande Bretagne}

  d  In total, au venit \{*două {\textbar} trei\} persoane, [\{\textbf{printre {\textbar} între}\}\textbf{ care} (şi) Maria şi Ion].

    \textit{Au total, sont venues \{deux {\textbar} trois\} personnes, parmi lesquelles Maria et Ion} 


\begin{enumerate}
\item \label{bkm:Ref295148292}a  Certains de mes amis, [\{\textbf{dont {\textbar} parmi lesquels}\} Marie], sont venus à la fête. 


\end{enumerate}
  b  Paul a parcouru cinq livres sur le sujet, [\{\textbf{dont {\textbar} parmi lesquels}\} le gros sur l'étagère].

  c  De nombreux pays, [\{\textbf{dont {\textbar} parmi lesquels}\} la France, le Royaume-Uni et l'Espagne], ont été frappés par des attentats terroristes.

  d  \{*Deux {\textbar} Trois\} personnes sont venues, [\{\textbf{dont {\textbar} parmi lesquels}\}\textbf{} Marie et Jean].

(ii) Interprétation partitionnante : Dans d'autres cas, le corps du VRA contient des syntagmes nominaux quantifiés par des pronoms~/ déterminants cardinaux ou proportionnels, quantifiant sur l'ensemble dénoté par l'antécédent. La relation sémantique qui s'établit entre l'antécédent et l'élément distingué est cette fois-ci plutôt une relation entre un ensemble et \textsc{un sous-ensemble}. Ce type de VRA \textsc{partitionne} l'ensemble dénoté par l'antécédent en sous-ensembles (disjoints) sur la base de restrictions supplémentaires sur la restriction ou la portée du quantifieur présent dans le VRA. Uniquement avec ce type d'interprétation, le VRA peut introduire une liste de sous-parties qui est coextensive à l'entité fractionable dénotée par l'antécédent. Ainsi, en roumain, l'introducteur compatible avec l'exhaustivité est par défaut \textit{dintre care}, la préposition \textit{dintre} englobant en elle-même une sémantique partitionnante, alors que \textit{printre care} et \textit{între care} ne permettent pas l'exhaustivité. En revanche, le français permet a  priori et \textit{dont} et \textit{parmi lesquel(le)s}, bien que les locuteurs manifestent une préférence nette pour l'emploi de \textit{dont}.  


\begin{enumerate}
\item a  In total, au venit trei persoane, [\{*\textbf{printre {\textbar}~*între {\textbar} dintre}\}\textbf{ care} doi bărbați şi o femeie].


\end{enumerate}
{\itshape
Au total, sont venues trois personnes, dont deux hommes et une femme}

  b  In total, au venit trei persoane, [\{*\textbf{printre {\textbar}~*între {\textbar} dintre}\}\textbf{ care} una ieri şi două azi-dimineață].

{\itshape
Au total, sont venues trois personnes, dont une hier et deux ce matin}

  c  In anul 2009, am cucerit 72 de medalii, [\textbf{dintre care} 25 de aur, 22 de argint şi 25 de bronz].

    \textit{En 2009, j'ai conquis 72 médailles, dont 25 d'or, 22 d'argent et 25 de bronze} 


\begin{enumerate}
\item a  Trois personnes sont venues, [\{\textbf{dont {\textbar} parmi lesquelles}\} une femme et deux hommes].


\end{enumerate}
  b  Trois personnes sont venues, [\{\textbf{dont {\textbar} parmi lesquelles}\} une lundi et deux mardi].

  c  \textstyleapplestylespan{Sur les 78 projets candidats cette année, 12 ont été retenus,}{~[\{}{\textbf{dont {\textbar}} }\emph{\textbf{\textup{parmi}}}\emph{\textup{} }\emph{\textbf{\textup{lesquels}}}\emph{\textup{\} 10}}{~}\textstyleapplestylespan{en sciences dures et 2 seulement en sciences humaines et sociales.}\emph{\textup{} }

Cependant, les données ne se laissent pas facilement décrire avec ce type d'interprétation. D'une part, en roumain, bien que \textit{dintre care} soit effectivement le plus fréquent dans ce type de contextes, il y a quelques occurrences de \textit{printre care} et \textit{între care} avec une interprétation partitionnante. D'autre part, en français, bien que \textit{dont} et \textit{parmi lesquel(le)s} puissent être utilisées en variation libre, il y a au moins un contexte où \textit{parmi lesquel(le)s} n'est pas approprié. Pour expliquer ces distributions étonnantes, on doit distinguer entre deux types de partition. Dans les VRA avec une interprétation partitionnante, l'élément distingué dénote une sous-partie qui n'est pas référentielle (c.-à-d. l'élément distingué n'est pas identifiable indépendamment de la référence de l'antécédent), mais qui peut être définie comme ayant certaines propriétés qui ne sont pas partagées par les autres sous-parties appartenant à la même entité fractionable. Il pourrait s'agir d'une propriété de la sous-partie elle-même (donc, une propriété de l'élément distingué et, dans ce cas, la partition s'établit entre des entités qui ne désignent pas des éventualités) ou bien d'une propriété de la sous-éventualité à laquelle participe la sous-partie dénotée par l'élément distingué (et, dans ce cas, la partition s'établit entre des éventualités). Si on prend en compte ces deux sous-types de partition, on observe qu'en roumain, si la sous-partie n'est pas exhaustive, \emph{\textup{l'interprétation partitionnante est possible avec} }\emph{printre care} \emph{\textup{et} }\emph{între care}\emph{\textup{ plutôt avec le premier sous-type (comparer} }\emph{\textup{\REF{ex:5:65}}}\emph{\textup{ et} }\emph{\textup{\REF{ex:5:67}}}\emph{\textup{). Une condition similaire est observée en français :} }\emph{parmi lesquel(le)s}\emph{\textup{, bien qu'il soit compatible avec une interprétation partitionnante et avec une sous-partie exhaustive, ne permet jamais que le corps du VRA décrive explicitement une situation, c.-à-d. une sous-éventualité de l'éventualité principale dénotée par la phrase hôte (comparer} }\emph{\textup{\REF{ex:5:66}}}\emph{\textup{ et} }\emph{\textup{\REF{ex:5:68}}}\emph{\textup{).}}


\begin{enumerate}
\item \label{bkm:Ref295319048}a  Șapte persoane, [\{\textbf{printre {\textbar} între {\textbar} dintre}\} \textbf{care} cinci polițişti], au murit într-o explozie.  


\end{enumerate}
{\itshape
Sept personnes, dont cinq policiers, sont mortes dans une explosion}

  b  Cele 20 de cadre medicale, [\{\textbf{printre {\textbar} între {\textbar} dintre}\} \textbf{care} şase chirurgi şi patru anestezişti], au încheiat intervenția chirurgicală după şase ore.

{\itshape
Les 20 cadres médicaux, dont six chirurgiens et quatre anesthésistes, ont fini l'intervention chirurgicale au bout de six heures}

  c  Paul a scris mai multe romane, [\{\textbf{printre {\textbar} între {\textbar} dintre}\} \textbf{care} două în limba franceză].

{\itshape
Paul a écrit plusieurs romans, dont deux en français}


\begin{enumerate}
\item \label{bkm:Ref295319097}a  Plusieurs personnes, [\{\textbf{dont {\textbar} parmi lesquelles}\} deux enfants], seront reconduites à la frontière.


\end{enumerate}
  b  Chaque année, l'université accueille des milliers d'étudiants, [\{\textbf{dont {\textbar} parmi} \textbf{lesquels}\} un tiers d'étudiants étrangers].

  c  Chaque année, des milliers d'étudiants s'inscrivent à l'université, [\{\textbf{dont {\textbar} parmi} \textbf{lesquels}\} plus de 75\% en informatique]\emph{\textup{.} }


\begin{enumerate}
\item \label{bkm:Ref295319065}a  In România, trăiesc aproximativ 8~000 de evrei, [\{\textbf{*printre {\textbar} *între {\textbar} dintre}\}\textbf{ care} jumătate în Bucureşti].  


\end{enumerate}
{\itshape
En Roumanie, vivent environ 8~000 juifs, dont la moitié à Bucarest}

  b  Anul acesta, Abi a citit cinci romane, [\{\textbf{*printre {\textbar} *între {\textbar} dintre}\} \textbf{care} două în vacanța de iarnă].

{\itshape
Cette année, Abi a lu cinq romans, dont deux pendant les vacances d'hiver}

  c  Săptămâna aceasta, am cheltuit 1~000 de euro, [\{*\textbf{printre {\textbar} *între {\textbar} dintre}\} \textbf{care} 800 numai ieri].

    \textit{Cette semaine, j'ai dépensé 1~000 euros, dont 800 seulement hier}


\begin{enumerate}
\item \label{bkm:Ref295319101}a  Paul a parcouru cinq livres sur le sujet, [\{\textbf{dont {\textbar} *parmi lesquels}\} la moitié hier]. 


\end{enumerate}
  b  A l'arrivée à La Toussuire dimanche 12 juin, les coureurs auront parcouru plus de 1~000 km, [\{\textbf{dont {\textbar} *parmi lesquels}\} la plupart en montagne.

  c  Pendant les vacances estivales, Marie a lu cinq romans, [\{\textbf{dont {\textbar} *parmi lesquels}\} deux sur la plage].

Cette contrainte sémantique sur l'impossibilité d'avoir une description d'éventualité explicite avec \textit{printre care, între care} en roumain et \textit{parmi lesquel(le)s} en français se traduit au niveau syntaxique par la présence ou non d'un cluster dans le corps du VRA. Ainsi, on observe qu'avec \textit{printre care} et \textit{între care} en roumain et \textit{parmi lesquel(le)s} en français, le corps du VRA se comporte comme une seule unité, le quantifieur ayant forcément le statut de déterminant dans un syntagme nominal complexe \REF{ex:5:69}b-\REF{ex:5:70}b, tandis que, avec \textit{dintre care} en roumain et respectivement \textit{dont} en français, le corps du VRA peut être un cluster composé d'au moins deux syntagmes, comme le montre la possibilité d'avoir un quantifieur avec un statut pronominal \REF{ex:5:69}a-\REF{ex:5:70}a.  


\begin{enumerate}
\item \label{bkm:Ref295319261}a  Patru jurnalişti, [\textbf{dintre} \textbf{care} \{unul {\textbar} un\} cetățean israelian] au pătruns în Gaza ieri.  


\end{enumerate}
{\itshape
Quatre journalistes, dont \{l'un {\textbar} un\} citoyen israélien, sont entrés à Gaza hier}

  b  Patru jurnalişti, [\textbf{printre {\textbar} între} \textbf{care} \{*unul {\textbar} un\} cetățean israelian] au pătruns în Gaza ieri.

{\itshape
Quatre journalistes, dont \{l'un {\textbar} un\} citoyen israélien, sont entrés à Gaza hier}


\begin{enumerate}
\item \label{bkm:Ref295319263}a  \textstyleapplestylespan{Quatre journalistes, dont \{l'un {\textbar} un\} citoyen israélien, sont entrés à Gaza hier}. 


\end{enumerate}
  b  \textstyleapplestylespan{Quatre journalistes, parmi lesquels \{*l'un {\textbar} un\} citoyen israélien, sont entrés à Gaza hier}.

De manière générale, si le VRA contient une coordination nominale, il suffit qu'un des syntagmes nominaux soit non-référentiel, pour que l'interprétation partitionnante soit disponible.


\begin{enumerate}
\item Papa a aflat despre eliberarea a 15 ostatici în Columbia, [\textbf{între care} franco-columbiana Ingrid Betancourt, trei americani şi trei locotenenți columbieni]. 


\end{enumerate}
{\itshape
Le pape a appris la libération de 15 otages en Colombie, dont la franco-colombienne Ingrid Betancourt, trois américains et trois lieutenants colombiens} 


\begin{enumerate}
\item Prends deux objets, [\textbf{dont} cette bouteille et \{un {\textbar} *ce\} couteau].


\end{enumerate}
Dans un VRA, la sémantique de la tête de l'introducteur (c.-à-d. la préposition) joue un rôle dans le choix d'interprétation disponible pour le VRA en question. Ainsi, en roumain, on observe que les mêmes propriétés lexicales des prépositions \textit{printre} et \textit{dintre} dans les VRA peuvent être observées dans d'autres emplois des prépositions : \textit{printre} peut être partitionnant ou exemplifiant, alors que \textit{dintre} est toujours partitionnant. 


\begin{enumerate}
\item a  Avem \{majoritatea {\textbar} spionii\} \textbf{printre} noi.  


\end{enumerate}
\textit{Nous avons} \{\textit{la majorité {\textbar} les espions}\} \textit{parmi nous}

  b  \{majoritatea {\textbar} *spionii\} \textbf{dintre} noi

\{\textit{la majorité {\textbar} les espions}\} \textit{parmi nous } 

On observe que tous les introducteurs en roumain et en français imposent des contraintes particulières au corps du VRA, sauf \textit{dont} en français. Pourquoi ? Les contraintes dérivent généralement du sémantisme de la préposition \textit{dintre, printre, între} en roumain et respectivement \textit{parmi} en français : ainsi, en dehors de leurs emplois dans les VRA, la préposition \textit{dintre} marque la partition, les prépositions \textit{printre} et \textit{între} marquent l'inclusion, tout comme la préposition \textit{parmi} en français, alors que l'item \textit{dont} en français ne possède pas de propriété lexicale particulière.  

\subsubsection{Interprétation d'éventualité}
Le VRA introduit de manière implicite ou explicite une sous-éventualité de l'éventualité décrite par la phrase hôte. Dans l'exemple \REF{ex:5:74}a, \textit{Maria est venue me voir} est une sous-éventualité de \textit{Plusieurs amis à moi sont venus me voir}. Il n'est pas suffisant de dire que le VRA exprime une relation entre \textit{Maria} et \textit{plusieurs amis à moi}. On doit spécifier dans la sémantique que \REF{ex:5:74}a signifie \textit{Maria est venue me voir}. Ainsi, on rend compte de l'agrammaticalité de l'exemple \REF{ex:5:74}b, où le VRA est enchâssé dans une non-éventualité, pour lequel on ne peut pas définir de sous-éventualité.


\begin{enumerate}
\item \label{bkm:Ref295403595}a  Mai mulți prieteni de-ai mei, [\textbf{printre care} şi Maria], au venit să mă vadă.  


\end{enumerate}
{\itshape
Plusieurs amis à moi, parmi lesquels aussi Maria, sont venus me voir}

  b  *Niciunul dintre prietenii mei, [\textbf{printre} \textbf{care} în mod particular Maria], nu a venit să mă vadă.

{\itshape
Aucun de mes amis, parmi lesquels en particulier Maria, n'est venu me voir } 

L'hypothèse selon laquelle un VRA décrirait une sous-éventualité est justifiée surtout par le deuxième sous-type de VRA avec une interprétation partitionnante, à savoir les cas dans lesquels la sous-partie n'est pas définie par une propriété de la sous-partie elle-même, mais par une propriété de la sous-éventualité ; elle peut être identifiée uniquement sur la base des propriétés de la nouvelle éventualité. La description explicite de la sous-éventualité est faite dans ces cas par des expressions spatiales ou temporelles, qui peuvent facilement prédiquer sur une sous-éventualité (en la délimitant dans l'espace ou dans le temps), d'où leur analyse en termes de prédicats ou modifieurs d'éventualité (\citet{Rothstein2004}). Ainsi, la sous-éventualité introduite par le VRA peut comporter un modifieur spatial comme les syntagmes prépositionnels \textit{în zona Bruxelles} \REF{ex:5:75}a  et \textit{en Colombie {\textbar} au Pérou} \REF{ex:5:76}a, ou bien un modifieur temporel comme les syntagmes nominaux \textit{vara} \REF{ex:5:75}b et \textit{ce matin} \REF{ex:5:76}b.


\begin{enumerate}
\item \label{bkm:Ref295407351}a  Comunitatea românească din Regatul Belgiei numără oficial aproximativ 20~000 de persoane, [\textbf{dintre} \textbf{care} mai mult de jumătate în zona Bruxelles].  


\end{enumerate}
{\itshape
La communauté roumaine du Royaume de Belgique compte officiellement environ 20~000 personnes, dont plus de la moitié autour de Bruxelles}

  b  Media precipitațiilor anuale este de circa 1~000 mm, [\textbf{dintre care} între 50 şi 60\% vara].

{\itshape
La moyenne des précipitations annuelles est d'environ 1~000 mm, dont entre 50\% et 60\% l'été } 


\begin{enumerate}
\item \label{bkm:Ref295407354}a  \textstyleapplestylespan{En Amérique latine, 23 journalistes ont trouvé la mort, [}\textstyleapplestylespan{\textbf{dont}}\textstyleapplestylespan{ 9 en Colombie et 7 au Pérou]}. 


\end{enumerate}
  b  Plusieurs personnes sont venues, [\textbf{dont} une ce matin].

Le modifieur temporel peut induire une interprétation télique (c.-à-d. ayant un terme défini par la nature même de l'événement) pour la sous-éventualité dans le VRA, cf. \REF{ex:5:77}a, ou bien atélique (c.-à-d. sans terme inhérent), cf. \REF{ex:5:77}b.


\begin{enumerate}
\item \label{bkm:Ref295414301}a  Ieri au escaladat vârful Cervin cel puțin 30 de atleți, [\textbf{dintre care} doi în mai puțin de patru ore].  


\end{enumerate}
{\itshape
Hier, ont escaladé le sommet Cervin au moins 30 athlètes, dont deux en moins de quatre heures}

  b  La terminarea conferinței, m-au ținut de vorbă mai mulți participanți, [\textbf{dintre} \textbf{care} unul timp de două ore].

{\itshape
A la fin de la conférence, je me suis entretenue avec plusieurs participants, dont un pendant deux heures } 

Si on accepte le fait que les VRA décrivent des sous-éventualités, on peut facilement rendre compte des cas difficiles comme \REF{ex:5:78} en roumain et \REF{ex:5:79} en français, dans lequel le VRA semble avoir plusieurs antécédents, le premier violant le principe énoncé dans la sous-section \ref{sec:5.3.2.1}, car \textit{un cadeau} dans la phrase hôte n'est pas une entité plurielle telle que définie dans la sous-section mentionnée. On peut donc réanalyser le VRA dans cet exemple comme ayant un seul antécédent \textit{plusieurs personnes} (qui respecte bien les contraintes établies dans la sous-section \ref{sec:5.3.2.1}) ; la relation \textit{un cadeau -- un livre} est une relation supplémentaire (de type somme -- sous-partie) qui se justifie par le fait que l'éventualité dans le VRA est une sous-éventualité de la situation décrite dans la phrase hôte, et par conséquent, le faux antécédent ne doit pas nécessairement obéir aux contraintes discutées dans la sous-section mentionnée ci-dessus). 


\begin{enumerate}
\item \label{bkm:Ref295411593}Ion a oferit câte un cadou mai multor persoane, [\textbf{printre} \textbf{care} şi o carte *(Mariei)]. 


\end{enumerate}
{\itshape
Ion a offert un cadeau à plusieurs personnes, dont un livre à Maria} 


\begin{enumerate}
\item \label{bkm:Ref295411595}Paul a offert un cadeau à plusieurs personnes, [\textbf{dont} un livre *(à Marie)]. 


\end{enumerate}
Ce type d'approche est en accord avec ce qu'on observe sur le comportement du VRA par rapport aux prédicats collectifs. Un prédicat collectif comme \textit{se rencontrer, écrire ensemble, s'embrasser (amoureusement)} ou encore \textit{sortir ensemble} dénote une éventualité auquel participe une somme d'(au moins deux) individus. Or, on observe que le VRA doit contenir la somme minimale requise par une telle relation. Ainsi, le corps du VRA ne peut pas contenir une entité plus petite que l'entité minimale participant à une sous-instance de l'éventualité dénotée par l'hôte, ce qui explique les différences d'acceptabilité observées dans les exemples suivants :


\begin{enumerate}
\item a  Foarte mulți foşti colegi au venit la petrecere, [\textbf{printre} \textbf{care} şi Maria].  


\end{enumerate}
{\itshape
Beaucoup d'anciens collègues sont venus à la fête, parmi lesquels Maria}

  b  Foarte mulți foşti colegi s-au întâlnit la petrecere, [\textbf{printre care} şi \{\#Maria {\textbar} Maria cu Ion\}].

\textit{Beaucoup d'anciens collègues se sont rencontrés à la fête, dont} \textit{\{}\textit{Maria {\textbar} Maria avec Ion\}}

  c  Mai multe persoane au scris împreună articole, [\textbf{printre care} şi \{\#Maria {\textbar} Maria cu Ion\}].

    \textit{Plusieurs personnes ont écrit ensemble des papiers, dont \{Maria {\textbar} Maria avec Ion\}}


\begin{enumerate}
\item a  Plusieurs personnes se sont embrassées (pour se dire au revoir), [\textbf{dont} Marie]. 


\end{enumerate}
  b  \#Plusieurs personnes se sont embrassées (amoureusement), [\textbf{dont} Marie].

  c  \#Plusieurs personnes sortent ensemble, [\textbf{dont} Marie].


\begin{enumerate}
\item a  Marie fait partie des personnes qui se sont embrassées (pour se dire au revoir).


\end{enumerate}
  b  \#Marie fait partie des personnes qui se sont embrassées (amoureusement).

  c  \#Marie fait partie des personnes qui sortent ensemble.

D'ailleurs, attribuer au VRA une interprétation d'éventualité et, en particulier, une interprétation de sous-éventualité, explique un des aspects mentionnés dans la sous-section \ref{sec:5.3.1}, à savoir le fait que le VRA est interprété sous la portée des verbes d'attitude propositionnelle dans la phrase hôte (voir, dans ce sens, l'exemple roumain \REF{ex:5:51}, adapté pour le français en \REF{ex:5:83}).


\begin{enumerate}
\item \label{bkm:Ref295412799}a  Pierre croit que certaines plantes, [\textbf{dont} la verveine], soignent les ulcères.


\end{enumerate}
={\textgreater}  b  Pierre croit que la verveine soigne les ulcères.

${\neq}${\textgreater}  c  La verveine soigne les ulcères.

Un dernier argument qu'on peut apporter pour une interprétation d'éventualité du VRA est le fait qu'en roumain le corps du VRA peut comporter une structure coordonnée par la conjonction \textit{iar} `et', qui est spécialisée pour la coordination d'éventualités {\guillemotleft}~contrastives~{\guillemotright} (voir les discussions dans le chapitre 2, section \ref{sec:2.9}, et chapitre 4, section \ref{sec:4.3.1} et 4.3.4.4).


\begin{enumerate}
\item a  Prins între două sticle, [\textbf{dintre care} una de vin, iar alta de borviz], se uita la dânsele lung. (cité par \citep[268]{Gheorghe2004})   


\end{enumerate}
{\itshape
Pris entre deux bouteilles, dont l'une de vin et l'autre d'eau minérale, il les regardait longuement}

  b  In Giurgiu, existau la sfârşitul lunii noiembrie 45~811 carduri, [\textbf{dintre care} în lei, 44~991, iar în euro, 820].

{\itshape
Dans le département de Giurgiu, il y avait à la fin du mois de novembre 45~811 cartes (bancaires), dont en lei, 44~991, et en euros, 820 } 

  c  In Irak, de la începutul conflictului au fost răpiți mai mult de 150 de cetățeni străini, [\textbf{dintre care} în 2004, 22 de jurnalişti, iar în 2005, alți cinci].

    \textit{En Irak, depuis le début du conflit ont été enlevés plus de 150 citoyens étrangers, dont en 2004, 22 journalistes, et en 2005, encore cinq}

\subsubsection{Synthèse}
Dans cette section, je me suis intéressée aux propriétés sémantiques des VRA. D'abord, on a observé que les VRA ont un comportement hybride quant à la distinction relative restrictive vs. relative non-restrictive. D'une part, les VRA ont une interprétation non-intersective, ne restreignant pas l'ensemble dénoté par l'antécédent à un sous-ensemble particulier, ce qui les rapproche des relatives non-restrictives. D'autre part, le contenu du VRA appartient au contenu asserté par la phrase hôte, ce qui les oppose aux relatives non-restrictives ordinaires. Ensuite, on a montré que les VRA expriment simultanément une double relation partitive : (i) une relation partitive au niveau des individus (c.-à-d. entre un légitimeur dénotant une entité plurielle fractionable et un élément distingué désignant une sous-partie) et (ii) une relation partitive au niveau des éventualités (c.-à-d. le VRA introduit explicitement ou implicitement une sous-éventualité de l'éventualité dénotée par la phrase hôte). Parallèlement, on a inventorié deux types d'interprétations dans les VRA : (i) une interprétation exemplifiante (dont les propriétés définitoires sont : l'élément distingué est référentiel, la somme des sous-parties n'est pas coextensive à l'entité fractionable, il s'agit plutôt d'une relation entre un ensemble et un élément de cet ensemble) et (ii) une interprétation partitionnante (dont les critères définitoires sont les suivants : l'élément distingué n'est pas identifiable indépendamment de la référence de l'antécédent, la somme des sous-parties est coextensive à l'entité fractionable, il s'agit plutôt d'une relation entre un ensemble et un sous-ensemble). Sur la base de ces contraintes, on peut maintenant décrire précisément les différences qu'on observe d'un introducteur à l'autre en roumain et en français : en roumain, l'introducteur \textit{dintre care} n'est utilisé qu'avec l'interprétation partitionnante, alors que \textit{printre care} et \textit{între care} sont compatibles avec les deux (bien qu'on note une préférence pour l'interprétation exemplifiante), à condition qu'il n'y ait pas de description explicite d'éventualité. En français, \textit{dont} peut avoir les deux interprétations, alors que \textit{parmi lesquel(le)s} obéit à une contrainte similaire à celle observée avec \textit{printre care} et \textit{între care} en roumain : ils ne sont pas préférés dans les VRA qui décrivent explicitement une sous-éventualité. 

\subsection{Les VRA ne sont pas des phrases relatives verbales}
\label{bkm:Ref294028428}Les VRA ont été décrits comme des phrases relatives elliptiques où manque la tête verbale (voir \citet{Grevisse1993} pour le français et Gheorghe (2004, 2005) pour le roumain). Dans une approche en termes d'ellipse, les VRA sont considérés comme des phrases relatives ordinaires qui se distinguent simplement par le fait qu'une partie du matériel phonologique manque. Dans cette perspective, le matériel qui manque (indiqué par des chevrons dans l'arbre simplifié en \REF{ex:5:85}) présente une structure syntaxique {\guillemotleft}~invisible~{\guillemotright}, ce qui nous rappelle les approches structurales discutées dans le chapitre 3. Pour qu'une telle analyse soit appropriée, elle doit remplir les deux conditions suivantes : (i) on doit pouvoir reconstruire une phrase relative à partir de tout VRA de façon régulière, et (ii) les mêmes propriétés sémantiques qui caractérisent les VRA doivent pouvoir s'appliquer aux phrases relatives aussi. On démontre par la suite qu'aucune des deux conditions n'est satisfaite : la reconstruction d'une forme verbale n'est pas toujours possible et la contribution sémantique d'un VRA n'est pas celle d'une phrase relative ordinaire. 


\begin{enumerate}
\item \label{bkm:Ref295496296}Analyse par reconstruction syntaxique 


\end{enumerate}
{   [Warning: Image ignored] % Unhandled or unsupported graphics:
%\includegraphics[width=3.4783in,height=1.6035in,width=\textwidth]{fe443409cd384d3fb0f6390ffd77f513-img74.svm}
} 

\subsubsection{Reconstruction syntaxique}
Si on adopte une approche elliptique à base de reconstruction syntaxique, il y a au moins trois stratégies pour la reconstruction verbale dans les VRA. On peut reconstruire : (i) une forme verbale du même lexème que le verbe de la phrase hôte, cf. \REF{ex:5:86}a-\REF{ex:5:87}a, dans les VRA qui contiennent un cluster imitant la syntaxe de l'hôte ; (ii) la forme d'un verbe existentiel ou d'un verbe qui paraphrase la relation d'appartenance à un ensemble (ex. roum. \textit{a se afla, a fi} `être', \textit{a se găsi} `se trouver' ou fr. \textit{être, figurer,} \textit{faire partie}), cf. \REF{ex:5:86}b-\REF{ex:5:87}b, ou bien (iii) la forme d'un verbe de citation (ex. roum. \textit{a cita} `citer'\textit{, a menționa} `mentionner' ou fr. \textit{citer, mentionner}), cf. \REF{ex:5:86}c-\REF{ex:5:87}c.


\begin{enumerate}
\item \label{bkm:Ref295725397}a  Ion a pictat mai multe tablouri, [\textbf{dintre} \textbf{care} două (sunt pictate) la mare].  


\end{enumerate}
{\itshape
Ion a peint plusieurs tableaux, dont deux (sont peints) à la mer}

  b  Israelul a omorât peste 700 de palestinieni, [\textbf{dintre care} 200 (sunt) copii].

{\itshape
L'état d'Israël a tué plus de 700 palestiniens, dont 200 (sont) des enfants}

  c  Marin Preda a scris mai multe romane, [\textbf{printre care} (cităm) \textit{Moromeții}].

    \textit{Marin Preda a écrit plusieurs romans, dont (nous citons) Moromeții}


\begin{enumerate}
\item \label{bkm:Ref295725399}a  Jean a peint beaucoup de tableaux, [\textbf{dont} deux (ont été peints) à la mer]. 


\end{enumerate}
  b  Paul a écrit cinq livres, [\textbf{dont} deux (sont) sur le même sujet.

  c  Zola a écrit beaucoup de romans, [\textbf{dont} (on peut citer) \textit{Germinal}].

Cependant, ces stratégies ne sont pas disponibles dans tous les contextes. Il n'y a pas un mécanisme général de reconstruction syntaxique pouvant s'appliquer à tous les VRA. Pour chaque exemple, on doit choisir une certaine stratégie. Le type de reconstruction syntaxique possible dépend : (i) des propriétés syntaxiques du corps du VRA, et (ii) du type de sémantique à paraphraser (exemple ou partition). 

Dans une approche à base de reconstruction syntaxique, le choix de la forme verbale dépend des contraintes lexicales, comme les propriétés de sous-catégorisation, qui ne sont pas corrélées avec les propriétés sémantiques. Si dans un VRA comme \REF{ex:5:88}a, la troisième stratégie marche avec les deux introducteurs \textit{dont} et \textit{parmi lesquel(le)s} en français, cf. \REF{ex:5:88}b, la deuxième option obéit à des contraintes particulières, imposées par le comportement syntaxique du verbe reconstruit. Ainsi, en \REF{ex:5:88}c on peut reconstruire le verbe \textit{figurer} dans un VRA introduit par \textit{parmi lesquel(le)s}, mais pas dans un VRA introduit par \textit{dont}, car le verbe \textit{figurer} peut sous-catégoriser un syntagme prépositionnel introduit par la préposition \textit{parmi}, et non un syntagme prépositionnel introduit par la préposition \textit{de}. En revanche, en \REF{ex:5:88}d on reconstruit une forme verbale comme l'expression \textit{faire partie}, qui est compatible avec \textit{dont} (car elle sous-catégorise un syntagme prépositionnel en \textit{de}), mais pas avec \textit{parmi lesquel(le)s} (car elle ne sous-catégorise pas un complément marqué par \textit{parmi}).


\begin{enumerate}
\item \label{bkm:Ref295727189}a  Plusieurs personnes sont venues, [\{\textbf{dont {\textbar} parmi lesquelles}\} Jean]. 


\end{enumerate}
  b  Plusieurs personnes sont venues, [\{\textbf{dont {\textbar} parmi lesquelles}\} on peut citer Jean].

  c  Plusieurs personnes sont venues, [\{*\textbf{dont {\textbar} parmi lesquelles}\} figure Jean].

  d  Plusieurs personnes sont venues, [\{\textbf{dont {\textbar} *parmi lesquelles}\} Jean fait partie].

Dans certains cas, aucune des trois stratégies mentionnées ci-dessus  n'est disponible dans les deux langues, car il y a trop de contraintes sur la sous-catégorisation du verbe reconstruit. Cela arrive surtout dans les VRA qui contiennent un cluster de syntagmes.


\begin{enumerate}
\item Media precipitațiilor anuale este de circa 1~000 mm, [\textbf{dintre care} (??) între 50 şi 60\% (??) vara]. 


\end{enumerate}
{\itshape
La moyenne des précipitations anuelles est d'environ 1~000 mm, dont entre 50\% et 60\% l'été} 


\begin{enumerate}
\item a  La petrecere, mai toți au vorbit cu câte cineva, [\textbf{printre care} şi Dan cu Ioana].  


\end{enumerate}
{\itshape
A la fête, presque tous ont parlé avec quelqu'un, parmi lesquels Dan avec Ioana}

  b  *La petrecere, mai toți au vorbit cu câte cineva, [\textbf{printre care} şi Dan a vorbit cu Ioana].

{\itshape
A la fête, presque tous ont parlé avec quelqu'un, parmi lesquels Dan aussi a parlé avec Ioana}

  c  *La petrecere, mai toți au vorbit cu câte cineva, [\textbf{printre care} şi Dan e cu Ioana].

{\itshape
A la fête, presque tous ont parlé avec quelqu'un, parmi lesquels Dan aussi est avec Ioana}

  d  *La petrecere, mai toți au vorbit cu câte cineva, [\textbf{printre care} menționăm şi Dan cu Ioana].

    \textit{A la fête, presque tous ont parlé avec quelqu'un, parmi lesquels on mentionne aussi Dan avec Ioana}


\begin{enumerate}
\item a  Plusieurs ont eu un cadeau, [\textbf{dont} Marie un livre]. 


\end{enumerate}
  b  *Plusieurs ont eu un cadeau, [\textbf{dont} Marie a eu un livre].

  c  *Plusieurs ont eu un cadeau, [\textbf{dont} Marie est un livre].

  d  *Plusieurs ont eu un cadeau, [\textbf{dont} on cite Marie un livre].

Pour sauver la possibilité de reconstruction syntaxique dans les clusters, on pourrait faire appel à une reconstruction plus complexe qui combine une des trois stratégies avec une phrase relative interne. Bien qu'elle marche pour certains exemples, cf. \REF{ex:5:92}c, cette nouvelle possibilité n'est pas une solution générale pour les deux interprétations disponibles dans les VRA (comparer \REF{ex:5:92}c-\REF{ex:5:93}c) et, en plus, elle ne rend pas compte des contraintes pesant sur les syntagmes du cluster dans un VRA, qui doivent être toujours des constituants de même niveau dans la phrase hôte \REF{ex:5:94}.


\begin{enumerate}
\item \label{bkm:Ref295811496}a  J'ai vendu 16 jeux, [\textbf{dont} certains à mes amis]. 


\end{enumerate}
  b  *J'ai vendu 16 jeux, [\textbf{dont} j'ai vendu\textbf{} certains à mes amis].

  c  J'ai vendu 16 jeux, [\textbf{dont} on peut citer\textbf{} certains\textbf{} que j'ai vendu à mes amis].


\begin{enumerate}
\item \label{bkm:Ref295811498}a  J'ai parlé à plusieurs personnes hier, [\textbf{dont} à Marie de linguistique]. 


\end{enumerate}
  b  *J'ai parlé à plusieurs personnes hier, [\textbf{dont} j'ai parlé à Marie de linguistique].

  c  *J'ai parlé à plusieurs personnes hier, [\textbf{dont} on peut citer que j'ai parlé\textbf{} à Marie de linguistique].


\begin{enumerate}
\item ~\label{bkm:Ref295811536}??/*Mes amis croient que la vie existe sur d'autres planètes, [\textbf{dont} Marie sur Mars].  


\end{enumerate}
Un argument supplémentaire contre la reconstruction verbale dans les VRA est dû aux propriétés du français \textit{dont} et des syntagmes nominaux sans tête lexicalisée, qui sont d'ailleurs très fréquents avec les VRA \REF{ex:5:95}a. Les syntagmes nominaux sans tête qui fonctionnent comme des compléments directs d'un verbe déclenchent la réalisation de l'affixe pronominal \textit{en} sur le verbe \REF{ex:5:95}b. Cependant, les VRA avec \textit{dont} ne permettent aucune reconstruction, qu'il s'agisse ou non de l'affixe \textit{en} cf. \REF{ex:5:95}c-d, car de toute façon \textit{dont} est incompatible avec la réalisation de l'affixe \textit{en} sur le verbe.


\begin{enumerate}
\item \label{bkm:Ref295745347}a  Il a offert trois livres, [\textbf{dont} deux à son frère]. 


\end{enumerate}
  b  Il *(en) a offert deux à son frère.

  c  *... \textbf{dont} il a offert deux à son frère.

  d  *... \textbf{dont} il en a offert deux à son frère. 

En même temps, si on envisage une reconstruction syntaxique dans les VRA, on ne peut pas expliquer pourquoi le marquage casuel ou prépositionnel en roumain est interdit avec les VRA introduits par \textit{dintre care} \REF{ex:5:96}, alors qu'il est possible avec les VRA introduits par \{\textit{printre {\textbar} între}\}\textit{ care} \REF{ex:5:97}, étant donné que les deux prépositions ont le même comportement syntaxique en dehors des VRA.


\begin{enumerate}
\item \label{bkm:Ref295810578}a  Ion a oferit flori mai multor persoane, [\textbf{dintre care} \{majoritatea {\textbar} *majorității\} fete]. 


\end{enumerate}
Ion a offert fleurs plusieurs\textsc{.dat} personnes, dont \{la-plupart {\textbar} la-plupart.\textsc{dat}\} filles 

  \textit{Ion a offert des fleurs à plusieurs personnes, dont la plupart des filles } 

  b  Dragoş lucrează cu şapte medici, [\textbf{dintre care} (*cu) doi israelieni].

    Dragoş travaille avec sept médecins, dont (avec) deux israéliens

    \textit{Dragoş travaille avec sept médecins, dont deux israéliens}


\begin{enumerate}
\item \label{bkm:Ref295810602}a  Ion a oferit flori mai multor fete, [\textbf{printre care} şi \{Maria {\textbar} Mariei\}]. 


\end{enumerate}
  Ion a offert fleurs plusieurs\textsc{.dat} filles, parmi lesquelles aussi \{Maria {\textbar} Maria.\textsc{dat}\} 

  \textit{Ion a offert des fleurs à plusieurs filles, parmi lesquelles Maria} 

  b  Ion a vorbit cu mai multe fete, [\textbf{printre care} şi (cu) Maria].

    Ion a parlé avec plusieurs filles, parmi lesquelles aussi (avec) Maria

    \textit{Ion a parlé avec plusieurs filles, parmi lesquelles Maria}

Dans d'autres cas, le choix de la stratégie est influencé par le type de sémantique observé dans le VRA. Ainsi, la troisième stratégie, qui utilise un verbe de citation comme moyen de reconstruction, est appropriée pour les VRA à interprétation exemplifiante, mais inacceptable avec une interprétation partitionnante.


\begin{enumerate}
\item a  Marin Preda a scris mai multe romane, [\textbf{printre care} (menționăm) \textit{Moromeții}].  


\end{enumerate}
{\itshape
Marin Preda a écrit plusieurs romans, dont (nous mentionnons) Moromeții}

  b  Săptămâna aceasta, am cheltuit 1~000 de euro, [\textbf{dintre} \textbf{care} (\#menționăm) 800 numai ieri].

{\itshape
Cette semaine, j'ai dépensé 1~000 euros, dont (nous mentionnons) 800 seulement hier } 


\begin{enumerate}
\item a  Zola a écrit beaucoup de romans, [\textbf{dont} (on peut citer) \textit{Germinal}]. 


\end{enumerate}
  b  Je vends 16 jeux, [\textbf{dont} (\#on peut citer) la plupart encore dans leur boîte].

De plus, la position du verbe reconstruit change en fonction de la stratégie utilisée : s'il s'agit d'un verbe de citation, il précède nécessairement le corps du VRA cf. \REF{ex:5:100}b-\REF{ex:5:101}b, alors qu'avec les autres stratégies, le verbe suit le (premier) syntagme du corps, \REF{ex:5:100}a-\REF{ex:5:101}a.


\begin{enumerate}
\item \label{bkm:Ref295811943}a  ... patru persoane, [\textbf{dintre care} una (este) cetățean american]  


\end{enumerate}
{\itshape
quatre personnes, dont l'une (est) citoyen américain}

  b  ... patru persoane, [\textbf{printre care} (amintim) un cetățean american]

{\itshape
quatre personnes, parmi lesquelles (on mentionne) un citoyen américain} 


\begin{enumerate}
\item \label{bkm:Ref295811946}a  ... quatre journalistes, [\textbf{dont} l'un (est) citoyen américain] 


\end{enumerate}
  b  ... quatre journalistes, [\textbf{dont} (on peut citer)\textbf{} un citoyen américain]  

On observe ainsi qu'il n'y a pas de mécanisme général de reconstruction qui s'applique à tous les VRA. Si ce mécanisme est disponible, des contraintes lexicales, syntaxiques ou sémantiques doivent être prises en compte pour chaque cas. Comme il s'agit d'un mécanisme ad-hoc et superflu, la reconstruction syntaxique doit être abandonnée.  

\subsubsection{Différences sémantiques}
Si on considère que les VRA sont des phrases relatives elliptiques dérivées à partir des phrases relatives verbales, on s'attend à ce que leur contribution sémantique soit la même. Or, on observe que les VRA n'ont pas les mêmes propriétés sémantiques que les phrases relatives ordinaires.

\paragraph[Contenu (non{}-)parenthétique]{Contenu (non-)parenthétique}
Dans la section \ref{sec:5.3.1}, on a vu que les phrases relatives non-restrictives se comportent sémantiquement comme des phrases indépendantes qui contiennent une proforme (\citet{Arnold2004}), se prêtant à une analyse en termes d'implicatures conventionnelles. Par conséquent, leur contribution sémantique est largement indépendante par rapport à celle de la phrase hôte. En revanche, le contenu des VRA n'est pas parenthétique, c.-à-d. il fait partie du contenu asserté de la phrase hôte, comme le montre le contraste en \REF{ex:5:102}. Si l'exemple \REF{ex:5:102}a, qui contient une phrase relative complète, est cohérent, la séquence des énoncés en \REF{ex:5:102}b, contenant un VRA, est contradictoire, car on assume à la fois la présence et l'absence des oreilles apparentes chez les baleines. 


\begin{enumerate}
\item \label{bkm:Ref295814041}a  \textstyleapplestylespan{Non, tu te trompes !} Bien que la plupart des mamifères, [\textbf{dont} les baleines font effectivement partie], aient des oreilles apparentes, les baleines, elles, n'en ont pas. 


\end{enumerate}
  b  \#\textstyleapplestylespan{Non, tu te trompes !} Bien que la plupart des mamifères, [\textbf{dont} les baleines], aient des oreilles apparentes, les baleines, elles, n'en ont pas. 

Un des corrélats empiriques de cette différence sémantique observée entre les VRA et les phrases relatives non-restrictives est le fait que les verbes d'attitude propositionnelle n'ont pas de portée sur le contenu d'une phrase relative non-restrictive, cf. \REF{ex:5:103}, alors que cela n'est pas le cas dans les VRA, c.-à-d. un VRA est interprété sous la portée d'un tel verbe (cf. la discussion dans la section \ref{sec:5.3.1} et l'exemple \REF{ex:5:104}). 


\begin{enumerate}
\item \label{bkm:Ref295817928}a  Paul crede că anumite plante, [\textbf{printre care} amintim şi sunătoarea], vindecă ulcerul. 


\end{enumerate}
{\itshape
Paul croit que certaines plantes, parmi lesquelles nous mentionnons aussi la verveine, soigne les ulcères}

${\neq}${\textgreater}  b  Paul crede că sunătoarea vindecă ulcerul.

{\itshape
Paul croit que la verveine soigne les ulcères}

={\textgreater}  c  Sunătoarea vindecă ulcerul.

    \textit{La verveine soigne les ulcères}


\begin{enumerate}
\item \label{bkm:Ref295817961}a  Paul crede că anumite plante, [\textbf{printre care} şi sunătoarea], vindecă ulcerul. 


\end{enumerate}
{\itshape
Paul croit que certaines plantes, dont la verveine, soigne les ulcères}

={\textgreater}  b  Paul crede că sunătoarea vindecă ulcerul.

{\itshape
Paul croit que la verveine soigne les ulcères}

${\neq}${\textgreater}  c  Sunătoarea vindecă ulcerul.

{\itshape
La verveine soigne les ulcères}

Contrairement à sa contrepartie verbale, le contenu du VRA fait nécessairement partie du contenu de la phrase hôte ; par conséquent, une relative verbale ne permet pas de faire la même inférence logique qu'un VRA. 


\begin{enumerate}
\item a  \textstyleapplestylespan{Certains de mes amis, [}\textstyleapplestylespan{\textbf{dont}}\textstyleapplestylespan{ Marie \{est {\textbar} était\} la plus drôle], sont venus à la fête}. 


\end{enumerate}
${\neq}${\textgreater}   b  Marie est venue à la fête.


\begin{enumerate}
\item a  \textstyleapplestylespan{Certains de mes amis, [}\textstyleapplestylespan{\textbf{dont}}\textstyleapplestylespan{ Marie], sont venus à la fête}. 


\end{enumerate}
={\textgreater}  b  Marie est venue à la fête.

Les phrases relatives non-restrictives peuvent localement faire un commentaire sur la phrase hôte, alors que les VRA ne le peuvent pas. C'est pour cela que la séquence \textstyleapplestylespan{\textit{Il est donc étrange que Balzac en particulier soit autant boudé des enfants} }\textstyleapplestylespan{est une continuation appropriée pour une phrase relative non-restrictive} \textstyleapplestylespan{\REF{ex:5:107}}\textstyleapplestylespan{a, mais non pour un VRA} \textstyleapplestylespan{\REF{ex:5:107}}\textstyleapplestylespan{b.}


\begin{enumerate}
\item \label{bkm:Ref295818394}a  \textstyleapplestylespan{Les grands auteurs du XIXe, [}\textstyleapplestylespan{\textbf{dont}}\textstyleapplestylespan{ Balzac est le plus célèbre], sont beaucoup lus par les enfants. Il est donc étrange que Balzac en particulier soit autant boudé des enfants}. 


\end{enumerate}
  b  \textstyleapplestylespan{Les grands auteurs du XIXe, [}\textstyleapplestylespan{\textbf{dont}}\textstyleapplestylespan{ Balzac], sont beaucoup lus par les enfants. \#Il est donc étrange que Balzac en particulier soit autant boudé des enfants}. 

A la lumière de ces différences, on ne peut pas non plus assimiler les VRA aux phrases relatives restrictives, car les VRA ne restreignent pas la dénotation de l'antécédent.

\paragraph[Rôle de l'introducteur]{Rôle de l'introducteur}
Un autre problème soulevé par une approche elliptique en termes de reconstruction est le fait qu'elle prédit la bonne formation de certains VRA, qui en réalité sont inacceptables pour des raisons sémantiques. Cette approche considère que la sémantique partitive des VRA dérive du prédicat verbal élidé plutôt que de l'introducteur ; or, cette hypothèse ne rend pas compte des contraintes observées avec certains syntagmes prépositionnels dans les VRA.

En roumain, le corps d'un VRA introduit par \textit{dintre care} ne peut pas contenir de syntagme nominal référentiel, par exemple un nom propre \REF{ex:5:108}a. Cependant, on peut facilement reconstruire une forme verbale dans le même contexte et obtenir une phrase relative bien formée \REF{ex:5:108}b. On observe donc que l'interprétation exemplifiante est exclue dans les VRA introduits par \textit{dintre care}, mais disponible avec une forme verbale. 


\begin{enumerate}
\item \label{bkm:Ref295815347}a  *Au venit mai multe persoane, [\textbf{dintre care} Maria].  


\end{enumerate}
\textsc{aux} venus plusieurs personnes, dont Maria

{\itshape
Plusieurs personnes sont venues, dont Maria}

  b  Au venit mai multe persoane, [\textbf{dintre care} o amintim pe Maria].

    \textsc{aux} venus plusieurs personnes, dont \textsc{cl.3sg} mentionnons \textsc{mrq.spec} Maria

\textit{Plusieurs personnes sont venues, dont on mentionne Maria }  

En français, la différence observée entre le VRA introduit par \textit{dont} en \REF{ex:5:109}a et le VRA introduit par \textit{parmi lesquelles} en \REF{ex:5:109}b est inattendue dans une approche à base de reconstruction syntaxique, car le prédicat gérant la relation entre le VRA et l'antécédent est supposé être la tête verbale qui manque dans la phrase relative et non l'introducteur. En revanche, dans une approche non-structurale des VRA, rien n'empêche de considérer que l'introducteur possède des propriétés de sélection en ce qui concerne la constituance du corps du VRA (ex. dans un VRA introduit par \textit{parmi lesquel(le)s}, le corps doit contenir au moins un syntagme nominal). 


\begin{enumerate}
\item \label{bkm:Ref295819431}a  J'ai parlé à plusieurs personnes, [\textbf{dont} à Marie de linguistique]. 


\end{enumerate}
  b  *J'ai parlé à plusieurs personnes, [\textbf{parmi lesquelles} à Marie de linguistique].

\paragraph[Sémantique partitive stricte]{Sémantique partitive stricte}
Le type de relations explicitées par les VRA est beaucoup plus contraint par rapport aux relations qu'on peut avoir quand on emploie une phrase relative partitive. 

On a vu que les relations méronymiques sont exclues dans les VRA (cf. l'exemple \REF{ex:5:59}b-\REF{ex:5:60}b de la section \ref{sec:5.3.2.1}), mais elles sont possibles avec une phrase relative partitive, comme le montre le contraste en \REF{ex:5:110}.


\begin{enumerate}
\item \label{bkm:Ref295822265}a  *Paul adore les suédoises, [\textbf{dont} leurs cheveux]. 


\end{enumerate}
  b  Les fenêtres de ce château, [\textbf{dont} la peinture est écaillée], doivent être remplacées.

On a vu également que l'élément distingué dans le corps du VRA doit être une sous-partie de l'ensemble dénoté par l'antécédent (cf. la section \ref{sec:5.3.2.1}). Bien que le `pied-piping' soit, en principe, possible dans un VRA, uniquement les relations directes ensemble/sous-partie sont autorisées, alors que les phrases relatives partitives n'obéissent pas à une contrainte aussi stricte.  


\begin{enumerate}
\item a  *J'ai reçu deux pétitions, [\textbf{parmi} les signataires\textbf{ desquelles} Jean]. 


\end{enumerate}
  b  J'ai reçu deux pétitions, [\textbf{parmi} les signataires\textbf{ desquelles} figure Jean]. 

Pour conclure, à part le fait que la reconstruction syntaxique ne peut pas s'appliquer de façon uniforme à tous les VRA, on observe que les propriétés sémantiques ne sont pas les mêmes dans les VRA et dans les relatives non-restrictives verbales. Par conséquent, l'approche structurale en termes d'ellipse syntaxique ne peut pas rendre compte des propriétés syntaxiques et sémantiques des VRA. 

\subsection{Une approche constructionnelle des VRA~en termes d'ajouts fragmentaires}
On propose une approche non-structurale, qui rend compte et de la forme et du contenu des VRA sans postuler de structure syntaxique {\guillemotleft}~invisible~{\guillemotright}. Cette approche alternative reprend et enrichit l'analyse proposée par Bîlbîie \& \citet{Laurens2009}. 

Principales idées pour comprendre l'analyse :

{}- Le corps du VRA est la tête de la construction.

{}- L'introducteur a des propriétés de sélection syntaxique et sémantique vis-à-vis de la tête.

{}- L'introducteur a également une contribution sémantique importante et marque la construction.

{}- Le VRA est un ajout fragmentaire.

Dans cette section, je présente tout d'abord la notion sémantique de \textit{fragment} et je montre comment elle permet la {\guillemotleft}~reconstruction~{\guillemotright} sémantique d'un contenu propositionnel, à partir de la phrase hôte. Dans un deuxième temps, je présente la notion syntaxique de \textit{cluster}, qui permet de dériver toute séquence de syntagmes sans tête verbale qui peut constituer le corps du VRA. Dans un troisième temps, je m'intéresse aux relations fonctionnelles dans le VRA, en particulier à la relation syntaxique qui s'établit entre le corps et l'introducteur, pour décider de la tête du VRA. Ensuite, je montre comment les contraintes de localité permettent, d'une part, l'accès à l'élément distingué dans le corps du VRA et, d'autre part, l'accès à l'antécédent dans la phrase hôte. Enfin, je présente l'analyse du VRA dans son ensemble avec ses deux sous-types, en fonction des deux types d'introducteurs employés : un syntagme prépositionnel contenant une forme \textit{qu-} ou bien l'introducteur \textit{dont}. Je donne les entrées lexicales pour chaque introducteur, ainsi que les contraintes qui définissent les deux sous-types de VRA. La section finit par une présentation des deux \textit{dont} en français : le \textit{dont} des VRA et le complémenteur \textit{dont} dans les relatives verbales. 

\subsubsection{Théorie des fragments}
\label{bkm:Ref299024693}Toutes les différences enregistrées entre les VRA et leurs contreparties verbales s'expliquent par le fait que les VRA sont des ajouts fragmentaires.


\begin{enumerate}
\item Analyse en termes de fragment 


\end{enumerate}
{   [Warning: Image ignored] % Unhandled or unsupported graphics:
%\includegraphics[width=2.7252in,height=1.6382in,width=\textwidth]{fe443409cd384d3fb0f6390ffd77f513-img75.svm}
} 

Un fragment est une expression dont le contenu sémantique n'est pas déductible de la forme prise en isolation. Le contenu sémantique d'un fragment dépend : (i) du type du fragment ; (ii) du contenu sémantique des constituants du fragment ; (iii) des informations contextuelles qui peuvent être de nature linguistique ou non (Ginzburg \& \citet{Sag2000}, Fern\'andez \textit{et al.} (2007)). 

Un fragment phrastique comme, par exemple, la question courte \textit{quand} en \REF{ex:5:113}a\footnote{Ce type d'ellipse est connu dans la littérature sous le nom de \textit{sluicing}.} , est interprété comme ayant le même contenu sémantique que la phrase \textit{quand elle va venir} en \REF{ex:5:113}b. Le contenu sémantique du fragment \textit{quand} dans l'exemple \REF{ex:5:113}a est une fonction (i) du type du fragment (les questions courtes ont le même type de contenu que les phrases interrogatives, c.-à-d. une abstraction propositionnelle), (ii) du contenu littéral du constituant dans le fragment (\textit{quand} fournit le paramètre pour l'abstraction propositionnelle, en particulier il introduit un paramètre temporel) et (iii) de l'information contextuelle (la phrase source \textit{Marie va venir}, qui contient l'antécédent du fragment, fournit la proposition dont on a besoin pour construire l'abstraction propositionnelle). Ainsi, le fragment \textit{quand} a un contenu similaire à la phrase \textit{quand elle va venir}.  


\begin{enumerate}
\item \label{bkm:Ref295832479}a  Marie va venir, mais personne ne sait [quand]. 


\end{enumerate}
  b  Marie va venir, mais personne ne sait [quand elle va venir].

Pour calculer la sémantique du fragment, on utilise le langage \textit{Minimal Recursion Semantics} (désormais, MRS) de Copestake \textit{et al.} (2005), car, outre le fait qu'il permet la description d'une sémantique {\guillemotleft}~plate~{\guillemotright}, il a l'avantage de permettre la description des représentations sémantiques partielles ou incomplètes (telles que celles d'un fragment avant résolution), en sous-spécifiant les représentations complètes.\footnote{On a choisi MRS, mais on pourrait utiliser aussi d'autres langages, comme \textit{Lexical Resource Semantics} (LRS, Richter \& \citet{Sailer2003}) ou encore \textit{Type Theory with Records} (TTR, \citet{Cooper2005}).}  La résolution du fragment consiste ainsi dans un système de méta-contraintes reliant quatre représentations sémantiques dont deux partielles : le contenu de la phrase hôte (= la source) et le contenu du fragment résolu (= la cible) sont des représentations complètes, alors que le contenu {\guillemotleft}~abstrait~{\guillemotright} obtenu à partir de l'hôte et le contenu des constituants du fragment sont des représentations partielles.

En MRS, les unités minimales de la représentation sémantique sont appelées des prédications élémentaires (angl. \textit{elementary predications}, abrégé EP). Elles représentent une relation sémantique accompagnée de ses arguments, p.ex. \textit{aimer(x, y)}. 

La représentation générale d'une structure MRS est donnée en \REF{ex:5:114}. Les traits qui nous intéressent ici sont les deux premiers. Le trait HOOK contient l'ensemble de l'information accessible de l'extérieur aux foncteurs sémantiques. En particulier, il contient le trait LTOP, qui stocke le label sur lequel aucun autre n'a portée au niveau local (cela correspond à la tête sémantique de la structure qui est décrite), et le trait INDEX, qui stocke l'indice de la relation accessible au foncteur. Le trait RELS est la liste de prédications élémentaires.


\begin{enumerate}
\item \label{bkm:Ref297843200}Structure générale MRS


\end{enumerate}
  [Warning: Image ignored] % Unhandled or unsupported graphics:
%\includegraphics[width=5.7693in,height=0.9374in,width=\textwidth]{fe443409cd384d3fb0f6390ffd77f513-img76.svm}
  

Les quatre représentations sémantiques dont on a besoin pour la résolution sémantique du fragment dans les VRA correspondent à quatre multi-ensembles de prédications élémentaires en MRS : le multi-ensemble A correspond au contenu de la phrase source (noté SOURCE par la suite) ; le multi-ensemble B correspond au contenu abstrait (noté ABSTRACT-CONT) obtenu à partir de la phrase source ; le multi-ensemble C représente~le contenu du fragment (noté FRAGMENT) et, finalement, le multi-ensemble D correspond au contenu du fragment résolu / la cible (noté TARGET). Cela est illustré en \REF{ex:5:115} et décrit formellement en \REF{ex:5:116}. Les multi-ensembles A et D sont des représentations sémantiques complètes, alors que les multi-ensembles B et C sont des représentations sémantiques partielles. 


\begin{enumerate}
\item \label{bkm:Ref290580025}a  Plusieurs personnes sont venues, [\textbf{dont} Marie hier]. 


\end{enumerate}
  b  Plusieurs personnes sont venues. (A = SOURCE) 

  c  X est venu (B = ABSTRACT-CONT)

  d  Marie hier (C = FRAGMENT)

  e  Marie est venue hier. (D = TARGET)


\begin{enumerate}
\item \label{bkm:Ref290580103}Représentation MRS d'un fragment


\end{enumerate}
  [Warning: Image ignored] % Unhandled or unsupported graphics:
%\includegraphics[width=4.7055in,height=2.7083in,width=\textwidth]{fe443409cd384d3fb0f6390ffd77f513-img77.svm}
 

Ces multi-ensembles sont reliés par deux méta-contraintes : la fusion et la distillation\footnote{Je remercie à Olivier Bonami pour m'avoir proposé ces deux termes.}, qui opèrent sur des prédications élémentaires du même type. 

\textbf{Fusion de listes :} Soient L\textsubscript{1}, L\textsubscript{2} et L\textsubscript{3} trois descriptions de listes. L\textsubscript{1} est une \textbf{fusion} de L\textsubscript{2} et L\textsubscript{3} si et seulement si pour chaque description d'élément E\textsubscript{} figurant dans L\textsubscript{1}, soit apparaît dans L\textsubscript{2} mais pas dans L\textsubscript{3}, soit  apparaît dans L\textsubscript{3} mais pas dans L\textsubscript{2}, soit est l'unification d'une des descriptions d'éléments figurant dans L\textsubscript{2} et d'une des descriptions d'éléments figurant dans L\textsubscript{3}.

\textbf{Distillation de listes :} Soient L\textsubscript{1}, L\textsubscript{2} et L\textsubscript{3} trois descriptions de listes. L\textsubscript{1} est une \textbf{distillation} de L\textsubscript{2} et L\textsubscript{3} si et seulement si (i) L\textsubscript{2} est une fusion de L\textsubscript{1} et L\textsubscript{2} et (ii) L\textsubscript{3} est une fusion de L\textsubscript{1} et L\textsubscript{3}.

\textbf{Distillation maximale de listes} : Soient L\textsubscript{1}, L\textsubscript{2} et L\textsubscript{3} trois descriptions de listes. L\textsubscript{1} est une \textbf{distillation maximale} de L\textsubscript{2} et L\textsubscript{3} si et seulement si (i) L\textsubscript{1} est une distillation de L\textsubscript{2} et L\textsubscript{3} et (ii) il n'existe pas de liste L\textsubscript{1}' qui soit une distillation de L\textsubscript{2} et L\textsubscript{3} qui contiennent plus d'éléments que L\textsubscript{1}.

La première contrainte opérant sur les quatre représentations sémantiques est donnée en \REF{ex:5:117}a : grosso modo, on fait la distillation du contenu de la source et du contenu du fragment résolu afin de récupérer le contenu {\guillemotleft}~abstrait~{\guillemotright} dont on a besoin pour interpréter le fragment. La distillation doit être maximale, afin de récupérer le maximum d'informations de la source (les VRA partagent tout avec leur hôte, excepté le contenu littéral du fragment). Ainsi, le contenu de la cible en \REF{ex:5:115} ne peut pas être quelque chose de moins précis, comme \textit{Marie a fait quelque chose hier}. 

La deuxième contrainte est donnée en \REF{ex:5:117}b : on obtient le contenu du fragment résolu en fusionnant le contenu {\guillemotleft}~abstrait~{\guillemotright} et le contenu littéral du fragment. A l'issue de la fusion, on obtient plusieurs résultats qui seront par la suite contraints par des contraintes supplémentaires à l'interface syntaxe-sémantique, imposées par les clusters ou par la construction VRA elle-même.  


\begin{enumerate}
\item \label{bkm:Ref298864939}\label{bkm:Ref290579737}a  B est une distillation maximale de A et D~ 


\end{enumerate}
  b  D est une fusion de B et C

Ce système de représentation sémantique en MRS peut être intégré dans une grammaire de type HPSG en utilisant un trait FRAGMENT dont la valeur correspond à deux traits : le trait SOURCE et le trait ABSTRACT-CONT qui sont de type \textit{sem-obj}. Dans l'approche qu'on adopte ici, la valeur du trait CONT de tous les signes est de type \textit{mrs}. Tous les signes possèdent également un trait C-CONT (\textit{constructional content}), permettant d'exprimer un éventuel contenu sémantique d'origine constructionnelle\footnote{Les contenus de type message y sont conçus comme des relations d'origine constructionnelle.}, la valeur de ce trait étant également de type \textit{mrs}. 


\begin{enumerate}
\item \label{bkm:Ref298864805}Représentation du fragment en HPSG avec MRS


\end{enumerate}
  [Warning: Image ignored] % Unhandled or unsupported graphics:
%\includegraphics[width=2.8327in,height=1.1839in,width=\textwidth]{fe443409cd384d3fb0f6390ffd77f513-img78.svm}
 

A la représentation du fragment donnée en \REF{ex:5:118}, on ajoute les deux contraintes mentionnées précédemment  en \REF{ex:5:117}, qui informellement peuvent être décrites comme suit : le contenu {\guillemotleft}~abstrait~{\guillemotright} est obtenu en distillant le contenu de la source et le contenu du fragment résolu ; le contenu du fragment résolu est obtenu en fusionnant le contenu {\guillemotleft}~abstrait~{\guillemotright} avec les constituants du fragment et l'éventuel contenu sémantique d'origine constructionnelle.  


\begin{enumerate}
\item a  4 est une distillation maximale de 3 et 1~ 


\end{enumerate}
  b  1 est une fusion de 4 avec 5 et 2

\subsubsection{Théorie des clusters}
\label{bkm:Ref299024699}Certains types de fragments sont sujets à des contraintes de forme qui instancient les informations lexicales sur les propriétés de sous-catégorisation des items lexicaux qui ne sont pas réalisés dans la séquence elliptique. Cette propriété a amené Ginzburg \& \citet{Sag2000} à analyser le fragment phrastique comme la branche unaire d'un syntagme ayant l'ensemble des propriétés d'une phrase, y compris la catégorie syntaxique VERBAL. Dans cette perspective, le fragment \textit{Marie}, conçu comme une réponse courte à une question comme \textit{Qui est venu}, peut être décrit en postulant un syntagme de type tête-fragment, présenté en \REF{ex:5:120} : 


\begin{enumerate}
\item \label{bkm:Ref298867494}Syntagme de type tête-fragment dans Ginzburg \& \citet{Sag2000}


\end{enumerate}
  [Warning: Image ignored] % Unhandled or unsupported graphics:
%\includegraphics[width=5.4638in,height=1.2465in,width=\textwidth]{fe443409cd384d3fb0f6390ffd77f513-img79.svm}
 

Cependant, l'approche que je présente ici est un peu différente de celle proposée par Ginzburg \& \citet{Sag2000}, car, contrairement aux types de fragments analysés par Ginzburg \& \citet{Sag2000}, les fragments dans les VRA n'ont pas la même distribution que leurs contreparties à tête verbale, comme le montrent les différences d'acceptabilité qu'on observe avec les exemples en \REF{ex:5:122}.


\begin{enumerate}
\item a  Je me demande [où]. 


\end{enumerate}
  b  Je me demande [où il est].


\begin{enumerate}
\item \label{bkm:Ref298867546}a  Plusieurs ont parlé à mes amis, [dont {\textless}Marie à Marc{\textgreater}]. 


\end{enumerate}
  b  *Plusieurs ont parlé à mes amis, [dont Marie a parlé à Marc].

Cela est vrai même en dehors des constructions avec VRA. Il a été noté que les constructions elliptiques n'ont pas toujours la même distribution que leur source. Ainsi, en français la conjonction lexicalisée \textit{ainsi que} ou encore l'adverbial \textit{non pas} peuvent être suivis d'une séquence de syntagmes ayant le contenu d'une phrase, mais ils ne peuvent jamais se combiner avec une phrase finie (Abeillé \& \citet{Godard1996}, Mouret (2006, 2007, 2008)).


\begin{enumerate}
\item \label{bkm:Ref298951685}a  Paul a mangé une pomme, [\textbf{ainsi que} {\textless}Marie une orange{\textgreater}]. 


\end{enumerate}
  b  *Paul a mangé une pomme, [\textbf{ainsi que} Marie a mangé une orange].


\begin{enumerate}
\item \label{bkm:Ref298951687}a  Paul a invité Marie [et \textbf{non pas} {\textless}Marie Paul{\textgreater}]. 


\end{enumerate}
  b  *Paul a invité Marie [et \textbf{non pas} Marie a invité Paul].

C'est pour cela qu'on choisit de représenter les fragments comme un sous-type de \textit{head-only-ph} dont la seule branche tête correspond à un \textit{cluster} (comme pour les constructions à gapping). Les clusters sont des séquences de syntagmes qui ne sont pas reliés entre eux par des relations fonctionnelles. La notion de cluster en HPSG a été proposée par Mouret (2006, 2007) pour rendre compte des coordinations de séquences (angl. \textit{Argument Cluster Coordination}) en français (voir aussi section \ref{sec:4.5.1.1} du chapitre 4). Comme il l'indique, la notion de cluster doit être définie indépendamment de la coordination dans la grammaire, vu le fait qu'elle est pertinente pour décrire non seulement des constructions apparaissant dans le domaine de la coordination, mais aussi de la subordination.

Mouret (2006, 2007) pose une construction spécifique \textit{cluster-ph} sans tête, qui est un sous-type de \textit{non-headed-ph}. Les syntagmes de type tête-fragment (\textit{head-fragment-ph}) et de type cluster (\textit{cluster-ph}) sont représentés dans la hiérarchie de syntagmes donnée en \REF{ex:5:125}. 


\begin{enumerate}
\item \label{bkm:Ref299020979}\label{bkm:Ref298949932}Hiérarchie de syntagmes en HPSG, incluant les syntagmes sans tête et le fragment


\end{enumerate}
  [Warning: Image ignored] % Unhandled or unsupported graphics:
%\includegraphics[width=6.2972in,height=1.472in,width=\textwidth]{fe443409cd384d3fb0f6390ffd77f513-img80.svm}
 

Le syntagme de type cluster, défini formellement en \REF{ex:5:126}, présente un trait de tête CLUSTER qui prend pour valeur la liste des \textit{synsem} associées aux constituants immédiats. Les propriétés syntaxiques et sémantiques de ses constituants sont ainsi rendues accessibles au niveau de la construction. Le cluster peut comporter un seul constituant immédiat ou plus (cf. n ${\geq}$ 1).\footnote{La construction est saturée pour ses traits de valence et amalgame les valeurs SLASH de ses constituants. Les autres propriétés sont sous-spécifiées, ce qui permet la combinaison avec des formes comme \textit{ainsi que} et \textit{non pas} qui ne sont pas compatibles avec des catégories finies.} 


\begin{enumerate}
\item \label{bkm:Ref298950323}Syntagme de type cluster (cf. Mouret (2006, 2007))


\end{enumerate}
  [Warning: Image ignored] % Unhandled or unsupported graphics:
%\includegraphics[width=2.8728in,height=1.0634in,width=\textwidth]{fe443409cd384d3fb0f6390ffd77f513-img81.svm}
 

A partir de cette définition du syntagme cluster, on assume que les VRA dont le corps contient un seul syntagme présentent un cluster unaire, ce qui nous permet de généraliser l'analyse de façon simplifiée. On va subsumer donc sous la même analyse les deux formes de réalisation du corps des VRA, décrites dans la section \ref{sec:5.2.2}. 

Une fois les deux syntagmes définis (\textit{hd-fragment-ph} et \textit{cluster-ph}), on peut représenter formellement le corps du VRA sous ses deux formes. Ainsi, \REF{ex:5:127} est la représentation simplifiée d'un corps contenant un cluster de deux constituants, alors que \REF{ex:5:128} est la représentation d'un corps contenant un cluster unaire. Le syntagme {\guillemotleft}~supérieur~{\guillemotright} (\textit{head-fragment-ph}) hérite du syntagme {\guillemotleft}~inférieur~{\guillemotright} (\textit{cluster-ph}) sa catégorie sous-spécifiée, lui permettant de se combiner avec des foncteurs qui sélectionnent une catégorie non-finie, comme c'est le cas de la conjonction \textit{ainsi que} ou encore de l'averbial \textit{non pas} présentés plus haut et exemplifiés en \REF{ex:5:123} et \REF{ex:5:124}.


\begin{enumerate}
\item \label{bkm:Ref298950556}Représentation simplifiée d'un fragment contenant un cluster de deux constituants


\end{enumerate}
{   [Warning: Image ignored] % Unhandled or unsupported graphics:
%\includegraphics[width=2.3236in,height=3.5244in,width=\textwidth]{fe443409cd384d3fb0f6390ffd77f513-img82.svm}
} 


\begin{enumerate}
\item \label{bkm:Ref298950558}Représentation simplifiée d'un fragment contenant un cluster unaire


\end{enumerate}
{   [Warning: Image ignored] % Unhandled or unsupported graphics:
%\includegraphics[width=2.2752in,height=3.4319in,width=\textwidth]{fe443409cd384d3fb0f6390ffd77f513-img83.svm}
} 

Les clusters peuvent être sujets à plusieurs contraintes, comme par exemple le fait qu'ils doivent contenir un syntagme nominal non-marqué ou bien que leur forme instancie l'information sur la sous-catégorisation d'un mot qui n'est pas présent dans la structure. Le premier cas de figure peut être décrit formellement par les contraintes données en \REF{ex:5:129} et \REF{ex:5:131}. La première contrainte rend compte du fait qu'en français le corps du VRA doit contenir un syntagme nominal non-marqué par une préposition \REF{ex:5:130}. Cette contrainte s'applique partiellement en roumain (c.à.d. dans les VRA introduits par \textit{dintre care}, cf. la section \ref{sec:5.2.2.1}) et doit contenir une spécification supplémentaire : le cas du syntagme nominal doit être direct, c.-à-d. il doit avoir une forme de nominatif-accusatif, et non de datif-génitif. La contrainte en \REF{ex:5:131} rend ainsi compte de l'exemple \REF{ex:5:132}. 


\begin{enumerate}
\item \label{bkm:Ref298951898}Cluster contenant un NP non-marqué en français


\end{enumerate}
  [Warning: Image ignored] % Unhandled or unsupported graphics:
%\includegraphics[width=3.0193in,height=0.7075in,width=\textwidth]{fe443409cd384d3fb0f6390ffd77f513-img84.svm}
 


\begin{enumerate}
\item \label{bkm:Ref298952628}J'ai parlé \textbf{à} plusieurs personnes, [\textbf{parmi lesquelles} (*à) Marie].

\item \label{bkm:Ref298953361}\label{bkm:Ref298951902}Cluster contenant un NP non-marqué en roumain


\end{enumerate}
  [Warning: Image ignored] % Unhandled or unsupported graphics:
%\includegraphics[width=3.4807in,height=0.9161in,width=\textwidth]{fe443409cd384d3fb0f6390ffd77f513-img85.svm}
 


\begin{enumerate}
\item   \label{bkm:Ref298953416}Impactul a dus la spitalizarea mai multor persoane, [\textbf{dintre care} (*a) şapte români]. 


\end{enumerate}
l'impact a mené à l'hospitalisation plusieurs\textsc{.gen} personnes, dont \textsc{mrq.gen} sept roumains 

{\itshape
L'impact a mené à l'hospitalisation de plusieurs personnes, dont sept roumains } 

A ne pas confondre les notions de \textit{fragment} et \textit{cluster~}: la notion de \textit{cluster} est une notion syntaxique, liée à la constituance, alors que la notion de \textit{fragment} est une notion plutôt sémantique, liée au fait que l'interprétation n'est pas autonome, elle dépend de l'interprétation de l'antécédent.

\subsubsection{Relations fonctionnelles dans le VRA}
On s'intéresse maintenant aux relations fonctionnelles qui s'établissent entre le corps du VRA et l'introducteur. Avant de lister les analyses envisageables pour le VRA, je veux présenter les grandes lignes de l'analyse proposée par \citet{Godard1988} et Abeillé \& Godard (2006, 2007) pour les relatives ordinaires du français. Deux grands types de relatives sont distingués en fonction du type d'introducteur : les relatives avec pronom relatif (\textit{pro-rel-clause}) et les relatives avec complémenteur (\textit{compr-rel-clause}).\footnote{Parmi ces dernières, on distingue les relatives avec gap (introduites par \textit{que {\textbar} qui} ou \textit{dont}; p.ex. (i)) des relatives sans gap (en \textit{que {\textbar} qui}, p.ex. (ii)). Comme les formes \textit{que} et\textit{ qui} ne sont pas employées dans les VRA, on s'intéresse ici uniquement aux relatives en \textit{dont}, donc on laisse de côté la discussion sur les relatives sans gap.
(i)  a  quelque chose [\textbf{que} je regarde]
  b  quelqu'un [\textbf{qui} chante]
  c  quelqu'un [\textbf{dont} on parle]
(ii)  a  une fille [\textbf{qu}'elle est sympa]
  b  des feux [\textbf{qu}'il faut appeler les pompiers]} Un exemple simplifié pour chaque type est donné en \REF{ex:5:133}. 


\begin{enumerate}
\item \label{bkm:Ref298794950}a  une personne [\textbf{à~laquelle} on parle] 


\end{enumerate}
  b  une personne [\textbf{dont} on parle] 

Les relatives à pronom relatif comme \REF{ex:5:133}a sont analysées comme des constructions à extraction ; dans cette configuration, l'introducteur \textit{à laquelle} a la fonction extrait (angl. \textit{filler}) et doit correspondre à un gap dans le reste de la relative. Ce sous-type de relatives contraint le syntagme extrait à être un syntagme prépositionnel contenant un pronom relatif. Ces relatives sont donc analysées comme des syntagmes de type tête-extrait (angl. \textit{head-filler-ph}, cf. la hiérarchie donnée en \REF{ex:5:125}). En ce qui concerne le deuxième sous-type de relatives, c.-à-d. celles avec complémenteur, elles sont analysées plutôt comme des syntagmes de type tête-complément (angl. \textit{head-complement-ph}). Ainsi, le complémenteur \textit{dont}\footnote{Voir les propriétés du complémenteur \textit{dont} présentées dans la section \ref{sec:5.2.1.2}.}  en \REF{ex:5:133}b est analysé comme une tête syntaxique prenant une phrase comme complément et introduisant un trait syntaxique MARQUE (abrégé MRKG dans les structures de traits en HPSG) qui sert à distinguer les relatives avec et sans complémenteur.

On observe donc que pour les relatives ordinaires on n'a pas d'analyse uniforme en français. Revenant aux VRA, on doit préciser que ces relatives `sans verbe' ne se prêtent a priori à aucune des deux analyses proposées pour les relatives verbales en français, car il n'y a pas de gap dans le VRA, c.-à-d. il n'y a pas d'élément manquant (en position canonique) correspondant à l'introducteur. Aucune relation d'extraction ne semble exister au sein des VRA. De plus, comme on l'a observé dans la section \ref{sec:5.2.1.2}, la forme \textit{dont} dans les VRA n'a pas exactement les mêmes propriétés que le complémenteur \textit{dont} dans les relatives ordinaires. Par conséquent, on doit chercher d'autres possibilités d'analyse pour les VRA. 

On commence d'abord à distinguer le constituant qui pourrait avoir la fonction de tête dans le VRA. Il y a trois analyses possibles : (i) soit c'est le corps du VRA qui est la tête (analyse A), (ii) soit c'est l'introducteur du VRA qui est la tête (analyse B), (iii) soit il n'y a pas de tête (analyse C). On montre par la suite que les analyses B et C doivent être éliminées, car le corps du VRA présente des propriétés de tête. 


\begin{enumerate}
\item   Analyse A : Le corps est la tête.    Analyse B : L'introducteur est la tête.


\end{enumerate}
  [Warning: Image ignored] % Unhandled or unsupported graphics:
%\includegraphics[width=2.8689in,height=1.5654in,width=\textwidth]{fe443409cd384d3fb0f6390ffd77f513-img86.svm}
     [Warning: Image ignored] % Unhandled or unsupported graphics:
%\includegraphics[width=2.8689in,height=1.5366in,width=\textwidth]{fe443409cd384d3fb0f6390ffd77f513-img87.svm}
 

{
Analyse C : Il n'y a pas de tête.
}

{   [Warning: Image ignored] % Unhandled or unsupported graphics:
%\includegraphics[width=2.8728in,height=1.5409in,width=\textwidth]{fe443409cd384d3fb0f6390ffd77f513-img88.svm}
} 

\paragraph[Le corps du VRA comme tête]{Le corps du VRA comme tête}
L'argument le plus convaincant pour attribuer la fonction de tête au corps du VRA est son emploi indépendant dans certains exemples. Les syntagmes constituant le corps du VRA peuvent fonctionner de la même manière qu'un VRA dans son ensemble, en l'absence d'un introducteur. Ils se comportent comme des ajouts incidents, avec une sémantique et une distribution similaires à celles qu'on retrouve dans les VRA. Ainsi, dans le corpus on trouve des ajouts incidents sans introducteur, avec les deux types d'interprétations : une interprétation exemplifiante, facilitée par la présence de l'adverbial \textit{în special} `notamment' en roumain \REF{ex:5:135}a ou \textit{notamment} en français \REF{ex:5:136}a, ou encore une interprétation partitionnante apparaissant surtout dans une structure coordonnée \REF{ex:5:135}c-\REF{ex:5:136}b-c.


\begin{enumerate}
\item \label{bkm:Ref296067471}a  \c{T}ările din Europa de Est, [(\textbf{printre} \textbf{care}) în special România], reprezintă piețele preferate de către mafioții italieni pentru spălarea banilor.  


\end{enumerate}
{\itshape
Les pays de l'Europe de l'Est, (parmi lesquels) notamment la Roumanie, représentent les marchés préférés par la maffia italienne pour le blanchiment  d'argent}

  b  6~000 de zboruri au fost anulate, [(\textbf{dintre} \textbf{care}) aproape 70 doar în Bucureşti].

{\itshape
6~000 vols ont été annulés, (dont) presque 70 seulement à Bucarest}

  c  Șase persoane, [(\textbf{dintre} \textbf{care}) două din Gorj şi patru din Mehedinți], ar avea legături cu gruparea infracțională din care face parte şi Emilian Ștefan.

    \textit{Six personnes, (dont) deux de Gorj et quatre de Mehedinți, auraient des relations avec le groupe infractionnel dont Emilian Ștefan fait partie}


\begin{enumerate}
\item \label{bkm:Ref296067476}a  De nombreuses éspèces animales, [(\textbf{dont}) notamment les oursins], ont souffert de la pollution. 


\end{enumerate}
  b  Plusieurs personnes, [(\textbf{dont}) une hier et deux ce matin], se sont plaintes de l'organisation.

  c  Plusieurs personnes, [(\textbf{dont})\textbf{} Marie hier et Jean ce matin], ont signalé le problème.

Par conséquent, la présence de l'introducteur n'est pas toujours nécessaire dans les syntagmes avec une interprétation fragmentaire, pour qu'ils fonctionnent comme ajouts avec une interprétation exemplifiante ou partitionnante. Il faut noter toutefois qu'en l'absence d'un introducteur, la sémantique de l'ajout n'est pas restreinte aux deux types d'interprétations mentionnées. Ainsi, les ajouts sans introducteur peuvent contenir un quantifieur universel ou négatif \REF{ex:5:137}{}-\REF{ex:5:138}, ce qui ne semble pas être le cas des VRA standard, comme le montrent les exemples \REF{ex:5:57}{}-\REF{ex:5:58} de la section \ref{sec:5.3.2.1}.\footnote{Les ajouts qui sont compatibles avec un quantifieur universel ou négatif permettent aussi l'emploi d'une conjonction (comme \textit{mais}), bien que l'introducteur \textit{dont} soit impossible.
(i)  a  La secrétaire du labo a commandé plusieurs livres, [\{mais {\textbar} *dont\} tous en anglais]. 
  b  La secrétaire du labo a commandé plusieurs livres, [\{mais {\textbar} *dont\} aucun en français].
 } 


\begin{enumerate}
\item \label{bkm:Ref298798949}a  Am scris trei cărți, [toate pe aceeaşi temă].  


\end{enumerate}
  \textit{J'ai écrit trois livres, tous sur le même thème} 

  b  Patru fete, [niciuna trecută de 23 ani], au decedat vineri seara în urma unui accident de circulație.

{\itshape
Quatre jeunes filles, aucune de plus de 23 ans, sont décédées vendredi soir à la suite d'un accident de la circulation}


\begin{enumerate}
\item \label{bkm:Ref298798951}a  Plusieurs livres, [tous sur le même thème], ont été commandés. 


\end{enumerate}
  b  \%Quatre filles, [aucune de plus de 23 ans], habitent dans cet immeuble délabré.

Un argument contre l'analyse de l'introducteur comme tête concerne la réalisation syntaxique des arguments sémantiques de la tête de l'introducteur. A part la forme \textit{dont} en français, dont la catégorie est sujette à discussion, l'introducteur est toujours un syntagme prépositionnel (roum. \{\textit{printre {\textbar} între {\textbar} dintre}\}\textit{ care} ou fr. \textit{parmi lesquel(le)s}). Une préposition, comme le roumain \textit{printre} `parmi', introduit une relation sémantique entre deux arguments (indépendamment de son emploi dans les VRA ou en dehors d'un VRA) : l'un des arguments (noté \textit{Arg1} dans les exemples en \REF{ex:5:139})  est typiquement réalisé comme complément de la préposition, alors que l'autre (noté \textit{Arg2})  n'est pas réalisé dans le syntagme prépositionnel lui-même (c.-à-d. il est un argument externe). Ce deuxième argument n'est pas sélectionné par la préposition elle-même, mais plutôt par le syntagme prépositionnel dans son ensemble. Dans le cas des VRA, cet argument externe de la préposition n'est pas identifié avec l'antécédent du VRA, mais plutôt avec l'élément distingué se trouvant dans le corps du VRA. Il est donc légitime d'assumer que les propriétés de sélection de l'introducteur et celles du VRA dans son ensemble sont distinctes et, par conséquent, on ne doit pas analyser l'introducteur comme tête.  


\begin{enumerate}
\item \label{bkm:Ref296071594}a  Avem [un spion]\textsubscript{Arg2} [\textbf{printre} [noi]\textsubscript{Arg1}].  


\end{enumerate}
  \textit{Nous avons un espion parmi nous}

  b  ... mai multe persoane, [\textbf{printre} [care]\textsubscript{Arg1}] [şi Maria]\textsubscript{Arg2}.

{\itshape
... plusieurs personnes, parmi lesquelles Maria}

\paragraph[L'introducteur du VRA comme foncteur]{L'introducteur du VRA comme foncteur}
L'introducteur du VRA impose certaines contraintes : (i) il doit précéder le corps du VRA (contrairement à des ajouts adverbiaux comme \textit{notamment}, cf. les exemples \REF{ex:5:10}{}-\REF{ex:5:11}) ; (ii) il présente des propriétés de sélection ; (iii) (au moins dans certains cas) il modifie la distribution du syntagme avec lequel il se combine.

Les relations fonctionnelles qui s'établissent entre l'introducteur et le corps d'un VRA rappellent les discussions portées autour de la relation fonctionnelle entre le déterminant et le nom à l'intérieur d'un groupe nominal. Ainsi, Van Eynde (2003, 2006, 2007) argumente en faveur d'une distinction qui doit être faite entre la notion de tête  et et la notion de sélecteur. En particulier, il considère que les prénominaux (déterminants, numéraux, etc.) de l'italien et du néerlandais doivent être analysés comme des sélecteurs qui prennent un nominal comme leur tête et non comme leur complément (par exemple, dans le syntagme \textit{whose house}, le pronom au génitif \textit{whose} sélectionne un nom commun comme tête ; le syntagme dans son ensemble reçoit le cas de la tête nominale, cf. le Principe des Traits de Tête, indépendamment du cas du pronom au génitif). Cette approche a l'avantage de rendre compte de manière satisfaisante des faits d'accord morpho-syntaxique dans un syntagme nominal (ou encore de l'accord sémantique dans une coordination nominale).

Par conséquent, Van Eynde introduit une nouvelle fonction syntaxique \textit{foncteur} (qui regroupe les fonctions de spécifieur, marqueur ou ajout pré-tête). Les foncteurs ne sont pas sous-catégorisés, au contraire ils sélectionnent une tête, à laquelle ils peuvent imposer un certain marquage. Par conséquent, une tête combinée avec un foncteur peut avoir une distribution différente de celle qu'elle pourrait avoir en emploi indépendant. 

On propose donc d'analyser l'introducteur du VRA comme un foncteur\footnote{Le fait que l'introducteur du VRA ne se comporte pas comme une tête ordinaire ou encore comme un ajout pourrait être pris en compte par des approches alternatives : l'introducteur du VRA pourrait être analysé soit comme une tête {\guillemotleft}~faible~{\guillemotright} (voir, dans ce sens, l'analyse proposée pour les conjonctions dans le chapitre 2, section \ref{sec:2.10.2}), soit comme un marqueur. A priori, le choix d'une ou l'autre analyse ne fait pas de différence empirique majeure.}, sans pour autant étendre cette fonction aux constituants prénominaux\footnote{Bien que les arguments de Van Eynde soient convaincants, on reprend son analyse uniquement pour les introducteurs dans les VRA. Il reste à vérifier si on peut l'étendre aux prénominaux en roumain (et en français) ou encore à d'autres catégories comme les conjonctions ou les complémenteurs. Pour l'instant, on ne remet pas en cause ici la distinction entre spécifieurs et modifieurs.}. La linéarisation stricte de l'introducteur par rapport au corps du VRA, qui ne caractérise pas les ajouts, est un argument en faveur de cette analyse, cf. les exemples repris en \REF{ex:5:140}{}-\REF{ex:5:141}.


\begin{enumerate}
\item \label{bkm:Ref296074963}a  Mai multe țări sud-americane, [\textbf{printre care} şi Brazilia], exportă cafea în Europa. 


\end{enumerate}
{\itshape
Plusieurs pays sud-américains, parmi lesquels aussi le Brésil, exportent du café en Europe}

  b  *Mai multe țări sud-americane, [şi Brazilia \textbf{printre care}], exportă cafea în Europa.

  c  Mai multe țări, [\textbf{printre care} în mod special Brazilia], exportă cafea în Europa.

{\itshape
Plusieurs pays, parmi lesquels en particulier le Brésil, exportent du café en Europe}

  d  *Mai multe țări, [în mod special \textbf{printre care} Brazilia], exportă cafea în Europa.


\begin{enumerate}
\item \label{bkm:Ref298760194}a  Plusieurs personnes sont venues, [\textbf{parmi lesquelles} Jean].


\end{enumerate}
  b  *Plusieurs personnes sont venues, [Jean \textbf{parmi lesquelles}].

  c  Plusieurs personnes sont venues, [\textbf{parmi lesquelles} notamment Jean].

  b  *Plusieurs personnes sont venues, [notamment \textbf{parmi lesquelles} Jean].

Plus important encore, cette analyse est justifiée par les propriétés de sélection de l'introducteur et par la contribution sémantique de celui-ci au type sémantique de la construction dans son ensemble (ex. l'introducteur \textit{dintre care} en roumain impose une interprétation partitionnante à l'ensemble du VRA).

Le foncteur et la tête qu'il sélectionne forment un syntagme de type \textit{tête-foncteur} (\textit{hd-funct-ph}), qui, dans la hiérarchie de syntagmes figurant en \REF{ex:5:142}, est un sous-type des syntagmes de type \textit{tête-ajout} (\textit{hd-adj-ph}), cf. Van \citet{Eynde2007}. 


\begin{enumerate}
\item \label{bkm:Ref299022036}Le syntagme tête-foncteur dans une hiérarchie de syntagmes


\end{enumerate}
  [Warning: Image ignored] % Unhandled or unsupported graphics:
%\includegraphics[width=6.2909in,height=1.5327in,width=\textwidth]{fe443409cd384d3fb0f6390ffd77f513-img89.svm}
  

Le syntagme tête-foncteur est défini en \REF{ex:5:143} et exemplifié en \REF{ex:5:144}.


\begin{enumerate}
\item \label{bkm:Ref299023339}Syntagme de type tête-foncteur  


\end{enumerate}
  [Warning: Image ignored] % Unhandled or unsupported graphics:
%\includegraphics[width=3.4811in,height=1.3953in,width=\textwidth]{fe443409cd384d3fb0f6390ffd77f513-img90.svm}
 


\begin{enumerate}
\item \label{bkm:Ref299023342}Arbre avec un syntagme tête-foncteur  


\end{enumerate}
{   [Warning: Image ignored] % Unhandled or unsupported graphics:
%\includegraphics[width=3.7362in,height=3.3819in,width=\textwidth]{fe443409cd384d3fb0f6390ffd77f513-img91.svm}
} 

Les foncteurs ont deux propriétés principales : ils sélectionnent leur tête (cf. le trait SELECT, qui est inclus dans la valeur de HEAD de la branche foncteur) et contribuent au marquage du syntagme (cf. le trait MRKG, qui apparaît dans la valeur du trait CAT). 

Les trois contraintes pesant sur le syntagme tête-foncteur sont :

(i) Principe des Traits de Tête : la valeur du trait HEAD d'un syntagme avec tête est identique à la valeur du trait HEAD de la branche tête (on capte ainsi l'intuition générale que la catégorie de la tête donne la catégorie du syntagme). 

(ii) Principe du Sélecteur : on reprend de Van Eynde (2003, 2006, 2007) le trait SELECT qui capte les contraintes qu'une branche non-tête peut imposer à sa s{\oe}ur tête. La valeur du trait SELECT de la branche non-tête doit être identique à la valeur SYNSEM de la branche tête. La valeur du trait SELECT est de type \textit{synsem} et non \textit{signe}, ce qui fait que le foncteur peut imposer des contraintes concernant les propriétés syntaxiques et sémantiques de la tête, mais pas concernant les propriétés phonologiques de celle-ci ou encore sa structure interne (ce qui rend possible les différentes constituances au sein du corps du VRA : un seul constituant ou un cluster, par exemple). La raison pour laquelle le trait SELECT est introduit dans la valeur du trait HEAD du foncteur est le fait que les propriétés de sélection d'un foncteur syntagmatique sont identiques à celles de sa branche tête.

(iii) Principe du Marquage Généralisé : l'information qui figure dans la valeur du trait MARKING est partagée entre la mère et sa branche non-tête (la valeur MARKING de la mère dans un syntagme de type \textit{hd-funct-ph} est identique à celle de sa branche non-tête).  

Le syntagme tête-foncteur présente ainsi deux branches : une branche tête (dont l'indice est 3 en \REF{ex:5:143} et \REF{ex:5:144}) qui donne le trait HEAD de la construction et une branche foncteur qui sélectionne sa tête et contribue au marquage de la construction. Par conséquent, la tête d'un VRA est le fragment tel qu'il a été défini dans les sections 5.5.1 et 5.5.2.

Après avoir défini les relations syntaxiques qui s'établissent (i) à l'intérieur du corps d'un VRA, (ii) entre le corps du VRA et l'introducteur, et (iii) entre le VRA dans son ensemble et la phrase hôte, on peut donner une représentation simplifiée de ces relations en français \REF{ex:5:145} et en roumain \REF{ex:5:146}. L'arbre \REF{ex:5:146} est une illustration de l'exemple \REF{ex:5:147}.  


\begin{enumerate}
\item \label{bkm:Ref299109023}Syntaxe simplifiée des constructions à VRA en français  


\end{enumerate}
{   [Warning: Image ignored] % Unhandled or unsupported graphics:
%\includegraphics[width=4.3063in,height=2.4799in,width=\textwidth]{fe443409cd384d3fb0f6390ffd77f513-img92.svm}
} 


\begin{enumerate}
\item \label{bkm:Ref299109040}Syntaxe simplifiée des constructions à VRA en roumain  


\end{enumerate}
{   [Warning: Image ignored] % Unhandled or unsupported graphics:
%\includegraphics[width=3.9583in,height=2.5154in,width=\textwidth]{fe443409cd384d3fb0f6390ffd77f513-img93.svm}
} 


\begin{enumerate}
\item \label{bkm:Ref299042029}Vin mai multe persoane, [\textbf{printre care} şi Maria].


\end{enumerate}
  viennent plusieurs personnes, parmi lesquelles aussi Maria

  \textit{Plusieurs personnes viennent, parmi lesquelles Maria } 

La relation syntaxique qui s'établit entre le VRA et la phrase hôte~définit un syntagme de type tête-ajout (\textit{hd-adjunct-ph}). La relation syntaxique qui existe entre le corps du VRA et l'introducteur définit un syntagme de type tête-foncteur (\textit{hd-functor-ph}). 

\subsubsection{Théorie de la localité de sélection sémantique}
\label{bkm:Ref299031107}Les propriétés de sélection du VRA et les propriétés de sélection de l'introducteur obéissent au même type de contraintes de localité. Plus précisément, il y a deux relations faisant intervenir un problème de localité de sélection : (i) la relation à distance entre le VRA et son antécédent, et (ii) la relation à distance entre l'introducteur et l'élément distingué. Concernant la première relation, si le VRA modifie une phrase, l'antécédent doit être un dépendant direct de la tête de la phrase (comparer les exemples \REF{ex:5:148}b et \REF{ex:5:148}d). Concernant la deuxième relation, si l'introducteur modifie un cluster, le syntagme exprimant la sous-partie de l'entité fractionable (c.-à-d. l'élément distingué) doit être un constituant immédiat du cluster, cf. \REF{ex:5:149}. On observe donc que, dans les deux cas, les contraintes sont du même ordre : soit il y a localité, soit le niveau d'enchâssement est limité à 1.


\begin{enumerate}
\item \label{bkm:Ref296092613}a  Plusieurs personnes, [\textbf{dont} Marie], sont venues. 


\end{enumerate}
  b  Plusieurs personnes sont venues, [\textbf{dont} Marie].

  c  Des représentants de plusieurs pays, [\textbf{dont} le Brésil], se sont réunis hier.

  d  *Des représentants de plusieurs pays se sont réunis hier, [\textbf{dont} le Brésil].


\begin{enumerate}
\item \label{bkm:Ref296092705}a  Plusieurs personnes sont venues, [\textbf{dont} Marie]. 


\end{enumerate}
  b  Plusieurs personnes sont venues, [\textbf{dont} hier le frère de Marie].

Pour formaliser ces contraintes de localité, on utilise un dispositif analogue à celui utilisé par \citet{Kiss2005} pour la relation de sélection sémantique à distance entre les subordonnées relatives extraposées et leur antécédent en allemand. De manière générale, ce mécanisme permet aux ajouts (phrastiques) de modifier des syntagmes dont les propriétés sont très variables, à condition qu'ils contiennent un élément particulier qui puisse être qualifié d'antécédent et que celui-ci ne soit pas trop enchâssé. 

On reprend le trait ANCHORS (dont la valeur est une liste) introduit par \citet{Kiss2005}. Ce trait contient les indices des entités sémantiques qui sont accessibles à la sélection opérée par l'ajout. Les expressions linguistiques introduisent des ancres sémantiques typées qui sont propagées selon certaines règles et sélectionnables. 

Les deux contraintes dont on a besoin pour la propagation d'ancres figurent en \REF{ex:5:150} et \REF{ex:5:151}. 


\begin{enumerate}
\item \label{bkm:Ref296095245}Contrainte de localité 1 


\end{enumerate}
  [Warning: Image ignored] % Unhandled or unsupported graphics:
%\includegraphics[width=5.0799in,height=0.4791in,width=\textwidth]{fe443409cd384d3fb0f6390ffd77f513-img94.svm}
 


\begin{enumerate}
\item \label{bkm:Ref296095248}Contrainte de localité 2


\end{enumerate}
  [Warning: Image ignored] % Unhandled or unsupported graphics:
%\includegraphics[width=5.0799in,height=0.4575in,width=\textwidth]{fe443409cd384d3fb0f6390ffd77f513-img95.svm}
 

La première contrainte~(en \REF{ex:5:150}) dit qu'au niveau d'un syntagme, tous les dépendants directs de la tête sont sélectionnables et accessibles via l'ensemble des ancres.

La deuxième contrainte (en \REF{ex:5:151}) garantit qu'au niveau d'un cluster, tous les constituants immédiats du cluster sont sélectionnables et accessibles via l'ensemble des ancres. Un autre avantage des contraintes d'ancre est qu'elles restreignent la sélection sémantique au matériel qui est littéralement introduit dans le cluster. Les relations sémantiques reconstruites ne sont donc pas disponibles pour la sélection sémantique.

Le résultat de l'application de ces deux contraintes sur un VRA sera donné plus tard, une fois qu'on aura introduit tous les éléments nécéssaires pour l'analyse.

\subsubsection{Théorie des VRA}
Cette section fait la synthèse des contraintes syntaxiques et sémantiques qui s'appliquent au VRA dans son ensemble et montre la manière dont elles intéragissent.  

Les contraintes syntaxiques dans les VRA dérivent : (i) des propriétés des clusters, (ii) des propriétés de sélection des introducteurs, et (iii) des propriétés constructionnelles des VRA. Les clusters ont des propriétés inhérentes qui peuvent entrer en conflit avec les propriétés des VRA. Par exemple, les syntagmes dans un cluster doivent être des dépendants directs de la tête de la phrase hôte et avoir le même marquage que les syntagmes parallèles dans l'hôte. Cette contrainte entrera en conflit avec les propriétés de sélection de certains introducteurs, comme \textit{parmi lesquel(le)s} en français, qui contraignent le syntagme exprimant la sous-partie (c.-à-d. l'élément distingué) à être un syntagme nominal non-marqué.


\begin{enumerate}
\item a  \%J'ai parlé à plusieurs personnes, [\textbf{dont} à Marie de linguistique. 


\end{enumerate}
  b  *J'ai parlé à plusieurs personnes, [\textbf{parmi lesquelles} à Marie de linguistique].

Les propriétés constructionnelles du VRA dans son ensemble incluent la présence d'un introducteur avec une sémantique partitive. Pour les VRA introduits par un syntagme \textit{qu-}, il doit être précisé que l'introducteur doit contenir une forme \textit{qu-} coréférentielle avec l'antécédent dans la phrase hôte.

Les contraintes sémantiques des VRA dérivent : (i) de la sémantique du fragment, et (ii) de la sémantique partitive des VRA. Les VRA ont un type de contenu similaire à celui d'une phrase. Ils se comportent comme des anaphores descriptives, par le fait qu'ils introduisent une nouvelle entité sémantique qui partage une partie de sa description avec l'antécédent. Dans le cas des VRA, cette nouvelle entité est une sous-éventualité. Les VRA introduisent simultanément deux relations partitives : la première relation partitive relie l'antécédent et une nouvelle entité introduite par un syntagme dans le corps du VRA. Cette relation est exprimée par l'introducteur. Cela est mis en évidence par le fait que les VRA peuvent avoir soit une interprétation exemplifiante, soit une interprétation partitionnante, mais tous les introducteurs ne sont pas compatibles avec les deux interprétations. La deuxième relation partitive relie l'éventualité dénotée par la phrase source et la sous-éventualité introduite par le VRA.

Le VRA est défini en \REF{ex:5:153}. La tête du syntagme est un fragment phrastique. Il est sélectionné par l'introducteur du VRA qui impose une relation ensemble/sous-partie (\textit{sum-subpart-rel}), relation sémantique qui est caractéristique de cette construction. La relation ensemble/sous-partie est supposée avoir deux sous-types : un sous-type exemplifiant (\textit{exemplifying-sum-subpart-rel}) et un sous-type partitionnant (\textit{partitioning-sum-subpart-rel}). La construction elle-même apporte une deuxième relation mettant en jeu une sous-partie : cette relation connecte l'éventualité dénotée par la phrase hôte, [SUM 6 \textit{event}],  avec l'éventualité dénotée par le fragment phrastique, [SUM 7 \textit{event}]. De plus, la construction sélectionne un antécédent nominal. Finalement, on note l'emploi de l'ensemble ANCHORS pour exprimer la localité de la sélection opérée par l'introducteur et par le VRA lui-même. Ainsi, [ANCHORS {\textless}..., [IND 2], ...{\textgreater}] nous permet d'avoir accès à l'antécédent, qui désigne une entité plurielle [SUM 2] dans la phrase hôte, alors que [ANCHORS {\textless}..., [IND 8], ...{\textgreater}] nous permet l'accès à l'élément distingué, qui dénote la sous-partie [SUBPART 8] dans le corps du VRA. 


\begin{enumerate}
\item \label{bkm:Ref296099845}La construction VRA  


\end{enumerate}
  [Warning: Image ignored] % Unhandled or unsupported graphics:
%\includegraphics[width=5.7283in,height=4.8252in,width=\textwidth]{fe443409cd384d3fb0f6390ffd77f513-img96.svm}
 

Selon la forme de l'introducteur, la construction VRA présente deux sous-types : un VRA dont l'introducteur contient un mot \textit{qu-} (\textit{WH-VRA-ph}) et un VRA (en français uniquement) dont l'introducteur est \textit{dont} (\textit{DONT-VRA-ph}).

En ce qui concerne le premier sous-type (\textit{WH-VRA-ph}), l'introducteur contient une forme \textit{qu-} qui est coréférentielle avec l'antécédent nominal dans la phrase hôte (cf. le partage d'indices).


\begin{enumerate}
\item La construction VRA avec un introducteur contenant une forme \textit{qu-}


\end{enumerate}
  [Warning: Image ignored] % Unhandled or unsupported graphics:
%\includegraphics[width=4.1063in,height=0.8543in,width=\textwidth]{fe443409cd384d3fb0f6390ffd77f513-img97.svm}
 

Les prépositions fonctionnant comme tête dans l'introducteur du VRA (p.ex. roum. \textit{dintre, printre, între} et fr. \textit{parmi}) ont les propriétés lexicales suivantes : elles ont une structure argumentale qui contient deux éléments, un argument interne réalisé comme complément (COMPS) de la préposition et un argument externe (XARG) qui n'est pas réalisé comme dépendant de la préposition. La préposition sélectionne un syntagme qui contient une ancre coïndicée avec son argument externe. La préposition doit aussi introduire une relation ensemble/sous-partie (\textit{sum-subpart-rel}) entre ses deux arguments : l'argument interne dénote l'ensemble (SUM), alors que l'argument externe dénote la sous-partie (SUBPART) de l'ensemble. Les prépositions peuvent différer selon qu'elles permettent les deux types d'interprétation discutés en 5.3.2.2, à savoir une interprétation exemplifiante et une interprétatation partitionnante, ou bien uniquement une de ces deux interprétations. Ainsi, la préposition \textit{dintre} en roumain est compatible uniquement avec une interprétation partitionnante, alors que les autres prépositions (roum. \textit{printre} et \textit{între} et fr. \textit{parmi}) introduisent une relation sous-spécifiée (c.-à-d. elles sont compatibles avec les deux interprétations). 


\begin{enumerate}
\item Entrée lexicale pour la préposition \textit{dintre} en roumain


\end{enumerate}
  [Warning: Image ignored] % Unhandled or unsupported graphics:
%\includegraphics[width=4.7937in,height=2.8752in,width=\textwidth]{fe443409cd384d3fb0f6390ffd77f513-img98.svm}
 


\begin{enumerate}
\item Entrée lexicale pour la préposition \textit{printre} en roumain (de même pour \textit{între})  


\end{enumerate}
  [Warning: Image ignored] % Unhandled or unsupported graphics:
%\includegraphics[width=4.789in,height=2.8555in,width=\textwidth]{fe443409cd384d3fb0f6390ffd77f513-img99.svm}
 


\begin{enumerate}
\item Entrée lexicale pour la préposition \textit{parmi} en français


\end{enumerate}
  [Warning: Image ignored] % Unhandled or unsupported graphics:
%\includegraphics[width=4.7516in,height=2.8547in,width=\textwidth]{fe443409cd384d3fb0f6390ffd77f513-img100.svm}
 

Une exemplification de cette analyse est donnée en \REF{ex:5:158}. La préposition \textit{parmi} en français introduit une relation partitive sous-spécifiée (\textit{sum-subpart-rel}), c.-à-d. elle est compatible et avec l'interprétation exemplifiante et avec l'interprétation partitionnante. Sa structure argumentale contient un argument externe (qui dénote la sous-partie et qui correspond donc au syntagme \textit{Marie}) et un complément (qui dénote l'ensemble et qui correspond donc au pronom relatif \textit{lesquelles}).  Cette représentation exemplifie aussi le résultat de l'application d'une des deux contraintes de localité discutées plus haut dans la section \ref{sec:5.5.4}, à savoir la relation à distance entre l'introducteur et l'élément distingué : l'élément distingué dans le corps du VRA est rendu accessible via le trait ANCHORS.


\begin{enumerate}
\item \label{bkm:Ref299031436}Relation à distance entre l'introducteur et l'élément distingué


\end{enumerate}
  [Warning: Image ignored] % Unhandled or unsupported graphics:
%\includegraphics[width=6.4209in,height=3.6854in,width=\textwidth]{fe443409cd384d3fb0f6390ffd77f513-img101.svm}
 

En \REF{ex:5:159}, on a une représentation simplifiée de la construction dans son ensemble, qui nous permet d'observer aussi le résultat de l'application de la deuxième contrainte de localité discutée dans la section \ref{sec:5.5.4}, à savoir la relation à distance entre le VRA et son antécédent dans la phrase hôte : l'antécédent (ici, le syntagme nominal \textit{plusieurs personnes}, qui est coïndicé avec le pronom relatif \textit{lesquelles})  est rendu accessible au VRA via le trait ANCHORS. 


\begin{enumerate}
\item \label{bkm:Ref299032355}Relation à distance entre le VRA et son antécédent


\end{enumerate}
  [Warning: Image ignored] % Unhandled or unsupported graphics:
%\includegraphics[width=6.4957in,height=4.6547in,width=\textwidth]{fe443409cd384d3fb0f6390ffd77f513-img102.svm}
 

En ce qui concerne le deuxième sous-type de VRA (\textit{DONT-VRA-ph}), il aura comme seule contrainte le fait que le n{\oe}ud racine de la construction possède un trait MRKG dont la valeur est \textit{dont}.


\begin{enumerate}
\item La construction VRA avec un introducteur contenant \textit{dont} en français  


\end{enumerate}
  [Warning: Image ignored] % Unhandled or unsupported graphics:
%\includegraphics[width=2.0193in,height=0.4583in,width=\textwidth]{fe443409cd384d3fb0f6390ffd77f513-img103.svm}
 

On analyse le français \textit{dont} comme un marqueur qui n'a pas de structure argumentale, introduisant une relation sémantique sous-spécifiée. \textit{Dont} sélectionne un syntagme qui (i) contient une ancre pour l'argument dénotant la sous-partie et (ii) sélectionne un syntagme contenant une ancre pour l'argument dénotant l'ensemble. L'entrée lexicale pour la forme \textit{dont} dans les VRA est donnée en \REF{ex:5:161}. Ainsi, [ANCHORS {\textless}..., [IND 4], ...{\textgreater}] nous permet d'avoir accès à l'antécédent, qui désigne une entité plurielle [SUM 4] dans la phrase hôte, alors que [ANCHORS {\textless}..., [IND 3], ...{\textgreater}] nous permet l'accès à l'élément distingué, qui dénote la sous-partie [SUBPART 3] dans le corps du VRA. Finalement, l'entrée lexicale en \REF{ex:5:161} contraint la catégorie de la tête du VRA : le corps d'un VRA introduit par \textit{dont} ne  peut pas être une phrase finie.


\begin{enumerate}
\item \label{bkm:Ref299034743}Entrée lexicale pour \textit{dont} dans les VRA 


\end{enumerate}
  [Warning: Image ignored] % Unhandled or unsupported graphics:
%\includegraphics[width=5.7744in,height=2.3728in,width=\textwidth]{fe443409cd384d3fb0f6390ffd77f513-img104.svm}
 

Les sous-types de VRA doivent inclure des contraintes sur la syntaxe de la tête. Par exemple, les VRA de type \textit{qu-} en français doivent contraindre un des syntagmes du cluster à être un syntagme nominal et ils doivent lier l'indice de ce syntagme nominal au trait sous-partie de la relation ensemble/sous-partie exprimée par l'introducteur.

L'entrée proposée en \REF{ex:5:161} pour l'introducteur \textit{dont} dans les VRA est différente de celle postulée pour le complémenteur \textit{dont} dans les relatives verbales en français (Godard (1988, 1989), Abeillé, Godard \& \citet{Sag2003}, Abeillé \& Godard (2006, 2007)). Le complémenteur\footnote{\citet{Sag1997} définit un super-type \textit{verbal} qui a comme sous-types le verbe et les complémenteurs.} \textit{dont} est défini en \REF{ex:5:162}.


\begin{enumerate}
\item \label{bkm:Ref299110275}Entrée lexicale pour le complémenteur \textit{dont} dans les relatives ordinaires


\end{enumerate}
  [Warning: Image ignored] % Unhandled or unsupported graphics:
%\includegraphics[width=3.3098in,height=1.8118in,width=\textwidth]{fe443409cd384d3fb0f6390ffd77f513-img105.svm}
 

Contrairement à l'introducteur \textit{dont} dans les VRA, le complémenteur \textit{dont} est une tête syntaxique\footnote{Plus précisément, on peut considérer que le complémenteur \textit{dont} est une tête {\guillemotleft} faible {\guillemotright}, c.-à-d. une tête syntaxique qui partage sa valeur de HEAD avec son complément (cf. \citet{Tseng2002}).}, qui prend comme complément une phrase tensée\footnote{Pour les phrases verbales en français, on peut faire la distinction entre phrase tensée et phrase finie (GGF \textit{en prép.}). Une phrase tensée contient nécessairement une forme verbale ayant une marque de temps (indicatif, subjonctif, etc.). En revanche, l'opposition fini/non-fini est pertinente en français à cause du placement de la négation : les participes présents se distinguent des infinitifs, cf. \textit{ne venant pas} vs. \textit{*ne venir pas}. On peut ainsi considérer l'infinitif comme une forme verbale non-finie et non-tensée et le participe présent comme une forme finie et non-tensée. Le complémenteur \textit{dont} n'est compatible qu'avec une phrase tensée en français.},~auquel manque un constituant prépositionnel. En français standard, ce gap est un syntagme prépositionnel en \textit{de}. Le trait BIND sert à lier la valeur du trait SLASH du complément phrastique. Comme la valeur est un syntagme prépositionnel en \textit{de}, le gap dans une relative standard en \textit{dont} est toujours un syntagme prépositionnel.

Je donne en \REF{ex:5:163} un exemple de relative verbale introduite par le complémenteur \textit{dont}.  Le verbe \textit{parle} possède un trait SLASH non-vide qui indique qu'un argument est manquant. Le trait SLASH est partagé par les catégories dominant le verbe jusqu'à la relative elle-même, qui le {\guillemotleft}~vide~{\guillemotright}. 


\begin{enumerate}
\item \label{bkm:Ref299039682}Arbre avec le complémenteur \textit{dont} dans les relatives verbales


\end{enumerate}
{   [Warning: Image ignored] % Unhandled or unsupported graphics:
%\includegraphics[width=4.8417in,height=3.7839in,width=\textwidth]{fe443409cd384d3fb0f6390ffd77f513-img106.svm}
} 

En revanche, un VRA en \textit{dont} n'a pas de trait SLASH, car il n'y a pas de constituant manquant. Je donne en \REF{ex:5:164} un exemple de VRA en \textit{dont}. Le point commun entre les deux relatives est le fait qu'elles sont des ajouts, qui modifient un constituant de la phrase hôte.


\begin{enumerate}
\item \label{bkm:Ref299041282}Arbre avec l'introducteur \textit{dont} dans les VRA 


\end{enumerate}
  [Warning: Image ignored] % Unhandled or unsupported graphics:
%\includegraphics[width=6.6665in,height=4.1425in,width=\textwidth]{fe443409cd384d3fb0f6390ffd77f513-img107.svm}
 

\subsection{Conclusion}
Les VRA se caractérisent par les propriétés suivantes : (i) ce sont des ajouts incidents ; (ii) ils ont une sémantique non-restrictive, bien qu'ils fassent partie du contenu asserté de l'hôte, et (iii) ils ont une double sémantique partitive : au niveau des individus et au niveau des éventualités. En fonction du statut de l'élément distingué par rapport à l'antécédent, cette relation partitive se prête à deux interprétations : une interprétation exemplifiante si l'élément distingué constitue un élément de l'ensemble dénoté par l'antécédent, ou bien une interprétation partitionnante si l'élément distingué constitue un sous-ensemble de l'ensemble dénoté par l'antécédent. 

Les VRA en roumain et en français ont été décrits comme des phrases relatives présentant une ellipse du verbe, en vertu de leur ressemblance avec les phrases relatives partitives non-restrictives. Bien qu'une analyse en termes de reconstruction syntaxique puisse rendre compte de certaines propriétés des VRA, on a présenté des arguments empiriques (provenant des deux langues) contre une telle approche. 

Une analyse des VRA comme des ajouts fragmentaires rend compte de l'ensemble des propriétés des VRA. Bien que cette nouvelle approche puisse sembler techniquement complexe, les parties individuelles qui la composent sont justifiées de manière indépendante pour d'autres constructions dans la grammaire des deux langues. 



{\bfseries
Conclusion générale
}

Dans cette thèse, j'ai développé un fragment de grammaire qui rend compte des propriétés majeures de deux constructions elliptiques (auxquelles {\guillemotleft}~manque~{\guillemotright} la tête verbale), appelées respectivement gapping \REF{ex:5:1}a et ajouts relatifs averbaux (abrégés VRA) \REF{ex:5:1}b.


\begin{enumerate}
\item \label{bkm:Ref307390740}a  Jean aime les pommes [\textbf{et} Marie les bananes]. 


\end{enumerate}
  b  Plusieurs personnes sont venues cette semaine, [\textbf{dont} Marie (hier)].

Au terme de cette thèse, je veux insister sur les aspects généraux engendrés par l'étude des phrases elliptiques, en laissant de côté les aspects discutés dans les deux premiers chapitres.  

{\bfseries
Ellipse et reconstruction}

La proposition centrale défendue~ici est que les phrases trouées dans les coordinations à gapping et les subordonnées relatives de type VRA ne peuvent pas être alignées sur le fonctionnement d'une phrase verbale ordinaire. Leurs propriétés syntaxiques et sémantiques montrent qu'elles ne sont pas dérivées à partir d'une phrase complète. Une analyse en termes de reconstruction syntaxique est donc inadéquate. Par conséquent, la phrase elliptique dans les deux types de structures mentionnés comporte un mode d'organisation syntaxique spécifique et doit avoir un statut indépendant dans la grammaire, à savoir le statut d'une \textit{phrase fragmentaire}, c.-à-d. une unité syntaxique qui a un contenu de type message, mais dont la syntaxe est incomplète. Cette thèse apporte de nouveaux arguments en faveur d'une reconstruction plutôt sémantique avec des contraintes de parallélisme, cf. Ginzburg \& \citet{Sag2000}, Culicover \& \citet{Jackendoff2005}.

Crucialement, la sémantique complète peut être obtenue à partir d'une syntaxe incomplète, en exploitant la notion de \textit{fragment}, comme l'avaient proposé Ginzburg \& \citet{Sag2000} pour les questions et les réponses courtes dans le dialogue en anglais. On se donne donc en syntaxe la notion de \textit{fragment} conçu comme une construction à laquelle sont associées des conditions de bonne formation syntaxiques et interprétatives. Cependant, contrairement à Ginzburg \& \citet{Sag2000} qui analysent les fragments phrastiques comme ayant l'ensemble des propriétés d'une phrase finie (cf. la catégorie VERBAL), j'ai choisi de représenter les fragments comme étant construits à partir d'un \textit{cluster}, notion reprise de Mouret (2006, 2007) et requise de manière indépendante pour les coordinations de séquences dans la portée syntaxique d'un prédicat. Le syntagme de type cluster réunit tous les éléments résiduels d'une phrase elliptique et rend accessibles leurs propriétés syntaxiques et sémantiques au niveau de la construction. En permettant au cluster de comporter un seul constituant immédiat ou plus, on peut obtenir une analyse uniforme des coordinations à gapping et des VRA dont le corps est composé d'un ou plusieurs constituants immédiats. Le fragment hérite la catégorie sous-spécifiée du cluster, ce qui lui permet de se combiner avec des foncteurs sélectionnant des catégories non-finies (p.ex. \textit{ainsi que} et \textit{non pas} en français).\footnote{Ces deux notions sont compatibles avec plusieurs traitements, comme on a pu le voir dans les chapitres 4 et 5: dans le chapitre 4, j'ai tout simplement utilisé une version constructionnelle de HPSG, alors que dans le chapitre 5, j'ai utilisé, en plus, le langage \textit{Minimal Recursion Semantics}.}  

Une fois les deux notions de \textit{fragment} et \textit{cluster} introduites dans la grammaire, on peut ajouter les contraintes spécifiques à chaque construction. Ainsi, une coordination à gapping se distingue des autres constructions elliptiques par le fait que la phrase trouée doit suivre la phrase source (en roumain et en français)~et par le fait que la relation discursive qui s'établit entre les conjoints est toujours une relation symétrique. Quant aux propriétés spécifiques des VRA, on note la présence d'un introducteur avec une sémantique partitive (c.-à-d. il impose une relation de type ensemble/sous-partie).

La description et l'analyse des deux constructions étudiées dans cette thèse montrent que, de manière générale, on peut envisager une grammaire syntagmatique simple, surfaciste, sans éléments vides, sans effacement, sans mouvement et sans postuler nécessairement d'homomorphisme syntaxe-sémantique.

{\bfseries
Ellipse et parallélisme}

Un des arguments majeurs qu'on mentionne habituellement en faveur d'une reconstruction syntaxique dans les constructions elliptiques est la présence des effets de~{\guillemotleft}~connectivité~{\guillemotright} discutés dans le chapitre 3, section \ref{sec:3.5.1.1}, c.-à-d. un parallélisme structural entre la phrase elliptique et la phrase source, en ce qui concerne les propriétés morpho-syntaxiques des éléments résiduels. 

Les résultats de cette thèse montrent que le parallélisme structural, tel qu'il est discuté dans les travaux sur l'ellipse, est moins strict que ce que l'on pense (\textit{contra} Culicover \& \citet{Jackendoff2005}). Ainsi, pour les constructions à gapping, on a vu que le parallélisme syntaxique n'est pas strict en ce qui concerne la catégorie grammaticale, le nombre de dépendants réalisés, ainsi que l'ordre dans lequel apparaissent les éléments résiduels par rapport à leurs corrélats dans la phrase source. Ce parallélisme syntaxique {\guillemotleft}~relâché~{\guillemotright} exige simplement que les éléments résiduels remplissent les conditions de sélection du prédicat antécédent dans la phrase source (cf. la généralisation de Wasow qui gère les coordinations de termes dissemblables). De même, dans les VRA, on observe des asymétries en ce qui concerne le marquage prépositionnel (et casuel en roumain) : ainsi, en français, de manière générale, l'élément distingué dans le corps du VRA ne peut pas recevoir le marquage prépositionnel de son légitimeur dans la phrase hôte ; en roumain aussi, l'élément distingué dans les VRA introduits par \textit{dintre care} `parmi lesquel(le)s' ne reçoit généralement pas de marque casuelle ou prépositionnelle, par rapport à son légitimeur.

En revanche, on doit accorder plus d'attention aux effets de parallélisme en plan sémantique (et discursif pour le gapping). De manière générale, ces constructions elliptiques mettent en jeu un parallélisme sémantique fort. Dans les constructions à gapping, il doit y avoir au moins deux contrastes sémantiques entre les éléments résiduels et les corrélats (c.-à-d. deux paires contrastives). Dans les VRA, la relation sémantique qui s'établit entre l'élément distingué dans le corps du VRA et son légitimeur dans la phrase hôte est toujours une relation partitive (c.-à-d. le légitimeur dans la phrase hôte doit être une entité fractionable exprimant une somme dont~les sous-parties soient accessibles, alors que l'élément distingué dans le VRA doit être interprété comme une sous-partie de l'entité fractionable). 

De plus, pour les coordinations à gapping, un parallélisme fort est observé aussi en plan discursif, le gapping privilégiant les relations symétriques de parallélisme et de contraste.

{\bfseries
Perspectives}

Je commence par un petit point technique. Cette thèse en l'état demande une homogénéïsation des analyses HPSG proposées dans les deux derniers chapitres. En particulier, je devrais préciser l'analyse du gapping avec le langage \textit{Minimal Recursion Semantics} utilisé pour les VRA.

Un deuxième point que je veux mentionner concerne la description du gapping. Afin d'avoir une analyse complète des coordinations à gapping, le travail de cette thèse devrait être suivie par une étude prosodique. On a vu que le parallélisme est strict surtout au niveau sémantique et discursif. Il reste à vérifier si l'intonation dans le gapping est plutôt sensible aux aspects sémantiques et discursifs et moins aux aspects syntaxiques, ce qui invaliderait l'hypothèse de Féry \& \citet{Hartmann2005}. Toujours pour les coordinations à gapping, on devrait vérifier de plus près, dans les deux langues, les différents facteurs qui influencent les préférences des locuteurs dans l'acceptabilité des exemples, et cela peut être mieux observé en faisant des études de corpus et en utilisant des méthodes expérimentales.

Une troisième piste de recherche concerne la parenté qui pourrait être établie entre les deux constructions étudiées dans cette thèse et d'autres constructions elliptiques. A plusieurs reprises, j'ai émis l'hypothèse selon laquelle certaines constructions elliptiques semblent se prêter à un même type d'analyse que celle proposée pour le gapping et les VRA. 

Un type d'ellipse proche du gapping est le stripping et en particulier les ellipses polaires, dans lesquelles l'élément résiduel est accompagné d'un adverbe polaire comme \textit{aussi} \REF{ex:5:2}a et \textit{non plus} \REF{ex:5:2}b en français. Comme je l'ai déjà précisé dans le chapitre 4, certains auteurs (Hankamer \& \citet{Sag1976}, \citet{Gardent1991}, \citet{Lobeck1995}, \citet{Hartmann2000}, \citet{Toosarvandani2011}, etc.) analysent ces exemples comme un sous-type de gapping. Une description détaillée doit être faite pour voir si l'on peut trouver une analyse uniforme pour le gapping étudié dans cette thèse et les ellipses polaires.


\begin{enumerate}
\item \label{bkm:Ref307424453}a  Jean viendra à la fête [\textbf{et} Marie aussi]. 


\end{enumerate}
  b  Jean n'est pas venu à la fête [\textbf{et} Marie non plus].

Les structures comparatives \REF{ex:5:3} constituent un autre type d'ellipse qui permet des séquences qui ressemblent au gapping dans la coordination (cf. Zribi-\citet{Hertz1986}, Culicover \& \citet{Jackendoff2005}, Amsili \& \citet{Desmets2008}). Toujours dans le chapitre 4, j'ai mentionné quelques éléments qui suggèrent la souplesse des contraintes sur les structures comparatives, par rapport à celles qui sont en jeu dans une structure coordonnée. Une étude détaillée reste à faire pour voir si l'on peut envisager une analyse uniforme.


\begin{enumerate}
\item \label{bkm:Ref307424498}a  Jean est autant doué en bricolage [\textbf{que} Marie en décoration]. 


\end{enumerate}
  b  Il s'ennuie chez lui, [\textbf{comme} moi au boulot].    (Amsili \& \citet{Desmets2008})

Enfin, en dehors des VRA étudiés dans le chapitre 5, il y a d'autres subordonnées elliptiques ayant la fonction d'ajout et permettant des séquences à deux éléments résiduels :~les ajouts additifs \REF{ex:5:4}a, les ajouts exceptifs \REF{ex:5:4}b ou encore les ajouts circonstanciels \REF{ex:5:4}c. Là encore, une étude reste à faire pour chaque construction, pour voir si l'analyse proposée dans cette thèse peut être étendue.


\begin{enumerate}
\item \label{bkm:Ref307424511}a  Tout le monde a apporté quelque chose, [\textbf{y compris} Marie un gâteau]. 


\end{enumerate}
  b  Personne n'a apporté quoi que ce soit, [\textbf{sauf} Marie un gâteau].

  c  [\textbf{Bien que} pour la première fois à l'étranger], Marie s'est très bien débrouillée.

Contrairement à ce que l'on peut penser, j'aimerais préciser que les résultats de cette thèse ne remettent pas en cause à eux seuls la nécessité d'un mécanisme de reconstruction syntaxique dans la grammaire. Les mises en facteur à droite (abrégées RNR), par exemple, semblent mettre en jeu un mécanisme d'ellipse syntaxique (cf. Abeillé \& \citet{Mouret2010}). Comme je l'ai déjà mentionné dans la conclusion du chapitre 3, je considère que dans une grammaire de l'ellipse les deux solutions, à savoir la reconstruction syntaxique et la reconstruction à l'interface syntaxe-sémantique, doivent être disponibles, afin de rendre compte des propriétés des différentes constructions elliptiques.


