%!TEX root = main.tex
\chapter{Kontra Vollständigkeit: Ellipse} \label{chap-ellipse}

Die zweite empirische Herausforderung für eine enge konzeptuelle Verzahnung von Syntax und Valenz ist die unvollständige Realisierung des Valenzrahmens, besser bekannt als \isi{Ellipse}. In Kapitel~\ref{sec-einfuehrung} wurde das anhand von Satz~\ref{ex-ellipse-einf-a} veranschaulicht, der je nach Diskurskontext in unterschiedlicher Weise gekürzt werden kann: 

  \ex. \label{ex-ellipse-einf}
  \a. \label{ex-ellipse-einf-a}Peter repariert heute den Kühlschrank. 
  \b. Peter repariert heute \sout{den Kühlschrank}.
  \c. \sout{Peter} Repariert heute \sout{den Kühlschrank}.
  \d. Peter \sout{repariert} den Kühlschrank.
  \e. Peter \sout{repariert den Kühlschrank}.
  \f. \sout{Peter repariert} Den Kühlschrank.
  \f. \sout{Peter repariert} Heute \sout{den Kühlschrank}.
  %\f. \ldots

In diesem Kapitel sollen die valenztheoretischen Bedingungen solcher Weglassungen\is{Weglassung} untersucht werden. Es geht also um die Frage: Welche Bestandteile eines Valenzrahmens\is{Valenzrahmen} können im günstigsten Fall, d.\,h.\ bei Wahl eines jeweils passenden Diskurskontexts, weggelassen werden?  

Neben diesem an der Valenzrahmenrealisierung orientierten Ellipsenbegriff findet man jedoch in der Literatur auch einen anderen Ellipsenbegriff, der sich an der Semantik orientiert. Der erste Abschnitt stellt daher zunächst diese beiden Konzepte unter den Schlagwörtern Weglassung und Mitverstehen\is{Mitverstehen} einander gegenüber. In Abschnitt \ref{sec-ellipse-taxonomie} folgt anschlie\ss end die Herleitung einer Taxonomie der Ellipsenphänomene anhand des unmittelbaren syntaktischen Kontextes und der Position des Antezedens, nämlich Koordinationsellipsen, Adjazenzellipsen und freie Ellipsen. Diese Klassen werden schließlich in den restlichen Abschnitten des Kapitels (\ref{sec-koordinationsellipsen}--\ref{sec-situative-ellipsen}) weiter spezifiziert und empirisch unterfüttert. Besonderen Stellenwert erhält dabei die Koordinationsellipse\is{Ellipse!Koordinations-}, denn sie wird uns als Grundlage für die Herausarbeitung der Gesetzmä\ss igkeiten der sogenannten Vorwärtsellipse\is{Ellipse!Vorwärts-}, d.\,h.\ im Wesentlichen Gapping\is{Ellipse!Gapping} und Vorfeldellipse\is{Ellipse!Vorfeld-}, dienen. Dafür ist allerdings eine exakte Unterscheidung zwischen Konstituentenkoordination\is{Koordination!Konstituenten-} und Satzkoordination\is{Koordination!Satz-} nötig, für die ich die sogenannte $\kappa$-Reduzierbarkeit\is{k-Reduzierbarkeit@$\kappa$-Reduzierbarkeit} heranziehen werde. Diese Unterscheidung wird es auch erlauben, Vorwärtsellipsen vom sogenannten Right-Node-Raising\is{Koordination!Right-Node-Raising} (RNR) abzugrenzen, das auf Koordinationen beschränkt scheint und bei dem sich nicht zwingend eine Ellipse feststellen lässt. Daher wird RNR (ebenso wie VP-Ellipsen\is{Ellipse!VP-} (VPE) und Sluicing\is{Ellipse!Sluicing}, aber hier aus anderen Gründen) bei der Ellipsenmodellierung in Kapitel~\ref{sec-ellipsenanalyse} kein Thema sein.       

\section{Weglassung versus Mitverstehen}\is{Weglassung|(}\is{Mitverstehen|(}

Diese Arbeit geht von einem valenztheoretisch fundierten Ellipsenbegriff aus: Eine \isi{Ellipse} ist die \textsc{Weglassung} des Valenzträgers\is{Valenzträger} oder seiner obligatorischen Ergänzungen\is{Ergänzung!obligatorische}, d.\,h.\ von eigentlich obligatorischen Bestandteilen eines realisierten Valenzrahmens. Die Weglassung fakultativer Ergänzungen\is{Ergänzung!fakultative} zählt demnach nicht als Ellipse. Um auch die Weglassung von funktionalen Elementen wie \emph{dass}-Komplementierer\is{Komplementierer} unter den Ellipse-Terminus zu fassen, bietet sich eine Erweiterung der folgenden Art an: Eine Ellipse ist die Weglassung von solchen Teilen der objektsprachlichen Kette, deren Vorhandensein gemäß allgemeiner syntaktischer Regeln (im Zusammenspiel mit lexikalischen Eigenschaften) eigentlich notwendig sind, die also nicht weggelassen werden dürften. Ellipsen stellen dann Spezialfälle der syntaktischen Lizenzierung dar.

Davon abzugrenzen ist ein zweiter, die (Diskurs-)Semantik einbeziehender Ellipsenbegriff, der sich in meinen Augen als \textsc{Mitverstehen} von nicht-realisiertem Material paraphrasieren lässt. Er kommt beispielsweise in der Charakterisierung von \citet[1]{Winkler:Schwabe:03} zum Ausdruck, "`ellipsis lacks form and still has meaning"', oder letztens bei \citet[1]{Aelbrecht:10}: "`Ellipsis, however, is the omission of elements that are inferable from the context and thus constitutes a mismatch between sound and meaning."'\footnote{Dieser semantisch induzierte Ellipsenbegriff findet sich bei einer Vielzahl von Autoren, z.\,B.\  bei \citet[175]{Lyons:68}, bei \cite{Kindt:85} im Konzept der Expandierbarkeit eines Satzes\is{Satz} unter Beibehaltung seiner Bedeutung und auch bei \citet[52]{Klein:81}: "`By regular ellipsis, I refer to the phenomenon that, under certain complex conditions, the meaning of an utterance is systematically completed by the meaning of expressions which are not uttered but whose meaning is derived from the context"'.} Hier wird nicht die Valenzverletzung problematisiert, sondern vielmehr der nicht-kompositionale\is{Kompositionalität} Charakter elliptischer Konstruktionen, in denen das Ellipsenantezedenz Bedeutung beiträgt. Die E/A-""Unter"-scheidung\is{E/A-Klassifikation}, geschweige denn die Unterscheidung zwischen obligatorischen und fakultativen Ergänzungen\is{o/f-Klassifikation}, ist dabei zunächst irrelevant: Auch das Mitverstehen von Angaben\is{Angabe} wird als elliptisch deklariert. Dies hat z.\,B.\  zur Folge, dass das Mitverstehen von {\it gestern} und {\it bei Peter} im zweiten Satz von \ref{ex-mitverstehen-1} als Ellipse gilt: 

\ex. \label{ex-mitverstehen-1}Susi war gestern bei Peter. \\ Sie schlief \sout{gestern} \sout{bei Peter} auf dem Sofa ein.

Ich kann mir nicht vorstellen, dass die erwähnten Autoren hier tatsächlich immer das Vorliegen einer Ellipse diagnostizieren würden. Träfe dies nämlich zu, dann würde in letzter Konsequenz nicht nur die Unterscheidung zwischen Denotation und Kontextbedeutung verloren gehen, sondern es würde auch dazu führen, dass selbst einfachste Sätze unendlich viele und beliebig umfangreiche syntaktische Analysen erhalten. Im Zuge einer Fundierung des Ellipsenbegriffs wäre es also unumgänglich, genauer zu bestimmen, was beim Mitverstehen wesentlich sein soll und was nicht.

Statt aber einen recht vagen Begriff irgendwie durch Einschränkungen unter Kontrolle zu bekommen, werde ich im Folgenden den valenztheoretisch fundierten Ellipsenbegriff, Ellipse als Weglassung, verwenden.
\is{Weglassung|)}\is{Mitverstehen|)}



\section{Taxonomie der Ellipsenphänomene} \label{sec-ellipse-taxonomie}

Bei der Klassifikation von Ellipsen hat sich ein Zweiteilung des Phänomenbereichs etabliert, welche den einbettenden Kontext als Klassifizierungsgrundlage heranzieht.\footnote{Bemerkenswert daran ist, dass für die Klassifikation von Ellipsen auf den Kontext Bezug genommen wird, aber nicht etwa auf die syntaktische Struktur der Ellipsen. Damit enthält die Klassifikation keine natürlichen Klassen hinsichtlich der elliptischen Strukturen selber.} Dabei ist die Unterscheidung zwischen \textsc{sprachlichem Kontext}\is{Kontext} und \textsc{nichtsprachlichem Kontext} wesentlich: Unter nichtsprachlichem Kontext versteht etwa \citet[766]{Klein:93} Weltwissen und situative Faktoren, unter sprachlichem Kontext dagegen unmittelbaren syntaktischen (d.\,h.\ "`expliziten"') Kontext. Das ellidierte Material kann auf diese beiden Kontexte referieren, d.\,h.\ es existiert ein Kontextkorrelat des ellidierten Materials. Handelt es sich bei diesem Kontextkorrelat um ein Teil des sprachlichen Kontexts, dann spricht man von einem \isi{Antezedens}. Darauf bezogen können schlie\ss lich die folgenden beiden Globalklassen unterschieden werden: \textsc{Ellipsen mit Antezedens}\is{Ellipse!mit Antezedens} und \textsc{Ellipsen ohne Antezedens}\is{Ellipse!ohne Antezedens}.\footnote{Wiewohl diese Klassifizierung üblich ist, wird die Terminologie variiert: Kontextgestützte vs. situative Ellipsen \citep{Schwabe:94}; umgebungsabhängige vs.\ umgebungsunabhängige Ellipsen \citep{Kindt:85}; kontextkontrollierte vs.\ kontextabhängige Ellipsen \citep{Klein:93}. Ähnliches findet man bei der Unterscheidung von tiefen Anaphern und Oberflächenanaphern ("`deep and surface anaphora"') von \cite{Hankamer:Sag:76}.} Des weiteren werden Ellipsen mit Antezedens in \textsc{Koordinationsellipsen}\is{Ellipse!Koordinations-} und \textsc{Adjazenzellipsen}\is{Ellipse!Adjazenz-} eingeteilt. In Arbeiten wie \citet[768]{Klein:93} liegt dieser Subklassifizierung kriterial allerdings der Satzbegriff\is{Satz} zugrunde, indem Ellipse und Antezedens bei Koordinationsellipsen "`innerhalb eines Satzes"' stehen sollen und bei Adjazenzellipsen in verschiedenen Sätzen ("`selbständige aber eng zusammengehörige Äu\ss erungen"'). Nun ist der Satzbegriff insbesondere bei Satzkomplexen notorisch unklar,\footnote{Siehe B.\,L.\,\cite{Mueller:85} und \citet[91ff]{Matthews:93} für eine Übersicht. \citet[86ff]{Ortner:87} geht spezifischer auf das Verhältnis von Satzbegriff und Ellipsenbegriff in der germanistischen Forschungsgeschichte ein.} und sein Einsatz hat bei \cite{Klein:93} wohl in erster Linie den Zweck, dieses Phänomen überhaupt in der Rubrik (Satz"~)""Syntax abhandeln zu können. Ich werde dem in dieser Arbeit eine formale Definition der Koordinationsellipse entgegenstellen, die von einer Kategorisierung der Koordination als intra- oder supersententiell weitgehend unabhängig ist. 

Was die Verteilung der Forschungsaktivitäten betrifft, fällt eine beträchtliche Schieflage auf: Während den Ellipsen ohne Antezedens\is{Ellipse!ohne Antezedens} vergleichsweise wenig Aufmerksamkeit entgegengebracht wird, entfachen Ellipsen mit Antezedens\is{Ellipse!mit Antezedens}, insbesondere Koordinationsellipsen, einen regen Forschungsdiskurs seit den Anfängen der generativen Grammatik bis in unsere Tage. Dies hängt wohl mit der Flüchtigkeit nichtsprachlicher Kontexte zusammen, und mit dem natürlichen Drang der Syntaktiker, die Ellipse-Antezedens-Relation innerhalb eines vertrauten Satzmodells zu behandeln. In dieser Hinsicht stellt die Koordinationsellipse sicherlich den ergiebigsten Teilbereich dar.\footnote{Symptomatisch für diese Einstellung ist folgende Bemerkung aus \cite{Klein:93}: "`Dem Linguisten, der sich für Aufbau und Bedeutung der Ellipsen und ihr Verhältnis zu \glq vollständigen\grq~ Äu\ss erungen interessiert, kann es nun nicht darum gehen, all diese Fälle zu sichten und ihre Verwendungsbedingungen anzugeben. Vielmehr mu\ss\ versucht werden, jene Regeln anzugeben, nach denen sich solche Äu\ss erungen aufbauen und nach denen sich die Bedeutung des gesamten Ausdrucks ergibt. Dies lä\ss t sich am ehesten für die kontextkontrollierten Fälle der Koordinationsellipse und der Adjazenzellipse leisten."' (S.\,768)} Auch ich werde im Analyseteil in Kapitel~\ref{sec-ellipsenanalyse} und~\ref{ch-ohne-valenz} vorrangig auf Koordinationsellipsen eingehen.  



\section{Koordinationsellipsen} \label{sec-koordinationsellipsen}\is{Ellipse!Koordinations-}

In diesem Abschnitt erläutere ich die Begriffe und Methoden, die nötig sind, um Umfang und Art der koordinationsgebundenden Ellipsen genau zu bestimmen. In den nachfolgenden Abschnitten~\ref{sec-gapping} und~\ref{sec-rnr} werden dann auf dieser Grundlage die spezifischen Gesetzmäßigkeiten von Gapping und RNR behandelt.

\subsection{Koordination und Satz}

Eine \isi{Koordination} besteht aus syntaktisch weitgehend gleichwertigen Konjunkten, die durch einen oder mehrere Konjunktoren verbunden sind.  Wir beschränken uns in dieser Arbeit auf Koordinationen, die aus zwei Konjunkten und dem Konjunktor {\it und} bestehen. Es gilt also folgende Definition:
\begin{definition}[Koordination] 
Seien $\kappa_1$, $\kappa_2$ nicht-leere Konstituentenketten und\linebreak \glq{\it und}\grq\ ein Konjunktor, dann ist \glq$\kappa_1$ {\it und} $\kappa_2$\!\grq~ eine Koordination, wobei gilt, dass sich $\kappa_1$ unmittelbar rechtsadjazent vom Konjunktor befindet und $\kappa_2$ unmittelbar linksadjazent vom Konjunktor (Adjazenzbedingung). 
\end{definition}
Die Adjazenzbedingung ist ein wichtiger Bestandteil dieser Definition, denn man könnte, so wie \citet[546ff]{Hudson:76} unter dem Schlagwort "`conjunct postposing"', alternativ davon ausgehen, dass in bestimmten Fällen $\kappa_1$ in der objektsprachlichen Kette dem Rest der Koordination {\it und} $\kappa_2$ nicht unmittelbar vorausgeht.\footnote{Vgl.\ auch die "`gespaltenen Konjunkte"' in \cite{Hoehle:83}.} Die Instanziierung von $\kappa_1$, $\kappa_2$ in einer objektsprachlichen Kette hei\ss t \textsc{$\kappa$-Instanziierung}\is{k-Instanziierung@$\kappa$-Instanziierung}. Man beachte, dass die grammatische Wohlgeformtheit einer Koordination in dieser Definition keine Rolle spielt. Sie beschreibt also auch ungrammatische Koordinationen. Die Auseinandersetzung mit syntaktischen Pa"-rallelitätsbedingungen für die Konjunkte, z.\,B.\  bezüglich ihrer syntaktischen Kategorie, wird in dieser Arbeit auf ein Minimum beschränkt, nämlich auf ihre $\kappa$-Reduzierbarkeit\is{k-Reduzierbarkeit@$\kappa$-Reduzierbarkeit}.\footnote{Einen ganz anderen Schwerpunkt legen in dieser Hinsicht z.\,B.\ \cite{Sag:etal:85}, indem sie die Parallelitätsbedingung entlang der internen syntaktischen Struktur der Konjunkte formulieren.} Dazu gleich mehr. 

Des weiteren werden in dieser Arbeit ausschlie\ss lich \isi{Koordination} auf Satzebene\is{Satz} berücksichtigt, d.\,h.\ die Konjunkte bestehen aus Satzkonstituenten oder Satzgliedern. Ignoriert wird also intranominale, intrapräpositionale und intraadjektivische Koordination. Was ist unter einem Satzglied zu verstehen? Hier geraten wir unweigerlich an den Satzbegriff, für den, wie eben erwähnt, kein kanonisches Verständnis vorliegt. Als Arbeitsdefinition verwende ich deshalb den folgenden valenztheoretisch motivierte Satzbegriff, wobei ein Dependenzfeld den Valenzträger, seine Ergänzungen und seine Angaben umfasst (siehe Definition~\ref{def-dependenzfeld}, S.\,\pageref{def-dependenzfeld}):

\ex. \label{ex-satzbegriff} {\bf Valenztheoretisch motivierter Satzbegriff:}\is{Satz} \\
Ein Satz ist der (kleinste) kontinuierliche Teil einer objektsprachlichen Kette, in dem das Dependenzfeld eines finiten Verbs realisiert wird.

Der Satz ist demnach der syntaktische Bereich, in dem die Valenzrealisierung beurteilt und die Nicht-Realisierung von Valenzrollen, also eine Ellipse, festgestellt werden kann. Entsprechend verstehe ich unter einem Satzglied das Folgende:

\ex. \label{ex-satzglied} {\bf Satzglied:}\is{Satzglied}\\
Ein Satzglied ist Bestandteil des Dependenzfelds des finiten Verbs oder Bestandteil des Dependenzfelds eines infiniten Verbs, das mit dem finiten Verb kohärent konstruiert.

Dieser Satzgliedbegriff wird auch bei der Formulierung der Ellipseneigenschaften eine wichtige Rolle spielen (siehe insbesondere Abschnitt~\ref{sec-gapping}).


\subsection{Konstituentenkoordination versus Satzkoordination} \label{sec-ellipse-kappa}

Da in dieser Arbeit die Ellipse von Satzgliedern\is{Satzglied} (oder die Nicht-Realisierung von Bestandteilen verbaler Valenzrahmen) im Vordergrund steht, ist die folgende Typisierung der Koordination zweckmä\ss ig und auch in der Forschungsliteratur üblich: Bestehen die Konjunkte einer Koordination aus Sätzen, so spricht man von einer \textsc{Satzkoordination}\is{Koordination!Satz-}; bestehen die Konjunkte einer Koordination dagegen aus Nicht-Sätzen, d.\,h.\ aus Satzgliedern, nennt man diese eine \textsc{Konstituentenkoordination}\is{Koordination!Konstituenten-}.\footnote{Statt Konstituentenkoordination liest man auch oft "`phrasale Koordination"'.} Dabei ist es nicht ausgeschlossen, dass ein und dieselbe Koordination unterschiedliche $\kappa$-Instanziierungen\is{k-Instanziierung@$\kappa$-Instanziierung} zulässt. Um die Plausibilität einer $\kappa$-Instanziierung als Satzkoordination oder als Konstituentenkoordination zu überprüfen, werde ich deren $\kappa$-Reduzierbarkeit\is{k-Reduzierbarkeit@$\kappa$-Reduzierbarkeit} zu Rate ziehen.
\begin{definition}[$\kappa$-Reduzierbarkeit] Eine Koordination $K$ mit Konjunkten $\kappa_1$ und $\kappa_2$ in einem \isi{Satz} $S$ ist $\kappa$-reduzierbar, gdw. $K$ in $S$ sowohl durch $\kappa_1$ als auch durch $\kappa_2$ ersetzt werden kann, ohne dass die daraus resultierenden Sätze $S_1$ und $S_2$ unter Ellipsenrekonstruktion ungrammatisch sind.
\end{definition}
Mit anderen Worten: Die $\kappa$-Reduzierbarkeit ist ein Kriterium für die Ähnlichkeit der Konjunkte, die sich indirekt in der Kompatibilität mit einem identischen syntaktischen Kontext zeigt. Diese der $\kappa$-Reduzierbarkeit zugrundeliegende Idee ist keinesfalls neu. \cite{Maxwell:Manning:96} erläutern sie anhand der Schemata in \ref{ex-mm-3-4} und sprechen gar von "`traditional idea"'. Die Grammatikalität von Instanzen von Schema \ref{ex-mm-3} soll die Grammatikalität von Instanzen der Schemata \ref{ex-mm-4-a} und \ref{ex-mm-4-b} implizieren, wobei A und B Konstituentenketten sind:\footnote{\cite{Maxwell:Manning:96} erwähnen auch, dass Chomskys Koordinationsregel \citep[35]{Chomsky:57} die umgekehrte Implikation ("`backward implication"') enthält. Siehe S. \pageref{sec-koordinationsellipsen-5}.}

\ex. \label{ex-mm-3-4} 
\a. \label{ex-mm-3}A $\kappa_1$ und $\kappa_2$ B  \hfill	(S)\phantom{$_1$}
\b. \label{ex-mm-4-a}A $\kappa_1$ B \hfill (S$_1$)	
\c. \label{ex-mm-4-b}A $\kappa_2$ B \hfill (S$_2$)
   
Auch die \isi{External Homogeneity Condition (EHC)} aus \citet{Hoehle:90,Hoehle:91} und die \isi{String Continuation Condition (SCC)} aus \cite{Kathol:99} zielen in diese Richtung. Auf EHC und SCC sowie auf andere verwandte Konzepte der $\kappa$-Reduzierbarkeit gehe ich am Ende des Abschnitts (S.\,\pageref{sec-koordinationsellipsen-5}) gesondert ein. 

Es ist fraglich, ob die $\kappa$-Reduzierbarkeit\is{k-Reduzierbarkeit@$\kappa$-Reduzierbarkeit} eine Mindestanforderung für Koordination darstellt, ob sich also für jeden grammatischen Satz mit einer Koordination eine $\kappa$-Instanziierung\is{k-Instanziierung@$\kappa$-Instanziierung} finden lässt, die $\kappa$-reduziert werden kann. Bestimmte Koordinationstypen, nämlich Koordination, die syntaktisch oder semantisch als pluralische Einheiten\is{Plural} auf"|tritt, und asymmetrische Koordination\is{Koordination!asymmetrische}, scheinen diesen Schluss nicht zuzulassen, wie wir gleich sehen werden. 

Eine \textsc{Koordinationsellipse}\is{Ellipse!Koordinations-} liegt genau dann vor, wenn Ellipse und \isi{Antezedens} in unterschiedlichen Konjunkten einer Koordination auf"|treten. Da ich den Ellipsenbegriff auf die unvollständige Realisierung eines Valenzrahmens einschränke und der Satzbegriff aus \ref{ex-satzbegriff} eng mit der Realisierung eines Valenzrahmens eines finiten Verbs zusammenhängt, gilt in dieser Arbeit folgendes Korollar:\footnote{Katharina Hartmann drückt dasselbe aus, wenn sie schreibt: "`Deletion is a process only allowed by the LCH"' \citep[33]{Hartmann:00}. Hinter dem Kürzel LCH (für "`Long Conjunct Hypothesis"') verbirgt sich die Annahme, dass Konjunkte aus Sätzen bestehen, die gegebenenfalls reduziert werden. Der erwähnte Tilgungsprozess ist also auf Satzkoordination beschränkt. } 
\begin{corollary}
Koordinationsellipsen können nur in Satzkoordinationen\is{Koordination!Satz-} auf"|treten,\linebreak nicht aber in Konstituentenkoordinationen\is{Koordination!Konstituenten-}.
\end{corollary}
Die Konjunkte einer Konstituentenkoordination können per definitionem keine vollständigen finiten Valenzrahmen enthalten, d.\,h.\ Teile solcher Valenzrahmen befinden sich immer au\ss erhalb der Koordination. Nimmt man dennoch die Existenz valenzbezogener Ellipsen bei einer Konstituentenkoordination an, widerspricht das nicht nur formal unserem Verständnis einer Koordinationsellipse, sondern dann muss auch jegliche Konstituentenkoordination als elliptisch gelten. Und nichts spricht dagegen, hier noch weiter zu gehen: Jegliche \isi{Ergänzung} kann demzufolge immer als elliptisch deklariert werden, allein deswegen, weil sie trivialerweise nie den ganze \isi{Valenzrahmen} darstellt. Es dürfte offensichtlich sein, dass dieser Ellipsenbegriff abwegig ist. Ellipsen bei der Valenzrealisierung sind erst dann interessant, wenn der ganze mögliche Realisierungsbereich betrachtet wird. 

Die Frage, ob \isi{Koordination} eher als Konstituentenkoordination oder als Satzkoordination modelliert werden sollte, wird in der Literatur schon seit Jahrzehnten kontrovers diskutiert, auch weil damit gravierende Entscheidungen für das Syntaxmodell einhergehen.\footnote{Siehe z.\,B.\  \citet[Chapter~1]{Oirsouw:87}, \citet[61ff]{Wilder:97} und \citet[32ff]{Hartmann:00} für ausführliche Übersichten.} Das vorgestellte Diagnoseverfahren in Form der $\kappa$-Reduzierbarkeit bildet diese Möglichkeit unterschiedlicher Sichtweisen auf die Koordination ab, indem sie mehrere parallele $\kappa$-Instanziierungen einer Koordination lizenzieren kann. Wir werden dafür gleich Beispiele sehen. Da ich jedoch in dieser Arbeit an elliptischen Strukturen interessiert bin, wird schlie\ss lich die Betrachtung auf Satzkoordinationen eingegrenzt, ohne damit jedoch irgendwelche neuen Entscheidungshilfen contra Konstituentenkoordination liefern zu wollen. 

\subsubsection*{Beispiele für $\kappa$-Instanziierungen}

Die $\kappa$-Instanziierung\is{k-Instanziierung@$\kappa$-Instanziierung} kann in vielen Fällen sowohl als Satzkoordination\is{Koordination!Satz-} als auch als Konstituentenkoordination\is{Koordination!Konstituenten-} $\kappa$-reduzierbar sein. Das Vorliegen einer Koordinationsellipse ist dann also abhängig von der Wahl der $\kappa$-Instanziierung. Ein Beispiel dafür ist in \ref{ex-koord-2} zu sehen: Die $\kappa$-Instanziierung als Konstituentenkoordination in \ref{ex-koord-2-a} führt zu Konjunkten mit Konstituentenketten, während die $\kappa$-Instanziierung als Satzkoordination in \ref{ex-koord-2-b} eine Ellipsenannahme im zweiten Konjunkt notwendig macht: 

\ex. Gestern sah der Jäger einen Hasen und der Tourist drei Waschbären. \label{ex-koord-2}
\a. \label{ex-koord-2-a} Gestern sah [$\kappa_1$ der Jäger einen Hasen] und [$\kappa_2$ der Tourist drei\linebreak Waschbären]. \\
$\leadsto$ Gestern sah der Jäger einen Hasen. \\
$\leadsto$ Gestern sah der Tourist drei Waschbären. 
\b. \label{ex-koord-2-b} [$\kappa_1$ Gestern sah der Jäger einen Hasen] und [$\kappa_2$ \sout{Gestern sah} der Tourist drei Waschbären]. \\
$\leadsto$ Gestern sah der Jäger einen Hasen. \\
$\leadsto$ Gestern sah der Tourist drei Waschbären. 

Entsprechendes lässt sich auch in Beispiel \ref{ex-koord-3} beobachten -- mit dem Unterschied allerdings, dass hier bei Konstituentenkoordination satzmedial einzelne Verben koordiniert und bei Satzkoordination in beiden Konjunkten Ellipsen angenommen werden müssen:

\ex. \label{ex-koord-3} Der Jäger sah und erschoss den Hasen.
\a. \label{ex-koord-3-a} Der Jäger [$\kappa_1$ sah] und [$\kappa_2$ erschoss] den Hasen. \\
$\leadsto$ Der Jäger sah den Hasen. \\
$\leadsto$ Der Jäger erschoss den Hasen. 
\b. \label{ex-koord-3-b} [$\kappa_1$ Der Jäger sah \sout{den Hasen}] und [$\kappa_2$ \sout{der Jäger} erschoss den Hasen]. \\
$\leadsto$ Der Jäger sah den Hasen. \\
$\leadsto$ Der Jäger erschoss den Hasen. 


Daneben gibt es aber auch Koordinationsdaten, die nur eine $\kappa$-Instanziierung zulassen, bei der also die Konjunkte entweder aus\-schlie\ss lich Sätze oder aus\-schlie\ss lich Nicht-Sätze sind. Oft reichen bereits kleine Umstellungen oder morphologische Veränderungen aus, um eine bestimmte $\kappa$-Instanziierung zu verhindern oder zu ermöglichen. Die Koordination in \ref{ex-koord-4} unterscheidet sich beispielsweise von der in \ref{ex-koord-2} allein im \isi{Numerus} der Nominativ-NP im zweiten Konjunkt. Dadurch ist jedoch die $\kappa$-Instanziierung als Konstituentenkoordination in \ref{ex-koord-4-a} nicht mehr $\kappa$-reduzierbar, denn es besteht nun ein Numeruskonflikt zwischen der Nominativ-NP  {\it die Touristen} und dem finiten Verb {\it sah}. Die $\kappa$-Instanziierung als Satzkoordination in \ref{ex-koord-4-b} kann dagegen $\kappa$-reduziert werden, jedoch nur unter der Voraussetzung, dass der Numerus von Ellipse und \isi{Antezedens} nicht identisch sein muss:    

\ex. \label{ex-koord-4} Gestern sah der Jäger einen Hasen und die Touristen drei Waschbären.
\a. \label{ex-koord-4-a} Gestern sah [$\kappa_1$ der Jäger einen Hasen] und [$\kappa_2$ die Touristen drei Waschbären]. \\
$\leadsto$ *Gestern sah die Touristen drei Waschbären.
\b. \label{ex-koord-4-b} [$\kappa_1$ Gestern sah der Jäger einen Hasen] und [$\kappa_2$ \sout{gestern sahen} die Touristen drei Waschbären]. \\
$\leadsto$ Gestern sah der Jäger einen Hasen. \\
$\leadsto$ Gestern sahen die Touristen drei Waschbären.

Eine Umstellung der Satzglieder kann ebenfalls zu einer Verminderung der $\kappa$-Instanziierungs\-optionen führen. Verschiebt man etwa in \ref{ex-koord-2} die Nominativ-NP in das Vorfeld, geht man also von dem Satz in \ref{ex-koord-5} aus, dann fällt die $\kappa$-Instanziierung als Konstituentenkoordination in \ref{ex-koord-5-a} weg. Die $\kappa$-Reduzierung erzielt dann nämlich einen Satz mit zwei Nominativ-NPs, eine im Vorfeld und eine im Mittelfeld, was sicher ungrammatisch ist. Das gilt aber nicht für die $\kappa$-Reduzierung der $\kappa$-Instanziierung als Satzkoordination in \ref{ex-koord-5-b}:

\ex. \label{ex-koord-5} Der Jäger sah einen Hasen und der Tourist drei Waschbären.
\a. \label{ex-koord-5-a} Der Jäger sah [$\kappa_1$ einen Hasen] und [$\kappa_2$ der Tourist] drei Waschbären. \\
$\leadsto$ *Der Jäger sah der Tourist drei Waschbären.
\b. \label{ex-koord-5-b} [$\kappa_1$ Der Jäger sah einen Hasen] und [$\kappa_2$ der Tourist \sout{sah} drei Waschbären]. \\  
$\leadsto$ Der Jäger sah einen Hasen. \\
$\leadsto$ Der Tourist sah drei Waschbären.

Die Beispiele bis hierhin vermitteln den Eindruck, dass eine $\kappa$-Instanziierung als Satzkooordination immer zur Verfügung steht. In der Literatur herrscht jedoch weitgehend Konsens darüber, dass bei bestimmten Koordinationstypen ausschlie\ss lich eine Konstituentenkoordination vorliegt (siehe \citealt[Abschnitt~2.3]{Hartmann:00}). Auf diese gehe ich im Folgenden ein. Daran anschlie\ss end finden Satzkoordinationen Erwähnung, die nicht als Satzkoordination $\kappa$-instanziierbar sind, nämlich manche Vertreter der sogenannten asymmetrischen Koordination.   


\subsubsection*{Eindeutige Konstituentkoordination durch Variablenbindung?}\is{Koordination!Konstituenten-|(}

Die Diagnose bestimmter Bindungsrelationen\is{Bindung} bei Indefinita\is{Indefinitum} und quantifikationellen Ausdrü\-cken gilt z.\,B.\  \citet[773]{Klein:93}, \citet[33ff]{Hartmann:00} und \citet[31ff]{Reich:09} als hinreichendes Indiz für das Vorliegen einer Konstituentenkoordination. Ausgangspunkt ist der Satz in \ref{ex-klein-34d}, der oberflächlich betrachtet problemlos sowohl als Satzkoordination als auch Konstituentenkoordination $\kappa$-instanzi"-ierbar ist: 

\ex. \label{ex-klein-34d}Jemand kam um vier Uhr und ging um fünf Uhr. \hfill \citep[(34d)]{Klein:93}
\a. [$\kappa_1$ Jemand kam um vier Uhr] und [$\kappa_2$ \sout{jemand} ging um fünf Uhr]. \\
$\leadsto$ Jemand kam um vier Uhr.\\
$\leadsto$ Jemand ging um fünf Uhr.
\b. Jemand [$\kappa_1$ kam um vier Uhr] und [$\kappa_2$ ging um fünf Uhr]. \\
$\leadsto$ Jemand kam um vier Uhr.\\
$\leadsto$ Jemand ging um fünf Uhr.

Dieser Satz soll jedoch nicht mit \ref{ex-klein-34c} semantisch äquivalent sein: 

\ex.  Jemand kam um vier Uhr und jemand ging um fünf Uhr.\label{ex-klein-34c} \hfill \citep[(34c)]{Klein:93}

Satz \ref{ex-klein-34d} hat also nur die Lesart, dass es sich bei demjenigen, der kam, und demjenigen, der ging, "`um dieselbe Person handelt"' \citep[773]{Klein:93}.  Anders verhalte es sich bei einer entsprechenden Tilgung im ersten Konjunkt (eine sogenannte RNR-Koordination\is{Koordination!Right-Node-Raising}, dazu gleich mehr): 

\ex. Um vier Uhr kam und um fünf Uhr ging jemand.
\a.[$=$] Jemand kam um vier Uhr und jemand ging um fünf Uhr.

Die genannten Autoren ziehen daraus den Schluss, dass nur Konstituentenkoordination vorliegen kann, also:

\ex. Jemand [$\kappa_1$ kam um vier Uhr] und [$\kappa_2$ ging um fünf Uhr].

Die Argumentation hat zwei Prämissen: (i) es gibt in \ref{ex-klein-34d} nur die Lesart mit \isi{Referenzidentität} und (ii) Ellipsenrekonstruktion führt immer zu einer Ambiguität zwischen Referenzidentität und Referenzunterschied (denn Ellipse unter Identität ist das Resultat einer Tilgungsoperation, die Syntax und Semantik  einer Koordination unangetastet lässt, siehe \citealt[33]{Hartmann:00}). Ergo liegt in \ref{ex-klein-34d} keine Ellipse und keine Satzkoordination vor. 

Es stellt sich dann die Frage, ob diese Prämissen stimmen. Interessanterweise finden sich in \cite{Hoehle:91} ganz entgegengesetzte Einschätzungen zu strukturanalogen Daten, siehe \ref{ex-hoehle91-122} und \ref{ex-hoehle91-125}:\footnote{Höhle nimmt trotzdem keine Satzkoordination mit Ellipse in \ref{ex-hoehle91-122} und \ref{ex-hoehle91-125} an, sondern schlägt eine ATB-Analyse\is{Bewegung!Across-the-Board} auf Grundlage einer Konstituentenkoordination vor. Eine erhellende Diskussion solcher Referenzlesarten liefert \citet[Section~3.2.4]{Oirsouw:87}, der zu dem Schluss kommt, dass nicht die syntaktische Struktur, sondern der Äu\ss erungskontext\is{Kontext} ausschlaggebend ist: "`possible differences in interpretation are not reflected in the structure of underlying represenations, but depend on context. [\ldots] The different readings are imposed by the contexts in which the sentences occur [\ldots]."' (S. 204)}

\ex. \label{ex-hoehle91-122}Einen Hund hat Karl gefüttert und hat Heinz gestreichelt. 
\a.[$=$] Es hat Karl einen Hund gefüttert und es hat Heinz einen Hund gestreichelt. 
\z. \citep[(122-a), (123)]{Hoehle:91} 

\ex. \label{ex-hoehle91-125}Ein Affe hat den Hund gefüttert und hat den Kater gefüttert.
\a.[$=$] Ein Affe hat den Hund gefüttert und ein Affe hat den Kater gefüttert.
\z. \citep[(125), (126)]{Hoehle:91}

Höhle zufolge sollen die Indefinita\is{Indefinitum} {\it einen Hund}/{\it ein Affe} hier gerade nicht zwingend konjunktübergreifend referenzidentisch sein. Eine zwingende Referenzidentität kann ich auch in den Beispielsäzen in \ref{ex-koord-referenz-1} nicht erkennen:

\ex. \label{ex-koord-referenz-1}
\a. Irgendetwas Teures kauft Hans im Baumarkt und bestellt Maria im Internet.
\b. Haferflocken sind gesund und isst Henning täglich.
\c. An einem lauen Sommerabend verliebten sie sich und heirateten sie später auch.
 
Dies Alles widerspricht den Generalisierungen zur Referenzidentität, die z.\,B.\  \cite{Klein:93} und \cite{Reich:09} mit \ref{ex-klein-34d} motivieren -- und es nährt auch Zweifel am Vorliegen einer zwingenden Referenzidentität in \ref{ex-klein-34d} selber. Ist hier wirklich kein Referenzunterschied möglich? Man muss sagen: Das Fehlen einer Lesart eines konkreten Satzes anhand der eigenen \isi{Intuition} zu behaupten, ist immer spekulativ. Diese Spekulation auf eine Menge semantisch heterogener Sätze qua ihres Konstruktionstyps zu extrapolieren, ist sogar hochspekulativ. 

Wie steht es um die zweite Prämisse? Führt Ellipsenrekonstruktion immer zu einer Ambiguität zwischen \isi{Referenzidentität} und Referenzdiversität? Zunächst gilt festzuhalten, dass bei der $\kappa$-Reduzierung\is{k-Reduzierung@$\kappa$-Reduzierung} rein deskriptiv verfahren wird, d.\,h.\ es wird nur festgestellt, dass eine \isi{Ellipse} besteht, und diese behelfsmä\ss ig rekonstruiert. Es wird also nichts darüber ausgesagt, wie ein konkretes Syntaxmodell vorgeht und ob z.\,B.\  die Ellipse phonologisch, morphologisch oder referentiell identisch mit dem \isi{Antezedens} rekonstruiert wird. Zudem will ich nicht generell ausschlie\ss en, dass in solchen Fällen auch im Zuge einer Ellipserekonstruktion eine Referenzidentität erzwungen werden kann.\footnote{Man beachte, dass sich die Ellipse in den hier betrachteten Fällen immer im \isi{Vorfeld} eines V2-Satzes\is{Satz!V2-} verorten lässt, und dass man daher Referenzidentität an diesem Merkmal festmachen könnte. Doch auch diesen Eindruck torpediert Höhle mit folgender Einschätzung:\\ 
\fnex{
\ex. da\ss\ ein Affe den Hund gefüttert hat und den Kater gefüttert hat.
\a.[$\neq$] Ein Affe hat den Hund gefüttert und ein Affe hat den Kater gefüttert.
\z. \citep[(127), (126)]{Hoehle:91} 

}}

Ein Vorteil dieser oberflächenorientierten, bezogen auf die Ellipserekonstruktion agnostischen Herangehensweise ist auch, dass Koordinationen wie in \ref{ex-koord-referenz-2} überhaupt eine $\kappa$-Instan\-ziierung zugewiesen werden kann:

\ex. \label{ex-koord-referenz-2}Um vier Uhr kam jemand und ging um fünf Uhr.

Falls nämlich die $\kappa$-Instanziierung als Satzkoordination aus referenz-semanti"-schen Gründen inadäquat ist, bleibt nur die $\kappa$-Instanziierung als Konstituentenkoordination. Eine solche ist aber (entsprechend der Valenzeigenschaften von {\it ging}) nicht $\kappa$-reduzierbar:

\ex. [$\kappa_1$ Um vier Uhr kam jemand] und [$\kappa_2$ ging um fünf Uhr]. \\
$\leadsto$ Um vier Uhr kam jemand. \\
$\leadsto$ *Ging um fünf Uhr.

Koordinationen wie in \ref{ex-koord-referenz-2} sind bekannt als SLF-Koordinationen\is{Koordination!SLF-} \citep{Hoehle:83}, die eine Teilklasse der asymmetrischen Koordination\is{Koordination!asymmetrische} bilden (siehe \citealt[1f]{Reich:09}). Wir werden weiter unten in diesem Abschnitt noch eine andere Teilklasse der asymmetrischen Koordination kennenlernen, deren Vertreter sich tatsächlich auch rein oberflächlich einer $\kappa$-Instanziierung als Satzkoordination und als Konstituentenkoordination widersetzen.   
\is{Koordination!Konstituenten-|)}


\subsubsection*{Problem: Die Koordination als pluralische Einheit} \label{sec-plural}\is{Plural|(}

Eine weitere Gruppe von Koordinationsdaten, bei denen die Annahme einer Satzkoordination\is{Koordination!Satz-} unplausibel sein soll, lässt sich folgenderma\ss en charakterisieren: Bei ihnen handelt es sich (im Sinne der Konstituentenkoordination) meist um eine Koordination einzelner Nominalphrasen, die syntaktisch oder semantisch als pluralische Einheit in Erscheinung tritt. Syntaktisch wird die Pluralbildungseigenschaft der Koordination mit {\it und} bei der Subjekt-Verb-Kongruenz deutlich, indem etwa eine Koordination mit singularischen Nominalkonjunkten (in der Subjektposition) mit einem pluralischen Finitum kongruiert, z.\,B.\ in \ref{ex-phraskoord1}:\footnote{Die Koordination singularischer Nominalphrasen ist in manchen Fällen jedoch auch mit einem singularischen Finitum verträglich:\\
\fnex{
\ex. Da stehen / steht ein Mann und eine Frau. \hfill \citep[(1)]{Steiner:09}

}
Diese "`partielle"' oder "`asymmetrische"' \isi{Kongruenz} (vgl.\ \citealt{Munn:00}; \citealt{Lorimor:07}; \citealt{Steiner:09}) ermöglicht sowohl eine $\kappa$-Instanziierung als Konstituentenkoordination als auch eine $\kappa$-Instanziierung als Satzkoordination. Bei freien Relativsätzen\is{Satz!Relativ-} ist die asymmetrische Kongruenz sogar der Normalfall (\citealt[143]{Oppenrieder:91}; \citealt[Abschnitt~10.4.1.1]{Mueller:99}).}

\exi. \label{ex-phraskoord1} Der Jäger und der Hase trafen sich im Wald.
\a. \label{ex-phraskoord1a} [$\kappa_1$ Der Jäger \sout{trafen sich im Wald}] und [$\kappa_2$ der Hase trafen sich im Wald].\\
$\leadsto$ *Der Hase trafen sich im Wald.
\b. \label{ex-phraskoord1b} [$\kappa_1$ Der Jäger] und [$\kappa_2$ der Hase] trafen sich im Wald. \\
$\leadsto$ *Der Hase trafen sich im Wald.

Die $\kappa$-Instanziierung als Satzkoordination in \ref{ex-phraskoord1a} muss also scheitern.

Sind die Konjunkte dagegen pluralisch wie in \ref{ex-phraskoord1c}, dann ist eine $\kappa$-Instanzi"-ierung sowohl als Satzkoordination als auch als Konstituentenkoordination $\kappa$-reduzierbar: 

\ex. \label{ex-phraskoord1c} [$\kappa_1$ Die Jäger \sout{trafen sich im Wald}] und [$\kappa_2$ die Hasen trafen sich im Wald].\\
$\leadsto$ Die Jäger \sout{trafen sich im Wald}. \\ 
$\leadsto$ Die Hasen trafen sich im Wald. \\
$\neq$ Die Jäger trafen sich im Wald und die Hasen trafen sich im Wald.

Man darf jedoch nicht übersehen, dass die Bedeutung der $\kappa$-reduzierten Ausdrücke nicht mehr dem des Ausgangsausdrucks entspricht. Konnte der Ausgangsausdruck in \ref{ex-phraskoord1c} nämlich noch bedeuten, dass die Jäger auf eine Gruppe von Hasen trafen, so wird diese Lesart durch die $\kappa$-reduzierten Ausdrücke nicht mehr unterstützt. Vielmehr trafen nun nur die Jäger auf Jäger, und die Hasen auf Hasen. Davon sind systematisch die Subjekt- und Objektpositionen sogenannter symmetrischer oder nicht-distributiver Prädikate\is{symmetrisches Prädikat} ({\it sich treffen}, {\it zusammensto\ss en}, {\it ähnlich sein}) betroffen. Ein ähnliches Verhalten zeigen zudem Koordinationen mit Reziprokpronomen und Rezprokadverbien:\footnote{Hier zeigt sich, dass auch die Koordination von Verben eine pluralische Einheit bilden kann:\\
\fnex{
\ex. Er liest und schreibt nacheinander. \hfill \citep[(3-140)]{Hesse:Kuestner:85} \\
$\leadsto$ \#Er liest nacheinander.

}} 

\ex. \label{ex-phraskoord2}
[$\kappa_1$ Peter hetzte den Hasen \sout{aufeinander}] und [$\kappa_2$ \sout{Peter hetzte} den Igel] aufeinander.\\
$\leadsto$ \#Peter hetzte den Hasen aufeinander. 

All diese Fälle vermeintlich eindeutiger Konstituentenkoordination\is{Koordination!Konstituenten-} haben allerdings einen Haken: Sie sind auch als Konstituentenkoordination nicht $\kappa$-in"-stan"-zi"-ierbar. Dies stellt natürlich die Brauchbarkeit der $\kappa$-Reduzierung\is{k-Reduzierung@$\kappa$-Reduzierung} in Frage, indem es die oben geäu\ss erte Vermutung konterkariert, dass "`es für jeden grammatischen Satz mit Koordination mindestens eine $k$-Instanziierung gibt, die $k$-reduzierbar ist"'. Ein offensichtlicher Ausweg aus dieser Misere besteht darin, diese Koordinationsfälle innerhalb eines Satzglieds\is{Satzglied} zu verorten. Da koordinieren also nicht vollständige Satzglieder, sondern Bestandteile eines komplexen Satzglieds. Dafür spricht, dass diese komplexen Satzglieder tatsächlich syntaktisch und semantisch als pluralische Einheit auf"|treten. Und daraus folgt, dass solche Koordinationsfälle bei der Betrachtung der Koordination von vollständigen Satzgliedern irrelevant sind. 

Aber nicht alle pluralbildenden Koordinationen können als satzgliedinterne Phänomene abgetan werden. Dies legen zumindest die englischen Koordinationsdaten in \ref{ex-phraskoord-4} nahe:\footnote{Siehe dazu auch \cite{Yatabe:02}.}

\ex. \label{ex-phraskoord-4}
\a. The pilot claimed that the first nurse, and the sailor proved that the second nurse, were spies. \hfill \citep[173]{Postal:98}
\b. John whistled and Mary hummed together. \hfill \citep[192]{Jackendoff:77}

Solche Daten scheinen sich im Deutschen zunächst nicht reproduzieren zu lassen, denn die Übersetzungen von \ref{ex-phraskoord-4} in \ref{ex-phraskoord-5} sind (nicht nur) meiner Einschätzung nach schlecht:   

\ex. \label{ex-phraskoord-5}
\a. *Der Pilot behauptete, dass die erste Krankenschwester, und der Matrose bewies, dass die zweite Krankenschwester Spione waren.   
\b. *Karl pfiff und Maria summte gemeinsam. \hfill \citep[52, Fußnote 5]{Wesche:95}

Dieser Anschein trügt jedoch; \citet[(34)]{Schwabe:Heusinger:01} liefern ein klares Gegenbeispiel:

\ex.  Bist du sicher, dass [$\kappa$ Hans den Saft] und [$\kappa$ Fritz den Wein] gestohlen haben?\label{ex-sh-34} \\
$\leadsto$ *\ldots dass Hans den Saft gestohlen haben? \\
$\leadsto$ *\ldots dass Fritz den Wein gestohlen haben?


Während hier also offensichtlich eine Anwendungslücke der $\kappa$"=Reduzierbarkeit besteht, bezweifle ich, dass diese Lücke für unsere empirische Perspektive relevant ist. Es ist nämlich fraglich, ob der Koordinationstyp in \ref{ex-sh-34} überhaupt als elliptische Satzkoordination, auf die es hier ja ankommt, auf"|treten kann. Der Versuch, \ref{ex-sh-34} zu einer eindeutig elliptischen Satzkoordination (mit Gapping) umzuformulieren, scheitert:

\ex. 
\a. Heute haben Hans den Saft und Fritz den Wein gestohlen.
\b. *Hans haben den Saft und Fritz den Wein gestohlen. 

\is{Plural|)} 


\subsubsection*{Problem: Asymmetrische Koordination} \label{sec:asymmetrische:koordination}\is{Koordination!asymmetrische|(}
 
Ein kritisches Ma\ss\ an Verschiedenheit der Konjunkte scheint bei der sogenannten asymmetrischen Koordination \citep{Hoehle:90} überschritten, die Höhle mit \ref{ex-hoehle-90-6} veranschaulicht:\footnote{Zur Empirie und Analyse asymmetrischer Koordination siehe auch z.\,B.\  \cite{Wunderlich:88}, \cite{Buering:Hartmann:98}, \cite{Kathol:99}, \citet[596ff]{Sternefeld:06}, \cite{Reich:09}.} 

\ex. \label{ex-hoehle-90-6} wenn [$\kappa_1$ jemand nach Hause kommt] und [$\kappa_2$ da steht der Gerichtsvollzieher vor der Tür], \ldots \\
$\leadsto$ *wenn da steht der Gerichtsvollzieher vor der Tür, \ldots \\
\citep[(6)]{Hoehle:90}

Die Konjunkte $\kappa_1$ und $\kappa_2$ unterscheiden sich im Satztyp, denn in $\kappa_1$ liegt VE\is{Satz!VE-} vor, in $\kappa_2$ dagegen V2\is{Satz!V2-}.\footnote{Zur Erläuterung der Satztypen siehe Abschnitt \ref{sec-feldermodell}.} Der einbettende Komplementierer {\it wenn} ist jedoch nur mit VE verträglich, weshalb die $\kappa$-Reduzierung mit $\kappa_2$ zu einem ungrammatischen Satz führt. Die asymmetrische Koordination ist deshalb auch im mathematischen Sinne asymmetrisch, weil nämlich deren Konjunkte nicht umgeordnet werden können:

\ex. *wenn [$\kappa_2$ da steht der Gerichtsvollzieher vor der Tür] und [$\kappa_1$ jemand nach Hause kommt], \ldots

Natürlich gibt es zur $\kappa$-Instanziierung in \ref{ex-hoehle-90-6} Alternativen, aber keine davon ist aus meiner Sicht $\kappa$-reduzierbar. Man könnte z.\,B.\  von der $\kappa$-Instanziierung in \ref{ex-ak-1} ausgehen, bei der der \isi{Komplementierer} {\it wenn} Teil des ersten Konjunkts ist und somit nicht in irregulärer Weise mit einem V2-Satz kombiniert wird:

\ex. \label{ex-ak-1} [$\kappa_1$ Wenn jemand nach Hause kommt] und [$\kappa_2$ da steht der Gerichtsvollzieher vor der Tür], steigt man eben durch das Küchenfenster ein. \\
$\leadsto$ *Da steht der Gerichtsvollzieher vor der Tür, steigt man eben durch das Küchenfenster ein.

Die daraus resultierende $\kappa$-Reduktion finde ich jedoch ebenfalls unakzeptabel.

Im Folgenden werde ich diesen Koordinationstyp weitgehend ignorieren. Obwohl die $\kappa$-Reduzierbarkeit hier offensichtlich eine Anwendungslücke aufweist, schätze ich diese als nicht sehr gro\ss\ ein, so dass damit trotzdem ein hinreichender Teil des Koordinationsphänomens abgedeckt werden kann. \citet[(173)]{Reich:09} zufolge sind Ellipsenprozesse bei asymmetrischer Koordination sowieso "`strikt ausgeschlossen"'.
\is{Koordination!asymmetrische|)}

\subsection{Alternativen zur $\kappa$-Reduzierbarkeit} \label{sec-koordinationsellipsen-5}\is{k-Reduzierbarkeit@$\kappa$-Reduzierbarkeit|(}

Die $\kappa$-Reduzierbarkeit ist in erster Linie ein diagnostisches Werkzeug. Sie dient dazu, Koordinationstypen\is{Koordination} zu unterscheiden und deren Eigenschaften festzustellen. Dafür setzt sie ein Mindestma\ss\ an Ähnlichkeit zwischen den Konjunkten voraus, nämlich im Fall der Konstituentenkoordination\is{Koordination!Konstituenten-} die jeweilige Kompatibilität mit dem Kontext der Koordination, und im Fall der Satzkoordination\is{Koordination!Satz-} die Rekonstruierbarkeit einer  Ellipse in einem Konjunkt anhand eines \isi{Antezedens} im anderen Konjunkt. Es fehlt dagegen vollständig der Bezug zu phrasenstrukturellen oder morpho-syntaktischen Eigenschaften der Konjunkte (z.\,B.\  deren Konstituentenhaftigkeit).  
Um dieses Wesen der $\kappa$-Reduzierbarkeit zu veranschaulichen, werde ich in diesem Abschnitt vergleichbare Diagnose- und Ähnlichkeitskonzepte aus anderen Arbeiten kontrastierend darstellen. 

In mehrfacher Hinsicht ist unser Ansatz z.\,B.\  von der abgeschwächten \isi{Koordinationsregel} in \citet[35]{Chomsky:57} abzugrenzen, auch wenn zunächst Ähnlichkeiten auf"|fallen. Chomsky schreibt nämlich:
\begin{quote}
If we had two sentences Z+X+W and Z+Y+W, and if X and Y are actually constituents of these sentences, we can generally form a new sentence Z-X+and+Y-W.\footnote{Der Unterschied zu der ausführlicheren, etwas strikteren Formulierung der Koordintionsregel \citep[(26)]{Chomsky:57} ist für uns unwesentlich. Deshalb zitiere ich hier die einfachere Formulierung.}
\end{quote}
Hier wird also ein Satz mit einer Koordination, schematisiert als "`Z-X+and+Y-W"', mit zwei anderen Sätzen mit den Schemata "`Z+X+W"' bzw.\ "`Z+Y+W"' in Verbindung gebracht. Dies tut in dieser allgemeinen Form auch die $\kappa$-Reduzierung, siehe \ref{ex-mm-3-4}, doch bei genauerem Hinsehen unterscheiden sich diese Verbindungen folgenderma\ss en:
(i) Da ist zunächst eine unterschiedliche Richtung in der Betrachtungsweise. Die Koordinationtsregel geht von koordinationsfreien Sätzen aus und generiert hiermit eine Koordination, während die $\kappa$-Reduzierung eine Koordination analysiert, indem daraus koordinationsfreie Sätze gebildet werden. (ii) Dann werden unterschiedliche empirische Vorhersagen getroffen. Die $\kappa$-Reduzierung impliziert eben nicht, dass eine Koordinaton, die zwei grammatische $\kappa$-Redukte erhält, selber wohlgeformt ist. Die empirischen Vorhersagen sind also wesentlich schwächer. Freilich wird angenommen, dass die $\kappa$"=Reduzierbarkeit eine der Wohlgeformtheitsbedingungen der \isi{Koordination} (von Satzgliedern) darstellt. (iii) Schlie\ss lich werden die Konjunkte durch die Koordinationsregel so beschränkt, dass sie nur einzelne Konstituenten enthalten.\footnote{Die striktere Formulierung der Koordinationsregel in \citet[(26)]{Chomsky:57} lässt darüberhinaus nur einen gemeinsamen Konstituententyp ("`type"') zu.} Die $\kappa$-Reduzierung macht dagegen keinerlei derartigen Einschränkungen hinsichtlich der Anzahl und Art der Konjunktkonstituenten. 

Einen flexibleren Ähnlichkeitsbegriff entwickelt \cite{Hudson:88} im Rahmen der \isi{Dependenzgrammatik}. Seinem \isi{DICS-Prinzip} ("`Dependency in Coordinate Structures"') entsprechend müssen die Konjunkte jeweils dieselbe Anzahl an Dependenzbeziehungen mit denselben Wörtern au\ss erhalb der Koordination eingehen.\footnote{\citet[323]{Hudson:88} bringt es folgenderma\ss en auf den Punkt: "`any dependency relation which crosses a conjunct boundary must be shared by all the conjuncts"'. Diese Art der Übereinstimmung von extrovertierten syntaktischen Relationen ist auch Ansatzpunkt des Sharing-Konzepts bei \citet[35]{Johnson:04}. } Flexibler ist dieses dependenzbasierte Vorgehen deswegen, weil ein Konjunkt aus mehreren distinkten Dependenzgraphen bestehen kann. Dadurch ist es beispielsweise möglich, die Koordination in \ref{ex-hudson88-35} zu erfassen, dessen Konjunkte je in drei Dependenzrelationen mit dem finiten Verb {\it gave} stehen:   

\ex. \label{ex-hudson88-35} He gave [$\kappa_1$ Mary a record for her birthday] and [$\kappa_2$ Sue some chocolates for Christmas].\hfill \citep[(35)]{Hudson:88}

Die Konjunkte können dagegen schwerlich als einzelne Konstituenten im Sinne von Chomskys \isi{Koordinationsregel} gelten.\footnote{Im Rahmen generativer, konstituentenbasierter Grammatiktheorien\is{Generative Grammatik} wurde und wird deswegen meist angenommen, dass hier die Überreste einer vollständigen Konstituente sichtbar sind, die durch Bewegung (z.\,B.\ ATB-Extraktion\is{Bewegung!Across-the-Board}) oder Tilgung transformiert wurde.}

Verglichen mit der $\kappa$-Reduzierbarkeit ist das DICS-Prinzip trotzdem ein (in den meisten Fällen)\footnote{Als weniger restriktiv erweist sich das DICS-Prinzip bei der NP-Koordination bei symmetrischen Prädikaten\is{symmetrisches Prädikat} und bei pluralbildendem {\it und}, die es anders als die $\kappa$-Reduzierbarkeit ohne Umwege zulässt.} restriktiverer Ähnlichkeitsbegriff. Der Unterschied zeigt sich bei den Koordinationen in \ref{ex-koord-26}, deren Konjunkte zwar nicht den Vorgaben des DICS-Prinzips entsprechen, aber als Konstituentenkoordination $\kappa$"=reduzierbar sind:

\ex. \label{ex-koord-26}
\a. *[$\kappa_1$ John] and [$\kappa_2$ Harry thinks that Mary] climbed the mountain. \hfill \citep[(47a)]{Pickering:Barry:93}\label{ex-koord-26-a}
\b. He gave [$\kappa_1$ Mary a record for her birthday] and [$\kappa_2$ Sue some chocolates].\label{ex-koord-26-b} 

Dass die unakzeptable Koordination in \ref{ex-koord-26-a} nicht mit dem \isi{DICS-Prinzip} vereinbar ist, da zwischen {\it climbed} und dem ersten Konjunkt eine Dependenzrelation, zwischen {\it climbed} und dem zweiten Konjunkt aber zwei Dependenzrelationen bestehen, ist ein willkommener Effekt. Das gilt sicherlich nicht für die Tatsache, dass auch die (in meinen Augen) wohlgeformte Koordination in \ref{ex-koord-26-b} dem DICS-Prinzip widerspricht. 

Einen anderen Ähnlichkeitsbegriff, der wiederum weniger restriktiv als das DICS-Prinzip ist, schlagen \cite{Hoehle:90,Hoehle:91} mit der \isi{External Homogeneity Condition (EHC)} und \citet[305]{Kathol:99} mit der \isi{String Continuation Condition (SCC)} vor. Tatsächlich sind EHC, SCC und $\kappa$-Reduzierbarkeit äquivalent, soweit es die Konstituentenkoordination betrifft. Für die Definition des EHC geht Höhle von dem Abfolgeschema in \ref{ex-hoehle-90-1} aus: 

\ex. \label{ex-hoehle-90-1} $^1\!A \ \ldots \ ^k\!A \ [($\&$) \ ^1\!B \ \ldots \ \& \ ^n\!B] \ ^{k+1}\!A \ \ldots \ ^m\!A$ \hfill \citep[(1)]{Hoehle:90}

$^i\!B$ mit $1 \leq i \leq n$ steht hier für eines der Konjunkte einer beliebig vielgliedrigen Koordination. Die EHC lautet demnach: 

\ex. \label{ex-hoehle-90-3}
{\bf External Homogeneity Condition} \hfill \citep[(3)]{Hoehle:90} \\
The combinatorial properties of each $^i\!B$ are satisfied by $^1\!A , \ldots , ^m\!A$ in the same way as the combinatorial properties of every $^j\!B$ are.

Die Ähnlichkeit der kombinatorischen Eigenschaften der Konjunkte hinsichtlich des Kontexts\is{Kontext} der Koordination kommt dann folgenderma\ss en zum Ausdruck: "`It follows, then, that each single conjunct $^i\!B$ may be substituted for the whole constituent \glq$($\&$) \ ^1\!B \ \ldots \ ^n\!B$\grq~ salva grammaticalitate."' \citep[222]{Hoehle:90} Die Analogie zur $\kappa$-Reduzierbarkeit ist offensichtlich, obgleich bei der $\kappa$-Reduzierbarkeit "`salva grammaticalitate"' nach der Ellipsenrekonstruktion gilt. Höhle nutzt die EHC, um die Besonderheit der asymmetrischen Koordination\is{Koordination!asymmetrische} zu veranschaulichen, nämlich gerade die Verletzung der EHC. Wie wir bereits gesehen haben, widersetzt sich dieser Koordinationstyp auch einer $\kappa$-Instanziierung, die sich $\kappa$-reduzieren lässt. Denselben Zweck hat die SCC\is{String Continuation Condition (SCC)} in \cite{Kathol:99}, die Kathol anhand der Koordination in \ref{ex-koord-27} erläutert:

\ex. \label{ex-koord-27}In den Wald [$\kappa_1$ ging der Jäger] und [$\kappa_2$ lief der Junge].\hfill \citep[(1)]{Kathol:99}   

Die Bedingung ist hier, dass die Konjunkte den koordinationsexternen String {\it in den Wald} jeweils so erweitern können, dass ein grammatischer deutscher Satz entsteht.\footnote{Im Original: "`both conjuncts can be used to extend the initial string {\it in den Wald} to yield the well-formed German sentences {\it In den Wald ging der Jäger} und {\it In den Wald lief der Junge}."' \citep[305]{Kathol:99}} 

Um ein Diagnoseinstrument ohne Ähnlichkeitsbegriff handelt es sich schlie\ss lich bei der S-Para\-phra\-sier\-barkeit\is{S-Paraphrasierbarkeit} ("`sentence-paraphrasable"') von \citet[13f]{Oirsouw:87}. In van Oirsouws Bestreben, jegliche Koordination aus der Koordination vollständiger Sätze abzuleiten, dient ihm dieses sehr vage bestimmte Kriterium dazu, NP-Koordinationen herauszufiltern, die nicht oder nur sehr schwer auf koordinierte Sätze zurückgeführt, also S-paraphrasiert, werden können. Die Rede ist von NP-Koordination bei symmetrischen Verben\is{symmetrisches Prädikat}. Als S-paraphrasierbar gelten Koordinationen, die durch die Vereinbarkeit mit bestimmten Triggern wie {\it in that order} ihre Asymmetriefähigkeit unter Beweis stellen.\footnote{Daher wäre wohl die Unterscheidung zwischen asymmetriefähigen und asymmetrieunfähigen Koordinationen terminologisch näher an der Sache.} Die folgenden Beispiele aus \citet[13f]{Oirsouw:87} demonstrieren eine S-paraphrasierbare Koordination in \ref{ex-oirsouw-1-a} und eine nicht-S-paraphrasierbare Koordination in \ref{ex-oirsouw-1-b}:   

\ex. \label{ex-oirsouw-1}
\a. John arrived and Mary left, in that order.\label{ex-oirsouw-1-a}
\b. *John and Mary are alike, in that order.\label{ex-oirsouw-1-b}

Die Partitionierung der Koordinationsdaten in S-paraphrasierbare und nicht-S-paraphrasier\-ba\-re stimmt nicht mit der entsprechenden Partitionierung durch die $\kappa$-Reduzierbarkeit überein. Als nicht-$\kappa$-reduzierbar gelten sowohl die nicht-S-paraphrasierbare Koordination in \ref{ex-oirsouw-1-b} als auch die S-paraphrasierbare Koordination in \ref{ex-asymmetriefähig}: 

\ex. \label{ex-asymmetriefähig} John and Mary have arrived, in that order.
 
Wie oben (S.\,\pageref{sec-plural}--\pageref{sec:asymmetrische:koordination}) bereits dargestellt ist die $\kappa$-Reduzierbarkeit bei der Koordination singularischer Nominalphrasen nur eingeschränkt einsetzbar.
\is{k-Reduzierbarkeit@$\kappa$-Reduzierbarkeit|)}



\subsection{Taxonomie der Satzkoordination}\is{Koordination!Satz-|(}\is{Ellipse!Koordinations-|(}

Eine Taxonomie der Kooridinationsdaten lässt sich auf Basis der $\kappa$-Instanziierung als Satzkoordination und der dortigen Ellipsendistribution erstellen. Hiermit werden die vier Koordinationstypen in Tabelle \ref{tab-koord-typen} unterscheidbar, die ich mit der Klein'schen Terminologie der Vorwärts-\is{Ellipse!Vorwärts-} und Rückwärtsellipse\is{Ellipse!Rückwärts-} \citep[770]{Klein:93} versehen habe: Koordinationen mit zwei vollständigen (d.\,h.\ ellipsenfreien) Konjunkten, Koordinationen mit einem vollständigen ersten Konjunkt und einem unvollständigen zweiten Konjunkt (Vorwärtsellipse), Koordinationen mit einem unvollständigen ersten Konjunkt und einem vollständigen zweiten Konjunkt (Rückwärtsellipse), und schlie\ss lich Koordinationen mit einem unvollständigen ersten Konjunkt und einem unvollständigen zweiten Konjunkt (Vorwärts"=Rückwärtsellipse). Die Nützlichkeit dieser Taxonomie wird darin deutlich, dass sie zwei elliptische Koordinationsformen trennt, deren Unterschiedlickeit bereits vielfach in der Literatur thematisiert wurden (siehe z.\,B.\  \citealt{Ross:70}; \citealt{Hudson:76}; \citealt{Hoehle:83b}; \citealt{Klein:93}; \citealt{Hartmann:00}): Gapping und Right-Node-Raising. Im Folgenden werde ich deren Gesetzmä\ss igkeiten mit Blick auf die Valenzrahmenrealisierung weiter herausarbeiten. 
\begin{table}

\begin{tabular}{cc|l|l}
&& \multicolumn{2}{c}{{\bf 2. Konjunkt} $~~~$} \\
&& \multicolumn{1}{c}{vollständig} & \multicolumn{1}{|c}{unvollständig} \\
\hline
\multirow{6}{*}{\rotatebox{90}{\textbf{1. Konjunkt}}}& vollst. & & Vorwärtsellipse \\
&         & & $~~~~$Gapping\is{Ellipse!Gapping} \\
&         & & $~~~~$Vorfeldellipse\is{Ellipse!Vorfeld-} \\
&         & & $~~~~~~~~$Subjektlücke\is{Ellipse!Subjektlücke} \\
\cline{2-4}
& unvollst. & Rückwärtsellipse & Vorwärts-Rückwärtsellipse \\
&         & $~~~~$Right-Node-Raising & $~~~~$Right-Node-Raising 
\end{tabular} 

\caption{\label{tab-koord-typen}Taxonomie der Satzkoordination}
\end{table}

Die Taxonomie in Tabelle \ref{tab-koord-typen} enthält nicht alle Formen der Ellipse, die mit Koordination in Verbindung gebracht werden können. Es fehlen zumindest VP-Ellipse\is{Ellipse!VP-} (VPE) und Sluicing\is{Ellipse!Sluicing}. Die VPE bezeichnet eine im Englischen häufig anzutreffende Konstruktion (siehe z.\,B.\  \citealt{Hardt:93,Johnson:01}), in der die durch ein Hilfs- oder Modalverb regierte Infinitiv-Verbalphrase fehlt. Beispiele dafür sind \ref{ex-vpe-1-a} und \ref{ex-vpe-1-b}:

\ex. \label{ex-vpe-1}
\a. Dan likes golf, and George does \sout{like golf} too. \\ \citep[(1)]{Dalrymple:etal:91}\label{ex-vpe-1-a}
\b. Ben can solve the problem, but I know that Peter can't \sout{solve the problem}. \hfill \citep[(32a), 134]{Winkler:05}\label{ex-vpe-1-b}

Im Deutschen scheint die VPE dagegen sehr viel eingeschränkter möglich (\citealt[158ff]{Lobeck:95}; \citealt[Chapter~3]{Winkler:05}).\footnote{Aufschlussreich könnte zudem die Studie von \citet[Chapter~2]{Aelbrecht:10} zu Modalkomplementellipsen ("`Modal Complement Ellipsis"') im Niederländischen sein, die mir in gro\ss en Teilen auf das Deutsche übertragbar zu sein scheint. Aelbrecht versucht zu zeigen, dass Modalkomplementellipsen andere Extraktionseigenschaften haben als VPE im Englischen.} Die deutsche Überstzung von \ref{ex-vpe-1-b} sei beispielsweise nur mit Proform möglich: 

\ex. Ben kann die Aufgabe lösen, aber ich weiß, dass Peter *(es) nicht kann. \hfill \citep[(33a), 134]{Winkler:05}\label{ex-vpe-2}

Sprachtypologisch universeller ist das Sluicing\is{Ellipse!Sluicing} (siehe z.\,B.\  \citealt{Ross:69}; \citealt{Merchant:01}). Hierbei soll, wie in \ref{ex-sluicing-1} exemplifiziert, eine \textit{Wh}-Konstituente das Überbleibsel eines eingebetteten Interrogativsatzes bilden: 
 
\ex. \label{ex-sluicing-1}Peter sah jemanden, aber ich wei\ss\ nicht, wen \sout{er sah}.

Diese VPE- und Sluicing-Instanzen können zwar als Vorwärtsellipsen klassifiziert werden, sie unterscheiden sich aber im Aufbau und im Verhältnis zum Antezedens erheblich von Gapping und Vorfeldellipse. Die Herausarbeitung ihrer Spezifika würde den Rahmen dieser Arbeit sprengen. Gleichwohl werden VPE und Sluicing bei der Untersuchung von Modellierungstechniken in Kapitel \ref{sec-ellipsenanalyse} da in Erscheinung treten, wo die referierte Literatur sich ihrer annimmt.
\is{Koordination!Satz-|)}\is{Ellipse!Koordinations-|)} 


\section{Gapping}\label{sec-gapping}\is{Ellipse!Gapping|(}

Gapping ist eine Form der Vorwärtsellipse\is{Ellipse!Vorwärts-}, bei der die Ellipse im zweiten Konjunkt das finite \isi{Verb} einschlie\ss t. Ich verwende hier eine weite Auslegung des Gappingbegriffs, der auf sämtliche Vorwärtsellipsen in \ref{ex-koord-6} anzuwenden ist und in dieser Hinsicht mit dem in \citet[144]{Hartmann:00} übereinstimmt.  In der Literatur (z.\,B.\  bei \citealt{Ross:70,Jackendoff:71}) findet sich auch ein enger Gappingbegriff, der nur solche Vortwärtsellipsen abdeckt, die wie \ref{ex-koord-6-a} über eine kontinuierliche, satzmediale, das finite Verb einschlie\ss ende Ellipse verfügen. Der weite Gappingbegriff umfasst dagegen auch Ellipsen mit einem einzelnen Überbleibsel wie in \ref{ex-koord-6-b} und \ref{ex-koord-6-c}, die gemeinhin als "`bare argument ellipsis"'\is{Ellipse!Bare Argument Ellipsis} (BAE) oder "`stripping"' vom Gapping (im engeren Sinne) unterschieden werden:\footnote{\citet[275f]{Culicover:Jackendoff:05} halten ebenfalls eine Bündelung von BAE und Gapping (im engeren Sinne) für angebracht, aber sie betrachten dabei das Gapping als Spezialfall der BAE, nämlich als "`double BAE"'.}  

\ex. \label{ex-koord-6}
\a. \label{ex-koord-6-a}[$\kappa_1$ Der Jäger sah einen Hasen] und [$\kappa_2$ der Tourist \sout{sah} drei Waschbären].
\b. \label{ex-koord-6-b}[$\kappa_1$ Der Jäger sah einen Hasen] und [$\kappa_2$ \sout{der Jäger sah} drei Waschbären].
\c. \label{ex-koord-6-c}[$\kappa_1$ Der Jäger sah einen Hasen] und [$\kappa_2$ der Tourist \sout{sah einen Hasen}].
\d. \label{ex-koord-6-c} dass [$\kappa_1$ der Jäger einen Hasen sah] und [$\kappa_2$ der Tourist drei Waschbären \sout{sah}]

Halten wir diesen kleinsten gemeinsamen Nenner der beiden Gappingbegriffe in der Regel G1\is{Gappingregel!G1} fest:
\begin{itemize}
  \item[] {\bf G1:} Die Ellipse in Satz $S$ schlie\ss t das finite Verb von $S$ ein.
%  \item[G1:] Wird das finite Verb eines V1/V2-Satzes $S$ realisiert, dann gibt es keine Ellipse in $S$.
\end{itemize}
Der weite Gappingbegriff erfasst jedoch nicht alle grammatischen Formen der Vorwärtsellipse im Deutschen. Unter gewissen strukturellen Voraussetzungen ist es nämlich möglich, ein Satzglied des zweiten Konjunkts zu tilgen, ohne dabei das finite Verb in die Ellipse miteinzubeziehen. Solche Ellipsen lassen beispielsweise die Koordinationsdaten in \ref{ex-koord-8} zu, deren Gemeinsamkeit in der VE-Konfiguration\is{Satz!VE-} des zweiten Konjunkts liegt (vgl.\ \citealt{Wilder:94,Wilder:97}). Man beachte, dass hier nicht nur Koordinationen wie \ref{ex-koord-8-a} betroffen sind, die auch als Konstituentenkoordination $\kappa$-instanziiert werden können, sondern dass mit \ref{ex-koord-8-b} und \ref{ex-koord-8-c} Beispiele angegeben sind, die ausschlie\ss lich eine $\kappa$-Instanziierung als Satzkoordination erlauben:\footnote{Bei der $\kappa$-Instanziierung von \emph{dass}-Nebensätzen steht der \isi{Komplementierer} au\ss erhalb der Koordination. Unter den Prämissen dieser Arbeit, insbesondere die Akzentuierung der valenzbezogenen Ellipse, erscheint mir dieser Schritt als folgerichtig.} 

\ex. \label{ex-koord-8}
\a. \label{ex-koord-8-a}dass [$\kappa_1$ der Jäger den Hasen sah] und [$\kappa_1$ \sout{der Jäger} ihn erschoss].
\b. \label{ex-koord-8-b}dass [$\kappa_1$ der Jäger den Hasen sah] und [$\kappa_1$ der Tourist \sout{ihn} erschoss].
\c. \label{ex-koord-8-c}dass [$\kappa_1$ den Hasen der Jäger sah] und [$\kappa_2$ ihn \sout{der Jäger} erschoss].

Doch nicht nur VE-Konfigurationen sind dafür empfänglich. Auch bestimmte V1- und V2-Konfigurationen\is{Satz!V1-}\is{Satz!V2-} können im Rahmen der Vorwärtsellipse\is{Ellipse!Vorwärts-} quasi am finiten Verb vorbeitilgen. Hier sind die Vorgaben allerdings spezifischer: (i) maximal ein \isi{Satzglied} kann getilgt werden; (ii) das Vorfeld ist nicht besetzt, bzw.\ es besteht eine V1-Konfiguration. Ich nenne diese Form der Ellipse daher \textsc{Vorfeldellipse}\is{Ellipse!Vorfeld-}.\footnote{Freie Vorfeldellipsen, die auch "`Null-Topik"' \citep{Fries:88} oder \isi{Topic Drop} genannt werden, werden unten in Abschnitt~\ref{sec-adjazenzellipsen} behandelt.} In der Literatur bekannt geworden ist aber ein Spezialfall der Vorfeldellipse, die sogenannte \textsc{Subjektlücke}\is{Ellipse!Subjektlücke}.\footnote{Der Terminus geht auf \cite{Hoehle:83b} zurück. Anders als in der vorliegenden Arbeit wird dort nur dem Koordinationstyp in \ref{ex-koord-7-b} eine "`Subjektlücke"' zugesprochen. Ausschlaggebend dafür ist das Fehlen einer "`phrasale Koordination"', d.\,h.\ im Grunde die Unverfügbarkeit einer $\kappa$-Instanziierung als Konstituentenkoordination. Desweiteren geht Höhle von einer V1-Konfiguration\is{Satz!V1-} im zweiten Konjunkt aus, das die Subjektlücke im Mittelfeld verortet. In der Fülle der nachfolgenden Forschungsliteratur wird dieser Koordinationstyp oft als Spezialfall der  asymmetrischen Koordination\is{Koordination!asymmetrische} abgehandelt. Siehe auch die Hinweise oben in Abschnitt~\ref{sec-ellipse-kappa}.} Dafür zwei Beispiele in \ref{ex-koord-7}:   

\ex. \label{ex-koord-7}
\a. \label{ex-koord-7-a}[$\kappa_1$ Der Jäger sah einen Hasen] und [$\kappa_2$ \sout{der Jäger} erschoss ihn].
\b. \label{ex-koord-7-b}[$\kappa_1$ Einen Hasen sah der Jäger] und [$\kappa_2$ \sout{der Jäger} erschoss ihn]   

Die Subjektlücke in \ref{ex-koord-7-b} ist hier besonders auf"|fällig, da sie, im Unterschied zu \ref{ex-koord-7-a}, nicht als Konstituentenkoordination $\kappa$-instanziiert werden kann. Interessanterweise stoßen entsprechende Vorfeldellipsen mit \isi{Objektlücke} auf sehr wenig Akzeptanz. Zu diesen Ausnahmen gehört \cite{Fortmann:05}, der die als akzeptabel eingestufte Objektlücke in \ref{ex-koord-9} als Beispiel anführt:\footnote{Wolfgang Sternefeld äu\ss ert jedoch ernsthafte Zweifel an der Grammatikalität der Fortmann-Daten (siehe sein Vortrag an der Universität Leipzip am 20.07.2007, \url{http://www.s395910558.online.de/Downloads/Leipzig2007.pdf}). Dem schlie\ss t sich \citet[48ff]{Reich:09} an.} 

\ex. \label{ex-koord-9}[Den Mitgliedern der Forstverwaltung widerfahren immer wieder neue Abenteuer.] \\
So entkam dem Förster jüngst in der Schonung ein Hase und begegnete ein Fuchs. \hfill
\citep[(44)]{Fortmann:05}
%\ex. Ich hab den Film schon und brauchst du mir deshalb nicht mitbringen.

Man beachte, dass die Objektlücke unproblematisch ist, falls eine $\kappa$-Instanzi"-ierung als Konstituentenkoordination\is{Koordination!Konstituenten-} zugrunde gelegt werden kann, z.\,B.\  in \ref{ex-koord-10}:

\ex. \label{ex-koord-10}Den Ochsen füttert Karl und tränkt Heinz. \hfill \citep[(29a)]{Hoehle:83b} 

Ich möchte es bei diesen Ausführungen zu anderen Vorwärtsellipsen bewenden lassen. Im Folgenden werden aus dieser Koordinationsklasse nur die Gappingfälle eine weitergehende Betrachtung erfahren.

%\subsubsection*{Interne Bedingungen}

Abgesehen von G1 unterliegt das Gapping weiteren internen Bedingungen, die von den externen, antezedensbezogenen Bedingungen zu unterscheiden sind (siehe S.\,\pageref{sec-ext-bedingungen}).\footnote{Diese Einteilung findet man auch bei \citet[(40)]{Wilder:97}, der zwischen "`conditions on ellipsis independent of the antecedent"' und "`conditions on the antecedent-ellipsis relation"' unterscheidet. Unter ersteren, die den internen Bedingungen entsprechen, befindet sich auch die "`Head Condition on FWD"' \citep[(54)]{Wilder:97}: "`An ellipsis site may not be c-commanded by an overt (non-deleted) head in its domain (=conjunct)."' Damit erfasst Wilder nicht nur Fälle von Gapping, sondern auch andere Fälle der Vorwärtsellipse. Ich formuliere hier bewu\ss t theorieneutralere Generalisierungen ohne Kopfbegriff\is{Kopf} und c-Kommando\is{c-Kommando}.} Die \textsc{internen Bedingungen} bestimmen die Bestandteile des zweiten Konjunkts, die getilgt werden dürfen bzw.\ übrig bleiben können. Letztere werden auch als Überbleibsel ("`remnants"') bezeichnet. Es gilt dabei, Daten wie in \ref{ex-koord-17} auszuschlie\ss en:   

\ex. \label{ex-koord-17}
\a. *Karl versteckt sich hinter einer Mülltonne und Peter \sout{versteckt sich hinter} einem Auto. \hfill \citep[148]{Hartmann:00}\label{ex-koord-17-a}
\b. \#[$\kappa_1$ Der Jäger sah drei Hasen] und [$\kappa_2$ der Tourist \sout{sah drei} Waschbären].\label{ex-koord-17-b}

Die PP\is{Pr\"aposition} {\it hinter einem Auto} wird in \ref{ex-koord-17-a} nur teilweise getilgt, ebenso in \ref{ex-koord-17-b} die Nominalphrase {\it drei Waschbären}. Problematisch daran ist, dass solche Tilgungen nicht rekonstruiert werden, obwohl jeweils im ersten Konjunkt ein passendes \isi{Antezedens} zur Verfügung steht. In \ref{ex-koord-17-a} geht der Bezug zwischen der Präposition {\it hinter} und dem Überbleibsel {\it einem Auto} vollständig verloren, so dass das Überbleibsel als einfaches Dativ-Objekt erscheint. Da jedoch {\it verstecken} mit einem Dativ-Objekt inkompatibel ist, erscheint infolgedessen Satz \ref{ex-koord-17-a} als ungrammatisch. Weniger schwer wiegt in dieser Hinsicht die fehlende Rekonstruktion des Numerals {\it drei} in \ref{ex-koord-17-a}. Das Überbleibsel {\it Wäschbären} gibt auch ohne das Numeral ein akzeptables Akkusativobjekt des Verbs {\it sah} ab. Eine mögliche Generalisierung der Eindrücke aus \ref{ex-koord-17} könnte also lauten:\is{Gappingregel!G2}
\begin{itemize}
  \item[] {\bf G2:} Die Ellipse in Satz $S$ schlie\ss t nur vollständige Satzglieder\is{Satzglied} von $S$ ein.\footnote{Im Rahmen der generativen Grammatik wird oft angenommen, dass die Überbleibsel sogenannte Hauptkonstituenten ("`major constituents"')\is{Major Constituent} des Satzes bilden müssen. Im Anschluss an \citet[Fußnote~2]{Hankamer:73} hat der Begriff der Hauptkonstituente diverse Konkretisierungen und Anpassungen erfahren z.\,B.\  bei \citet[110ff]{Neijt:79}, \citet[12ff]{Chao:87}, \citet[Fußnote 10]{Wilder:97}, \citet[145ff]{Hartmann:00}. Im Grunde handelt es sich bei einer Hauptkonstituente um eine Implementierung des Satzgliedbegriffs\is{Satzglied}, d.\,h.\ um eine regelhafte Identifizierung derjenigen Teilstrukturen der generativen Phrasenstruktur, die mögliche Überbleibsel repräsentieren können. Ich möchte an dieser Stelle nicht näher darauf eingehen und verweise auf Kapitel \ref{sec-ellipsenanalyse}, wo technischen Details mehr Platz eingeräumt werden soll. In diesem Kapitel gilt weiterhin der Satzgliedbegriff aus \ref{ex-satzglied} auf S.\,\pageref{ex-satzglied}.}
\end{itemize} 
Dies trifft tatsächlich in vielen Fällen zu und erfasst beispielsweise auch das Gappingverhalten bei kohärenten Konstruktionen\is{kohärente Konstruktion} wie in \ref{ex-koord-18}: 

\ex. \label{ex-koord-18}[$\kappa_1$ Peter hat das Fahrrad zu reparieren versprochen] und [$\kappa_2$ Susi \sout{hat} das Auto \sout{zu reparieren versprochen}].

Eine alternative Generalisierung mit Bezug auf einen einzelnen Valenzrahmen, etwa den des maximal übergeordneten Verbs, würde hier nicht ausreichen: Das Überbleibsel {\it das Auto} ist nur ein Teil der \isi{Ergänzung} des infiniten Verbs {\it versprochen}. Man beachte au\ss erdem, dass der \isi{Verbalkomplex} partiell tilgbar zu sein scheint:

\ex. [$\kappa_1$ Peter hat das Fahrrad zu reparieren versprochen]\label{ex-koord-verbalkomplex}
\a. und [$\kappa_2$ Susi \sout{hat das Fahrrad zu reparieren} versucht].
\b. ?und [$\kappa_2$ Susi \sout{hat das Fahrrad} zu benutzen \sout{versprochen}].
\c. ??und [$\kappa_2$  Susi \sout{hat} das Auto \sout{zu reparieren} versucht].

Dies zeigt, dass auch die Verbalkomplexglieder zumindest in manchen Fällen als Satzglieder\is{Satzglied} klassifiziert werden sollten, was sie meiner Satzglied-Defini"-tion in \ref{ex-satzglied} entsprechend auch werden.  

Aber auch die vorgeschlagene Bezugnahme auf Satzglieder in G2 kann nicht alle Gappingfälle erfassen. Damit sind aber nicht die "`Quasi-Pronominalisierung"'\is{Quasi-Pronominalisierung} \citep{Kunze:72} in \ref{ex-koord-19-a} gemeint, welche so $\kappa$-instanziiert werden könnte, dass die Nominalphrase {\it andere Bücher} partiell getilgt zu sein scheint. Die Annahme der partiellen Tilgung von Nominalphrasen (auch bekannt als NP-Ellipse\is{Ellipse!NP-}) ist nicht zwingend notwendig und zudem nicht immer plausibel (vgl.\ \citealt[781f]{Klein:93}), siehe \ref{ex-koord-19-b}:

\ex. \label{ex-koord-19}
\a. \label{ex-koord-19-a}[$\kappa_1$ Einige Bücher stehen hier] und [$\kappa_2$ andere \sout{Bücher stehen} dort]. \hfill \citep[(3-160)]{Hesse:Kuestner:85}
\b. \label{ex-koord-19-b}Mein Buch steht hier und *dein/deines dort. \hfill \citep[(3-161)]{Hesse:Kuestner:85}
%\c. Er unterstützt mein und hilft deinem Kind. \hfill \citep[(3-163)]{Hesse:Kuestner:85}

Ungleich problematischer ist, dass die partielle Tilgung von PPs in manchen Fällen möglich zu sein scheint. In \ref{ex-koord-28} sollen beispielsweise nur noch die Präpositionen\is{Pr\"aposition} der PPs zu sehen sein. Während man in \ref{ex-koord-28-a} noch argumentieren kann, dass {\it ohne} keine Präposition ist, sondern den Status einer adverbialen Bestimmung hat, ist das in \ref{ex-koord-28-b} und \ref{ex-koord-28-c} unplausibel:\footnote{Den spiegelbildlichen Fall, bei dem die Nominalphrase realisiert ist und die Präposition fehlt, kann ich für das Deutsche nicht bestätigen. Für das Englische kursieren in der Literatur dagegen die Daten in \ref{ex-koord-29}:\\
\fnex{
\ex. \label{ex-koord-29}
\a. [$\kappa_1$ Fred has been working on semantics] and [$\kappa_2$ Bill \sout{has been working on} syntax]. \hfill \citep[(6)]{Hudson:89}
\b. [$\kappa_1$ Many men like golf] and [$\kappa_2$ \sout{many} women like gardening]. \hfill \citep[(59-a)]{Hudson:88}

}}

\ex. \label{ex-koord-28}
\a. \label{ex-koord-28-a} Peter hat das Auto mit Werkzeug repariert und Susi \sout{hat das Auto} ohne \sout{Werkzeug repariert}.
\b. \label{ex-koord-28-b}Karl versteckt sich hinter einer Mülltonne und Peter \sout{versteckt sich} in \sout{einer Mülltonne}.
\b. \label{ex-koord-28-c} Peter sagt, die Rohre gehören über den Putz, und Martin sagt, \sout{die Rohre gehören} unter \sout{den Putz}. \hfill \citep[149, Fußnote 6]{Hartmann:00}
%\ex. \label{ex-koord-28-c} weil Peter einsteigt und Hans aus\sout{steigt}. \hfill \cite[151]{Hartmann:00}

Um solche partiellen PP-Ellipsen zu erfassen, müsste und könnte G2 angepasst werden. Darauf möchte ich hier aber verzichten, da dies eine Verkomplizierung bedeuten würde, die in Kauf zu nehmen wäre, ohne dieses Phänomen genauer verstanden zu haben. Abgesehen von Hinweisen findet sich nämlich in der Literatur meines Wissens keine umfangreichere Untersuchung zu diesem Thema.  

Ähnlich problematisch für G2 ist ein Phänomen, das \cite{Little:10} als \textsc{satzübergreifendes Gapping}\is{Ellipse!Gapping} ("`cross-clausal gapping"') bezeichnet und mit den Daten in \ref{ex-koord-20} veranschaulicht: 

\ex. \label{ex-koord-20}
\a. \label{ex-koord-20-a}[$\kappa_1$ John hopes the Bills win], and [$\kappa_2$ Fred \sout{hopes} the Colts \sout{win}]. \hfill \citep[(20a)]{Little:10}
\b. \label{ex-koord-20-b}[$\kappa_1$ Robin knows a lot of reasons why dogs are good pets], and [$\kappa_2$ Leslie \sout{knows a lot of reasons why} cats \sout{are good pets}].	\hfill \citep[273]{Culicover:Jackendoff:05}

Die Überbleibsel {\it the Colts} und {\it cats} sind keine Satzglieder des Konjunktsatzes, sondern Satzglieder eines darin eingebetteten Gliedsatzes. Dieses Gappingpotential kann auch im Deutschen beobachtet werden: 

\ex. \label{ex-koord-21}
\a. \label{ex-koord-21-a}[$\kappa_1$ Peter versprach, das Fahrrad zu reparieren] und [$\kappa_2$ Susi \sout{ver"-sprach} das Auto \sout{zu reparieren}].
\b. \label{ex-koord-21-b}[$\kappa_1$ Peter versprach, dass das Fahrrad repariert wird] und [$\kappa_2$ Susi \sout{versprach, dass} das Auto \sout{repariert wird}].
\c. \label{ex-koord-21-c}[$\kappa_1$ Peter freut sich, wenn das Fahrrad repariert wird], und [$\kappa_2$ Susi \sout{freut sich, wenn} das Auto \sout{repariert wird}].    

In den Beispielen in \ref{ex-koord-21} ist jeweils {\it das Auto} Satzglied eines eingebetteten Gliedsatzes, nämlich eines inkohärenten Verbalfelds\is{inkohärente Konstruktion} in \ref{ex-koord-21-a}, eines \emph{dass}-Satzes\is{Satz!\textit{dass}-} in \ref{ex-koord-21-b}, oder eines Konditionalsatzes\is{Satz!Konditional-} in \ref{ex-koord-21-c}. Diese Beobachtungen (bzw.\ Grammatikalitätsurteile) scheinen in einem Widerspruch zu den Einschätzungen zu stehen, die z.\,B.\  bei \cite{Neijt:79} geäu\ss ert werden:

\ex. \label{ex-koord-22}
\a. *John discussed the question of which roses are to be planted and Peter \sout{discussed the question} (of) which appletrees \sout{are to be planted}.\label{ex-koord-22-a}
\b. *John asked what to write to Mary and Peter \sout{asked what to write} to Sue.\label{ex-koord-22-b}
\z. \citep[153]{Neijt:79}  

Neijt sieht in der Ungrammatikalität der Daten in \ref{ex-koord-22} einen Beleg dafür, dass Bewegungsinseln\is{Bewegungsinsel} (z.\,B.\  \textit{wh}-Sätze und komplexe NPs) auch beim Gapping eine Rolle spielen, indem diese nur vollständig getilgt werden können. Doch das ist insbesondere mit \ref{ex-koord-20-b} überhaupt nicht vereinbar. Das satzübergreifende Gapping  im Englischen ist also einigerma\ss en umstritten:\footnote{Die Uneinheitlichkeit der Grammatikalitätsurteile setzt sich beim Gapping mit nur einem Überbleibsel, auch "`bare argument ellipsis"'\is{Ellipse!Bare Argument Ellipsis} oder "`stripping"' genannt, fort. \citet[688f]{Merchant:04} findet Adjazenzellipsen\is{Ellipse!Adjazenz-}, in denen das Überbleibsel aus Bewegungsinseln\is{Bewegungsinsel} stammt, ungrammatisch, während  \citet[244f]{Culicover:Jackendoff:05} solche Fälle bei bestimmter Akzentuierung des \isi{Antezedens} akzeptieren. Streitpunkt ist etwa das Datum in \ref{ex-bae-islands-1}:\\
\fnex{
\ex. \label{ex-bae-islands-1}
\a. Does Abby speak the same Balkan language that Ben speaks?
\b. (*) No, Charly.
\c. No, she speaks the same Balkan language that Charly speaks.
\z. (\citealt[(87)]{Merchant:04}; \citealt[245]{Culicover:Jackendoff:05})

}
Die deutschen Entsprechungen sind in meinen Augen akzeptabel.} Da ich aber an den deutschen Exemplaren in \ref{ex-koord-21} nichts Schlechtes finden kann, schlage ich eine Anpassung von G2 vor, welche satzübergreifendes Gapping zulässt:\is{Gappingregel!G2$^+$} 
\begin{itemize}
  \item[] {\bf G2$^+$:} Die Ellipse in Satz $S$ schlie\ss t nur vollständige Satzglieder\is{Satzglied} von $S$ ein, au\ss er es handelt sich um Teilsätze.
\end{itemize}
Die partielle Ellidierung satzförmiger Satzglieder schlie\ss t z.\,B.\  die partielle Ellidierung von Relativsätzen\is{Satz!Relativ-} aus, wenn sie als Bestandteil einer komplexen NP betrachtet werden können. Bestätigt wird das durch die Unakzeptabilität der Sätze in \ref{ex-koord-relativsatz}:

\ex. \label{ex-koord-relativsatz}
\a. ??Claudia, die einen Schäferhund hat, soll mitkommen und Annika, \sout{die} einen Beagle \sout{hat, soll mitkommen}.
\b. ??Jeder, der einen Schäferhund hat, soll mitkommen und jeder, \sout{der} einen Beagle \sout{hat, soll mitkommen}.
\c. ??Jeder soll mitkommen, der einen Schäferhund hat, und jeder \sout{soll mitkommen, der} einen Beagle \sout{hat}.\label{ex-koord-relativsatz-c}

Hier zeigt sich auch, dass es keine Rolle spielt, ob der Relativsatz restriktiv oder nicht-restriktiv ist, oder ob er extraponiert wurde wie in \ref{ex-koord-relativsatz-c}. 

Abschlie\ss end soll eine Beobachtung im Zusammenhang mit \emph{dass}"=Komplementierern\is{Komplementierer} in die Form einer Gappingregel gebracht werden:\is{Gappingregel!G3}    
\begin{itemize}  
  \item[] {\bf G3:} Die Ellipse in Satz $S$ schlie\ss t den \emph{dass}-Komplementierer in $S$ ein.
%  \item[G3:] Wird der \emph{dass}-Komplementierer in Satz $S$ realisiert, dann gibt es keine Ellipse in $S$.  
\end{itemize}
Die Gappingregel G3 erklärt die Unakzeptabilität der Daten in \ref{ex-koord-6-c2}, für die jeweils der overte \emph{dass}-Komplementierer im zweiten Konjunkt verantwortlich zu machen ist:  

\ex. \label{ex-koord-6-c2}
\a. *[$\kappa_1$ dass der Jäger einen Hasen sah] und [$\kappa_2$ dass der Tourist drei Waschbären \sout{sah}]\label{ex-koord-6-c2-a}
\b. *[$\kappa_1$ Peter versprach, dass das Fahrrad repariert wird] und [$\kappa_2$ Susi \sout{versprach}, dass das Auto \sout{repariert wird}].\label{ex-koord-6-c2-b} 

Ich beschränke mich hier ausdrücklich auf \emph{dass}-Komplementierer und bleibe bis auf Weiteres agnostisch in Bezug auf andere Exemplare der Komplementiererklasse. 

%\subsubsection*{Externe Bedingungen}

Die \textsc{externen Bedingungen} \label{sec-ext-bedingungen} für das Gapping bestimmen die Parallelität von Ellipse und Antezedens, also diejenigen morpho-syntaktischen Aspekte, in denen Ellipse und Antezedens divergieren können oder identisch sein müssen.\footnote{Auf der anderen Seite legt eine interessante Beobachtung bei \cite{Oirsouw:87} nahe, dass sich Überbleibsel und Antezedens in bestimmten Aspekten unterscheiden müssen:\\
\fnex{
\ex. *John schlägt den Hund und John die Katze. \hfill \citep[151]{Oirsouw:87}\label{ex-oirsou-151}

}
Ausschlagebend für die Unakzeptabilität von \ref{ex-oirsou-151} ist offensichtlich, dass das Überbleibsel {\it John} über ein koreferentielles Antezedens im ersten Konjunkt verfügt und beide dieselbe semantische Rolle innehaben. Divergiert die \isi{semantische Rolle} dagegen, dann stellt \isi{Koreferenz} kein Problem mehr dar:\\  
\fnex{
\ex. John schlägt den Esel und der Esel daraufhin John.

}} Verben können in folgender Hinsicht divergieren: \isi{Numerus} (in \ref{ex-koord-23-a}), \isi{Person} (in \ref{ex-koord-23-b}), \isi{Genus Verbi} (in \ref{ex-koord-23-c}) und \isi{Valenzrahmen} (in \ref{ex-koord-23-d}):

\ex. 
\a. \label{ex-koord-23-a}[$\kappa_1$ Gestern sah der Jäger einen Hasen] und [$\kappa_2$ die Touristen \sout{sahen} drei Waschbären].
\b. \label{ex-koord-23-b}[$\kappa_1$ Gestern sah der Jäger einen Hasen] und [$\kappa_2$ du \sout{sahst} drei Waschbären]. 
\c. \label{ex-koord-23-c}[$\kappa_1$ Das Fahrrad wird repariert] und [$\kappa_2$ Peter \sout{wird} dafür bezahlen].
\d. \label{ex-koord-23-d}[$\kappa_1$ Philipp dehnte sich zweimal von Kopf bis Fuß], und [$\kappa_2$ Timm \sout{dehnte sich} noch nicht mal den kleinen Zeh].    

Dagegen scheint das Tempusmerkmal\is{Tempus} immer identisch rekonstruiert zu werden, d.\,h.\ in \ref{ex-koord-24} kann das Präteritum des Antezedens-Verbs nicht in der Ellipse als Präsens rekonstruiert werden:

\ex.  \#[$\kappa_1$ Letztes Semester nahm Peter an einem Tanzkurs teil] und [$\kappa_2$ dieses Semester \sout{nimmt Peter} vielleicht auch \sout{an einem Tanzkurs teil}].\label{ex-koord-24}

Bei den Nominalphrasen können Divergenzen im \isi{Kasus} beobachtet werden, sofern wie in \ref{ex-koord-25-a} und \ref{ex-koord-25-b} ein entsprechender Kasus-Synkretismus vorliegt. Es hat also den Anschein, dass Nominalphrasen, anders als Verben, phonetisch identisch rekonstruiert werden. Satz \ref{ex-koord-25-c} ist also deswegen inakzeptabel, weil die Antezendens-NP ebenso wie die Ellipse-NP eindeutig die Kasusmorphologie des Dativs trägt, während {\it erfreut} ein Akkusativobjekt benötigt:

\ex. \label{ex-koord-25}
\a. Käse mag ich nicht und ist auch nicht gut für mich. \hfill \citep[(36)]{Oirsouw:93}\label{ex-koord-25-a}
\b. [$\kappa_1$ Die Spielsachen haben Johannes gefallen] und [$\kappa_2$ \sout{die Spielsa"-chen haben Johannes} später auch erfreut].\footnote{Das Beispiel ist eine Abwandlung von (3-165) aus \cite{Hesse:Kuestner:85}. }\label{ex-koord-25-b} 
\c. *[$\kappa_1$ Die Spielsachen haben dem Kind gefallen] und [$\kappa_2$ \sout{die Spielsa"-chen haben dem Kind} später auch erfreut].\label{ex-koord-25-c} 

Auch eine Linearisierungsdiskrepanz zwischen Ellipse und Antezedens ist möglich, anders als \cite{Ross:70} behauptet. \cite{Oirsouw:87} beweist das mit folgendem Datum:

\ex. Den Gro\ss en nehme ich, und du \sout{nimmst} den Kleinen. \hfill \citep[262]{Oirsouw:87}

\is{Ellipse!Gapping|)}


\section{Right-Node-Raising}\label{sec-rnr}\is{Koordination!Right-Node-Raising|(}

Die zweite prominente Koordinationsellipse der Taxonomie in Tabelle \ref{tab-koord-typen} ist das sogenannte Right-Node-Raising (RNR).\footnote{Die erste Beschreibung des Phänomens im Rahmen der generativen Grammatik findet man bei \citet[174ff]{Ross:67}, obgleich zuvor schon Noam Chomsky darüber gestolpert war und es als ungrammatisch abtat \citep[35, Fußnote 2]{Chomsky:57}. Der Terminus selber stammt aus \cite{Postal:74} und reflektiert die dort (und bei Ross) vorgeschlagene Bewegungsanalyse.} Die wesentliche Eigenschaft des RNR ist die Unvollständigkeit des ersten Konjunkts bei einer $\kappa$-Instanziierung als Satzkoordination.\is{Ellipse!Rückwärts-} Die Vollständigkeit oder Unvollständigkeit des zweiten Konjunkts ist dabei zunächst unerheblich.\footnote{Diese Bestimmung von RNR verhält sich meiner Einschätzung nach äquivalent zur Bestimmung bei \cite{Hartmann:00}, wo es hei\ss t: "`Characteristically, some element is shared by all the conjuncts. This element appears at the right periphery of the last conjunct and is phonologically empty in the same position (i.e. the right edge) of the preceding conjuncts."' (S.\,53)} Zwei Beispiele für RNR sind in \ref{ex-koord-11} mit vollständigem zweiten Konjunkt und in \ref{ex-koord-12} mit unvollständigem zweiten Konjunkt angegeben: 

\exi. \label{ex-koord-11} Der Tourist darf und der J\"ager muss die Wildschweine erschie\ss en.
\a. [$\kappa_1$ Der Tourist darf] und [$\kappa_2$ der J\"ager muss] die Wildschweine erschie\ss en.
\b. [$\kappa_1$ Der Tourist darf \sout{die Wildschweine erschie\ss en}] und [$\kappa_2$ der J\"ager muss die Wildschweine erschie\ss en].

\exi. \label{ex-koord-12} Der J\"ager hat heute einen Hasen und gestern drei Waschb\"aren gesehen.
\a. Der J\"ager hat [$\kappa_1$ heute einen Hasen] und [$\kappa_2$ gestern drei Waschb\"aren] gesehen.
\b. [$\kappa_1$ Der J\"ager hat heute einen Hasen \sout{gesehen}] und [$\kappa_2$ \sout{der Jäger hat}  gestern drei Waschb\"aren gesehen].

Beide Beispiele sind sowohl als Satzkoordination als auch als Konstituentenkoordination $\kappa$-instanziierbar. Dies muss jedoch nicht so sein, wie das Beispiel in \ref{ex-koord-13} beweist. Hier steht nur eine $\kappa$-Instanziierung als Satzkoordination zur Verfügung: 

\ex. \label{ex-koord-13}Der Jäger hat einen Hasen und der Tourist drei Waschbären gesehen.
\a. [$\kappa_1$ Der Jäger hat einen Hasen] und [$\kappa_2$ der Tourist drei Waschbären] gesehen. \\
$\leadsto$ *Der Tourist drei Waschbären gesehen.
\b. [$\kappa_1$ Der Jäger hat einen Hasen \sout{gesehen}] und [$\kappa_2$ der Tourist \sout{hat} drei Waschbären gesehen].

Da aber beide Konjunkte in \ref{ex-koord-13} unvollständig sind, könnte man vermuten, dass zumindest RNR-Fälle mit einem vollständigen zweiten Konjunkt immer als Konstituentenkoordination\is{Koordination!Konstituenten-} $\kappa$-instanziierbar sind. Ich sehe nichts, was gegen diese Vermutung spricht.\footnote{Folgendes RNR-Datum aus \citet[(23a)]{Cann:etal:05b} könnte aus der Reihe tanzen:\\
\fnex{
\ex. \label{ex-koord-14}John read, but he hasn't understood any of my books.
\a. [$\kappa_1$ John read], but [$\kappa_2$ he hasn't understood] any of my books.\\
$\leadsto$ *John read any of my books.
\b. [$\kappa_1$ John read \sout{some of my books}], but [$\kappa_2$ he hasn't understood any of my books].\\
$\leadsto$ John read some of my books.

}
Die $\kappa$-Instantiierung als Konstituentenkoordination scheitert an der fehlenden Lizenzierung des NPIs {\it any of my books} durch das erste Konjunkt. Andererseits muss bei der Satzkoordination eine neuartige Diskrepanz zwischen Ellipse und Antezedens stipuliert werden, damit die $\kappa$-Instanziierung funktioniert: Ellipse und Antezendent unterscheiden sich in der \isi{Polarität}, so dass die Ellipse im ersten Konjunkt das PPI {\it some of my books} enthält. \citet[81f]{Ha:08} sieht dieses Vorgehen durch vergleichbare VPE-Daten gestützt und verweist au\ss erdem auf \cite{Klima:64}, der für ein morphologisches Verständnis der Polarität eintritt. Damit rückte die Polaritätsdiskprepanz in die Nähe der Numerusdiskrepanz, die in \ref{ex-koord-4} zu sehen ist. Übrigens nutzen diese Flexibilität der Satzkoordination auch \cite{Cann:etal:05b}, um \ref{ex-koord-14} im Rahmen ihres dynamischen Ansatzes zu modellieren, wenngleich dort Ellipse und Antezedens nur semantisch verlinkt werden.
}

Der Ellipse des ersten Konjunkts einer RNR-Koordination kann man au\ss erdem folgende Eigenschaften zusprechen:
\begin{enumerate}
  \item[] {\bf RNR1} Die Ellipse befindet sich an der rechten Peripherie des ersten Konjunkts.
  \item[] {\bf RNR2} Die Ellipse hat einen beliebigen Umfang und ignoriert Konstituentengrenzen.
  \item[] {\bf RNR3} Das Antezedens der Ellipse ist ein kontinuierlicher Bereich an der rechten Peripherie des zweiten Konjunkts.  
\end{enumerate}
RNR1 drückt sich in der Unakzeptabilität der RNR-Konstruktion in \ref{ex-koord-151} mit einer medialen Weglassung des finiten Verbs {\it hat} aus:\footnote{Ein vermeintliches Gegenbeispiel ist die Antwort in \ref{ex-rnr-gegenbsp}:\\
\fnex{
\ex. \label{ex-rnr-gegenbsp}Wer hat was gesehen?  \\
     Der Jäger \sout{hat} einen Hasen \sout{gesehen} und der Tourist \sout{hat} drei Waschbären \sout{gesehen}.

}
Bei den Ellipsen im ersten Konjunkt handelt es sich aber nicht um Koordinationsellipsen, sondern um Adjazenzellipsen\is{Ellipse!Adjazenz-} (siehe Abschnitt \ref{sec-adjazenzellipsen}), denn das \isi{Antezedens} befindet sich nicht im zweiten Konjunkt, sondern in der vorangehenden Frage.}

\ex. *[$\kappa_1$ Der Jäger \sout{hat} einen Hasen \sout{gesehen}] und [$\kappa_2$ der Tourist hat drei Waschbären gesehen].\label{ex-koord-151}

Interessanterweise gilt diese Einschränkung nicht für das zweite Konjunkt, wie man in \ref{ex-koord-13} sehen konnte.\footnote{Die Tilgungen im zweiten Konjunkt sind wahrscheinlich ähnlichen Einschränkungen unterworfen wie beim Gapping\is{Ellipse!Gapping}. Dies lässt beispielsweise \ref{ex-koord-15i} vermuten, wo (analog zu \ref{ex-koord-17-a}) das Überbleibsel {\it einem Mülleimer} kein vollständiges \isi{Satzglied} bildet und der Satz dadurch unakzeptabel erscheint:\\  
\fnex{
\ex. *\label{ex-koord-15i}[$\kappa_1$ Hinter einer Mülltonne hat sich Karl \sout{versteckt} und [$\kappa_2$ \sout{hinter} einem Mülleimer \sout{hat sich} Peter versteckt].

}} Die zweite Eigenschaft, RNR2, erlaubt es, einen beliebigen Bereich an der rechten Peripherie des ersten Konjunkts zu tilgen, ohne auf Konstituentengrenzen Rücksicht zu nehmen. Das führt dazu, dass auch Konstituententeile getilgt werden können, also z.\,B.\  ein Teil des einbetteten Relativsatzes\is{Satz!Relativ-} in \ref{ex-koord-15-a}, oder ein Teil des \emph{dass}-Satzes\is{Satz!\textit{dass}-} in \ref{ex-koord-15-b}, oder auch ein Teil einer PP\is{Pr\"aposition} in \ref{ex-koord-15-c}:  

\ex. \label{ex-koord-15} 
\a. \label{ex-koord-15-a}[$\kappa_1$ Er führt zu einem Kunden, dessen Radio \sout{defekt ist}], und [$\kappa_2$ \sout{er führt} zu einer Kundin, dessen Fernsehgerät defekt ist]. \\ \citep[(3-101)]{Hesse:Kuestner:85}
\b. \label{ex-koord-15-b}[$\kappa_1$ Ich glaube, dass Hans \sout{den Berg bestieg}], und [$\kappa_2$ Susi behauptet, dass Maria den Berg bestieg].  
\c. \label{ex-koord-15-c}[$\kappa_1$ Das eine Buch lag auf \sout{dem Tisch}] und [$\kappa_2$ das andere \sout{lag} unter dem Tisch]. 	\hfill \citep[150]{Lobin:93}

Solche Ellipsen, wie sie im ersten Konjunkt auf"|treten, sind beim Gapping\is{Ellipse!Gapping} vollkommen ausgeschlossen: \ref{ex-koord-15-a} und \ref{ex-koord-15-c} versto\ss en gegen G1\is{Gappingregel!G1} und G2\is{Gappingregel!G2},\footnote{Wobei man streiten kann, ob die partielle Ellipse der PP wirklich mit dem Gapping unvereinbar ist. Siehe die Diskussion oben im Zusammenhang mit \ref{ex-koord-28}.} und \ref{ex-koord-15-b} verstö\ss t gegen G3\is{Gappingregel!G3}.  Die dritte Eigenschaft, RNR3, beschränkt schlie\ss lich das Antezedens im zweiten Konjunkt derart, dass die Unakzeptabilität von \ref{ex-koord-16} nur folgerichtig ist:

\ex. *Der Tourist darf \sout{die Wildscheine erschie\ss en} und die Wildschweine muss der J\"ager erschie\ss en.\label{ex-koord-16} 

RNR1 und RNR3 wirken an sich willkürlich und unmotiviert. Einen Zusammenhang erkennt man erst, wenn man auf die damit lizenzierten $\kappa$"=Instanziierungen blickt: Diese Ellipseeigenschaften verhindern RNR-Koordinationen, die nicht als Konstituentenkoordination $\kappa$-instanziierbar sind. Mit anderen Worten: Man gewinnt den Eindruck, dass die RNR-Koordina\-tion eigentlich eine originäre Konstituentenkoordination ist.\footnote{Möglicherweise ist diese Generalisierung zu strikt. \cite{Hudson:76} zeigt anhand der folgenden Daten für das Englische, dass RNR nicht auf Koordinationsellipsen\is{Ellipse!Koordinations-} beschränkt ist:\\
\fnex{
\ex. \label{ex-hudson76-5}
\a. \label{ex-hudson76-5-a}  Of the people questioned, those who liked outnumbered by two to one those who disliked the way in which the devaluation of the pound had been handled. 
\b. \label{ex-hudson76-5-b} I'd have said he was sitting on the edge of rather than in the middle of the puddle.
\c. \label{ex-hudson76-5-c} It's interesting to compare the people who like with the people who dislike the power of the big unions. 
\z. \citep[(5)]{Hudson:76}  
  
}  
Satz \ref{ex-hudson76-5-b} enthält eine Komparationsellipse\is{Ellipse!Komparations-}, die man auch als Koordinationsellipse mit dem Koordinator {\it rather than} auf"|fassen könnte. In den Sätze \ref{ex-hudson76-5-a} und \ref{ex-hudson76-5-c} lässt sich beim besten Willen kein Koordinator identifizieren, trotzdem ist der jeweils erste Relativsatz\is{Satz!Relativ-} elliptisch. Vorausgesetzt, dass diese RNR-Form auch im Deutschen möglich ist (eine strukturanaloge Übersetzung von \ref{ex-hudson76-5-a} und \ref{ex-hudson76-5-c} will mir nicht gelingen), müsste man also RNR-Ellipse als originäre Ellipsen mit Antezedens bezeichnen. Dadurch würde jedoch der grundsätzliche Unterschied zwischen RNR-Ellipse und Vorwärtsellipsen\is{Ellipse!Vorwärts-} nicht verschwinden.} Diesen Eindruck trübt jedoch die Möglichkeit einer medialen Ellipse im zweiten Konjunkt (vgl.\ \ref{ex-koord-13} und \ref{ex-koord-15-c}), durch die eine $\kappa$-Instanziierung als Konstituenten verhindert wird. In dieser Hinsicht fehlt also eine vierte Eigenschaft, nämlich die Ellipse im zweiten Konjunkt und das Antzedens im ersten Konjunkt jeweils auf die linke Peripherie zu beschränken. Wie bereits erwähnt, handelt es sich hier möglicherweise um ein Mischphänomen aus RNR im ersten Konjunkt und Gapping\is{Ellipse!Gapping} im zweiten Konjunkt.
\is{Koordination!Right-Node-Raising|)}





\section{Adjazenzellipsen} \label{sec-adjazenzellipsen}\is{Ellipse!Adjazenz-|(}

Üblicherweise wird der Gegensatz zwischen Koordinationsellipse und Adjazenzellipse in der relativen Platzierung von Ellipse und Antezedens festgemacht: Während sich Ellipse und \isi{Antezedens} bei Koordinationsellipsen im selben \isi{Satz} befinden sollen, tun sie das bei Adjazenzellipsen nicht (siehe Abschnitt \ref{sec-ellipse-taxonomie}). Hier sind sie auf unterschiedliche, meist adjazente Sätze oder zumindest "`eng zusammengehörige Äu\ss erungen"' \citep[768]{Klein:93} verteilt. Dazu zählt \cite{Klein:93} konventionalisierte Äu\ss erungsfolgen wie die \isi{Frage-Antwort-Folge}\footnote{Der Satzbegriff wird bisweilen so weit gedehnt, dass auch Frage-Antwort-Folgen als Sätze kategorisiert werden. Ein Beispiel dafür liefert \cite{Kindt:85} mit der folgenden Satzdefinition: "`Der Satz ist die kleinste selbständige sprachliche Einheit, d.\,h.\ da\ss \ deren Vertreter in Texten frei vorkommen können."' (S.185)} in \ref{ex-adjazenz-1-a}, die partielle Korrektur in \ref{ex-adjazenz-1-b}, die partielle Bestätigung in \ref{ex-adjazenz-1-c}, und die parallele Fortführung in \ref{ex-adjazenz-1-d}:  

\ex. \label{ex-adjazenz-1}
\a. \label{ex-adjazenz-1-a} Wer schlug wen wo? \\
Alexander \sout{schlug} die Perser bei Issos.
\b. \label{ex-adjazenz-1-b} Otto hat hundert Mark gewonnen.\\
(Nein,) Peter \sout{hat} tausend verloren.
\c. \label{ex-adjazenz-1-c} Otto hat im Lotto gewonnen. \\
(Ja,) \sout{Otto hat} fast eine Million \sout{gewonnen}.
\d. \label{ex-adjazenz-1-d} Ich komme heute abend. \\
Ich \sout{komme} auch. \\
Ich \sout{komme} nicht.
\z. \citep[768]{Klein:93} 

In all diesen Beispielen gibt es eine Art Vorwärtsellipse\is{Ellipse!Vorwärts-}, da das Antezedens der Ellipse vorausgeht. Vermutlich ist das bei Adjazenzellipsen immer der Fall. Die interessante Frage ist nun, ob die Adjazenzellipsen denselben Einschränkungen unterworfen sind, die wir bei den Vorwärtsellipsen in Koordinationen, d.\,h.\ bei Gapping\is{Ellipse!Gapping} und Vorfeldellipsen\is{Ellipse!Vorfeld-}, feststellen konnten. Die Antwort fällt jedoch uneinheitlich aus. 

Für Frage-Antwort-Folgen scheint dies weitgehend zu gelten. Zunächst bestätigen die Fra\-ge-Antwort-Folgen in \ref{ex-adjazenz-2-1}--\ref{ex-adjazenz-2-3}, dass sowohl \isi{Valenzträger} als auch Ergänzungen\is{Ergänzung} weggelassen werden können: 

\ex. \label{ex-adjazenz-2-1} Alexander schlug wen wo? \\
\sout{Alexander schlug} die Perser bei Issos.

\ex. \label{ex-adjazenz-2-2} Wer schlug die Perser wo? \\
 Alexander \sout{schlug die Perser} bei Issos.

\ex. \label{ex-adjazenz-2-3} Alexander schlug die Perser wo? \\
 \sout{Alexander schlug die Perser} bei Issos.

Gilt dabei G1\is{Gappingregel!G1}? Umfassen Ellipsen also immer auch das finite Verb (ausgenommen Vorfeldellipsen)? Um dies im Rahmen von Frage-Antwort-Folgen\is{Frage-Antwort-Folge} klären zu können, muss nach dem Finitum gefragt werden, was man beispielsweise mit Hilfe der Konstruktion {\it etwas machen mit jemandem} tun kann.\footnote{Eine Bemerkung zur syntaktischen Parallelität zwischen Frage und Antwort: Die Frage-Antwort-Folgen mit der Konstruktion {\it etwas machen mit jemanden} deuten an, dass Fragen und Antworten keinesfalls immer syntaktisch parallel aufgebaut sind. Diese syntaktische Divergenz ist noch ausgeprägter in Fällen, wo Antworten aus Fragen inferiert werden. \cite{Schlangen:03} gibt hierfür das Beispiel in \ref{ex-schlangen}:\\
\fnex{
\ex. \label{ex-schlangen}Warum sollte ich Seegra\ss\ essen?\\
\sout{Es enthält} Viele Vitamine. \\
(vgl.\ \citealt[(1)]{Schlangen:03}) 

}
\cite{Schlangen:03} nennt diesen Fall "`resolution-via-inference"' und grenzt ihn gegenüber "`resolution-via-identiy"' ab. Bei partiellen Korrekturen ist eine kausale \isi{Inferenz} dagegen nur schwer herzustellen:\\ 
\fnex{
\ex. Seegra\ss\ ist ungesund. \\
\#Nein, \sout{es enthält} viele Vitamine!

}} Die Unakzeptabilität der entsprechenden Antwortsätze in \ref{ex-adjazenz-2-7}, \ref{ex-adjazenz-2-8} und \ref{ex-adjazenz-2-6-b} ist offensichtlich:   

\ex. \label{ex-adjazenz-2-7} Was machte Alexander mit den Persern bei Issos? \\
*\sout{Alexander} schlug \sout{die Perser bei Issos}. 

\ex. \label{ex-adjazenz-2-8} Was machte Alexander mit den Persern wo? \\
*\sout{Alexander} schlug \sout{die Perser}  bei Issos.

\ex. \label{ex-adjazenz-2-6} Was machte Alexander bei Issos? 
\a.  *Bei Issos schlug \sout{Alexander} die Perser.\label{ex-adjazenz-2-6-b}
\b.  \sout{Alexander} schlug die Perser.\label{ex-adjazenz-2-6-a}

Tritt das Finitum als Überbleibsel in Erscheinung, dann liegt entweder, wie in \ref{ex-adjazenz-2-6-a}, eine Vorfeldellipse\is{Ellipse!Vorfeld-} oder, wie in  \ref{ex-adjazenz-3-a}, ein VE-Satz\is{Satz!VE-} vor:

\ex. \label{ex-adjazenz-3}Kallisthenes berichtet, dass Alexander was mit wem bei Issos machte?
\a. \sout{dass Alexander} die Perser schlug.\label{ex-adjazenz-3-a}
\b. *dass \sout{Alexander} die Perser schlug.\label{ex-adjazenz-3-b}

An dem Antwortsatz in \ref{ex-adjazenz-3-b} wird zudem deutlich, dass die Ellipse immer auch den \isi{Komplementierer} {\it dass} umfasst, dass also G3\is{Gappingregel!G3} gilt.

Auf den Zusammenhang zwischen Vorfeldellipsen in Koordinationen, dort besser bekannt als Subjektlücken\is{Subjektlücke}, und freien Vorfeldellipsen wie in \ref{ex-fries-88} wird überraschenderweise recht selten hingewiesen:\footnote{Für solche freien Vorfeldellipsen\is{Ellipse!Vorfeld-} kursieren unterschiedliche Bezeichnungen: \cite{Ross:82} nennt dieses Phänomen "`pronoun zap"' (siehe \citealt{Huang:84}), \cite{Oirsouw:87} "`topic drop"'\is{Topic Drop} und \cite{Fries:88} "`Null-Topik"'\is{Null-Topik}.}\footnote{Eine erfreuliche Ausnahme macht diesbezüglich \citet[136ff]{Oirsouw:87}. \citet[153]{Reich:09} sieht jedoch drei wesentliche Unterschiede: (i) Vorfeldellipsen in Koordinationen\is{Koordination!SLF-} ("`SLFK"') erlauben keinen Sprecherwechsel, (ii) sie erfordern ein \isi{Antezedens} und (iii) und lizenzieren "`nur die Auslassung des Subjekts"'\is{Subjekt}, während die freie Vorfeldellipse ("`Topik Drop"') auch "`Objekte [\ldots] oder gar Adverbiale [\ldots] ellidieren"' kann. Der dritte Punkt ist sicher der schwerwiegendste (vgl.\ Abschnitt~\ref{sec-gapping}).}  

\ex. \label{ex-fries-88}Was ist denn mit Sofia?
\a. \sout{Sofia} ist mir fremd gegangen!
\b. \sout{Sofia} hab' ich seit drei Wochen nicht mehr geseh'n!
\z. \citep[(1)--(3)]{Fries:88} 

Meiner Einschätzung nach liegt hier tatsächlich ein und dieselbe Form der Ellipse vor. Natürlich eignet sich nicht jede freie Vorfeldellipse für jede Koordination, denn es bestehen bei Objektlücken recht deutliche Beschränkungen für die Position des  Antezedens im ersten Konjunkt, wie wir oben gesehen haben. Es kann aber prinzipiell für jede freie Vorfeldellipse ein erstes Konjunkt gefunden werden, das die resultierende Koordination akzeptabel macht.  

Gilt auch G2\is{Gappingregel!G2} (bzw.\ G2$^+$\is{Gappingregel!G2$^+$})? Können also nur vollständige Satzglieder weggelassen werden? Nein und Ja. Bei dem Antwortsatz in \ref{ex-adjazenz-2-4} erstreckt sich die Ellipse nur über die \isi{Pr\"aposition} der PP {\it bei Issos}:  

\ex. \label{ex-adjazenz-2-4} Alexander schlug die Perser bei welcher Stadt? \\
 \sout{Alexander schlug die Perser bei} Issos.

Enthält der Antwortsatz jedoch mehr als eine overte Konstituente wie in \ref{ex-adjazenz-2-5}, dann geht die Teilellidierung der PP {\it bei Issos} deutlich zu Lasten der Akzeptabilität: 

\ex. Alexander schlug wen bei welcher Stadt?\label{ex-adjazenz-2-5}  \\
 *\sout{Alexander schlug} die Perser \sout{bei} Issos.
 
Die in G2$^+$ aufgenommene Ausnahme für Teilsätze lässt sich möglicherweise auch bei Frage-Antwort-Folgen reproduzieren:

\ex. \label{ex-adjazenz-2-9} Wer berichtet, dass Alexander die Perser wo schlug? \\
?Kallisthenes \sout{berichtet, dass Alexander die Perser} bei Issos \sout{schlug}?

Es lässt sich also festhalten, dass die oben für die Vorwärtsellipse\is{Ellipse!Vorwärts-} in Koordinationen (G1--G3 und Vorfeldellipse)  formulierten Gesetzmä\ss igkeiten auch bei Adjazenzellipsen im Rahmen von Frage-Antwort-Folgen beobachtbar sind. %\\

Im Vergleich zu Frage-Antwort-Folgen sind die Ellidierungsmöglichkeiten bei partiellen Korrekturen\is{partielle Korrektur} sehr viel grö\ss er. Die Daten in \ref{ex-adjazenz-4} führen beispielsweise vor Augen, dass zwar \ref{ex-adjazenz-4-b} mit einem overten Finitum eher unakzeptabel ist, dass in \ref{ex-adjazenz-4-c} jedoch das Finitum als alleiniges Überbleibsel durchzugehen scheint:   

\ex. \label{ex-adjazenz-4}Der Jäger streichelte den Hasen.
\a. Nein, der Jäger erschoss den Hasen!
\b. *Nein, Peter erschoss \sout{den Hasen}!\label{ex-adjazenz-4-b}
\c. ?Nein, \sout{der Jäger} erschoss \sout{den Hasen}!\label{ex-adjazenz-4-c}
\d. Nein, \sout{der Jäger} erschoss ihn!

Was VE-Konstruktionen\is{Satz!VE-} betrifft, gibt es dagegen wieder eine gewisse Parallelität zwischen Antwortsätzen (siehe \ref{ex-adjazenz-3}) und Korrektursätzen: 

\ex. \label{ex-adjazenz-5}Man sagt, dass der Jäger den Hasen streichelte.
\a. Nein, \sout{dass der Jäger den Hase} erschoss!
\b. Nein, \sout{dass der Jäger} die Waschbären erschoss!
\c. *Nein, dass \sout{der Jäger} die Waschbären erschoss!

Möglicherweise liegt der Schlüssel zum Verständnis von \ref{ex-adjazenz-4-c} in der Eigenschaft partieller Korrekturen\is{partielle Korrektur} bzw.\ Bestätigungen, einzelne Wörter (!) in korrigierender oder bestätigender Weise fokussieren zu können, oder eine bestehende  Fokussierung\is{Fokus} aufzunehmen. Augenfällig wird das bei der partiellen Bestätigung einer Ja/Nein-Frage in \ref{ex-adjazenz-6-b}, die durch die Fokussierung des finiten Verbs die Tilgung aller anderen Ko-Konstitutenten ohne Schwierigkeit ermöglicht. Dies scheint bei der entsprechenden partiellen Korrektur in \ref{ex-adjazenz-6-a} zwar nicht gleicherma\ss en ausgeprägt zu sein, aber es ist doch eher möglich als in den Antworten zu den W-Fragen oben:  

\ex. \label{ex-adjazenz-6}Der Jäger streichelte den Hasen?
\a. ?Nein, \sout{der Jäger} erschoss \sout{den Hasen}!\label{ex-adjazenz-6-a}
\b. Ja, \sout{der Jäger} streichelte \sout{den Hasen}!\label{ex-adjazenz-6-b}

Dies lässt sich auch auf Teile von PPs\is{Pr\"aposition} anwenden, wodurch G2$^+$\is{Gappingregel!G2$^+$} konterkariert wird:

\ex. \label{ex-adjazenz-7} Der Jäger lauert auf seinem Jägersitz?
\a. \label{ex-adjazenz-7} Nein, \sout{der Jäger lauert} unter \sout{seinem Jägersitz}. \b. Nein, \sout{der Jäger lauert auf} Manfreds \sout{Jägersitz}.
\c. Ja, \label{ex-adjazenz-7} \sout{der Jäger lauert} auf \sout{seinem Jägersitz}.
\d. Ja, \label{ex-adjazenz-7} \sout{der Jäger lauert auf} seinem \sout{Jägersitz}.

Die Ellipsen im Zusammenhang mit partiellen Korrekturen\is{partielle Korrektur} sind also prinzipiell flexibler als bei Frage-Antwort-Folgen\is{Frage-Antwort-Folge}, womit wir abschlie\ss end festhalten können, dass  Adjazenzellipsen zumindest nicht eingeschränkter in ihren Tilgungsmustern sind als Vorwärtsellipsen\is{Ellipse!Vorwärts-} in Koordinationen\is{Koordination}.
\is{Ellipse!Adjazenz-|)}


\section{Ellipsen ohne Antezedens} \label{sec-situative-ellipsen}\is{Ellipse!ohne Antezedens|(}

Koordinationsellipsen und Adjazenzellipsen besitzen ein \isi{Antezedens}, d.\,h.\ ein Kontextkorrelat im sprachlichen \isi{Kontext}. Was dabei als relevanter sprachlicher Kontext zählt, kann unterschiedlich festgelegt sein: Mal ist es das andere Konjunkt, mal eine vorausgehende Frage oder sonstwie beschaffene Äu\ss erung. Das Vorhandensein eines passenden Antezedens ist jedoch nicht Voraussetzung für das Vorhandensein einer Ellipse. So wird beispielsweise die unvermittelte Frage in \ref{ex-situativ-1} sicher nicht als irgendwie ungrammatisch oder unverständlich empfunden:

\ex. \label{ex-situativ-1} \sout{Möchtest du} Kaffee?

Dasselbe gilt, um ein weiteres Beispiel zu geben, für den Fall, dass die Frage in \ref{ex-fries-88} nonverbal gestellt wird, etwa indem der Fragende auf Sofias Portrait zeigt, und die Antwort lautet:

\ex. \label{ex-fries-88-2}\sout{Sofia} hab' ich seit drei Wochen nicht mehr geseh'n!

Solche antezedensfreien Ellipsen besitzen ein Kontextkorrelat im nichtsprachlichen Kontext oder \isi{Common Ground}, d.\,h.\ im geteilten \isi{Weltwissen} oder in den geteilten, unausgesprochenen Annahmen hinsichtlich der aktuellen Diskurssituation. \cite{Schwabe:94} nennt sie deshalb \textsc{situative Ellipsen}.\footnote{Für weitere Beispiele verweise ich ebenfalls auf \cite{Schwabe:94}, wo eine Vielzahl situativer Ellipsen gesammelt und systematisiert wird. Eine neuere Übersicht über situative Ellipsen im Englischen (und anderen Sprachen) liefert z.\,B.\ \cite{Merchant:04}. \citet[767]{Klein:93} nennt in seiner Aufzählung "`kontextabhängiger"' Ellipsen weniger solche relativ produktiven dialogischen Ellipsen, sondern vor allem recht konventionalisierte Ausdrucksformen wie Textsortellipsen, elliptische Auf"|forderungen ({\it Ins Bett mir dir!}), rituelle Formeln ({\it Aus den Augen -- aus dem Sinn}) und lexikalische Ellipsen ({\it Otto sitzt \sout{im Gefängnis}}), die als Teil des Sprachwissens gesehen werden müssen und daher ganz anders zu bewerten sind -- ganz zu schweigen von verarbeitungsbedingten und entwicklungsbedingten Ellipsen, die Klein hier auch nennt.} 

Was aber die Form der Ellipse betrifft, kann man nur geringfügige Unterschiede zwischen situativen Ellipsen und Adjazenzellipsen\is{Ellipse!Adjazenz-} feststellen. So halte ich die situativen Ellipsen in \ref{ex-situativ-2} mit ungetilgtem finiten Verb grundsätzlich für ungrammatisch:

\ex. \label{ex-situativ-2}
\a. *\sout{Sofia} sah \sout{ich} letzten Donnerstag.\label{ex-situativ-2-a}
\b. *Sofia sah \sout{ich} letzten Donnerstag.

G1\is{Gappingregel!G1} scheint also auch bei Ellipsen ohne Antezedens wirksam zu sein. Was allerdings G2$^+$\is{Gappingregel!G2$^+$} und G3\is{Gappingregel!G3} betrifft, insbesondere die Tilgungsmöglichkeiten innerhalb eines Teilsatzes, lassen sich passende Daten (mit passenden Kontexten) nur schwer konstruieren. Ich finde aber, dass die Akzeptabilitätsunterschiede in \ref{ex-situativ-7} recht deutlich hervortreten:

\ex. \label{ex-situativ-7}
\a. \label{ex-situativ-7-a}\sout{Du fragst mich,} ob ich Sofia gesehen habe?
\b. *\sout{Du fragst mich, ob} ich Sofia gesehen habe?\label{ex-situativ-7-b}
\c.  \label{ex-situativ-7-c}\sout{Du fragst mich, ob} ich Sofia gesehen \sout{habe}?
\d.  \label{ex-situativ-7-d}\sout{Du fragst mich, ob ich} Sofia gesehen \sout{habe}? 
\e. *\sout{Du fragst mich,} ob ich Sofia gesehen \sout{habe}?\label{ex-situativ-7-e}

Während sich die Grammatikalität von \ref{ex-situativ-7-a}, \ref{ex-situativ-7-c} und \ref{ex-situativ-7-d} auf G2$^+$\is{Gappingregel!G2$^+$} zurückführen lässt, ist die Ungrammatikalität von \ref{ex-situativ-7-a} mit G1\is{Gappingregel!G1} und die Ungrammatikalität von \ref{ex-situativ-7-e} mit G3\is{Gappingregel!G3} zu erklären. Die bei Koordinationsellipsen\is{Ellipse!Koordinations-} beobachtbare Ausnahme von VE-Sätzen\is{Satz!VE-} bei G1 scheint bei situativen Ellipsen also nicht zu bestehen. 

Auch die satzübergreifende Ellipse in \ref{ex-situativ-3} ist nur mit Mühe als situative Ellipse rekonstruierbar, da hier sehr spezifische, schwer nachzuvollziehende Annahmen über den Diskurskontext gemacht werden müssen:

\ex. ?\label{ex-situativ-3}Susi \sout{freut sich, wenn} das Auto \sout{repariert wird}?

Mit Blick auf die angeführten Daten lässt sich also sagen, dass situative Ellipsen zwar im Detail über ein eingeschränkteres Ellipsenpotential verfügen als Vorwärtsellipsen\is{Ellipse!Vorwärts-} und Adjazenzellipsen\is{Ellipse!Adjazenz-}, dass jedoch die Tilgungsformen von Vorwärtsellipsen, die sich im Einklang mit G1--G3 und der Vorfeldellipse befinden, auch bei situativen Ellipsen gefunden werden können.   

Andererseits gibt es ein Ellipsenpotential, dass der situativen Ellipse alleine gegeben zu sein scheint. Die singularischen, zählbaren Nomen in \ref{ex-situativ-4} und \ref{ex-situativ-5} kommen ohne \isi{Artikel} aus, sofern sie satzinitial stehen:\footnote{Solche Artikel-Tilgungen im Englischen werden beispielsweise in \citet[736]{Morgan:73}, \citet[492]{Yanofsky:78}, \citet[64ff]{Barton:90}, \citet[728f]{Merchant:04} thematisiert.}

\ex. \label{ex-situativ-4}
\a. \sout{Ein} Nettes Kleid.
\b. \sout{Ein} Nettes Kleid hast du da!
\c. *Du hast da \sout{ein} nettes Kleid!

\ex. \label{ex-situativ-5}
\a. \sout{Ein} Idiot!
\b. *So \sout{ein} Idiot!

Die Ursache hierfür ist wohl in spezifischen Eigenschaften von Exklamationskonstruktionen\is{Exklamativkonstruktion} zu suchen, auf die ich hier nicht weiter eingehen möchte. Fakt ist auf jeden Fall, dass nach diesen verkürzten, exklamativen NPs nicht gefragt werden kann, und deshalb zumindest die Adjazenzellipsen\is{Ellipse!Adjazenz-} in \ref{ex-situativ-6} nicht wohlgeformt sind:    

\ex. \label{ex-situativ-6}Was hast du da? / Was soll ich anziehen?
\a. *\sout{Ein} Nettes Kleid.
\b. Ein nettes Kleid.

\is{Ellipse!ohne Antezedens|)}


\section{Zusammenfassung}

Die Weglassung obligatorischer Valenzrahmenbestandteile ist offensichtlich eine empirische Herausforderung für eine Verschränkung von Syntax und \isi{Valenz}, die mit der Idealisierung der  Vollständigkeit\is{Idealisierung!der Vollständigkeit} von Valenzrahmenrealisierungen einhergeht. Dieses Kapitel hatte den Zweck, die allgemeinen Gesetzmäßigkeiten der Weglassung zu bestimmen. Ausgangspunkt dafür war die aus der Forschungsliteratur bekannte kontextbezogene Ellipsentaxonomie mit Koordinationsellipse, Adjazenzellipse und antezedensfreier bzw.\ situativer Ellipsen. 

Die Untersuchung der Koordinationsellipse\is{Ellipse!Koordinations-} setzt eine saubere Unterscheidung zwischen Konstituentenkoordination und Satzkoordination voraus, denn ob bei einer \isi{Koordination} eine \isi{Ellipse} überhaupt vorliegt, ist ma\ss geblich davon abhängig, welche Abmessungen den Konjunkten zugesprochen werden. Da die für uns interessante Koordinationsform allein die Satzkoordination ist, denn nur sie kann per definitionem Ellipsen enthalten, werden die Konjunkte jeder betrachteten Koordination als Sätze betrachtet bzw.\ $\kappa$-instanziiert. Jede $\kappa$"=Instanziierung muss dabei dem Kriterium der $\kappa$-Reduzierbarkeit genügen. Es stellt sich heraus, dass sich für die meisten Koordinationen auf Satzebene (mit Ausnahme der pluralbildenden Koordination und der asymmetrischen Koordination) eine $\kappa$-reduzierbare\is{k-Reduzierbarkeit@$\kappa$-Reduzierbarkeit} $\kappa$-Instanziierung als Satzkoordination finden lässt. Die mit dieser Methode gebildeten elliptischen Satzkoordinationen wurden dann entsprechend der Taxonomie aus \cite{Klein:93} in Vorwärtsellipsen, Rückwärtsellipsen und Vorwärts-Rückwärtsellipsen eingeteilt.  

Im nächsten Schritt wurden spezifische Formen der Koordinationsellipse, nämlich Vorwärtsellipsen (Gapping, Vorfeldellipse) und Rückwärtsellipsen (Right-Node-Raising) eingehend untersucht. Die für das Gapping gewonnenen Generalisierungen G1--G3 konnten schlie\ss lich auch bei Adjazenzellipsen und antezedensfreien Ellipsen beobachtet werden. Die Ergebnisse lassen sich folgendermaßen zusammenfassen:

\begin{enumerate}
	\item  Es deutet vieles darauf hin, dass Vorwärtsellipsen, Adjazenzellipsen und situative Ellipsen im Wesentlichen denselben Wohlgeformtheitsregeln unterworfen sind. Gapping ist also nicht nur auf Koordination beschränkt.
  \item Daraus folgt, dass jede linguistische Modellierung der Ellipse, die eines satzimmanenten Antezedens bedarf, empirisch lückenhaft ist. 
  \item Dagegen sind RNR-Ellipsen, d.\,h.\ Ellipsen im ersten Konjunkt von RNR"=Koordinationen, auf Koordination beschränkt. Sie scheinen zudem immer als Konstituentenkoordination $\kappa$-instanziierbar zu sein, was dafür spricht, RNR in erster Linie als Koordinationsphänomen und nicht als Ellipsenphänomen zu betrachten.

\end{enumerate}

Den Modellierungsstrategien für elliptische Strukturen unter Beibehaltung der Idealisierung der Vollständigkeit werden wir uns in Kapitel \ref{sec-ellipsenanalyse} zuwenden. Wie bei der Modellierung kohärenter Konstruktionen sollen dabei TAG und seine Varianten im Vordergrund stehen, ohne jedoch allgemeingültige Tendenzen aus dem Blick zu verlieren. Dazu gehört etwa, dass die Fixierung auf Koordinationsellipsen bei der Betrachtung der Ellipsenphänomene zu Modellen führen kann, die nur einen Teilbereich des Phänomens abdecken. Aber auch, wie eine Ellipsenmodellierung aussehen muss, die die Idealisierung der Vollständigkeit vermeidet.

 











