%!TEX root = main.tex
\chapter{Was ist Valenz?} \label{ch-mit-valenz}

Der Valenzbegriff und die Valenztheorie wurden in der neueren Forschung ma\ss geblich durch \cite{Tesniere:59} initiiert und können mittlerweile auf eine reichhaltige Rezeptionsgeschichte zurückblicken.\footnote{Siehe die Literaturangaben in \citet[9]{Agel:00}.} Darauf in dieser Arbeit in gebührendem Ma\ss e einzugehen, wäre nicht nur sehr aufwändig, es wäre auch nicht zielführend. Das Ziel ist nämlich, ein bestimmtes Valenzverständnis formal möglichst eindeutig zu charakterisieren, welches in den betrachteten Syntaxmodellen (in oft impliziter Weise) zum Ausdruck kommt. Damit werde ich mich in den Abschnitten \ref{sec:valenz:hinführung} und \ref{sec-valenzbegriff} beschäftigen. Eine natürliche Folge dieser formalen Charakterisierung ist auch, den Valenzbegriff von verwechslungsgefährdeten Konzepten wie \isi{Dependenz}, \isi{Rektion} oder \isi{logische Stelligkeit} abzugrenzen.% und damit die Frage zu beantworten, die in einer Arbeit mit dem gewählten Titel beantwortet werden muss: "`Wozu Valenz?"'\footnote{Aus \citet[378f]{Jacobs:03}.}

Auf die formale Charakterisierung des Valenzbegriffs folgt in Abschnitt~\ref{sec-valenzermittlung} eine Diskussion der Valenzermittlungsmethoden. Diese werden zwar im weiteren Verlauf der Arbeit keine zentrale Rolle spielen, aber daran spiegeln sich nochmals Eigenschaften und auch Schwierigkeiten des Valenzbegriffs.  

Schlie\ss lich werden in Abschnitt \ref{sec-valenzrealisierung} verschiedene Idealisierungen der Valenzrealisierung formuliert, die die Amalgamierung von Valenz und Syntax begleiten und mit denen wir uns in der restlichen Arbeit intensiv auseinandersetzen werden.

\section{Hinleitung zum Valenzbegriff} \label{sec:valenz:hinführung}

Beginnen möchte ich, um einen etwas weiteren Anlauf zu nehmen, mit der "`Valenzidee"' von Vilmos \'Agel:
 
\ex. Valenzidee \citep[105]{Agel:00}: 
\a. Wörter -- vor allem Verben -- prädeterminieren die Satzstruktur.\\ \citep[7]{Agel:00} 
\b. Relationale Sprachzeichen, die der Kategorie Verb angehören, haben qua ihres Aktantenpotentials die Fähigkeit/Potenz, die semantische und syntaktische Organisation des Satzes zu prädeterminieren.

In der Literatur wird auf die Valenz von nicht-verbalen Kategorien, etwa Substantiven und Adjektiven, nur selten eingegangen. Vermutlich weil dort die Prädetermination der syntaktischen Umgebung weniger ausgeprägt ist und etwa bei Substantiven im Allgemeinen ein "`Valenzverlust durch Nominalisierung"' \citep[61]{Agel:00} vorliegt. Außerdem steht meist, wie auch bei anderen Syntaxtheorien, der \isi{Satz} im Mittelpunkt des Interesses und damit verbunden auch das Verb, denn ihm wird bei der syntaktischen und semantischen Prädetermination des Satzes eine zentrale Rolle zugeschrieben. Die vorliegenden Arbeit macht da keine Ausnahme.

Wie die Prädetermination qua Valenz zu verstehen ist, führt Joachim Jacobs weiter aus:
\begin{quote}
Mit dem Valenzkonzept sollen bestimmte mit einzelnen Wörtern verbundene und für sie spezifische Informationen darüber erfasst werden, in welchen Satzumgebungen sie unter welchen inhaltlichen Bedingungen vorkommen können. \citep[378]{Jacobs:03}
\end{quote}
Prädeterminiert ist also der Satzgliedbestand bzw.\ die "`mögliche Satzumgebung"' oder die Verträglichkeit mit einer Satzumgebung, wie ich es hier nennen möchte, nicht aber die Linearisierung der Satzglieder, welche ja primär von deren syntaktischen Kategorie abhängt. Jacobs verdeutlicht dies anhand der folgenden Satzumgebungen:

\ex. \label{ex-valenz-1}
\a. \label{ex-valenz-1-a} Er --- .
\b. \label{ex-valenz-1-b} Ihm --- .
\c. \label{ex-valenz-1-c} Er --- die Tür.
\d. \label{ex-valenz-1-d} Er --- der Tür.
\e. \label{ex-valenz-1-e} Er --- die Tür mit einem Trick.
\f. \label{ex-valenz-1-f} Er --- ihr die Tür.
\f. \label{ex-valenz-1-g} Die Tür --- ihn.
\z. \citep[(1)]{Jacobs:03}  

Es lässt sich beobachten, so Jacobs, dass Verben in ihrer Verträglichkeit mit den Satzumgebungen in \ref{ex-valenz-1} variieren. Das Verb {\it öffnet} lasse die Satzumgebungen \ref{ex-valenz-1-a}, \ref{ex-valenz-1-c}, \ref{ex-valenz-1-e} und \ref{ex-valenz-1-f} zu, wohingegen andere Verben wesentlich eingeschränkter platziert werden können: {\it erblickt} sei nur mit der Satzumgebung \ref{ex-valenz-1-c} verträglich, und {\it schnarcht} nur mit \ref{ex-valenz-1-a}.\footnote{Grund für die Unverträglichkeit von {\it erblickt} mit Satzumgebung \ref{ex-valenz-1-g} sei, so Jacobs, eine Folge der "`semantischen Repräsentation"' des Verbs, welche einen belebten Agens fordere. Wir werden diese Eigenschaft weiter unten als Valenzbeziehung INSP\is{Valenzbeziehung!INSP} behandeln. "`Zunächst rätselhaft"' sei dagegen die Unverträglichkeit von {\it erblickt} und \ref{ex-valenz-1-e}. Jacobs erklärt diese mit einer Unverträglichkeit der "`sortalen Forderung"' bezüglich der beschriebenen Situationen: {\it erblickt} beschreibe Situationen ohne Handlungsmoment, d.\,h.\ "`ihr Eintreten wird vom jeweiligen Subjekt nicht aktiv und intentional herbeigeführt"' \citep[382]{Jacobs:03}. Dagegen könne {\it mit einem Trick} "`sinnvoll nur auf Situationen angewandt werden, die Handlungen sind"'.} Diese Verträglichkeit bzw.\ Unverträglichkeit mit Satzumgebungen führt Jacobs auf Valenzeigenschaften der Verben zurück, die von anderen Faktoren der Satzumgebungsselektion, etwa der Subjekt-Verb-Kongruenz, unterschieden werden müsse. Wie wir gleich sehen werden, muss man auch das blo\ss e Rektionspotential als eigenständigen Selektionsfaktor begreifen. 

Führt man die Jacobs'sche Herleitung des Valenzkonzepts anhand möglicher Satzumgebungen fort, so kann man feststellen, dass für die Verträglichkeit eines Verbs mit einer bestimmten Satzumgebung die anderen Satzglieder nicht gleicherma\ss en relevant sind. Fügt man beispielsweise den Sätzen in \ref{ex-valenz-1} die temporale Bestimmung {\it nach Mitternacht} hinzu, dann ändert sich am Verträglichkeitseindruck für {\it öffnet}, {\it erblickt} und {\it schnarcht} nichts:

\ex. \label{ex-valenz-2}
\a. \label{ex-valenz-2-a} Er --- nach Mitternacht.
\b. \label{ex-valenz-2-b} Ihm --- nach Mitternacht.
\c. \label{ex-valenz-2-c} Er --- die Tür nach Mitternacht.
\d. \label{ex-valenz-2-d} Er --- der Tür nach Mitternacht.
\e. \label{ex-valenz-2-e} Er --- die Tür mit einem Trick nach Mitternacht.
\f. \label{ex-valenz-2-f} Er --- ihr die Tür nach Mitternacht.
\f. \label{ex-valenz-2-g} Die Tür --- ihn nach Mitternacht.

{\it nach Mitternacht} ist also weniger relevant für die Satzumgebungsselektion als z.\,B.\ {\it die Tür}. Diese Relevanzdiskrepanz schlägt sich in einer dichotomischen Kategorisierung der das Verb umgebenden Satzglieder nieder: Das verträglichkeitsrelevante Satzglied {\it die Tür} wird als \is{Ergänzung}\textsc{Ergänzung} (bzw.\ Komplement) bezeichnet, und das verträglichkeitsirrelevante Satzglied {\it nach Mitternacht} als \textsc{Angabe}\is{Angabe} (bzw.\ Adjunkt), wobei dann die Zuordnung \textsc{E/A-Klassifikation}\is{E/A-Klassifikation} genannt wird.\footnote{Die Terminologie ist alles andere als einheitlich, siehe \citet[766]{Storrer:03}. Tesni\`ere unterscheidet im Rahmen seiner Drama-Metaphorik zwischen valenzgebundenen "`actants"' und valenzungebundenen "`circonstants"'. In der englischsprachigen Literatur trifft man dagegen in der Regel auf "`arguments"' und "`complements"' auf der einen und "`adjuncts"' und "`modifiers"' auf der anderen Seite. Zu beachten ist hier, dass das Subjekt oft nicht zu den "`complements"' gerechnet wird \citep[25]{Mueller:10}, dass also die "`complements"' in den "`arguments"' echt enthalten sein können (siehe dazu auch Fu\ss note~\ref{fn-subkat-valenz}). In der germanistischen Forschungstradition ist dagegen die Unterscheidung zwischen Ergänzungen und Angaben weit verbreitet, insbesondere in den Arbeiten, die dieser Darstellung zugrundeliegen. Im Folgenden verwende ich die Termini Ergänzung, Komplement und (syntaktisches) Argument synonym.} Demzufolge besitzt {\it öffnet} in der Satzumgebung \ref{ex-valenz-2-c} zwei Ergänzungen, {\it er} und {\it die Tür}, und die Angabe {\it nach Mitternacht}. Betrachtet man aber auch das Verb {\it öffnet} in der Satzumgebung \ref{ex-valenz-2-a}, mit der es ebenfalls verträglich ist, fällt sogleich auf, dass eine Ergänzung nicht immer realisiert sein muss, denn nun fehlt das Akkusativobjekt {\it die Tür}. Man spricht in diesem Fall von einer \textsc{fakultativen Ergänzung}\is{Ergänzung!fakultative}. Dass es daneben auch \textsc{obligatorische Ergänzungen} gibt, zeigt sich in der für {\it öffnet} unverträglichen Satzumgebung \ref{ex-valenz-2-b}, dessen Unverträglichkeit wohl mit dem Fehlen eines Subjekts zu erklären ist. Als obligatorische Ergänzungen können auch andere Kategorien auf"|treten, etwa Akkusativobjekte, was sich bei Einsetzung von {\it erblickt} beobachten lässt: Fehlt hier die Ergänzung {\it die Tür} wie in der Satzumgebung \ref{ex-valenz-2-a}, dann erscheint der Satz nicht wohlgeformt. Die Entscheidung, ob eine Ergänzung obligatorisch oder fakultativ ist, wird gemeinhin als \textsc{o/f-Klassifikation}\is{o/f-Klassifikation} bezeichnet. 

Die Realisierungsnotwendigkeit als Kriterium der o/f-Klassifikation von Ergänzungen ist allerdings mit Vorsicht zu genießen. Konstruktionsbedingt kann auch eine vermeintlich obligatorische Ergänzungen weggelassen werden, z.\,B.\ das Subjekt bei der \isi{Vorfeldellipse} in \ref{ex-valenz-3-b} und beim \isi{Imperativ} in \ref{ex-valenz-imperativ}:

\ex. Und was macht Peter? 
\a. Er öffnet die Tür.
\b. Öffnet die Tür.  \label{ex-valenz-3-b}

\ex. Öffne die Tür!\label{ex-valenz-imperativ}

Darüber hinaus gibt es noch umfassendere Ellipseformen wie das \isi{Gapping} in \ref{ex-valenz-4-b}, die jegliche Obligatheitsansprüche von Bestandteilen des Valenzrahmens zu konterkarieren scheinen:

\ex. Und was öffnet Peter? 
\a. Er öffnet die Tür.
\b. Die Tür. \label{ex-valenz-4-b}

Solche Ellipsephänomene stellen eine besondere Herausforderung für Valenztheorien und deren Syntaxmodellierung dar und stehen daher in dieser Arbeit im Zentrum des  Interesses. Eine eingehende Darstellung des Phänomens findet sich in Kapitel \ref{chap-ellipse}. %\\

Die Betrachtung von Satzumgebungsalternativen stellt nur eine der Möglichkeiten dar, die \isi{E/A-Klassifikation} empirisch-theoretisch zu fundieren. Im nächsten Abschnitt werde ich diesen Ansatz der \isi{Subklassenspezifik} (SUBKLASS) zuordnen. Darüber hinaus kursiert in der Literatur eine Vielzahl weiterer Valenzkriterien bzw.\ Valenzbeziehungen\is{Valenzbeziehung} -- \cite{Jacobs:94} nennt derer sieben -- was Ausdruck der Schwierigkeit ist, den Valenzbegriff auf eine einzelne empirisch"=theoretische Grö\ss e zurückzuführen. Die exakte Bestimmung, Rechtfertigung und Anwendung jeder dieser Valenzbeziehungen ist keine leichte Aufgabe und soll in dieser Arbeit auch gar nicht geleistet werden. Ich werde stattdessen überblicksartig markante Begriffe, Probleme und Perspektiven vortragen, um der Vielschichtigkeit und Unschärfe des Valenzbegriffs gerecht zu werden. Ich strukturiere die weiteren Ausführungen zur Valenztheorie anhand der drei Problemkomplexe, die \citet[23]{Storrer:92} identifiziert hat:

\begin{enumerate}\setlength{\itemsep}{-.5ex}
  \item Problem der Begriffsklärung
  \item Problem der Valenzermittlung (E/A-Klassifikaktion)
  \item Problem der Valenzrealisierung (o/f-Klassifikation)
\end{enumerate}   
Das Problem der Begriffsklärung wird im nächsten Abschnitt Thema sein. An dessen Ende wird trotz der widrigen Umstände eine exakte Formalisierung der Valenzeigschaften von Verben unternommen, die die wesentlichen Intuitionen des Jacobs'schen Valenzbegriffs aufnimmt. In engem Zusammenhang mit dem Problem der Begriffsklärung steht das Problem der \isi{Valenzermittlung}, dem wir uns in Abschnitt \ref{sec-valenzermittlung} zuwenden. Schlie\ss lich wird in Abschnitt \ref{sec-valenzrealisierung} auf das Problem der \isi{Valenzrealisierung} eingegangen, woran sich die Formulierung dreier Idealisierungen der Valenzrealisierung\is{Idealisierung} anschließt. 





\section{Problem der Begriffsklärung} \label{sec-valenzbegriff}


\'Agels Valenzidee und Jacobs "`gemeinsamer Nenner"' des Valenzkonzepts gehen von einer irgendwie gearteten verbbestimmten Strukturierung der Satzsyntax bzw.\ Satzsemantik aus. Diese Strukturierung vollzieht sich anhand der dichotomischen Klassifizierung der Ko-Satzglieder des Verbs in Ergänzungen\is{Ergänzung} oder Angaben\is{Angabe} (\isi{E/A-Klassifikation}). Die Klärung des Valenzbegriffs bedeutet also die Klärung folgender Fragen:

\begin{enumerate}
	\item Was ist der Unterschied zwischen Ergänzungen und Angaben? D.\,h.\ was ist die spezifische Beziehung zwischen Ergänzung und Valenzträger\is{Valenzträger}?
	\item Warum gibt es diesen Unterschied?   
\end{enumerate}

Was die erste Frage betrifft, so möchte ich mich hauptsächlich an die Arbeit von Joachim Jacobs, vor allem \cite{Jacobs:94}, halten.\footnote{Für einen kurzen historischen Überblick über die Entwicklung der verschiedenen Valenzbegriffe ist \citet[208-215]{Helbig:73} zu empfehlen. Eine neuere, gerafftere Darstellung des Jacobs'schen Ansatzes findet man auch in \citet[369ff]{Zifonun:03}.} Jacobs entwirft und verteidigt hier einen multidimensionalen Ansatz, der sich deutlich von dem bis dahin in der germanistischen Forschungstradition vorherrschenden Ziel, Valenz auf eine einzelne \isi{Valenzbeziehung} zu reduzieren, absetzt. Die dort formulierten mono- und bidimensionalen Erklärungsversuche seien letztlich erfolglos gewesen, haben sie doch zu einer "`Aufsplitterung des Valenzbegriffs"' geführt, ohne dass eine dominante Valenzbeziehung gefunden worden sei. Als Ursache für diese "`Valenz\-misere"' diagnostiziert Jacobs das Ausgehen von falschen Prämissen, nämlich von der Homogenität des Valenzphänomens.\footnote{Eine andere falsche Prämisse sei es, von einer klaren Intuition auszugehen \citep[Fußnote 8]{Jacobs:94}.} Stattdessen, so Jacobs, muss von einer Heterogenität des Valenzphänomens, also von einer Bündelung verschiedener Phänomene unter den Valenzbegriff ausgegangen werden, weshalb eine "`Aufsplitterung des Valenzbegriffs"' nur folgerichtig sei. Er favorisiert deshalb eine Nebeneinanderstellung von sieben gleichwertigen Valenzbeziehungen\is{Valenzbeziehung}, die nicht notwendigerweise simultan vorliegen müssen, und die bereits aus mo\-no\-dimensionalen Erklärungsversuchen bekannt sind: Notwendigkeit (NOT), formale Spezifität (FOSP), inhaltliche Spezifität (INSP), Argumenthaftigkeit (ARG), Beteiligtheit (BET), Assoziiertheit (ASSOZ) und Exozentrizität (EXO).\footnote{\label{fn-valenz-vater}Eine gegenteilige Schlussfolgerung aus dieser "`Valenzmisere"' zieht \cite{Vater:78}, der die Unterscheidung von Ergänzungen und Angaben als solche in Frage stellt. Vater zufolge gibt es keine Angaben, nur Ergänzungen, die verbspezifisch als obligatorisch oder fakultativ differenziert werden. In Jacobs' Terminologie: Vater beschränkt sich auf die NOT-Beziehung und lehnt also den ARG-gestützten Valenzbegriff ab. In eben diese Kerbe schlägt auch \cite{Storrer:92} mit ihrem Konzept der Situationsvalenz (siehe Abschnitt \ref{sec-stug-implikationen}). Später rückt Vater von dieser Position allerdings wieder ab und zeigt sich in \citet[227]{Vater:81} von einem anderen Valenzkriterium überzeugt, das auf \cite{Hoehle:78} zurückgeht: "`Ergänzungen sind die Konstituenten, deren Auftreten im Zusammenhang mit einem Verb nicht prädiktabel sind."' Jacobs subsumiert dieses Kriterium unter FOSP \citep[44]{Jacobs:94}.} Davon rückt Jacobs NOT, FOSP, INSP und ARG in das Zentrum seiner Betrachtungen, und so möchte ich es im Folgenden auch halten.\footnote{Es ist unklar, so Jacobs, wie die BET-Beziehung einzuordnen ist. "`Es liegt nahe, zu vermuten, da\ss \ sich BET deswegen nicht in einer konsistenten Weise in einer [Inklusions-]Hierarchie der B-Bindungen [d.\,h.\ Valenzbindungen] einordnen lä\ss t, weil diese Relation eine qualitativ anderer Art der B-Bindung darstellt als die anderen Beziehungen."' \citep[66]{Jacobs:94}} Diese vier Valenzbeziehungen sollen in gebotener Kürze und anhand der Explikationen in \citet[Kapitel~4]{Jacobs:94} skizziert werden. Hinzunehmen werde ich allerdings die bereits erwähnte Subklassenspezifik (SUBKLASS).

\subsection{NOT}
\is{Valenzbeziehung!NOT|(}

NOT expliziert eine spezifischere Form der im letzten Abschnitt erwähnten Realisierungsnotwendigkeit, wobei Jacobs vorausschickt, dass $S$ für "`strukturell und semantisch disambiguierte Sätze"' stünde.

\ex. {\bf Notwendigkeit (NOT):} Es "`gilt für alle $S$ und alle Konstituenten $X$ und $Y$ von $S$: NOT($X$,$Y$) in $S$ gdw. $X$ aufgrund der lexikalischen Füllung von $Y$ in $S$ nicht weggelassen werden kann, ohne da\ss \ die dadurch entstehende Struktur bei gleichbleibender Interpretation von $Y$ ungrammatisch wird"'. \citep[14]{Jacobs:94}  

Welchen syntaktischen Kategorien $X$ und $Y$ angehören, lässt Jacobs dagegen offen, so dass im Prinzip diese und die folgenden Explikationen nicht blo\ss \ auf die Verbvalenz anwendbar sind. Wenn man die Verbvalenz zugrunde legt (also $X$ als Ergänzung und $Y$ als Verb versteht), fallen zwei wesentliche Einschränkungen auf: Zum einen soll die Etablierung der NOT-Bezie\-hung zwischen Ergänzung und Verb von der "`lexikalischen Füllung"' des Verbs abhängig sein, und nicht etwa von konstruktionsbedingten Eigenheiten des Satzes wie der Möglichkeit zur Ellipse; zum anderen ist NOT auch an eine bestimmte Interpretation des Verbs geknüpft. Lie\ss e man, so Jacobs, beispielsweise das Objekt {\it Gerda} in \ref{ex-NOT} weg, dann änderte sich auch die Interpretation des Verbs. Dieses Datum sei deshalb ungeeignet für die Falsifikation von NOT({\it Gerda}, {\it erschrecken$^a$}).

\ex. \label{ex-NOT}
\a. Peter wird Gerda erschrecken$^a$.
\b. Peter wird erschrecken$^p$.
\z. \citep[14f]{Jacobs:94}

Zu beachten ist au\ss erdem, dass NOT von einer Reihe anderer Notwendigkeitsbeziehungen abgegrenzt werden muss, die sich nicht an der Grammatikalität eines sprachlichen Ausdrucks festmachen, damit aber bei der Valenzermittlung verwechselt werden können. \citeauthor{Storrer:92} (1992: 105) zählt hierzu SEM-NOT (semantische Notwendigkeit), KOM-NOT (kommunikative Notwendigkeit) und TEX-NOT (textuelle Notwendigkeit).\footnote{Storrers Notwendigkeitstaxonomie erinnert an die Taxonomie der Unvollständigkeit bzw.\ der "`Expansionserwartung"' bei \cite{Kindt:85}. Kindt unterscheidet zwischen syntaktischer, semantischer und pragmatischer Expansionserwartung, die mit ebensolchen Formen der Unvollständigkeit korrelieren. Auch andere Autoren haben die Vielschichtigkeit des linguistischen Notwendigkeitskonzepts erkannt (vgl.\ \citealt[226]{Vater:81}).} Schlie\ss lich macht Storrer auch auf die grundsätzliche Problematik des Prädikats \glq ungrammatisch\grq\ aufmerksam. Der Anschein trügt also, dass NOT per Eliminierungstest\is{Valenzermittlung!Eliminierungstest} leicht zu überprüfen sei (siehe Abschnitt \ref{sec-valenzermittlung-sprecherbefragung}).
\is{Valenzbeziehung!NOT|)}

\subsection{FOSP und INSP} 
\is{Valenzbeziehung!FOSP|(}\is{Valenzbeziehung!INSP|(}

FOSP und INSP greifen die Idee auf, dass der Valenzträger eine Ergänzung "`formal"' (d.\,h.\ morphosyntaktisch) oder "`inhaltlich"' (d.\,h.\ semantisch"=konzeptionell) in idiosynkratischer, von der Grammatik und dem Lexikon nicht vorhergesagter Weise spezifiziert:

\ex. {\bf Formale Spezifität (FOSP):}  Es "`gilt für alle $S$ und alle Konstituenten $X$ und $Y$ von $S$: FOSP($X$,$Y$) in $S$ gdw. es mindestens ein Formmerkmal $M$ von $X$ gibt, für das gilt: Da\ss \ ein Ausdruck mit dem Formmerkmal $M$ als Begleiter von $Y$ in $S$ fungieren kann, ist eine spezifische Eigenschaft von $Y$"'. \citep[22]{Jacobs:94}

\ex. {\bf Inhaltliche Spezifität (INSP):} Es "`gilt für alle $S$ und alle Konstituenten $X$ und $Y$ von $S$: INSP($X$,$Y$) in $S$ gdw. es mindestens ein Inhaltsmerkmal $M$ von $X$ gibt, für das gilt: Da\ss \ ein Ausdruck mit dem Inhaltsmerkmal $M$ als Begleiter von $Y$ in $S$ fungieren kann, ist eine spezifische Eigenschaft von $Y$"'. \citep[22]{Jacobs:94}

Die Einschränkung auf eine "`spezifische Eigenschaft"' des Valenzträgers bedeutet hier, dass diese Eigenschaft nicht aus allgemeineren, z.\,B.\ lexikalischen Regeln abgeleitet ist, also in Anbetracht der übrigen Eigenschaften des Valenzträgers nicht "`prädiktabel"' ist. Davon betroffen ist, so Jacobs, beispielsweise das Akkusativobjekt {\it Herrn Maier} in \ref{ex-fosp-1}:

\ex. \label{ex-fosp-1}(weil) Peter Herrn Maier rasiert 

"`Dass {\it rasieren} eine Akkusativ-NP als Objekt nimmt"', sei eine Eigenschaft, die "`sich nicht durch irgendwelche grammatischen Regeln vorhersagen [lässt]"' \citep[23]{Jacobs:94}.  Es besteht hier demnach die Valenzbeziehung FOSP({\it Herrn Maier}, {\it rasiert}). Darüber hinaus liegt auch INSP({\it Herrn Maier}, {\it rasiert}) vor, da das Akkusativobjekt ein "`semantisches Merkmal [+belebt]"'  enthalten muss.\ifdraft{\footnote{Stefan Müller hat mich darauf aufmerksam gemacht, dass ein Versto\ss\ gegen INSP-Vorgaben unter Negation (immer?) möglich ist:

\ex. Peter rasiert den Stein nicht, denn Steine kann man nicht rasieren.

}}{} Dass nicht nur das Akkusativobjekt von einer solcherart motivierten INSP-Bezie\-hung betroffen sein kann, haben wir bereits in \ref{ex-valenz-1-g} gesehen, als es um die Unverträglichkeit von {\it erblicken} mit der Satzumgebung \textit{die Tür --- ihn} ging. Übrigens schließt Jacobs, zumindest für das Englische, eine FOSP-Beziehung zur Subjekt-NP grundsätzlich aus, weil "`keines ihrer formalen Merkmale spezifisch für ein bestimmtes Verb ist"' \citep[37]{Jacobs:94}.

Die Tücken einer "`spezifischen Eigenschaft"' werden an der Jacobs'schen Analyse von \ref{ex-fosp-2} deutlich:

\ex. \label{ex-fosp-2}(weil) Peter Herrn Maier den Kopf rasiert

Die Valenzbeziehungen FOSP({\it Herrn Maier}, {\it rasiert}) und FOSP({\it den Kopf}, {\it rasiert}) bestünden hier nämlich nicht, also auch nicht die Valenzbeziehung zwischen {\it rasiert} und seinem Akku\-sativ\-objekt wie noch in \ref{ex-fosp-1}. Jacobs begründet dies mit einer denkbare Regel, die ein "`NPnom-NPakk-Verb"' wie in \ref{ex-fosp-1} mit einer "`NPnom-NPdat-NPakk-Variante"' korreliert. Daran wird deutlich, dass die Rücksichtnahme auf jegliche Regelhaftigkeit der Formmerkmalsverteilung den Bestand der FOSP- und INSP-Beziehun\-gen stark einschränkt -- möglicherweise zu stark.\footnote{Aus diesem Grund nimmt Jacobs zusätzlich noch die Valenzbeziehungen DFOSP und DINSP an \citep[25]{Jacobs:94}, auf die ich hier nicht eingehen will. Möglicherweise ist es besser, kein dichotomisches Kategorieverständis der Regelhaftigkeit bzw.\ Vorhersagbarkeit anzusetzen, sondern ein "`Stufenmodell"' \citep[181]{Agel:00}, wie es in \cite{Breindl:89} entwickelt wurde.} Im Epilog wählt Jacobs deshalb eine schwächere Formulierung: keine "`spezifische"', d.\,h.\ "`unprädiktable"' Eigenschaft soll hier ausschlaggebend sein, sondern eine "`inhärente"' Eigenschaft des Valenzträgers, im Unterschied zu Eigenschaften der syntaktischen Konstruktion \citep[71f]{Jacobs:94}. Im Weiteren werde ich diese schwächere Formulierung von FOSP und INSP vor Augen haben.

Zuletzt sei noch darauf hingewiesen, dass FOSP zwar an den Rektionsbegriff\is{Rektion} im Sinne der Formfestlegung (vgl.\ \citealt[Abschnitt~2.1]{Zifonun:03}; \citealt[33ff]{Eisenberg:06}) erinnert, doch es kann keine vollständige Entsprechung bestehen. Jacobs berücksichtigt in der FOSP-Explikation ausdrücklich nur Satzglieder \citep[14]{Jacobs:94}, was etwa die Rektionsbeziehung zwischen \isi{Pr\"aposition} und Nomen in einer PP ausklammert. Und es wäre zumindest erklärungsbedürftig, wenn Präpositionen durch die Kasusrektion eines Nomens pauschal der Status eines Valenzträgers zugesprochen würde, wo doch Präpositionen "`in der Regel nicht als selbständige Valenzträger eingeordnet werden"' \citep[371]{Zifonun:03}.\footnote{Präpositionen gelten nicht als typische Valenzträger, da sie "`nicht selbst einen Sachverhalt entwerfen, sondern der verbalen oder nominalen Stütze bedürfen"' \citep[374]{Zifonun:03}. Ähnliches wird für Hilfsverben, Kopulaverben und Modalverben angenommen, die zwar in einer FOSP-Beziehung zu einer infinitivischen Verbform stehen, die aber nicht das "`Zentrum eines Sachverhaltsentwurfs"' \citep[375]{Zifonun:03} bilden. Ihr valenztheoretischer Stellenwert ist dadurch dem von adverbialen Bestimmungen vergleichbar.} Nur so ist es auch verständlich, dass Jacobs bei der Systematisierung der Valenzbeziehungen (siehe Abschnitt~\ref{sec:valenz:system}) ohne weitere Kommentierung von der Gültigkeit der Inklusion der FOSP-Beziehungen unter die ARG-Beziehungen ausgeht. Stünden nämlich alle Präpositionen in einer FOSP-Beziehung zum kasusregierten Nomen, und damit auch dank der Inklusion in einer ARG-Beziehung, dann wäre zu erklären, worin z.\,B.\ die eigenständige Prädikation von {\it auf} in {\it warten auf} oder die von {\it um} in {\it bitten um} bestünde. Ich kann sie nicht erkennen. Weder sind also Rektionsbeziehungen notwendigerweise Valenzbeziehungen (kontra Eisenberg, siehe \citealt[360]{Zifonun:03}), noch Valenzbeziehungen notwendigerweise FOSP-Beziehungen. 

Im Übrigen berührt Jacobs in seinen Ausführungen die interne Struktur von Präpositionalphrasen überhaupt nicht, so weit ich erkennen kann. Damit soll Präpositionen, ebenso wie Nomen, keinesfalls der Valenzträgerstatus pauschal abgesprochen sein. Dieser steht einfach nicht im Fokus seiner Arbeit.\footnote{Allerdings klammert Jacobs auch Kopulakonstruktionen aus, die, zumindest \citet[374f]{Zifonun:03} zufolge, im Zusammenspiel mit dieser FOSP-Explikation zu einem "`Widerspruch der Befunde"' führen.}
 
%\footnote{Für \cite{Eisenberg:99} ist FOSP/Rektion notwendig und hinreichend für das Vorliegen einer Valenzbeziehung (siehe \citet[356f]{Zifonun:03}).}

\is{Valenzbeziehung!FOSP|)}\is{Valenzbeziehung!INSP|)}


\subsection{ARG} \label{sec-arg}
\is{Valenzbeziehung!ARG|(}

Die Besetzung einer Argumentstelle im semantischen Prädikat des Valenzträgers konstituiert die Valenzbeziehung ARG:

\ex. {\bf Argumenthaftigkeit (ARG):} Es "`gilt für alle $S$ und alle Konstituenten $X$ und $Y$ von $S$: ARG($X$,$Y$) in $S$ gdw. $X$ in $S$ in einer von $Y$ ausgehenden Prädikation als Argument integriert ist"'. \citep[17]{Jacobs:94}

Jacobs geht hier offensichtlich von der Existenz einer sprecher- und intuitionsunabhängigen Prädikation einer konkreten Verbbedeutung aus, die durch spezielle Tests, etwa den Gesche\-hen-Test\is{Valenzermittlung!Geschehen-Test} (siehe Abschnitt \ref{sec-valenzermittlung-sprecherbefragung}), ermittelt werden kann. Dagegen sei der intuitionsgeleiteten Bestimmung einer semantischen Repräsentation nicht zu trauen, weil in manchen Fällen mehrere unterschiedliche Prädikationen denkbar sind. Beispielsweise hält Jacobs für den Satz

\ex. Er trägt das Fahrrad in den Keller. \label{ex-valenz-arg-1}

die zwei Prädikationen in \ref{ex-valenz-arg-2} mit zwei- oder dreistelligem {\tt trägt}-Prädikat für intuitiv plausibel:

\ex. \label{ex-valenz-arg-2}
\a. {\tt trägt(er,das Fahrrad, in den Keller)}
\b. {\tt in den Keller(trägt(er,das Fahrrad))} \label{ex-valenz-arg-2-b}
\z. \citep[(6),(7)]{Jacobs:94}

Die für ARG relevante Prädikation soll allerdings nur die zweistellige in \ref{ex-valenz-arg-2-b} sein, weswegen in Satz \ref{ex-valenz-arg-1} hinsichtlich {\it trägt} also genau die ARG-Beziehungen ARG({\it in den Keller}, {\it trägt}), ARG({\it trägt}, {\it er}) und ARG({\it trägt}, {\it das Fahrrad}) bestehen. Unstrittig ist also der Argumentstatus der Agens- und Patiensrollen, wohingegen "`bei lokalen, temporalen, instrumentalen und direktionalen Begleiterrollen [\ldots] das Vorliegen von Argumenthaftigkeit im Einzelfall geprüft werden [muss]"' \citep[768]{Storrer:03}.

Die Gleichsetzung von Valenzträger und Prädikat hat übrigens zur Folge, dass entsprechend der Prädikation in \ref{ex-valenz-arg-2-b} die Angabe {\it in den Keller} den Status eines Valenzträgers erhält, als dessen Ergänzung das finite Verb {\it trägt} fungiert. Dies ist für die E/A-Unterscheidung beim Valenzträger {\it trägt} selber unerheblich (siehe \citealt[59]{Jacobs:94}), hat aber gravierende Folgen für die Richtung der Dependenzrelation\is{Dependenz}: Da auf Grundlage der Valenzeigenschaften die Dependenzrelationen etabliert werden, dominiert nun in der Dependenzstruktur die Angabe {\it in den Keller} den Valenzträger {\it trägt}. Die  Dependenzliteratur geht jedoch im Allgmeinen vom umgekehrten Dominanzverhältnis aus.\footnote{Siehe aber \citet[5--7]{Melcuk:09}, der unterschiedliche Dependenzstrukturen\is{Dependenzgraph} ("`Semantic Structure"', "`Deep-Syntactic Structure"' und "`Surface-Syntactic Structure"') annimmt und damit die semantische Dependenz von der syntaktischen entkoppelt.} Um dem zu genügen und trotzdem die ARG-Beziehung als "`zentrales Kriterium der Valenzgebundenheit"' \citep[768]{Storrer:03} beizubehalten, muss also die semantische Repräsentation sorgfältig gewählt werden. Man könnte z.\,B.\ statt \ref{ex-valenz-arg-2-b} die Repräsentation in \ref{ex-valenz-arg-3} \`a la Davidson zur Ermittlung der ARG-Beziehung heranziehen, in der {\it trägt} kein Argument der Prädikation {\it in den Keller} darstellt:  

\ex. \label{ex-valenz-arg-3} {\tt $\exists$e [trägt(er,das Fahrrad,e) $\wedge$ in den Keller(e)]}

\citet[19f]{Jacobs:94} macht auf eine ähnliche Problematik im Zusammenhang mit der semantischen Repräsentation von Quantoren aufmerksam:\footnote{Mit den Worten von Arnim von Stechow und Wolfgang Sternefeld wird hier "`das janusköpfige Verhalten von Nominalien"' deutlich, "`die sich semantisch als Prädikate, syntaktisch als Argumente verhalten."' \citep[71]{Stechow:Sternefeld:88}}    

\ex. \label{ex-valenz-arg-4} weil er alle Gläser reinigt \hfill \citep[(12)]{Jacobs:94}
\a. \label{ex-valenz-arg-4-a} {\tt Für alle Gläser x gilt: reinigt (er,x)}     \hfill \citep[(12a)]{Jacobs:94}
\b. \label{ex-valenz-arg-4-b} {\tt $\forall$ x [Glas(x) $\to$ reinigt(er,x)]}

Nimmt man die Prädikation in \ref{ex-valenz-arg-4-a} oder die etwas formalere Variante in \ref{ex-valenz-arg-4-b} ernst, dann kann {\it alle Gläser} nicht als Ergänzung von {\it reinigt} fungieren, da die dafür nötige Prädikat-Argument-Struktur nicht vorliegt. Allenfalls könne die erwünschte ARG-Beziehung irgendwie über die Variablenbindung zustande kommen, so Jacobs.

\is{Valenzbeziehung!ARG|)} 


\subsection{SUBKLASS}
\is{Valenzbeziehung!SUBKLASS|(}

Einer Erwähnung wert ist eine (umstrittene) Valenzbeziehung, die bei Jacobs zwar kurz diskutiert \citep[26]{Jacobs:94}, aber nicht in den Valenzbeziehungskatalog aufgenommen wird: die Subklassenspezifik. Sie ist bei \citet[187]{Agel:00} folgenderma\ss en expliziert:\footnote{\citet[190f]{Agel:00} kann, ebenso wie \cite{Jacobs:94} und \cite{Vater:78},  in der SUBKLASS-Beziehung keine Valenzbeziehung sehen. Vielmehr sei es eine Dependenzbeziehung. Ich belasse die SUBKLASS-Beziehung trotzdem im Kreis der Valenzbeziehungen, weil sie sich gut auf die eingangs beobachtete Satzumgebungs(un)verträglichkeit anwenden lässt.}

\ex. {\bf Subklassenspezifik (SUBKLASS):}  Es gilt für alle $S$ und alle Konstituenten $X$ und $Y$ von $S$: SUBKLASS($X$,$Y$) in $S$ gdw. die Klasse [von] $X$ nicht bei jedem beliebigen Element der Klasse [von] $Y$ in $S$ korrekt ist und das negative Sprecherurteil bei abweichenden $S$ nicht auf blo\ss er Inkongruenz basiert. 

SUBKLASS kann also spiegelbildlich zu NOT verstanden werden, denn es problematisiert nicht die Abwesenheit, sondern die Anwesenheit einer bestimmten Ko-Konstituente. Insofern ist auch eine kontrastive Ähnlichkeit zu FOSP und INSP vorhanden, da bei SUBKLASS nach der grammatischen Inkompatibilität mit bestimmten Formmerkmalen gefragt wird. Zur Verdeutlichung sieht man in \ref{ex-subclass} den Kontrast zwischen den Verben {\it warten} und {\it erwarten}, der auf die grammatischen Inkompatibilität von {\it erwarten} mit einem Präpositionalobjekt\is{Pr\"aposition} mit der Präposition {\it auf} zurückzuführen ist:  

\ex. \label{ex-subclass}
\a. Beckett wartet auf Godot.
\b. *Beckett erwartet auf Godot.
\z. \citep[187]{Agel:00}

Dieser Kontrast ist im Sinne der SUBKLASS-Beziehung nun weniger für die Valenzeigenschaften von {\it erwarten} interessant, sondern sagt vielmehr etwas über die Valenzgebundenheit des Präpositionalobjekts von {\it warten} aus. Es liegt also, bezogen auf die Datenlage in \ref{ex-subclass}, die Beziehung SUBKLASS({\it auf Godot}, {\it wartet}) vor, womit sich für den Status einer Ergänzung beim Präpositionalobjekt argumentieren lie\ss e. Die Etablierung der SUBKLASS-Beziehung per Substitutionstest ist allerdings mit Vorsicht anzugehen, denn, wie \'Agel in seiner Explikation auch betont, das negative Sprecherurteil darf sich nicht durch Inkongruenz bezüglich des Weltwissens ergeben, sondern nur durch "`Inkorrektheit"' bezüglich der "`grammatischen und lexikalischern Regeln einer Einzelsprache"' \citep[188]{Agel:00}. Damit berühren wir jedoch bereits das Problem der \isi{Valenzermittlung}, auf das in Abschnitt~\ref{sec-valenzermittlung} genauer eingegangen wird.

\is{Valenzbeziehung!SUBKLASS|)}

\subsection{System der Valenzbeziehungen} \label{sec:valenz:system}
\is{Valenzbeziehung|(}

Soweit die knappe Darstellung der Valenzbeziehungen und die ebensolche Beantwortung der Frage nach dem Unterschied von Ergänzung und Angabe. Es sollte sich der Eindruck eingestellt haben, dass die skizzierten Valenzbeziehungen tatsächlich sehr unterschiedlich sind, z.\,B.\ hinsichtlich ihrer linguistischen Beschreibungsebene: Während NOT und FOSP (und auch SUBKLASS) auf einer syntaktischen bzw.\ morphologischen Ebene operieren, beziehen sich INSP und ARG auf eine logisch-semantisch-konzeptuelle Ebene. Da noch hinzukommt, dass sie sich extensional nicht vollkommen decken \citep[Kapitel~5]{Jacobs:94},\footnote{Jacobs spricht zunächst von einer "`Unabhängigkeit der Beziehungen"', doch er meint eigentlich die "`Kontravalenz"' (d.\,h.\ die negierte Äquivalenz $\neq$) der Beziehungen. Es ist sehr wohl möglich, dass eine logische Abhängigkeit in Form einer Folgebeziehung besteht, dass also $P \neq Q \; \wedge \; P \to Q$. Damit ist nicht ausgeschlossen, dass es eine Beziehung $R$ gibt, die alle anderen Beziehungen $R_1 \ldots R_n$ inkludiert, d.\,h.\ für jede $R_i$ gilt $R_i \to R$. Siehe \cite[52, 64]{Jacobs:94}.} erhält Jacobs' These der Heterogenität des Valenzphänomens unter diesen Umständen etwas zwingendes.  In der Konsequenz kann also Valenz als "`relativ inhaltsleere Sammelbezeichnung"' \citep[55]{Jacobs:94} verstanden werden, die verglichen mit dem traditionellen Verständnis doch sehr an linguistischer Strahlkraft verliert. Andererseits, so Jacobs im Epilog \citep[71]{Jacobs:94}, stünden die Phänomene doch in einem Zusammenhang und lägen "`in prototypischen Fällen"' gemeinsam vor. Diese "`empirisch sehr häufige Zusammenballung"' gelte es zu erklären -- und damit den Valenzbegriff zu rehabilitieren. 

Hierfür beobachtet \citet[64--68]{Jacobs:94} zunächst die Inklusionsverhältnisse zwischen den Valenzbeziehungen und erkennt darin u.\,a.\ den Zusammenhang, dass eine NOT-, FOSP- oder INSP-Beziehung nur dann besteht, wenn auch die entsprechende ARG-Beziehung vorliegt.\footnote{\citet[227f]{Vater:81} nennt Beispiele für den für diese Inklusionssystematik problematischen Fall, dass zwar keine ARG-Beziehung, aber sehr wohl eine NOT-, FOSP- oder INSP-Beziehung besteht:\\
\fnex{\ex. \label{ex-vater-81-1} Dann kam es zu Gewalttätigkeiten. \hfill \citep[(22')]{Vater:81}

}
In diesem Satz habe das unpersönliche Verb {\it kam} "`logisch nur ein Argument, nämlich das Ereignis, zu dem es kam"'. Das Expletivpronomen {\it es} sei also nur in der "`kategorialen Charakterisierung (KC)"' enthalten und nicht mit Elementen der "`logischen Charakterisierung (LC)"' koindiziert, was den Jacobs'schen Valenzbeziehungen NOT und FOSP entspricht. Soweit ich erkennen kann, geht \cite{Jacobs:94} auf diesen Fall nicht ein.} Das bedeutet jedoch nicht, dass man Valenz mit der logischen Stelligkeit des Denotatprädikats gleichsetzen kann. Dagegen sprechen Fälle, in denen Argumente, d.\,h.\ semantische Rollen, die konzeptuell recht deutlich vorliegen, nicht realisiert werden können, d.\,h.\ "`blockiert"' \citep[36]{Steinitz:92} sind. Ein Beispiel dafür liefert das Verb {\it lügen}, dessen Patiens-Rolle sich einer Realisierung widersetzt -- anders als etwa beim Derivat {\it belügen}, wo die Patiensrolle zu den obligatorischen Ergänzungen zu rechnen ist.\label{ex-luegen}\footnote{Dass es vielleicht nicht völlig ausgeschlossen ist, die Patiens-Rolle und die Thema-Rolle mit {\it lügen} zu realisieren, zeigt \cite{Storrer:96} mit folgenden Daten:\\
\fnex{\ex. 
\a. "`Die Hausaufgaben sind schon fertig"', log Peter. \hfill \citep[(2)]{Storrer:96}
\b. ?Peter log gegenüber der Mutter. \hfill \citep[(3)]{Storrer:96}

}} 
Im Epilog systematisiert und verfeinert Jacobs diese Inklusionsbeziehungen folgendermaßen:\footnote{"`>"' sei eine transitive, irreflexive und asymmetrische Relation: "`Wenn es $P > Q$, dann gilt für alle Sätze $S$ und alle Konstituenten $X$, $Y$ von $S$: Falls $P(X,Y)$ in $S$, dann $Q(X,Y)$ in $S$."' \citep[65]{Jacobs:94}}

\ex. \{ NOT, FOSP \} $>$ INSP $>$ ARG \hfill \label{ex:inklusion}\citep[71]{Jacobs:94}

Daran wird deutlich, dass die ARG-Beziehung eine grundlegende, irgendwie dominante Valenzbeziehung darstellt. 

Wie lässt sich nun diese Inklusionsdominanz der ARG-Beziehung erklären? Oder um auf die zweite Eingangsfrage zurückzukommen: Warum gibt es den Unterschied zwischen Ergänzungen und Angaben? Jacobs Erklärungsvorschlag zielt auf eine Grammatikalisierungsabstufung der ARG-Beziehung entlang der Inklusionshierarchie in \ref{ex:inklusion} ab.\footnote{In \cite{Jacobs:94a} werden die Valenzbeziehungen nochmals gebündelt, so dass eine Zweiteilung aus  "`syntaktischer Valenz"' (FOSP und NOT) und "`semantischer Valenz"' (INSP und ARG) entsteht. Eine Besonderheit in Jacobs' Theorie ist, dass die syntaktische Valenz immer realisiert werden muss und die Fakultativität bestimmter Ergänzungen mit der lexikalischen Varianz von syntaktischen Valenzen eines Valenzträgers erklärt wird, denen wiederum eine Varianz in der semantischen Valenz entspricht. Vgl.\ auch Abschnitt \ref{sec-existenzfrage}.}   
Eine ARG-Beziehung kann also zu einer INSP-Beziehung grammatikalisieren, und eine INSP-Bezie\-hung zu einer NOT-Bezie\-hung und/""oder zu einer FOSP"=Be\-ziehung. Valenz ist demnach Ausdruck der\linebreak Grammatikalisierung der ARG"=Bezie\-hung, oder etwas allgemeiner mit den Worten Vaters "`als syntaktischer Reflex eines logischen Tatbestands zu sehen"' \citep[225]{Vater:81}.\footnote{Siehe \citet[205ff]{Agel:00} für eine kritische Entgegnung.}
  
Diese Grammatikalisierungshypothese lässt sich funktional präzisieren, indem der Valenz darüber hinaus die Rolle eines Vermittlers zuerkannt wird: Valenz vermittelt zwischen semantischen Argumenten und deren syntaktischen Realisierungen. Jacobs bringt das in einer neueren Arbeit folgenderma\ss en auf den Punkt: 

\begin{quote}
Valenz ist ein relationales morphosyntaktisches Merkmal von Wortformen, das kodiert, wie semantische Argumente des Lexems, zu dem die jeweilige Wortform gehört, zu realisieren sind. \citep[504]{Jacobs:09}
\end{quote}
Dieser Funktionsaspekt wird insbesondere in der englischsprachigen Literatur als \isi{Argument Linking} bezeichnet.

\is{Valenzbeziehung|)}


\subsection{Mathematisch-formale Explikation}

Die Valenz einer Verbform als die für die Prädetermination einer Satzumgebung relevante Grö\ss e fasse ich als Menge von Tripeln aus semantischer Rolle, Formmerkmal und Notwendigkeitsmarker auf:
\begin{definition}[Valenz] \label{def-valenz}\is{Valenz}\is{semantische Rolle}
Die Valenz einer Verbform $V$ ist die Menge $\mathsf{Val}(V) \subseteq \{ \langle R,$ $F, not \rangle |$ 
\begin{itemize}
	\item $R$ ist eine semantische Rolle, 
	\item $F$ ist ein morphosyntaktisches Merkmal,
	\item und $not \in \{+,-\}$ zeigt die Notwendigkeit an, einer Konstituente mit Merkmal $F$ die semantische Rolle $R$ zuzuwei\-sen $\}$.
\end{itemize} 
\end{definition}
Eine Verbform ist die morphologisch ausspezifizierte und semantisch disambiguierte Instanziierung eines Verblexems, wie sie in einem konkreten Satz auftritt. Dementsprechend wäre beispielsweise die Valenz der Verbform \textit{rasiert} mit seinen zwei obligatorischen Ergänzungen die Menge $\mathsf{Val}(\text{\textit{rasiert}}) = \{\langle agens, nomi$\-$nativ, $""$+\rangle, \langle patiens, akkusativ, +\rangle\}$. Die Mengen von semantischen Rollen und\linebreak Formmerkmalen sehe ich hier als gegeben an und möchte sie vorerst nicht weiter konkretisieren. Durch den Bezug auf Verbformen erlaubt Definition \ref{def-valenz}, den Aktiv- und Passivformen der Verben unterschiedliche Valenzen zuzuordnen, ohne dies allerdings zu erzwingen. Darüber hinaus lässt die Definition zu, dass eine Verbform mehr als eine Valenz besitzt. 

Die Menge der Valenztripel als Valenz einer Verbform $V$ ist weitgehend unrestringiert und man könnte zusätzlich die folgenden Einschränkungen ins Auge fassen:\is{semantische Rolle} 
\begin{enumerate}\setlength{\itemsep}{-.5ex}
  \item Jede semantische Rolle wird genau einem Formmerkmal zugeordnet, d.\,h.\ die Menge $\{\langle R,F \rangle | \langle R,F,not \rangle$ $\in \mathsf{Val}(V) \}$ ist eine (partielle) Funktion.
  \item Jedes Formmerkmal wird genau einer semantischen Rolle zugeordnet, d.\,h.\ die Menge $\{\langle F,R \rangle | \langle R,F,not \rangle$ $\in \mathsf{Val}(V) \}$ ist eine (partielle) Funktion.
  \item Jede Rollenrealisierung ist entweder notwendig oder nicht notwendig, d.\,h.\ die Menge $\{\langle R, not \rangle | \langle R,F,not \rangle \in \mathsf{Val}(V) \}$ ist eine (partielle) Funktion ist.
\end{enumerate}
Die erste und zweite Einschränkung greifen jedoch empirisch zu kurz.\footnote{Z.\,B.\ Valenzalternationen (\textit{Ich schreibe den Brief Peter/an Peter}) und doppelter Akkusativ (\textit{Ich nenne ihn meinen Freund} oder \textit{Er lehrt ihn den Tango}).} Eine Ausnahme zur dritten Einschränkung ist dagegen nur schwer vorstellbar, zumindest solange \glq notwendig\grq\ einen binären Wertebereich hat. 

Nur kurz erwähnen möchte ich zwei Aspekte, die bei der hier vorgebrachten Valenz-Defini\-tion unberücksichtigt bleiben: Zum einen die mögliche Unverträglichkeit eines Valenzträgers mit bestimmten Angabentypen\is{Angabe}, die im Zusammenhang mit der SUBKLASS-Bezie\-hung\is{Valenzbeziehung!SUBKLASS} thematisiert wurde und werden wird (siehe die Diskussion des Substitionstests im Anschluss auf S.~\pageref{par-subsitutionstest}); zum anderen die mögliche Interdependenz hinsichtlich der Realisierung verschiedener Valenzrollen (\citealt[306ff]{Jacobs:94a}; \citealt[397f]{Jacobs:03}).\footnote{Eine Interdependenz zwischen der Realisierung von Akkusativ- und Dativobjekt drückt sich z.\,B.\ in folgenden Gegenüberstellungen aus:\\
\fnex{\ex. 
\a. dass er (dem Boten) das Paket übergibt
\b. dass er dem Boten *(das Paket) übergibt
\z. \citep[(29)]{Jacobs:03}

\ex. 
\a. dass er (seinem Nachbarn) das Haus renoviert
\b. dass er seinem Nachbarn *(das Haus) renoviert
\c. dass er (seinem Nachbarn das Haus) renoviert
\z. \citep[(30)]{Jacobs:03}

}} Der valenztheoretische Status beider Aspekte scheint mir in der Literatur nicht gesichert zu sein. Au\ss erdem hätte deren Berücksichtigung keine substantielle Auswirkung auf den Valenzbegriff, zumindest was die vorliegenden Arbeit betrifft.     

Eine Trennung der morphosyntaktischen und semantischen Ebene der Valenz kann man in der Unterscheidung von syntaktischer und semantischer Valenz vollziehen:  
\begin{definition}[Syntaktische Valenz]\is{Valenz}
Die syntaktische Valenz einer Verbform $V$ ist \linebreak $\mathsf{synVal}(V) = \{\langle F,not \rangle | \langle R,F,not \rangle \in \mathsf{Val}(V) \}$.
\end{definition}
\begin{definition}[Semantische Valenz]
Die semantische Valenz einer Verbform $V$ ist \linebreak $\mathsf{semVal}(V) = \{R | \langle R,F,not \rangle \in \mathsf{Val}(V) \}$.
\end{definition}
Valenz ist also die Verknüpfung von syntaktischer und semantischer Valenz, wobei semantische Valenz nicht mit logischer Stelligkeit gleichzusetzen ist.\footnote{Dank der Entkoppelung von semantischer Valenz und logischer Stelligkeit ist es z.\,B.\ möglich, den Verben {\it lügen} und {\it belügen} dieselbe logische Stelligkeit zuzuweisen, obwohl {\it lügen}, im Unterschied zu {\it belügen}, die Realisierung der Patiensrolle blockiert.}  Der Valenzrahmen bündelt schlie\ss lich Verb und Valenz:
\begin{definition}[Valenzrahmen]\is{Valenzrahmen}
Der Valenzrahmen einer Verbform $V$ ist $\widehat{\mathsf{Val}}(V) =$ $\langle V, \mathsf{Val}(V) \rangle$.
\end{definition}

Die bis hier definierten Valenztermini sind lexikalische Grö\ss en und von einer Instanziierung im konkreten Satz abstrahiert. Die Valenzbeziehung zwischen Satzgliedern ist folgenderma\ss en zu bestimmen:

\begin{definition}[Valenzbeziehung] \label{def:valenzbeziehung}\is{Valenzbeziehung}
Seien die Verbform $V$ und die Konstituente $K$ \linebreak Satzglieder von Satz $S$. Zwischen $V$ und $K$ besteht eine Valenzbeziehung $\langle V,K,R \rangle$ an einer semantischen Rolle $R$, falls gilt:
\begin{itemize}
\item $K$ hat die semantische Rolle $R$ bzgl. der Prädikation von $V$  und 
\item $K$ hat das Formmerkmal $F$ für beliebiges $F$ und
\item $\langle R,F,not \rangle \in \mathsf{Val}(V)$ für beliebiges $not$.
\end{itemize}
\end{definition}
Hinsichtlich der Menge der Valenzbeziehungen für einen Satz $S$, nennen wir diese $\mathsf{ValR}(S)$, könnte man, wie schon bei den Valenztripelmengen aus Definition~\ref{def-valenz}, bestimmte Einschränkungen in Form von funktionalen Beziehungen durchspielen. Eine solche Einschränkung ist etwa, dass jede Konstituente $K$ in einer Valenzbeziehung zu maximal einem Verb $V$ steht, d.\,h.\ $\{\langle K,V \rangle | \langle V,K,R \rangle \in \mathsf{ValR}(S) \}$ ist eine partielle Funktion. Wir werden uns diese und andere Einschränkungen später in Abschnitt~\ref{sec-strukturfrage} ansehen. 

Auf die so definierten Valenzbeziehungen können wir bei der Bestimmung von Ergänzung und Angabe zurückgreifen:
\begin{definition}[Ergänzung und Angabe]\is{Ergänzung}\is{Angabe}
Seien die Verbform $V$ und die Konstituente $K$ Satzglieder von Satz $S$. Besteht eine Valenzbeziehung $\langle V,K,R\rangle$ mit beliebigen $R$, dann ist $K$ eine Ergänzung von $V$. Andernfalls ist $K$ eine Angabe von $V$, wenn die Bedeutung von $K$ an der Prädikation von $V$ unmittelbar beteiligt ist.
\end{definition}
Die Bedingung der Prädikationsbeteiligung für den Angabenstatus ist mit Blick auf kohärente Konstruktionen hinzugefügt, wo Satzglieder Angaben unterschiedlicher Verben sein können (siehe Kapitel~\ref{chap-kohaerenz}).\ifdraft{\marginpar{St.Müller: Was hei\ss t das?}}{}

Schlie\ss lich fasse ich den Valenzträger und seine Ergänzungen und Angaben unter dem Terminus Dependenzfeld zusammen:
\begin{definition}[Dependenzfeld] \label{def-dependenzfeld}\is{Dependenzfeld}
Sei V ein verbaler Valenzträger, E die Menge der Ergänzungen von V und A die Menge der Angaben von V. Das Dependenzfeld von V ist die Menge $\{$V\,$\} \ \cup$ E $\cup$ A.
\end{definition}




\section{Problem der Valenzermittlung} \label{sec-valenzermittlung}
\is{Valenzermittlung}

Angesichts der Unklarheit des traditionellen Valenzbegriffs wundert es nicht, dass eine Reihe unterschiedlichster Valenzermittlungsverfahren ersonnen wurden, deren Aufgabe darin besteht, die Anzahl und Spezifität der Ergänzungen eines konkreten Valenzträgers (meist eines Verbs) zu bestimmen. Um eine gewisse Ordnung zu erhalten, orientiere ich mich in der folgenden Darstellung an den Valenzermittlungsfamilien aus \citet[7.2.2]{Storrer:92}. Die \textsc{systeminternen Methoden} greifen auf nicht beobachtbare, abstrakte Entitäten der "`System-Ebene"' zurück und sehen hierin die entscheidenden Kriterien für das Vorliegen einer Valenzbeziehung. Dazu zählen im Folgenden logische Repräsentationen und Regeln der Grammatik. Dagegen gehen \textsc{konstruktive Methoden} davon aus, dass sich Valenzeigenschaften aus statistischen Eigenschaften eines Individuenkollektivs herleiten lassen.



\subsection{Valenzermittlung anhand der logischen Repräsentation} 

In seiner Explikation der ARG-Beziehung\is{Valenzbeziehung!ARG} stützt sich \cite{Jacobs:94} auf Prädikat-Argument-Strukturen, die, seinem Beispiel in \ref{ex-valenz-arg-2-b} nach zu urteilen, nicht beliebig, sondern irgendwie festgelegt sind: Das Verb {\it trägt} denotiert eine zweistellige, keine dreistellige Prädikation. Wie kommt Jacobs darauf? Eine Erklärung könnte sein, dass Jacobs Prädikat-Argument-Strukturen als extrasubjektiv gegeben betrachtet, dass er ihnen, mit anderen Worten, ontologische Realität zuspricht. Des weiteren muss von einem Homomorphismus zwischen Prädikat-Argument-Strukturen und Valenzrahmen ausgegangen werden, der die Prädikate auf Valenzträger und die Argumente auf Ergänzungen abbildet. Eine elegante, sprecherunabhängige Methode der Valenzermittlung bestünde dann darin, die Valenz eines Verbs direkt und unzweifelhaft von der naturgegebenen Stelligkeit\is{logische Stelligkeit} seiner Prädikation abzulesen.

Einen vermeintlichen Fürsprecher dieser Methode findet man in der Person Gottlob Freges. Zum einen bringt Frege mehrmals zum Ausdruck, dass der Sinn eines sprachlichen Ausdrucks, der "`Gedanke"', ontologisch unabhängig vom Denkenden ist, d.\,h.\ "`Gedanken existieren nach Frege unabhängig davon, ob sie jemals von einem Menschen gefa\ss t und sprachlich ausgedrückt werden"' \citep[492]{Schirn:92}. Zum anderen lassen sich diese Gedanken in gesättigte und ungesättigte Gedankenteile zerlegen \citep[491]{Schirn:92}. Die ungesättigten Gedankenteile sind der Sinn sogenannter Begriffswörter, die wie mathematische Funktionen im Verbund mit sättigenden Argumenten einen Sinn bzw.\ eine Bedeutung ergeben \citep[476f]{Schirn:92}. Man ist versucht, in dieser Gedankenzerlegung die besagte Prädikat-Argument-Struktur zu erkennen, die Begriffswörter als Valenzträger zu verstehen und damit den oben geforderte Homomorphismus zwischen Prädikat-Argument-Strukturen und Valenzrahmen herzustellen. 

Leider liegt diese Deutung der Fregeschen Position an entscheidender Stelle falsch: Die Gedanken im Fregeschen Sinne enthalten per se keine Prädikat-Argument-Struktur; sie sind zwar durch Begriffe (auf vielfache Weise) in gesättigte und ungesättigte Gedankenteile zerlegbar, bringen aber per se keine Zerlegungen mit (\citealt{Kemmerling:90,Kemmerling:10}; \citealt[492]{Schirn:92}). Gedankenzerlegungen sind also sprach- und sprecherabhängig, was sie als sprecherunabhängige Entscheidungshilfe für die Valenzermittlung offensichtlich unbrauchbar macht. Diese "`Amorphie"' der Gedanken wurde jedoch, so \cite{Kemmerling:90}, in der Frege-Rezeption von einflussreichen Autoren nicht erkannt und stattdessen die Gedankenzerlegung als eine apriorische tradiert. Insofern werden die ontologischen Annahmen, die ich Jacobs hier unterstelle, zwar nicht von Frege geteilt, aber ihm zugeschrieben und sind in der sprachwissenschaftlichen Community wohl weit verbreitet.\footnote{Als Beispiel eines Autors, der diese Position und sogar ein isomorphisches Abbildungsverhältnis zwischen "`Tatsachen"' und Sätzen vertritt, ist natürlich der Ludwig Wittgenstein des \textit{Tractatus logico-philosophicus} zu nennen \citep[53]{Kutschera:75}. Die vielleicht schärfste Entgegnung formulierte Ludwig Wittgenstein selber in seinen \textit{Philosophischen Untersuchungen}, sowohl hinsichtlich der Existenz einer "`Wirklichkeit an sich"', als auch hinsichtlich des Abbildungscharakters der Sprache (siehe dazu \citealt[133]{Kutschera:75}; \citealt[22]{Ortner:87}).}

Doch egal, wie es um die sprachphilosophische Motivierung naturgegebener ARG-Bezie\-hun\-gen\is{Valenzbeziehung!ARG} steht: Selbst unter der Annahme naturgegebener ARG-Bezie\-hun\-gen, die man Jacobs nach allem, was er sagt, unterstellen kann, sieht er sich dazu genötigt, auf ein Testverfahren auf Basis von Grammatikalitätsurteilen zurückzugreifen, nämlich den sogenannten Geschehen-Test.           



\subsection{Valenzermittlung anhand von Grammatikalitäts\-ur\-teilen} \label{sec-valenzermittlung-sprecherbefragung}

Geht man davon aus, dass für die Grammatikalität\is{Grammatikalität} einer Konstruktion die Valenz\-eigenschaften der Teilkonstituenten eine Rolle spielen, dass sich also Valenzeigenschaften in Grammatikalitäturteilen widerspiegeln, dann liegt eine Valenzermittlung anhand von Grammatikalitätsurteilen nahe. In diesem Abschnitt soll eine Auswahl der hierfür verwendeten Tests dargestellt werden, die alle dem Schema genügen, einen \glqq grammatischen\grqq\ Satz nach einer durch den Test bestimmten Regel zu modifizieren und zu prüfen, ob er seine \glqq Grammatikalität\grqq\ beibehalten hat. Zwei Aspekte der Testverfahren gilt es hier genauer zu prüfen: (1) die Zuordnung zu einem der Valenzbeziehungen und (2) der Bezug zum Begriff der Grammatikalität, der nicht nur theoretisch problematisch ist, sondern auch praktisch zu unbefriedigenden Ergebnissen der Testanwendungen führt. Letzterem Punkt werden wir uns am Ende dieses Abschnitts widmen.%\\

Bei der Überprüfung einer NOT-Beziehung\is{Valenzbeziehung!NOT} bietet sich der \textsc{Eliminierungstest}\is{Valenzermittlung!Eliminierungstest} an, d.\,h.\ eine Konstituente des grammatischen Ausgangssatzes wird entfernt:  

\ex. 
\a. Peter verzehrt eine Pizza.
\b. *Peter verzehrt.

Die Konstituente {\it eine Pizza} ist also nicht weglassbar und steht damit in einer NOT-Beziehung zum Verb. Man beachte dabei aber die zwei zusätzlichen Bedingungen, die in der Jacobs'schen Explikation der NOT-Beziehung genannt sind, nämlich die Abhängigkeit zur "`lexialischen Füllung"'  und die Kontinuität der Verb\-interpretation. Au\ss erdem muss, entsprechend Storrers Unterteilung, SYN-NOT ausschlaggebend sein, nicht SEM-NOT, KOM-NOT oder TEX-NOT (siehe \citealt[105]{Storrer:92}). Schlie\ss lich müssen elliptische Konstruktionen\is{Ellipse}, die ein Weglassen obligatorischer Ergänzungen zulassen (z.\,B.\ Vorfeldellipsen), vermieden werden. Der auf den ersten Blick simple Eliminierungstest wird damit zu einer komplexen Angelegenheit, denn bei der Erzeugung des Testsatzes muss ein gehöriges Ma\ss \ an linguistischem Wissen eingesetzt werden, um all diese Faktoren zu berücksichtigen.\footnote{Ähnlich sieht das \citet[36f]{Heringer:84}: "`[\ldots] die Frage der Notwendigkeit von Satzgliedern ist eben so vielfältig, da\ss \ sie nicht ohne weiteres operationalisierbar oder gar mit einer einfachen Einteilung [in syntaktischer, semantischer und kommunikativer Notwendigkeit] lösbar ist."' Siehe auch z.\,B.\ \citet[226]{Vater:81}.} %\\

Ein populärer Test für die ARG-Beziehung\is{Valenzbeziehung!ARG} ist (so auch bei \citealt[17f]{Jacobs:94}) der \textsc{Gesche\-hen-Test}\is{Valenzermittlung!Geschehen-Test}.\footnote{Zusätzlich zum Geschehen-Test finden sich in der Literatur eine Reihe weiterer ARG-Tests. Da sich diese, was die Grundidee und die grundsätzlichen Probleme betrifft, nicht wesentlich vom Geschehen-Test unterscheiden, verweise ich für eine Überblick auf \citet[178]{Agel:00}.} Hierbei werden Konstituenten des Ausgangssatzes in einen separaten, nachfolgenden Satz ausgelagert, welcher mit dem Verb {\it geschehen} (oder einem Äquivalent) gebildet wird:

\ex. 
\a. Die Kinder spielen hinter dem Hause.
\b. Die Kinder spielen. Das Spielen ist (=geschieht) hinter dem Hause.
\z. \citep[63]{Storrer:92}

\ex. 
\a. Der Obstgarten liegt hinter dem Hause.
\b. *Der Obstgarten liegt. Das Liegen ist (=geschieht) hinter dem Hause.
\z. \citep[63]{Storrer:92}

Demnach soll {\it hinter dem Hause} beim Verb {\it spielt} eine Angabe und beim Verb {\it liegt} eine Ergänzung sein. Die Grundidee dabei ist, dass eine Auslagerung von Argumenten die Prädikation des Ausgangssatzes zerstört und somit zu einer ungrammatischen Konstruktion führt. An kritischen Bemerkungen zu diesem Test mangelt es nicht.\footnote{Siehe der ausführlichere Überblick in \citet[178f]{Agel:00}.} Bereits \cite{Jacobs:94} hat angemerkt, dass der Geschehen-Test "`nicht scharf genug zwischen Argumenthaftigkeit und Notwendigkeit"' \citep[Fußnote 16]{Jacobs:94} trennt, dass wir es hier also im ersten Satz mit einem "`versteckten Eliminierungstest"' \citep[178]{Agel:00} zu tun haben, wodurch natürlich auch die eben geschilderten Schwierigkeiten geerbt werden. Was diesen "`versteckten Eliminierungstest"' übersteht, also fakultative Ergänzungen und Angaben, wird im zweiten Satz indirekt auf die Verträglichkeit mit dem Verb {\it geschehen} hin geprüft. Direktionalbestimmungen und Akkusativobjekte haben es hier, unabhängig vom ersten Testsatz und der darin vorkommenden Prädikation, schwerer, so dass die erzeugten Testsätze oft "`künstlich und konstruiert klingen"' \citep[76]{Storrer:92} und keine zuverlässigen Sprecherurteile erwarten lassen. Möglicherweise weniger gravierend, aber dennoch von Bedeutung ist die Einschränkung Jacobs, dass sich der Geschehen-Test nicht auf nicht-verbale Prädikationen anwenden lässt. Warum sollte aber die syntaktische Kategorie bei einem Test, der die ARG-Beziehung an sich offenlegen soll, eine Rolle spielen? Schlie\ss lich ist generell fragwürdig, ob mit einem Grammatikalitätsurteil eine ARG-Beziehung\is{Valenzbeziehung!ARG}, die au\ss ergrammatisch motiviert sein soll, ermittelbar sein kann.\footnote{\'Agel hat dazu eine klare Meinung: 
\begin{quote}
Aus der Tatsache, dass ARG ein formallogisch linguistisches System-Konstrukt ist, das [\ldots] an konkretem sprachlichem Material unmittelbar weder nachweisbar noch nachvollziehbar ist, folgt [\ldots], dass es methodisch schlicht absurd ist, ARG überhaupt testen zu wollen. Denn warum sollte ein parolebezogenes Angemessenheitsurteil eines Sprachteilhabers irgendeine Beziehung haben zu einem [\ldots] formal\-logisch-linguistischen System-Konstrukt? \citep[179]{Agel:00}
\end{quote} 
Siehe auch \citet[235]{Storrer:92}.}  %\\




Für die Ermittlung von FOSP- und INSP-Beziehungen\is{Valenzbeziehung!FOSP}\is{Valenzbeziehung!INSP} schlägt \citet[180]{Agel:00} die \textsc{Ersatzprobe}\is{Valenzermittlung!Ersatzprobe} vor. Durch Ersetzung eines Form- bzw.\ Inhaltsmerkmals einer Konstituente des Ausgangssatzes durch ein anderes soll ein ungrammatischer Testsatz erzeugt werden. Gelingt dies wie in \ref{ex-valenzermittlung-FOSP1}, so besteht zwischen dem ursprünglichen Formmerkmal und dem Verb eine FOSP-Beziehung. 

\ex. \label{ex-valenzermittlung-FOSP1}
\a. Anja will auf den FWB nicht verzichten.
\b. *Anja will über das/dem FWB nicht verzichten.
\z. \citep[179]{Agel:00}

Eine andere Situation finden wir in \ref{ex-valenzermittlung-FOSP2} vor, wo keine der Ersetzungen einen ungrammatischen Testsatz hervorruft.

\ex. \label{ex-valenzermittlung-FOSP2} Tobias stellt die Lampe auf/unter/neben/hinter/vor den Tisch. \\ \citep[180]{Agel:00}

Doch auch bei der Ersatzprobe kann es, meiner Ansicht nach, zu Überschneidungen mit dem Eliminierungstest\is{Valenzermittlung!Eliminierungstest} kommen, denn bei der Ersetzung von Formmerkmalen einer obligatorischen Ergänzung mit Formmerkmalen einer Angabe vollzieht man ebenfalls einen "`versteckten Eliminierungstest"'. Beispielsweise kann, wie in \ref{ex-valenzermittlung-FOSP4} zu sehen ist, das obligatorische Objekt {\it eine Pizza} durch die Angabe {\it nach einer Pizza} ersetzt werden. Der resultierende Testsatz ist jedoch wegen der Weglassung des obligatorischen Objekts ungrammatisch. 

\ex. \label{ex-valenzermittlung-FOSP4}
\a. Peter verzehrte eine Pizza.
\b. *Peter verzehrte nach einer Pizza. 
%\c. Peter verzehrte nach einer Pizza ein Tiramsu.

Ich sehe au\ss erdem nicht, was verhindern sollte, mit Hilfe der Ersatzprobe\is{Valenzermittlung!Ersatzprobe} aus der Ungrammatikalität von \ref{ex-valenzermittlung-FOSP3} die FOSP-Beziehungen der jeweiligen Präpositionen von \ref{ex-valenzermittlung-FOSP2} zum Verb {\it stellt} zu folgern. Sicher wäre das nicht das Ergebnis, das \'Agel im Sinn hat: 

\ex. *Tobias stellt die Lampe des Tisches.\label{ex-valenzermittlung-FOSP3}

Die Ersatzprobe funktioniert also nicht mit beliebigen Ersetzungen der Formmerkmale, sondern mit ganz bestimmten, die jedoch von \'Agel nicht genau spezifiziert werden. Welche Ersetzungen sind wohl gemeint? Mir scheint die folgende Eingrenzung am sinnvollsten: Die Ersetzung des Formmerkmals $F_1$ mit dem Formmerkmal $F_2$ im Satz $S$ gilt nur dann als Ersatzprobe\is{Valenzermittlung!Ersatzprobe}, wenn $F_2$ an dieselbe semantische Rolle geknüpft werden kann wie $F_1$ in $S$. Dementsprechend wäre also die Ersetzung in \ref{ex-valenzermittlung-FOSP3} keine Instanz einer Ersatzprobe, weil der Genitiv nicht als Formmerkmal einer Richtungsangabe benutzt werden kann. Analoges gilt für die Ersetzung in \ref{ex-valenzermittlung-FOSP4} und auch für \'Agels Beispiel in \ref{ex-valenzermittlung-FOSP1}. Letzteres kann jedoch leicht angepasst werden:

\ex. \label{ex-valenzermittlung-FOSP5}
\a. Anja will auf den FWB nicht verzichten.\label{ex-valenzermittlung-FOSP5-a}
\b. *Anja will den FWB nicht verzichten.\label{ex-valenzermittlung-FOSP5-b}
\c. Anja will den FWB nicht missen.\label{ex-valenzermittlung-FOSP5-c}

Der ungrammatische Satz \ref{ex-valenzermittlung-FOSP5-b} zeigt, dass das Formmerkmale der Patiensrolle von {\it verzichten} (auf-PP) nicht durch das Formmerkmal der Patiensrolle von {\it missen} (Akkusativ-NP) in \ref{ex-valenzermittlung-FOSP5-b}  ersetzt werden kann, obwohl beide Verben eine identische Rollensemantik haben.\footnote{Weitere Paarungen dieser Art sind {\it helfen}/{\it unterstützen} \citep[402]{Mueller:10} und {\it treffen}/{\it begegnen} \citep[126]{Pollard:Sag:87}. Auch Paarungen sogenannter Besitzwechselverben wie {\it geben}/{\it nehmen} \citep[Kapitel~7]{Kunze:91} können in diesem Zusammenhang genannt werde. Vgl.\ auch die Zusammenfassung in \citet[Abschnitt~11.11.5]{Mueller:10}.} Demnach besteht zwischen beiden Verben und ihren Patiensrollenrealisierungen eine FOSP-Beziehung\is{Valenzbeziehung!FOSP}. Die Ersatzprobe\is{Valenzermittlung!Ersatzprobe} erfasst also die Wahl eines Valenzträgers aus einer Menge möglicher Formmarkierungen für eine bestimmte Rolle.  %\\



\label{par-subsitutionstest} Schlie\ss lich soll noch der \textsc{Substitutionstest}\is{Valenzermittlung!Substitutionstest} für die Ermittlung der SUB-\linebreak KLASS-Beziehun\-gen\is{Valenzbeziehung!SUBKLASS} erwähnt werden. Anders als bei der Ersatzprobe\is{Valenzermittlung!Ersatzprobe} wird hier der Valenzträger ersetzt, so dass Ungrammatikalität dann auftritt, wenn die betroffenen Valenzträger unterschiedliche Valenzrahmen aufweisen. \'Agels Beispiel aus \ref{ex-subclass} sei hier wiederholt, in dem {\it wartet} durch {\it erwartet} substituiert und dadurch ein ungrammatischer Testsatz erzeugt wird: 

\ex. \label{ex-subclass2}
\a. Beckett wartet auf Godot.
\b. *Beckett erwartet auf Godot.

Leider gibt es auch hier neben den klaren Fällen wie in \ref{ex-subclass2} solche, die einer weiteren Einschränkung bedürfen. Wie schon in der Explikation der SUBKLASS-Beziehung\is{Valenzbeziehung!SUBKLASS} oben angesprochen wurde, müssen die erzeugten Testsätze "`inkorrekt"', also ungrammatisch sein, während "`Inkongruenz"' nicht ausreicht:

\ex. 
\a. Peter schläft auf der Wiese.
\b. ??Peter ist auf der Wiese tot.

Der Testsatz im Beispiel zeige, so \citet[188]{Agel:00}, eine Inkongruenz von Lokalbestimmung und dem Ausdruck {\it ist tot}. Dass SUBKLASS\is{Valenzbeziehung!SUBKLASS} also die Valenzgebundenheit der Lokalbestimmung postuliert, was \citet[26]{Jacobs:94} behauptet, entspricht nicht der \'Agel'schen Explikation. Neben der Schwierigkeit der Unterscheidung von Inkongruenz und Inkorrektheit findet sich bei \citet[66f]{Storrer:92} eine weitere Schwierigkeit für die Durchführung des Substitutionstests\is{Valenzermittlung!Substitutionstest}: Wie viele Verben müssen substituiert werden, bis die beliebige Substituierbarkeit und damit der Angabe-Status nachgewiesen ist? Die unbefriedigende Antwort lautet: Alle! In der konkreten Anwendung der Substitution bedeutet dies, dass die SUBKLASS-Beziehung\is{Valenzbeziehung!SUBKLASS} zwar verifiziert, aber nicht falsifiziert werden kann. Das will jedoch nicht hei\ss en, dass die SUBKLASS-Beziehung immer verifiziert werden kann, wo eine Valenzbeziehung angenommen wird. Eine solche testimmanente Lücke bilden diejenigen Ergänzungen, die die Form und Funktion von Lokalitäts- oder Direktionalbestimmungen haben und andernorts als Angaben gelten (siehe \citealt[360f]{Zifonun:03}).\footnote{\citet[16f]{Vater:81} diskutiert den Substitutionstest unter anderem Blickwinkel als "`Kriterium der freien Hinzufügbarkeit"': "`Es besagt, da\ss\ Angaben dann vorliegen, wenn sie einem Satz frei hinzufügbar sind."' Wie Jacobs kommt Vater unter der Annahme eines weiten Inkorrektheitsbegriffs, der semantische Inkompatibilität einschlie\ss t, zu dem Ergebnis, dass u.\,a.\ Lokalitäts- und Direktionaladverbien vorschnell als Ergänzungen klassifiziert werden.}    %\\






Es gibt eine Reihe weiterer Tests, die sich jedoch nicht einer Valenzbeziehung zuordnen lassen bzw.\ in der Valenzliteratur nicht zugeordnet werden. Stellvertretend sollen zwei Tests kurz dargestellt werden, die prototypisch für andere Tests stehen, nämlich der Permutationstest und der Dialogtest. 

Der \textsc{Permutationstest}\is{Valenzermittlung!Permutationstest} gehört zu jenen Tests, bei denen die Linearisierung des Ausgangssatzes verändert wird. Die Grundannahme ist also, dass sich Angaben und Ergänzungen in ihrer Bewegungsflexibilität unterscheiden, dass z.\,B.\ Angaben in das Nachfeld bewegt werden können, Ergänzungen aber nicht. In manchen Fällen kann man das tatsächlich beobachten: 
 
\ex. 
\a. Du hast das Buch am Vormittag dorthin gelegt.
\b. \hspaceThisNegative{(*)}(*)Du hast das Buch dorthin gelegt am Vormittag.\label{ex-storrer-63-b}
\c. *Du hast das Buch gelegt dorthin.\label{ex-storrer-63-c}
\z. \citep[63]{Storrer:92}

Die \isi{Extraposition} der Ergänzung {\it dorthin} in \ref{ex-storrer-63-b} soll deutlich schlechter sein als die Extraposition der Angabe {\it am Vormittag} in \ref{ex-storrer-63-c}. Das muss aber nicht so sein, wie \citet[Abschnitt~13.1.1]{Mueller:99} anhand von Korpusdaten zeigt. Die Extraponierbarkeit stellt also keinen Reflex des valenztheoretischen Status dar.

Im Gegensatz zum Permutationstest betrachtet man beim \textsc{Dialogtest}\is{Valenzermittlung!Dialogtest} eine Folge von Sätzen, beispielsweise die alternativen Frage-Antwort-Folgen in \ref{ex-dialogtest}. Hierbei wird angenommen, dass die Antwort \textit{das wei\ss \ ich nicht} auf eine Nachfrage nur dann möglich (bzw.\ angemessen) ist, wenn nach einer Angabe gefragt wird:

\ex.\label{ex-dialogtest}
Mitteilung: Die Neubauers sind gekommen. \\
Frage 1: Wohin sind sie gekommen? \\
Frage 2: Wann sind sie gekommen? \\
Antwort: Das wei\ss \ ich nicht. \\
\citep[68]{Storrer:92}

Die Antwort ist bei Frage 1 wesentlich schlechter als bei Frage 2 (vermutlich weil die Verwendung des Verbs \textit{gekommen} präsupponiert, dass der Sprecher das Ziel kennt). Daraus wird gefolgert, dass es sich bei der Richtungsangabe beim Verb {\it gekommen} um eine (fakultative) Ergänzung handelt, bei der Temporalbestimmung dagegen um eine Angabe. Abgesehen von der Fragwürdigkeit dieser Zuordnung funktioniert der Dialogtest in dieser Form nicht für alle Verben gleichermaßen gut. Beispielsweise könnte hier auch nach dem Akkusativobjekt von \textit{essen} gefragt werden, was demnach Angabenstatus hätte.  %\\


Valenzbeziehungen\is{Valenzbeziehung} ohne geeignete Sprechertests sind BET und ASSOZ. BET richtet sich nach der semantischen Rolle\is{semantische Rolle}, die lexikalisch/konzeptuell vorgegeben und damit einem Grammatikalitätsurteil nicht zugänglich ist. Ähnlich verhält es sich bei ASSOZ, welche auf Basis eines Assoziationstest und gleichsam grammatikextern ermittelt wird (siehe Abschnitt \ref{sec-valenzermittlung-konstruktiv}).

%\subsubsection*{Valenzermittlung und Grammatikalität}
\is{Grammatikalität|(}

Die Auseinandersetzung mit den Ermittlungsverfahren für NOT, FOSP, INSP, SUBKLASS und ARG hat gezeigt, dass sie alle mit spezifischen und allgemeinen Problemen behaftet sind, was Konzeption und Durchführung betrifft. \citet[215]{Storrer:92} fasst die daraus resultierende Situation so zusammen: 

\begin{enumerate}
  \item "`Für dieselbe verbspezifische Rolle werden bei verschiedenen, zur parallelen Verwendung vorgeschlagenen Tests unterschiedliche Me\ss ergebnisse erzielt."'
  \item "`Für dieselbe verbspezifische Rolle werden bei einem mit demselben Sys\-tem-Satz [kontextlose Beispielsätze, T.\,L.] durchgeführten Test bei unterschiedlichen Probanden verschiedene Me\ss ergebnisse erzielt."' 
\end{enumerate}
Was den ersten Punkt betrifft, lässt sich die Uneinheitlichkeit der Testergebnisse noch relativ leicht mit der Zugehörigkeit zu unterschiedlichen Valenzbeziehungen erklären, die ja naturgemä\ss\ eine unterschiedliche Extension haben können. Verglichen damit ist der zweite Punkt schwerwiegender und nicht so leicht aus der Welt zu schaffen -- wenn überhaupt. Schuld daran hat das grundsätzlich problematische Verhältnis zwischen Grammatikalitätsbegriff und Sprecherintuition.

Das entscheidende Kriterium bei Sprecherbefragungen anhand von System"=Sätzen ist der introspektive Wohlgeformtheitseindruck.\footnote{System-Sätze sind gewisserma\ss en "`Sätze in Vacuo"' (Clemens Knobloch, zitiert nach \citealt[12]{Ortner:87}).} Wie \cite{Storrer:92} betont, sei dieser jedoch nicht mit einem der Termini Grammatikalität und Akzeptabilität gleichzusetzen. Nach \cite{Chomsky:65} wird Grammatikalität durch das bereits vorliegende (abstrakte, kompetenzseitige) Grammatikmodell bestimmt, wogegen Akzeptabilität durch Performanzfaktoren gesteuert wird. Ein grammatischer Satz kann also sehr wohl unakzeptabel sein.\footnote{Der spiegelbildliche Fall, nämlich dass ein ungrammatischer Satz akzeptabel erscheint, kann ebenfalls eintreten \citep{Frazier:85,Gibson:Thomas:99}.} Bei der Sprecherbefragung zur Valenzermittlung ist das Prädikat \glq grammatisch\grq\ also deswegen fehl am Platz, weil gerade das zugrundeliegende Grammatikmodell unbekannt ist und Sprecherurteile generell als performanzbeeinflusst betrachtet werden müssen.\footnote{Chomsky erkennt das grundlegende Problem seines Ansatzes sehr wohl: "`no adequate formalizable techniques are known for obtaining reliable information concerning the facts of linguistic structure"' \citep[19]{Chomsky:65}. Mit anderen Worten: Wie kann ich etwas über das Grammatikmodell in Erfahrung bringen, wenn kompetenzreine, grammatische Sätze nicht zweifelsfrei bestimmt werden können, d.\,h.\ wenn alle Sprecherurteile potentiell performanzbeeinflusst sind? Sein Lösungsansatz besteht darin, von den \glqq klaren Fällen\grqq\ ("`crucial and clear cases"') auszugehen und daraus eine vorläufige Grammatik abzuleiten, welche dann auf \glqq schwierigere Fälle\grqq\ ("`unclear and difficult cases"') angewandt wird. Leider ist nicht immer offensichtlich, gerade wenn es um lexikalische Eigenschaften geht, was ein \glqq klarer Fall\grqq\ und ein \glqq schwieriger Fall\grqq\ sein soll und wie das eine aus dem anderen (nicht) abgeleitet werden kann.} Andererseits kann aber auch das Prädikat \glq akzeptabel\grq\ nicht gemeint sein, da die Sprecherbefragung kontextlose System-Sät\-ze verwendet und Performanzeinflüsse zu vermeiden sucht. Die Unschärfe vieler Valenzermittlungsverfahren lässt sich dann damit erklären, dass die Performanzeinflüsse nicht hinreichend eingeschränkt werden, etwa durch die Beigabe eines detaillierten Kontextes, sondern der Proband einem freien Kontextassoziationsspiel ausgesetzt wird, dessen Ausgang stark von valenzunabhängigen Faktoren abhängt.\footnote{Für eine allgemeine Auseinandersetzung mit den Problemen einer Gleichsetzung von introspektiven Sprecherurteilen und Grammatikalität (bzw.\ mit dem Grammatikalitätsbegriff an sich) siehe \cite{Schuetze:96} und \cite{Sampson:07}.}

Storrers Weg aus diesem Dilemma ist einerseits die strikte Trennung von Valenz und Grammatikalität und andererseits der explizite Bezug zu Kontextfaktoren. Dieser nichtsyntaktische Valenzbegriff, den Storrer in einem "`Modell der Situationsvalenz"' einführt, findet sich auch im STUG-Framework und soll daher in Abschnitt \ref{sec-situationsvalenz} genauer dargestellt werden. 

\is{Grammatikalität|)}


\subsection{Valenzermittlung mit statistischen Methoden} \label{sec-valenzermittlung-konstruktiv}

Anders als die Valenzermittlungsverfahren anhand einzelner Sprecherurteile und System-Sät\-ze gründen "`konstruktive Methoden"' \citep[228ff]{Storrer:92} auf Datenkollektiven und deren statistischen Merkmalen. Storrer zufolge muss man zwei Typen von Datenkollektiven unterscheiden: Korpora beobachteter Valenzrealisierungen ("`Korpussätze"') und psycholinguistische Versuchsergebnisse ("`Sprecherkollektivierung"'). 

Die Statistik von Textkorpora ist für die Valenzermittlung nur von Belang, wenn sich Valenzverhältnisse in der Häufigkeitsverteilung\is{Valenzermittlung!Korpusstatistik} ihrer Bestandteile niederschlagen. Bestandteile eines Textkorpus können nicht nur Worttoken, sondern auch z.\,B.\ Lemmatisierungen, morphosyntaktische Marker oder Phrasenstrukturmarker sein.\footnote{\cite{SchulteImWalde:02} betrachtet beispielsweise die Häufigkeitsverteilung lexikalisierter CFG-Regeln, die beim Parsen eines Textkorpus verwendet werden.} Storrer nennt als Beispiel Arbeiten zur historischen Valenzlexikographie, wo auf Basis kleiner historischer Korpora überaus simple statistische Methoden zur Valenzermittlung herangezogen werden: Es wird gemessen, wie häufig Komplementtypen zusammen mit einem bestimmten Valenzträger auftreten, und anschlie\ss end anhand eines mehr oder weniger willkürlichen prozentualen Schwellenwertes über den Ergänzungsstatus des Komplementtyps entschieden. Dieses Vorgehen wäre heutzutage, da sich die Methodik der statistischen Kookkurrenzanalyse wesentlich weiterentwickelt hat und gewisse Standards üblich geworden sind,\footnote{Siehe z.\,B.\ \cite{Lemnitzer:97}, \cite{Krenn:99}, \cite{Evert:05}.} sicherlich nicht mehr zu rechtfertigen. Allerdings gilt für die heute üblichen Valenzermittlungverfahren qua Korpusstatistik dieselbe Grundannahme wie damals -- die Grundannahme nämlich, dass Valenzträger mit Ergänzungen häufiger oder zumindest statistisch auffälliger kookkurieren als mit Nicht-Ergänzungen. Und auch heute muss in den meisten Fällen irgendwie ein statistischer Schwellenwert gesetzt werden, der signifikante von insignifikanten Kookkurrenzen trennt.\footnote{Siehe \cite{SchulteImWalde:09} für eine aktuelle Übersicht über korpusbasierte Valenzermittlungsverfahren.} Andernfalls erhielte man bestenfalls "`das komplette Rolleninventar"' \citep[232]{Storrer:92} ohne Valenzeingrenzung, d.\,h.\ sowohl Ergänzungsrollen als auch Angaberollen.

Als Beispiel einer Sprecher-Kollektivierung\is{Valenzermittlung!Sprecher-Kollektivierung} dient \citet[93ff, 232ff]{Storrer:92} Heringers Assoziationstest\is{Valenzermittlung!Assoziationstest} \citep{Heringer:85}. Hier wird nicht von System-Sätzen oder Korpussätzen ausgegangen, sondern von Verben in deren Infinitivform, zu denen Probanden Fragewörter möglicher Ergänzungsfragen nennen sollen. Aufgrund der Häufigkeit und Latenzzeit der Nennung eines Frageworts wird eine graduelle Valenzbindung zum Verb errechnet, die in einem folgenden Schritt anhand eines zu bestimmenden Schwellenwerts der Valenzbindungsstärke evaluiert werden kann.\footnote{Die zugrundeliegende Valenzbeziehung\is{Valenzbeziehung} wird von \citet[29]{Jacobs:94} als ASSOZ(IIERTHEIT) bestimmt, was \citet[209]{Agel:00} ablehnt und lieber von PRÄSUPP(OSITION) spricht. \'Agel nennt au\ss erdem Argumente dafür, dass "`PRÄSUPP kein psychologisches Korrelat von ARG sein kann"' \citep[210]{Agel:00}.}  Obwohl Heringers Studie schon ein paar Jahre zurückliegt, kann man nicht sagen, dass sich an den Grundsätzen der Herangehensweise in neueren Arbeiten wie z.\,B.\ \cite{SchulteImWalde:08} etwas wesentliches verändert hat.

Die Statistik von Textkorpora und Sprecher-Kollektivierungen\is{Valenzermittlung!Sprecher-Kollektivierung} der Valenzermittlung zugrunde zu legen, ist Storrer zufolge prinzipiell problematisch: Zum einen müsse der Schwellenwert festgelegt werden, ab dem ein statistisches Kookkurrenzverhältnis\is{Valenzermittlung!Korpusstatistik} als Valenzverhältnis gilt. Diese Festlegung ist in Storrers Augen künstlich und graduelle Valenzbindungsma\ss e damit kein Ausweg aus der Valenzermittlungsproblematik. Zum anderen müssen die anhand der Statistik eines Datenkollektivs gewonnenen Ergebnisse auf deren Individuen übertragen werden, womit eine unerlaubte "`Re-Individualisierung"' \citep[233]{Storrer:92} vorliege, die nämlich nur dann erlaubt sei, wenn sich alle Individuen des Datenkollektivs tatsächlich gleich verhielten. Eine dritte Schwierigkeit speziell für statistische Verfahren sei die potentielle Polysemie der Verben \citep[Fußnote 174]{Storrer:92}. Was das betrifft, könnten aber möglicherweise die Methoden der Word Sense Disambiguation etwas ausrichten, die in den letzten 20 Jahren spürbare Fortschritte gemacht haben (siehe \citealt{McCarthy:09}; \citealt[Chapter 20]{Jurafsky:Martin:09}).

Wenn aber die Festlegung eines Schwellenwerts zur E/A-Abgrenzung\is{E/A-Klassifikation!graduelle} prinzipiell problematisch ist, drängt sich natürlich die Frage auf, ob nicht stattdessen eine graduelle E/A-Abgrenzung angenommen werden sollte. Nur ist so eine graduelle Abgrenzung schwer mit Syntaxmodellen vereinbar, die von einer dichotomischen E/A-Abgrenzung ausgehen. Die Anerkennung der Gradualität führt nämlich letztlich zu einer kategorialen Nivellierung von Ergänzungen und Angaben, was sich bei \cite{Heringer:84} und im Situationsvalenzmodell\is{Situationsvalenz} von \cite{Storrer:92} auch ganz offen zeigt (siehe Abschnitt~\ref{sec-situationsvalenz}).	        


\section{Problem der Valenzrealisierung} \label{sec-valenzrealisierung}
\is{Valenzrealisierung}

Nehmen wir an, dass die Valenzermittlung (halbwegs zufriedenstellend) durchgeführt wurde und das Ergänzungsinventar eines bestimmten Verbs $V$ vorliegt. Zwei Fragestellungen hinsichtlich der Realisierung dieses Valenzrahmens in einem konkreten Satz $S$, d.\,h.\ hinsichtlich des Verhältnisses zwischen Valenz und Syntax, treten nun in den Vordergrund: 
\begin{enumerate}
  \item Welche Valenz(rahmen)bestandteile von $V$ werden in $S$ realisiert?
  \item Wie wird Valenz von $V$ strukturell in $S$ realisiert?
  %\item Werden die Valenzrahmenbestandteile von $V$ eindeutig realisiert oder sind sie in $S$ auch Bestandteile andere Valenzrahmen?
\end{enumerate}
Die erste Fragestellung möchte ich als Existenzfrage bezeichnen, die zweite dagegen als Strukturfrage.\footnote{\cite{Storrer:92} selber konzentriert sich bei ihrer Auseinandersetzung mit diesem Problemkomplex auf Teilaspekte der Existenzfrage, nämlich auf die obligatorisch/fakultativ-Klassifizierung.} Das Ergebnis der folgenden Auseinandersetzung mit diesen Fragestellungen wird die Formulierung dreier Idealisierungen der Valenzrealisierung sein.\footnote{Wie die Diskussion im vorangegangenen Abschnitt gezeigt hat, kann man auch in der  dichotomischen E/A-Abgrenzung\is{E/A-Klassifikation} eine Idealisierung\is{Idealisierung} des Valenzbegriffs erkennen. Diesen Hinweis verdanke ich Kim Gerdes.}

Der Terminus Idealisierung\is{Idealisierung} ist hier im Sinne von \cite{Stokhof:Lambalgen:11} zu verstehen. Stokhof und van Lambalgen unterscheiden zwischen Idealisierung ("`idealisation"') und Abstraktion ("`abstraction"') als Mittel der wissenschaftlichen Konzeptualisierung: Während die Abstraktion, die in den Naturwissenschaften vorherrsche, zu einer reversiblen Neutralisierung eines quantitativen Parameters führt, wird bei der Idealisierung ein qualitativer Parameter ignoriert. Die Idealisierung gehe also mit einem ontologischen Wechsel ("`ontological change"' oder "`ontological shift"') einher, welcher durch das resultierende Modell allein nicht rückgängig gemacht werden kann -- anders als bei der Abstraktion. Als Beispiele nennen Stokhof und van Lambalgen u.\,a.\ die Kompetenz"=Performanz"=Unterscheidung, die Unterscheidung zwischen semantischer und pragmatischer Bedeutung und die Annahme der Unendlichkeit natürlicher Sprache (zu Letzterem siehe auch \citealt{Pullum:Scholz:10}). 

\subsection{Existenzfrage} \label{sec-existenzfrage}

Auch wenn nach der Valenzermittlung eine Liste von Ergänzungen für ein Verb vorliegt, hei\ss t das nicht, dass alle diese Ergänzungen tatsächlich in einem konkreten Satz realisiert werden müssen. Es ist eben nicht allein das Notwendigkeitskriterium NOT\is{Valenzbeziehung!NOT} valenzbestimmend. In der Literatur ist deshalb eine Einteilung in obligatorische und fakultative Ergänzungen üblich, die zum Problem der Valenzrealisierung führt, d.\,h.\ zum Problem der o/f-Klassifikation\is{o/f-Klassifikation} der Ergänzungen des Valenzrahmens (siehe \citealt[95ff]{Storrer:92}). 

Das übliche Verfahren zur o/f-Klassifizierung ist der schon erwähnte Eliminierungstest\is{Valenzermittlung!Eliminierungstest} an kontextlosen System-Sätzen, wobei das Kriterium der introspektive Wohlgeformtheitseindruck von Sprechern ist. Dieses Verfahren ist jedoch denselben Einschränkungen (Abhängigkeit von der "`lexikalischen Füllung"', Kontinuität der Verb\-interpretation, Beschränkung auf SYN-NOT) unterworfen wie bei der E/A-Klassifikation\is{E/A-Klassifikation}. Und natürlich kann nicht erwartet werden, dass die intersubjekte Unschärfe mit einem Mal verschwunden ist, die damit zu erklären ist, dass zwar kein Kontext angegeben, aber implizit gefordert und sprecherspezifisch imaginiert wird. Die o/f-Klassifizierung erbt also die methodologischen Schwächen der E/A-Klassifikation \citep[240ff]{Storrer:92}.

Ungeachtet der methodologischen Schwächen der o/f-Klassifikation wird auf die Frage, welche Valenzrahmenbestandteil zu realisieren sind, gewöhnlich mit der folgende Idealisierung geantwortet:\footnote{Vgl.\ Storrers "`Teilregel 1: In einem Satz mit dem Verb Vi mu\ss\ jeder obligatorische Valenzpartner zu Vi realisiert sein."' \citep[110]{Storrer:92}}  

\begin{idealisierung}[Idealisierung der Vollständigkeit]\is{Idealisierung!der Vollständigkeit} 
Der Valenzträger und die obligatorischen Ergänzungen werden realisiert.
\end{idealisierung} 
Diese Idealisierung hat zwei Bestandteile, nämlich die Realisierung der Valenzträger und die Realisierung der obligatorischen Ergänzungen, die ich im Folgenden anhand prominenter Beispiele veranschaulichen möchte. %\\

Eine Idealisierung der Valenzträgerrealisierung wird besonders dann deutlich, wenn diese aus formalen Eigenschaften einer Repräsentationsstruktur folgt. Ein konsequentes Beispiel liefert etwa die \isi{Dependenzgrammatik} (siehe z.\,B.\ \citealt{Engel:88,Heringer:93,Eroms:00}), deren Repräsentationsstrukturen aus Graphen über Wörtern bestehen, in denen dem \isi{Valenzträger} eine zentrale Rolle zukommt: Als Knoten, der die Ergänzungen und Angaben dominiert, ist er in den meisten Fällen für das Zustandekommen eines verbundenen und daher wohlgeformten Dominanzgraphen notwendig. Ohne einen Valenzträger hängen die Dependenten quasi in der Luft. 

Auch in einer prominenten Konstituentengrammatik wie der Generativen\linebreak Grammatik\is{Generative Grammatik} findet man die Realisierung des Valenzträgers im Repräsentationsformalismus vorausgesetzt. In Chomskys ersten Beschreibung des X-bar-Schemas\is{X-bar-Schema} \citep[210]{Chomsky:70} treten beispielsweise die Valenzträger als Kopf\is{Kopf} ("`head"') derjenigen Phrase in Erscheinung, in der die Valenzrealisierung in der Tiefenstruktur lokalisiert ist (mehr dazu im nächsten Abschnitt). Zu den formalen Eigenschaften dieser Phrasenstrukturen (als Parsebäume kontextfreier Grammatiken) gehört, dass es strenggenommen keine leeren Köpfe\is{Kopf} geben kann, d.\,h.\ es kann keine nicht-terminalen Blätter geben. Man kann sich allerdings in der Generativen Grammatik\is{Generative Grammatik} mit phonetisch leeren Terminalen behelfen, den sogenannten leeren Kategorien\is{leere Kategorie} ("`empty categories"') -- ein Mittel, das Dependenzgrammatiken\is{Dependenzgrammatik} der obigen Art normalerweise nicht zur Verfügung steht, da leere Kategorien dort keinen Wortstatus haben (wobei es Ausnahmen gibt wie z.\,B.\ die "`zero verb forms"' in \citealt{Melcuk:09}).  %\\

Während die Valenzträgerrealisierung durch Eigenschaften des Repräsentationsformalismus in vielen Fällen bereits vorausgesetzt ist, kommt die Idealisierung\is{Idealisierung!der Vollständigkeit} der Valenzrealisierung meist expliziter zum Ausdruck. Als Beispiel möchte ich aus Gründen der Prominenz wiederum die Generative Grammatik\is{Generative Grammatik} herausgreifen, genauer gesagt die Government-and-Binding-Theorie\is{GB-Theorie} (GB-Theorie) in \cite{Chomsky:81}. Dort regelt das sogenannte Projektionsprinzip\is{Projektionsprinzip} ("`projection principle"') das Verhältnis zwischen Valenzeigenschaften ("`subcategorization properties"') und Syntax, welches lautet:\footnote{\label{fn-subkat-valenz}Ich setze hier der Einfachheit halber Valenz\is{Valenz} (bzw.\ "`$\theta$-marking"') mit Subkategorisierung\is{Subkategorisierung} gleich, obwohl Chomsky ein echtes Inklusionsverhältnis zwischen $\theta$-Markierung und Subkategorisierung stipuliert: Das Subjekt\is{Subjekt} sei zwar vom Verb $\theta$-markiert, es sei aber nicht Teil der Subkategorisierungsmerkmale des Verbs \citep[37]{Chomsky:81}. Das Projektionsprinzip müsse daher intuitiv folgenderma\ss en lauten: "`every syntactic representation (i.e. LF and D- and S-structure) should be a projection of the thematic structure and the properties of subcategorization of lexical entries"' \citep[36]{Chomsky:81}. Chomsky formuliert daraufhin eine angepasste, aber auch umständlichere Version des Projektionsprinzips \citep[38]{Chomsky:81}, deren Darstellung uns an dieser Stelle keine neuen Einsichten bieten würde.}  
\begin{quote}
Representations at each syntactic level (i.e. LF and D- and S-structure) are projected from the lexicon, in that they observe the subcategorization properties of lexical items. \citep[29]{Chomsky:81}
\end{quote}
Die Valenzrollen müssen also vollständig in der syntaktischen Struktur repräsentiert sein, wobei Chomsky keinen Unterschied zwischen obligatorischen und fakultativen Valenzrollen\is{Ergänzung} macht. Was man sich unter dieser Repräsentation genauer vorstellen muss, legt Chomsky im sogenannten $\theta$-Kriterium\is{theta-Kriterium@$\theta$-Kriterium} ("`$\theta$-criterion"') für die Tiefenstruktur fest:\footnote{\cite{Chomsky:81} ersetzt später die Bewegungstransformation durch Beschränkungen auf Ketten, die direkt auf der S-Struktur definiert sind: "`Eine Kette besteht aus den Positionen, die eine NP bei Bewege-$\alpha$ durchläuft"' \citep[263]{Stechow:Sternefeld:88}. Damit wird auch die Tiefenstruktur obsolet. In unnachahmlicher Weise benutzt Chomsky jedoch beide Begrifflichkeiten, Bewegung und Kette, parallel. Ich beschränke mich hier auf die Bewegungsmetapher.}  
\begin{quote}
Each argument bears one and only one $\theta$-role, and each $\theta$-role is assigned to one and only one argument. \citep[36]{Chomsky:81}
\end{quote} 
Zwischen syntaktischen Argumenten ("`argument[s]"') und semantischen Rollen\is{semantische Rolle} ("`$\theta$-role[s]"') soll ein eineindeutiges bzw.\ bijektives Abbildungsverhältnis bestehen. Nichts geht also verloren oder wird mehrfach verwendet. Ein syntaktisches Argument entspricht hier einem phrasalen Knoten im Syntaxbaum, wobei dieser Knoten auch eine leere Kategorie\is{leere Kategorie} dominieren könnte, also nicht overt realisiert sein muss.\footnote{Manche Autoren machen deshalb einen Unterschied zwischen in der Tiefenstruktur "`notwendigen"' Ergänzungen und in der Oberflächenkette "`obligatorischen"' Ergänzungen\is{Ergänzung} (siehe \citealt[100]{Storrer:92}). Die Mengen der "`notwendigen"' und "`obligatorischen"' Ergänzungen müssen dort nicht deckungsgleich sein, was die Möglichkeit für fakultative Ergänzungen eröffnet.} Soweit ich aber sehen kann, sieht \cite{Chomsky:81} eine Basisgenerierung leerer Kategorien, mit Ausnahme von PRO\is{PRO} in der Subjektposition\is{Subjekt} von nicht-finiten Komplementsätzen\is{Komplementsatz}, nicht vor. Das hat zur Folge, dass in der Regel alle thematischen Rollen\is{theta-Rolle@$\theta$-Rolle} eines Verbs, sein sogenanntes $\theta$-Raster\is{theta-Raster@$\theta$-Raster}, auch eineindeutig oberflächenrealisiert werden müssen.

Wir haben bereits festgestellt, dass Chomsky in einem gegebenen $\theta$-Raster\is{theta-Raster@$\theta$-Raster} nicht zwischen obligatorischen und fakultativen $\theta$-Rollen\is{theta-Rolle@$\theta$-Rolle} unterscheidet. Bezogen auf ein gegebenes $\theta$-Raster ist also Chomskys Ansatz der Valenzrealisierung strikter als die Idealisierung der Vollständigkeit\is{Idealisierung!der Vollständigkeit}, da letztere immerhin das Fehlen der fakultativen Ergänzungen zulässt. Allerdings lässt sich das Konzept der fakultativen Ergänzung\is{Ergänzung!fakultative} auch in Chomskys Ansatz implementieren, wenn es möglich ist, einem Verb mehrere unterschiedliche $\theta$-Raster zuzuweisen. Dann wären nämlich nur diejenigen $\theta$-Rollen obligatorisch, die in allen $\theta$-Rastern vertreten sind, während alle anderen $\theta$-Rollen qua $\theta$-Raster-Auswahl einen fakultativen Status haben.\footnote{Etwas Analoges schlägt beispielsweise \cite{Jacobs:94a} vor.} Eine solche Wahlfreiheit unter verschiedenen $\theta$-Rastern\is{theta-Raster@$\theta$-Raster} scheint aber bei Chomsky nicht gegeben zu sein. Vielmehr scheint es so zu sein, dass sich das $\theta$-Raster eines Verbs exakt aus seiner Bedeutung ableiten lässt, dass nämlich "`ein n-stelliges Prädikat n verschiedene $\theta$-Rollen vergibt"' \citep[258]{Stechow:Sternefeld:88}.\footnote{Ich drücke mich hier vorsichtig aus, da ich in \cite{Chomsky:81} keine eindeutigen Aussagen diesbezüglich finden kann.} Ein Kernmerkmal dieser sowie anderer Grammatiktheorien aus dem Bereich der Generativen Grammatik\is{Generative Grammatik} ist nach \cite{Culicover:Jackendoff:05} deren \isi{Schnittstellenuniformität} ("`Interface Uniformity"'):

\begin{quote}
The syntax-semantics interface is maximally simple, in that meaning maps transparently onto syntactic structure; and it is maximally uniform, so that the same meaning always maps onto the same syntactic structure. \citep[47]{Culicover:Jackendoff:05}
\end{quote}
Die eindeutige Zuordnung von Bedeutung und Valenzrahmen fehlt in dem in Abschnitt \ref{sec-valenzbegriff} entwickelten Valenzbegriff, bei dem die Bedeutung die Valenz zwar motiviert, aber nicht vorgibt. Die Folgen der \isi{Schnittstellenuniformität} sind weitreichend und betreffen keinesfalls nur fakultative Ergänzungen\is{Ergänzung!fakultative}. Ein Beispiel: Das Verb {\it lügen} würde dann auch die blockierte Patiensrolle in die syntaktische Struktur projizieren, weil sie Bestandteil des semantischen Rolleninventars von \textit{lügen} ist, wie die overte Realisierung bei {\it belügen} zeigt (siehe S.\,\pageref{ex-luegen}). Man erhielte bei \textit{lügen} also zwangsweise zu viele Argumentpositionen in der syntaktischen Struktur, die mittels leerer Kategorien\is{leere Kategorie} gefüllt werden müssten. Genau das (mittels PRO\is{PRO}) wird zumindest bei der Subjektposition\is{Subjekt} von Infinitiven üblicherweise gemacht. Die Alternative kann nur darin bestehen, für {\it lügen} und {\it belügen} unterschiedliche Bedeutungen zu stipulieren, d.\,h.\ bei \textit{lügen} keine Patiensrolle zuzulassen. Das wird aber der konzeptionellen Übereinstimmung von \textit{lügen} und \textit{belügen} nicht gerecht -- und kein Synaxmodell wäre dieses Opfer wert.\footnote{Ein weiteres interessantes Problemfeld für Grammatiktheorien mit \isi{Schnittstellenuniformität} sind polyvalente Verben wie {\it rollen}. Siehe \citet{Vogel:98}.} 
%
Nehmen wir also an, dass \textit{lügen} eine blockierte Patiensrolle und \textit{essen} eine fakultative Ergänzung\is{Ergänzung!fakultative} in die Syntax projiziert (was Chomsky meines Wissens weder offen befürwortet noch ausschließt). Wo steckt dann die Idealisierung der Vollständigkeit\is{Idealisierung!der Vollständigkeit}? Kann nicht durch die Verwendung von leeren Kategorien\is{leere Kategorie} jegliche Valenzrealisierung modelliert werden? Das ist jedoch nicht der entscheidende Punkt. Entscheidend ist, dass es in der GB-Theorie\is{GB-Theorie} leere Kategorien geben {\it muss}, weil die Tiefenstruktur\is{Tiefenstruktur} das Ideal der Vollständigkeit erfüllt.  %\\

Dass die Idealisierung der Vollständigkeit\is{Idealisierung!der Vollständigkeit} empirisch zu kurz greift, steht eigentlich au\ss er Frage: Sowohl \isi{Valenzträger} als auch obligatorische Ergänzungen\is{Ergänzung!obligatorische} können weggelassen werden, was in den zweiten Konjunkten der folgenden Koordinationen\is{Koordination} beispielhaft zu sehen ist:

\ex. \label{ex-existenz}
\a. \label{ex-existenz-a} [$\kappa_1$ Der Jäger sah einen Hasen] und [$\kappa_2$ der Tourist einen Waschbären].
\b. \label{ex-existenz-b} [$\kappa_1$ Einen Hasen sah der Jäger] und [$\kappa_2$ erschoss ihn]
\c. \label{ex-existenz-c} [$\kappa_1$ Einen Hasen sah der Jäger] und [$\kappa_2$ einen Waschbären]

Das Weglassen des Valenzträgers in \ref{ex-existenz-a} wird als Gapping\is{Gapping} bezeichnet und das Fehlen einer obligatorischen Ergänzung in \ref{ex-existenz-b} findet im Rahmen einer sogenannten asymmetrischen Koordination\is{Koordination!asymmetrische} statt, wohingegen das Fehlen sowohl des Valenzträgers als auch der obligatorischen Ergänzung in \ref{ex-existenz-c} eine sogenannte Bare Argument Ellipsis\is{Ellipse!Bare Argument Ellipsis} konstituiert. Alle diese Phänomene werden in Kapitel~\ref{chap-ellipse} einer näheren Betrachtung unterzogen.



\subsection{Strukturfrage} \label{sec-strukturfrage}

Die Strukturfrage hat eine horizontale und eine vertikale Dimension: Die horizontale Dimension umfasst Aspekte der \isi{Linearisierung} der Valenzrahmen als Teil von Dependenzfeldern\is{Dependenzfeld}, also die lineare (Dis-)Kontinuität der Dependenzfelder und die relative lineare Positionierung von Regent und Dependenten. Die vertikale Dimension betrifft dagegen Aspekte der Valenzrahmenzugehörigkeit bzw.\ Dependenzfeldzugehörigkeit. Im Folgenden werde ich kurz auf die (Dis-)""Kontinuität der Dependenzfelder und auf einen Aspekt der Dependenzfeldzugehörigkeit eingehen. Zur Lokalisierung von Valenzrahmenbestandteilen in einer hierarchischen Repräsentationsstruktur, z.\,B.\ in Dependenzgraphen oder Konstituentenstrukturen, werde ich mich ausführlicher in Kapitel \ref{sec-valenz-tag} äu\ss ern, wo die TAG-Modellierung von Valenzrahmen im Vordergrund steht.

Was die Linearisierung der Dependenzfelder (und der darin enthaltenen Valenzrahmen) betrifft, erweist sich die folgende Idealisierung als grammatiktheoretisch wirkungsmächtig:\footnote{Alternativ könnte man auch von einer Idealisierung der Überschneidungsfreiheit sprechen und sie folgenderma\ss en formulieren: Valenzrealisierungen bzw.\ Dependenzfelder überschneiden sich nicht linear.}  
\begin{idealisierung}[Idealisierung der Kontinuität]\is{Idealisierung!der Kontinuität}
Die Linearisierung eines Dependenzfelds\is{Dependenzfeld} ist zusammenhängend.
\end{idealisierung}  
Die lineare Kontinuität von Dependenzfeldern im Satz wurde insbesondere im Rahmen der Dependenztheorie in den 1960ern unter dem Stichwort der \textsc{Projektivität}\is{Projektivität} herausgestellt \citep[73ff]{Dikovsky:Modina:00} und verschiedentlich formal charakterisiert \citep[17ff]{Kuhlmann:10}. Sehr allgemein gesprochen kann man mit \citet[51]{Brettschneider:78} das der Projektivität zugrunde liegende Prinzip folgenderma\ss en auf den Punkt bringen: "`alles Zusammengehörende hat zusammenzustehen"'.\footnote{Das erinnert natürlich stark an das erste Behaghel'sche Gesetz, "`daß das geistig eng Zusammengehörige auch eng zusammengestellt wird"' \citep[\S 1426]{Behaghel:32}.} Das bedeutet jedoch nicht, dass der Dependent immer neben dem Kopf steht. Beispielsweise kommt das in der folgenden Linearisierungsregel ("`sequence rule"') aus \citet{Hudson:80} zum Ausdruck:

\ex. The modifiers [= Dependenten, T.\,L.] of a head should not be separated from it by any other items except other modifiers of the same head. \citep[(9)]{Hudson:80}
 
Die Attraktivität der Annahme, dass alle Dependenzstrukturen\is{Dependenzgraph} projektiv sind, beruht auch auf ihrem restringierenden, ordnenden Einfluss auf die Theoriebildung, "`weil sie die rein algebraische Kombinatorik spektakulär einschränkt und damit fundamentale Eigenschaften der Sprache schon durch die Form der Beschreibungssprache erfasst"' \citep[250]{Eroms:Heringer:03}. Das hat bei der Zuweisung einer konkreten Dependenzstruktur den Effekt, dass versucht wird, Dependenzen\is{Dependenz} so zu bestimmen, dass die resultierende Dependenzstruktur\is{Dependenzgraph} dem Ideal der Projektivität genügt \citep[251]{Eroms:Heringer:03}. Ist die Nichtprojektivität einer Dependenzstruktur dagegen unvermeidbar, so wird dies als "`Signal für die Markiertheit dieser Typen"' \citep[259]{Eroms:Heringer:03} interpretiert.   

Zusätzlich zur Attraktivität projektiver Dependenzstrukturen\is{Dependenzgraph} beigetragen hat in meinen Augen zweierlei: Zum einen wei\ss\ man seit \cite{Gaifman:65}, dass eine schwache Äquivalenz zwischen projektiven Dependenzstrukturen und kontextfreie Grammatiken\is{kontextfreie Grammatik} (CFG) besteht. Sie können also dieselben Stringsprachen erzeugen und haben eine vergleichsweise niedrige computationelle Komplexität. Zum anderen ist die Klasse der nicht-projektiven Dependenzstrukturen computationell schwer zu beherrschen, d.\,h.\ die Erkennung der entsprechenden Stringsprachen ist NP-vollständig \citep{Neuhaus:Broeker:97}.\footnote{In jüngeren Arbeiten (z.\,B.\ \citealt{Holan:etal:98,Dikovsky:Modina:00,Bodirsky:etal:05,Havelka:07,Maier:Lichte:11,Gomez:etal:11}) wird der Frage nachgegangen, wie sich derjenige Teilbereich der Nicht-Projektivität formal bestimmen lässt, der für natürliche Sprache tatsächlich relevant ist, und ob dieser Teilbereich computationell effizient verarbeitbar ist. Die Projektivität erweist sich jedoch empirisch betrachtet als dependenzstruktureller Normalfall (\citealt{Kuhlmann:Nivre:06,Maier:Lichte:11}).} %\\

Doch die Idealisierung der Kontinuität\is{Idealisierung!der Kontinuität} von Dependenzfeldern findet sich auch in der \isi{Tiefenstruktur} der Generativen Grammatik\is{Generative Grammatik}. Laut \citet[122, 215]{Chomsky:65} sollen dort die Valenzrelationen ("`strict subcategorization"', wobei das Subjekt ausgenommen ist) streng lokal ("`strictly local"') realisiert werden, d.\,h.\ nur innerhalb einer sogenannten lexikalischen Kategorie ("`lexical category"'). Die lexikalische Kategorie kann man sich als denjenigen Phrasenstrukturknoten vorstellen, dessen Kopf\is{Kopf} der Valenzträger\is{Valenzträger} bildet.\footnote{Den Terminus Kopf ("`head"') benutzt \cite{Chomsky:65} an dieser Stelle allerdings nicht.} Bei der Einführung des X-bar-Schemas\is{X-bar-Schema} in \cite{Chomsky:70} kommt dieser Aspekt deutlicher zum Vorschein:
\begin{quote}
To introduce a more uniform notation, let us use the
symbol $\overline{X}$ for a phrase containing $X$ as its head. Then the base rules introducing $N$, $A$, and $V$ will be replaced by a schema (48), where in place
of \ldots\ there appears the full range of structures that serve as complements
and $X$ can be any one of $N$, $A$, or $V$:

\ex.[(48)] $\overline{X} \to X$ \ldots

Continuing with the same notation, the phrases immediately dominating $\overline{N}$,
$\overline{A}$ and $\overline{V}$ will be designated $\overline{\overline{N}}$, $\overline{\overline{A}}$, $\overline{\overline{V}}$ respectively. \citep[210]{Chomsky:70} 
\end{quote}    
Die Valenzrahmenbestandteile ("`head"' und "`complements"')\is{Kopf}\is{Komplement} bilden also die rechte Seite einer kontextfreien Regel. Damit ist ihre Kontinuität in der abgeleiteten Tiefenstruktur zwingend. Varianten dieser Kontinuitätsidealisierung setzen sich in neuere Arbeiten fort, meist wenn Grundbegriffe wie Phrase\is{Phrase} oder Konstituente\is{Konstituente} erklärt werden:
\begin{quote}
Der Kopf einer Wortgruppe/""Konstituente/""Phrase/""Projektion ist dasjenige Element, das die wichtigsten Eigenschaften der Wortgruppe/""Konstituente/""Phrase/""Projektion bestimmt. Gleichzeitig steuert der Kopf den Aufbau der Phrase, d.\,h.\ der Kopf verlangt die Anwesenheit bestimmter anderer Elemente in seiner Phrase. \citep[19]{Mueller:10}
\end{quote}
Entscheidend ist hier, dass Müller eine "`Wortgruppe/""Konstituente/""Phrase/""Projektion"' als "`linear zusammenhängende Folge von Wörtern"' \citep[4]{Mueller:10} bzw.\ als "`Schachtel"' \citep[Abbildung~1.1]{Mueller:10} versteht. Etwas technischer drückt sich \cite{Jacobs:09} aus:
\begin{quote}
Wenn $X$ als valenzgebundenes Element und $Y$ als Valenzträger analysiert wird, ist $X$ in der syntaktischen Struktur mit $Y$ oder einer Phrase, deren Kopf $Y$ ist, verschwestert, wobei diese $Y$ enthaltende Phrase wiederum der Kopf der Phrase ist, die $X$ und $Y$ umfasst. \citep[497]{Jacobs:09}
\end{quote}
Da Phrasen hier wohl als linear kontinuierlich begriffen werden (denn Jacobs verwendet keine kreuzenden Kanten in den Phrasenstrukturbäumen), können sich Valenzrealisierungen nicht linear überschneiden.  

Eine abgeschwächte Form der Kontinuitätsidealisierung\is{Idealisierung!der Kontinuität} findet sich auch in den Wohlgeformtheitsbedingungen des TAG-Frameworks, denen wir uns in Abschnitt \ref{sec-tag-ling} zuwenden werden. %\\

Die empirischen Unzulänglichkeiten der Idealisierung der Kontinuität\is{Idealisierung!der Kontinuität} sind hinlänglich bekannt. Die Sätze in \ref{ex-realisierung-1} enthalten allesamt diskontinuierliche Dependenzfelder\is{Dependenzfeld} bzw.\ Valenzrahmenrealisierungen\is{Valenzrahmenrealisierung}:

\ex. \label{ex-realisierung-1}
\a. \label{ex-realisierung-1-b} dass den Kühlschrank Peter heute zu reparieren versucht
\b. \label{ex-realisierung-1-c} Wen meint Doris, dass Gerhard liebt? \hfill \citep[(2c)]{Featherston:05}
\c. \label{ex-realisierung-1-d} Über Syntax hat Sarah sich ein Buch ausgeliehen. \\ \citep[(1)]{DeKuthy:02}  

Satz \ref{ex-realisierung-1-b} zeigt eine sogenannte kohärente Konstruktion\is{kohärente Konstruktion} und Satz \ref{ex-realisierung-1-c} eine Brückenkonstruktion\is{Brückenkonstruktion}. Dass selbst nominale Valenzrahmen diskontinuierlich sein\linebreak können, verdeutlicht schließlich Satz \ref{ex-realisierung-1-d}. In dieser Arbeit werde ich daraus das Kohärenzphänomen herausgreifen und in Abschnitt \ref{chap-kohaerenz} detailliert darstellen. Um solche Diskontinuitäten analysieren zu können, ohne die Idealisierung der Kontinuität aufzugeben, wird beispielsweise in transformationsgrammatischen Theorien auf Bewegungstransformationen zurückgegriffen (siehe Abschnitt \ref{sec-ttmctag-modellierungsstrategien}). %\\


Abschlie\ss end möchte ich eine Idealisierung formulieren, die auf die vertikale Dimension der Strukturfrage (Wie sind Valenzbeziehungen in einem Satz hierarchisch organisiert?) Bezug nimmt, nämlich auf die Dependenzfeldzugehörigkeit oder, etwas eingegrenzter, auf die Zugehörigkeit zu einer Valenzrahmenrealisierung. Entsprechend dieser Eingrenzung findet man in vielen Syntaxtheorien die folgende Idealisierung: 

\begin{idealisierung}[Idealisierung der Funktionalität]\is{Idealisierung!der Funktionalität} 
Die Valenzbeziehungen in einem Satz sind hinsichtlich der daran beteiligten Verben und Konstituenten funktional. 
\end{idealisierung} 
Oben in Definition~\ref{def:valenzbeziehung} habe ich die Valenzbeziehungen\is{Valenzbeziehung} in einem Satz $S$, genannt $\mathsf{ValR}(S)$, als Menge von Tripeln $\langle V, K, R \rangle$ einer Verbform $V$, eines Satzglieds $K$ und einer semantischen Rolle\is{semantische Rolle} $R$ definiert. Diese Valenzbeziehungen können auf unterschiedliche Weise funktional eingeschränkt sein. Eine bereits erwähnte funktionale Einschränkung ist etwa, dass jede Konstituente $K$ in einer Valenzbeziehung zu maximal einem Verb $V$ steht, d.\,h.\ $\{\langle K,V \rangle | \langle V,K,R \rangle \in \mathsf{ValR}(S) \}$ ist eine partielle Funktion. Mit anderen Worten: Es gibt in einem Satz keine zwei Verbformen $V_1$, $V_2$ und ein Satzglied $K$ mit den Valenzbeziehungen $\langle V_1, K, R_i \rangle$ und $\langle V_2, K, R_j\rangle$ (für beliebige $R_i$, $R_j$). Diese Einschränkung wird im Rahmen der \isi{Dependenzgrammatik} durch den Ausschluss von Multidominanz in den Dependenzgraphen erreicht. In der \isi{GB-Theorie} ist die Idealisierung der Funktionalität dagegen Folge des $\theta$-Kriteriums\is{theta-Kriterium@$\theta$-Kriterium}, wonach eine Argumentposition genau eine $\theta$-Markierung haben muss.

Die Idealisierung der Funktionalität\is{Idealisierung!der Funktionalität} mag auf den ersten Blick harmlos erscheinen, doch es handelt sich dabei tatsächlich um eine Idealisierung. Das wird an Sätzen wie \ref{ex-injetiv-a} mit einer sogenannten Kontrollkonstruktion\is{Kontrolle} deutlich: 

\ex. \label{ex-injetiv} 
\a. \label{ex-injetiv-a} Peter versucht zu schlafen.
\b. \label{ex-injetiv-b} {\tt versucht(Peter,} {\tt schlafen(Peter))}

Das Nomen {\it Peter} ist hier nämlich das Argument zweier Prädikationen, angedeutet in \ref{ex-injetiv-b}, und aufgrund zweier bestehenden ARG-Beziehung\is{Valenzbeziehung!ARG} und entsprechend des oben dargestellten Valenzbegriffs auch Bestandteil zweier Valenzrahmenrealisierungen. Gegen eine Valenzbeziehung zwischen {\it zu schlafen} und {\it Peter} könnte man zwar vorbringen, dass gleichzeitig keine FOSP-\is{Valenzbeziehung!FOSP} oder NOT-Beziehung\is{Valenzbeziehung!NOT} besteht, und dieses als Bedingung in obigen Valenzbegriff integrieren. Aber genau dies wird in der \isi{GB-Theorie} durch die Idealisierung der Vollständigkeit\is{Idealisierung!der Vollständigkeit} qua \isi{Schnittstellenuniformität} ausgeschlossen. Die Kombination aus Idealisierung der Funktionalität und Idealisierung der Vollständigkeit führt also angesichts von Kontrollkonstruktionen wie in \ref{ex-injetiv-a} zu einem Dilemma: {\it zu schlafen} kann {\it Peter} nicht $\theta$-markieren, weil es bereits von {\it versucht} $\theta$-markiert ist; die Argumentposition, die {\it zu schlafen} $\theta$-markieren muss, bleibt also leer. Der Lösungsvorschlag von \cite{Chomsky:81} besteht darin, in der Argumentposition von {\it zu schlafen} ein phonetisch leeres Pronomen \isi{PRO} anzunehmen, das durch das Nomen {\it Peter} "`kontrolliert"' wird. Um PRO wieder loszuwerden, muss also entweder die Idealisierung der Vollständigkeit qua Schnittstellenuniformität oder die Idealisierung der Funktionalität, oder beide, angepasst bzw.\ fallengelassen werden (\citealt{Culicover:Wilkins:86}; \citealt[46, \S3.1.1]{Culicover:Jackendoff:06}). 

Man kann die Funktionalität von Valenzbeziehungen aber auch anders bestimmen und die semantische Rolle\is{semantische Rolle} miteinbeziehen. Eine solche Einschränkung ist beispielsweise, dass jedes $K$ nicht mehr als eine semantische Rolle in $V$ innehaben kann, dass also $\{\langle (V,K), R \rangle | \langle V,K,R \rangle \in \mathsf{ValR}(S) \}$ funktional ist. Empirisch scheint das nur schwer widerlegbar zu sein. Betrachtet man dagegen die Menge der Paare $\langle (V,R),K \rangle$, dann impliziert deren Funktionalität, dass jede semantische Rolle von $V$ nur einmal besetzt werden kann. Mit anderen Worten: Ergänzungen\is{Ergänzung} sind (ohne Koordination) nicht iterierbar. Das \textsc{Kriterium der Iterierbarkeit}\is{Iterierbarkeit} wird in der Literatur gerne zur Unterscheidung von Ergänzungen und Angaben\is{E/A-Klassifikation} herangezogen (siehe z.\,B.\ \citealt[22f]{Mueller:10}) und flie\ss t auch in Chomskys $\theta$-Kriterium\is{theta-Kriterium@$\theta$-Kriterium} ein.\footnote{Ähnlich auch das  "`Stratal Uniqueness Law"' aus \citet[255]{Kracht:02}: "`For a given predicate there can be at most one constituent bearing a particular relation to that predicate."'} Dem widersprechen zumindest oberflächlich betrachtet die Jacobs-Daten in \ref{ex-iterierbar-jacobs}, die er als Fälle von "`Akkumulation"' behandelt:

\ex.\label{ex-iterierbar-jacobs}
\a. Er wohnt in München in der Innenstadt in einem Altbau.\label{ex-iterierbar-jacobs-a}
\b. Sie trug den Korb in den Garten unter den Apfelbaum.
\c. Das Unglück ereignete sich am Sonntag um vier Uhr früh.
\d. Die Besprechung dauerte den ganzen Nachmittag bis in die frühen Abendstunden.
\e. Er benahm sich wie üblich miserabel.
\z. \citep[(11), (13)--(15)]{Jacobs:94}

Die hier iterierten Bestimmungen der Position, Richtung, Zeit etc.\ stünden in einer "`Spezifikationsbeziehung"' zueinander, seien aber jeweils durch das Verb valenzgebunden (siehe auch z.\,B.\ \citealt{Maienborn:91, Steinitz:92}). Folglich wären etwa die semantische Rolle der Positionsbestimmungen in \ref{ex-iterierbar-jacobs-a} dreimal realisiert, nämlich durch \textit{in München}, \textit{in der Innenstadt} und \textit{in einem Altbau}. Es ist aber auch denkbar, dass diese Realisierungen syntaktisch direkter zusammenhängen, dass also in \ref{ex-iterierbar-jacobs-a} nur \textit{in einem Altbau} die Ergänzung  von \textit{wohnt} ist, während es sich bei \textit{in München in der Innenstadt}  um ein disloziertes Adjunkt von \textit{in einem Altbau} handelt -- darin vielleicht der Diskontinuität bei NPs (NP-PP-Splits) nicht unähnlich, die oben anhand von \ref{ex-realisierung-1-d} erläutert wurde. Oder man könnte stattdessen argumentieren, dass \textit{in München in der Innenstadt} ein  "`frame adjunct"' \citep{Maienborn:01,Frey:03} ist, das den Wahrheitsanspruch bzw.\ propositionalen Gehalt von \ref{ex-iterierbar-jacobs-a} wesentlich allgemeiner einschränkt als die Ergänzungen des Verbs. 

An diesen Überlegungen wird deutlich, dass sich die Valenzbeziehungen\is{Valenzbeziehung}, die im Zusammenhang mit der Idealisierung der Funktionalität\is{Idealisierung!der Funktionalität} interessant sind, anscheinend schwieriger beurteilen lassen und dass hier in stärkerem Maße zusätzliche Annahmen eine Rolle spielen. Ich möchte daher diesen Bereich weitgehend ausklammern und mehr Gewicht auf die Auseinandersetzung mit den Idealisierungen der Vollständigkeit und Kontinuität legen.


\section{Zusammenfassung}

Das Valenzkonzept, so der Ausgangspunkt dieser Arbeit, ist ein wesentlicher Bestandteil heutiger Syntaxtheorien. Dieses Kapitel diente der Klärung des Fragekomplexes, was Valenz überhaupt ist, wie sie ermittelt und syntaktisch realisiert werden kann. 

Bei der Klärung des traditionellen Valenzbegriffs habe ich mich des multikriterialen Ansatzes von \cite{Jacobs:94} bedient, der zur Explikation der Valenzintuition die Valenzkriterien\is{Valenzkriterium} NOT, FOSP, INSP und ARG ins Feld führt und damit dem sehr uneinheitlichen Bild der Valenzforschung Rechnung trägt. Darauf aufbauend habe ich die Valenz als Menge von 3-Tupeln aus semantischer Rolle, morphosyntaktischem Formmerkmal und Notwendigkeitsmerkmal definiert.  

Diese Uneinheitlichkeit, d.\,h.\ das Fehlen eines überzeugenden monokriterialen Ansatzes, bedingt eine Diversifizierung der Valenzermittlungsmethoden\is{Valenzermittlung}, ohne leider in den meisten Fällen passgenau und zuverlässig eine bestimmte Valenzbeziehung identifizieren zu helfen. Methoden wie z.\,B.\ der Eliminierungstest für NOT und der Geschehen-Test für ARG, die auf introspektiven Grammatikalitätseindrücken beruhen, scheinen grundsätzlich durch Performanzeinflüsse beeinträchtigt. Es ist zudem oft strittig, insbesondere beim Geschehen-Test, ob mit einem bestimmten Testverfahren die anvisierte Valenzbeziehung überhaupt getestet werden kann. Statistische Methoden  versprechen demgegenüber einen objektiveren Standpunkt durch die Erstellung von Kookkurrenzmustern aus einer Vielzahl von Valenzrealisierungsinstanzen. Hier gilt es jedoch zu klären, ob Kookkurrenz ein hinreichend eindeutiger Reflex von Valenzbeziehungen darstellt, wie der Verbpolysemie mit statistischen Mitteln begegnet werden kann, und vor allem, wie sich der zwingend benötigte statistische Schwellenwert, der die Grenze zwischen Ergänzung und nicht-Ergänzung definiert, nicht ad hoc festlegen lässt.

Mit der Unklarheit des Valenzkonzepts und dem Problem der Valenzermittlung hängt natürlich auch das Problem der \isi{Valenzrealisierung}, insbesondere das Problem der o/f-Unterschei\-dung, eng zusammen. Würden der gesamte Valenzrahmen oder ein bestimmter Teil des Valenzrahmens immer realisiert sein, dann hätte man bei der Klärung des Valenzkonzepts und bei der Valenzermittlung leichtes Spiel. Tatsächlich, so meine These, gehen die meisten Syntaxmodelle bei der Integrierung des Valenzkonzepts genau diese Idealisierung\is{Idealisierung} der Vollständigkeit ein, zusammen mit der Idealisierung der Kontinuität und der Idealisierung der Funktionalität. Die nächsten beiden Kapitel werden jedoch zeigen, dass die Realität anders aussieht: Zum einen können beliebige Bestandteile eines realisierten Valenzrahmens abhängig vom Äu\ss erungskontext weggelassen werden; zum anderen können Valenzrahmen linear diskontinuierlich (bzw.\ überlappend) realisiert sein. Zur Modellierung solcher Phänomene müssen dann Ausweichstrategien implementiert werden, auf die ich in den Kapiteln \ref{sec-kohaerenz-tag} und \ref{sec-ellipsenanalyse} eingehe. 

Es ist klar, dass in dieser Hinsicht Syntaxmodelle von Vorteil sind, die keine Realisierungsidealisierungen\is{Idealisierung} inkorporieren, oder zumindest nicht alle. Solche Syntaxmodelle und ihre sowohl theoretische als auch empirische Tragfähigkeit stehen im Zentrum des Interesses dieser Arbeit: In Kapitel \ref{sec-ttmctag} werde ich ein Syntaxmodell auf Grundlage von TT-MCTAG ausarbeiten, das ohne Idealisierung der Kontinuität auskommt, während in Kapitel \ref{ch-ohne-valenz} eine etwas programmatischere Darstellung eines Syntaxmodells (STUG) folgt, dass zudem die Idealisierung der Vollständig umgeht. Das letztere Syntaxmodell greift Storrers Idee der Situationsvalenz\is{Situationsvalenz} auf, das wir in Abschnitt \ref{sec-situationsvalenz} kennenlernen werden. Darin wird Valenz primär im Schnittstellenbereich zwischen Lexikon, Semantik und Informationsstruktur angesiedelt, was für das Syntaxmodell natürlich gravierende Folgen hat, bedeutet dies doch nichts weniger als eine Trennung von Syntax und Valenz.  

