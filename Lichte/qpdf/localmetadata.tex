\title{Syntax \newlineCover und Valenz}
\subtitle{Zur Modellierung kohärenter und elliptischer Strukturen mit Baumadjunktionsgrammatiken}  
\author{Timm Lichte}
\renewcommand{\lsISBNdigital}{978-3-944675-51-0}
\renewcommand{\lsISBNhardcover}{978-3-944675-62-6}
\renewcommand{\lsISBNsoftcover}{978-3-944675-63-3}
\renewcommand{\lsISBNsoftcoverus}{978-1-523743-78-0}
%\dedication{Change dedication in localmetadata.tex}
\typesetter{Timm Lichte, Sebastian Nordhoff}
\proofreader{Armin Buch, Constantin Freitag, Martin Haspelmath, Sebastian Nordhoff, Daniela Schröder, Aviva Shimelman, Charlotte van Tongeren}
\BackTitle{Syntax und Valenz} % Change if BackTitle != Title
\BackBody{%
	So unterschiedlich die etablierten Syntaxmodelle auch sein mögen, sie alle haben gemeinsam, dass sie eng mit lexikalischen Valenzeigenschaften verzahnt sind: Valenztheoretisch fundierte Kategorien wie Valenzträger, Ergänzung und Angabe spielen in ihnen zentrale, klar differenzierte Rollen.
	%%%
	Das vorliegende Buch hat den Zweck, diesen gemeinhin ausgeblendeten Konsens aufzugreifen, hinsichtlich seiner Grundlagen und Auswirkungen zu untersuchen und schließlich auch in Frage zu stellen. Der empirische Schwerpunkt wird dabei auf kohärenten Konstruktionen und Ellipsen im Deutschen liegen, während auf der Theorieseite die Familie der Baumadjunktionsgrammatiken (TAG) im Vordergrund steht. TAG ist bisher vor allem für seine besonderen computerlinguistischen Eigenschaften bekannt; dieses Buch zeigt anhand zahlreicher Analysen, dass TAG auch valenztheoretisch heraussticht und den Weg zu empirisch neutraleren Syntaxmodellen weist.
}

\renewcommand{\lsSeries}{eotms} % use lowercase acronym, e.g. sidl, eotms, tgdi
\renewcommand{\lsSeriesNumber}{1} %will be assigned when the book enters the proofreading stage
\renewcommand{\lsURL}{http://langsci-press.org/catalog/book/20} % contact the coordinator for the right number