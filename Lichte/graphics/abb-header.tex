\documentclass[
	%pstricks=true
	,crop=true
	,varwidth=\maxdimen
	]{standalone}

\usepackage{tree-dvips,pst-tree, pst-node,color,graphics}
%\usepackage{vaucanson-g}

\usepackage{libertine}
\usepackage[T1]{fontenc}
\usepackage{ucs}
\usepackage[utf8x]{inputenc}
%\fontsize{10}{12}\selectfont
\usepackage[scaled=0.83]{beramono}
\usepackage[libertine]{newtxmath}

\usepackage{latexsym,amsmath,amssymb,wasysym}  
%\usepackage{ifsym}  	% fuer das Blitzsymbol  
\usepackage{marvosym}	% fuer das Blitzsymbol

\usepackage{ulem}  	% ulem für Durchstreichungen, Unterstreichungen etc.
\normalem			% Macht eine Änderung von ulem an \emph rückgängig.


\usepackage{german}

\usepackage{../LSP/lsp-styles/avm}
\avmfont{\sc} 
\avmvalfont{\it}

\newcommand{\dotted}[0]{\makedash{2pt}}
\avmfont{\sc} \avmvalfont{\rm} 
\newenvironment{avmnode}[1] { 
	\avmoptions{} 
	\avmvskip{.1ex}
	\avmfont{\sc} \avmvalfont{\rm}
	\hspace{-5mm}
	\begin{tabular}{c}
	#1 \\[-2.5ex]
	\begin{avm}
	\avml
	}
	%%%
	{
	\avmr
        \end{avm}\\[-0,5ex] 
	\end{tabular}
}
\newcommand{\svar}[1]
   {\setbox2=\hbox{$\scriptstyle #1$}\lower.2ex\vbox{\hrule
     \hbox{\vrule\kern1.25pt 
     \vbox{\kern1.25pt\box2\kern1.25pt}\kern1.25pt\vrule}\hrule}}
\newcommand{\circled}[1]{\textcircled{\raisebox{-0.4pt}{\small #1}}}
\newlength{\stmueTmp}
\newcommand*{\hspaceThis}[1]{\settowidth{\stmueTmp}{#1}\hspace*{\stmueTmp}}
